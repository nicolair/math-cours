%<dscrpt>Fichier de déclarations Latex à inclure au début d'un élément de cours.</dscrpt>

\documentclass[a4paper]{article}
\usepackage[hmargin={1.8cm,1.8cm},vmargin={2.4cm,2.4cm},headheight=13.1pt]{geometry}

%includeheadfoot,scale=1.1,centering,hoffset=-0.5cm,
\usepackage[pdftex]{graphicx,color}
\usepackage[french]{babel}
%\selectlanguage{french}
\addto\captionsfrench{
  \def\contentsname{Plan}
}
\usepackage{fancyhdr}
\usepackage{floatflt}
\usepackage{amsmath}
\usepackage{amssymb}
\usepackage{amsthm}
\usepackage{stmaryrd}
%\usepackage{ucs}
\usepackage[utf8]{inputenc}
%\usepackage[latin1]{inputenc}
\usepackage[T1]{fontenc}


\usepackage{titletoc}
%\contentsmargin{2.55em}
\dottedcontents{section}[2.5em]{}{1.8em}{1pc}
\dottedcontents{subsection}[3.5em]{}{1.2em}{1pc}
\dottedcontents{subsubsection}[5em]{}{1em}{1pc}

\usepackage[pdftex,colorlinks={true},urlcolor={blue},pdfauthor={remy Nicolai},bookmarks={true}]{hyperref}
\usepackage{makeidx}

\usepackage{multicol}
\usepackage{multirow}
\usepackage{wrapfig}
\usepackage{array}
\usepackage{subfig}


%\usepackage{tikz}
%\usetikzlibrary{calc, shapes, backgrounds}
%pour la présentation du pseudo-code
% !!!!!!!!!!!!!!      le package n'est pas présent sur le serveur sous fedora 16 !!!!!!!!!!!!!!!!!!!!!!!!
%\usepackage[french,ruled,vlined]{algorithm2e}

%pr{\'e}sentation du compteur de niveau 2 dans les listes
\makeatletter
\renewcommand{\labelenumii}{\theenumii.}
\renewcommand{\thesection}{\Roman{section}.}
\renewcommand{\thesubsection}{\arabic{subsection}.}
\renewcommand{\thesubsubsection}{\arabic{subsubsection}.}
\makeatother


%dimension des pages, en-t{\^e}te et bas de page
%\pdfpagewidth=20cm
%\pdfpageheight=14cm
%   \setlength{\oddsidemargin}{-2cm}
%   \setlength{\voffset}{-1.5cm}
%   \setlength{\textheight}{12cm}
%   \setlength{\textwidth}{25.2cm}
   \columnsep=1cm
   \columnseprule=0.5pt

%En tete et pied de page
\pagestyle{fancy}
\lhead{MPSI-\'Eléments de cours}
\rhead{\today}
%\rhead{25/11/05}
\lfoot{\tiny{Cette création est mise à disposition selon le Contrat\\ Paternité-Pas d'utilisations commerciale-Partage des Conditions Initiales à l'Identique 2.0 France\\ disponible en ligne http://creativecommons.org/licenses/by-nc-sa/2.0/fr/
} }
\rfoot{\tiny{Rémy Nicolai \jobname}}


\newcommand{\baseurl}{http://back.maquisdoc.net/data/cours\_nicolair/}
\newcommand{\urlexo}{http://back.maquisdoc.net/data/exos_nicolair/}
\newcommand{\urlcours}{https://maquisdoc-math.fra1.digitaloceanspaces.com/}

\newcommand{\N}{\mathbb{N}}
\newcommand{\Z}{\mathbb{Z}}
\newcommand{\C}{\mathbb{C}}
\newcommand{\R}{\mathbb{R}}
\newcommand{\D}{\mathbb{D}}
\newcommand{\K}{\mathbf{K}}
\newcommand{\Q}{\mathbb{Q}}
\newcommand{\F}{\mathbf{F}}
\newcommand{\U}{\mathbb{U}}
\newcommand{\p}{\mathbb{P}}


\newcommand{\card}{\mathop{\mathrm{Card}}}
\newcommand{\Id}{\mathop{\mathrm{Id}}}
\newcommand{\Ker}{\mathop{\mathrm{Ker}}}
\newcommand{\Vect}{\mathop{\mathrm{Vect}}}
\newcommand{\cotg}{\mathop{\mathrm{cotan}}}
\newcommand{\sh}{\mathop{\mathrm{sh}}}
\newcommand{\ch}{\mathop{\mathrm{ch}}}
\newcommand{\argsh}{\mathop{\mathrm{argsh}}}
\newcommand{\argch}{\mathop{\mathrm{argch}}}
\newcommand{\tr}{\mathop{\mathrm{tr}}}
\newcommand{\rg}{\mathop{\mathrm{rg}}}
\newcommand{\rang}{\mathop{\mathrm{rg}}}
\newcommand{\Mat}{\mathop{\mathrm{Mat}}}
\newcommand{\MatB}[2]{\mathop{\mathrm{Mat}}_{\mathcal{#1}}\left( #2\right) }
\newcommand{\MatBB}[3]{\mathop{\mathrm{Mat}}_{\mathcal{#1} \mathcal{#2}}\left( #3\right) }
\renewcommand{\Re}{\mathop{\mathrm{Re}}}
\renewcommand{\Im}{\mathop{\mathrm{Im}}}
\renewcommand{\th}{\mathop{\mathrm{th}}}
\newcommand{\repere}{$(O,\overrightarrow{i},\overrightarrow{j},\overrightarrow{k})$}
\newcommand{\cov}{\mathop{\mathrm{Cov}}}

\newcommand{\absolue}[1]{\left| #1 \right|}
\newcommand{\fonc}[5]{#1 : \begin{cases}#2 \rightarrow #3 \\ #4 \mapsto #5 \end{cases}}
\newcommand{\depar}[2]{\dfrac{\partial #1}{\partial #2}}
\newcommand{\norme}[1]{\left\| #1 \right\|}
\newcommand{\se}{\geq}
\newcommand{\ie}{\leq}
\newcommand{\trans}{\mathstrut^t\!}
\newcommand{\val}{\mathop{\mathrm{val}}}
\newcommand{\grad}{\mathop{\overrightarrow{\mathrm{grad}}}}

\newtheorem*{thm}{Théorème}
\newtheorem{thmn}{Théorème}
\newtheorem*{prop}{Proposition}
\newtheorem{propn}{Proposition}
\newtheorem*{pa}{Présentation axiomatique}
\newtheorem*{propdef}{Proposition - Définition}
\newtheorem*{lem}{Lemme}
\newtheorem{lemn}{Lemme}

\theoremstyle{definition}
\newtheorem*{defi}{Définition}
\newtheorem*{nota}{Notation}
\newtheorem*{exple}{Exemple}
\newtheorem*{exples}{Exemples}


\newenvironment{demo}{\renewcommand{\proofname}{Preuve}\begin{proof}}{\end{proof}}
%\renewcommand{\proofname}{Preuve} doit etre après le begin{document} pour fonctionner

\theoremstyle{remark}
\newtheorem*{rem}{Remarque}
\newtheorem*{rems}{Remarques}

\renewcommand{\indexspace}{}
\renewenvironment{theindex}
  {\section*{Index} %\addcontentsline{toc}{section}{\protect\numberline{0.}{Index}}
   \begin{multicols}{2}
    \begin{itemize}}
  {\end{itemize} \end{multicols}}


%pour annuler les commandes beamer
\renewenvironment{frame}{}{}
\newcommand{\frametitle}[1]{}
\newcommand{\framesubtitle}[1]{}

\newcommand{\debutcours}[2]{
  \chead{#1}
  \begin{center}
     \begin{huge}\textbf{#1}\end{huge}
     \begin{Large}\begin{center}Rédaction incomplète. Version #2\end{center}\end{Large}
  \end{center}
  %\section*{Plan et Index}
  %\begin{frame}  commande beamer
  \tableofcontents
  %\end{frame}   commande beamer
  \printindex
}


\makeindex
\begin{document}
\noindent

\debutcours{Propriétés globales des fonctions continues}{0.1 \tiny{ du \today} }


Ce document traite des fonctions continues sur un intervalle. L'étude des \href{\baseurl C2063.pdf}{fonctions d'une variable réelle} est réparti entre divers documents.
\section{Opérations. Espace des fonctions continues}
Rappelons qu'une fonction définie sur un intervalle $I$ est dite continue sur $I$ lorsque, pour tout $a\in I$, $f$ est continue en $a$.
\begin{nota}
 L'ensemble des fonctions continues sur un intervalle est noté $\mathcal C (I,\R)$ ou $\mathcal C^0 (I,\R)$ ou plus simplement $\mathcal C (I)$ ou $\mathcal C^0 (I)$. 
\end{nota}
\begin{prop}
 Soit $f$ et $g$ deux fonctions continues sur $I$ et $\lambda$ un nombre réel. Les fonctions $|f|$, $f+g$, $\lambda f$, $fg$, $\sup(f,g)$, $\inf(f,g)$ sont continues sur $I$.
\end{prop}
\begin{rem}
 Cela prouve en particulier que $\mathcal C (I)$ est un sous-espace vectoriel de $\mathcal F(I,\R)$. 
\end{rem}
\begin{demo}
 Un théorème analogue relatif aux fonctions convergentes en un point a été vu dans la partie étude locale. Il suffit de l'applique en chaque point de l'intervalle.
\end{demo}
\begin{prop}
 Soit $f$ une fonction continue dans un intervalle $I$ et  à valeurs dans un intervalle $J$. Soit $g$ une fonction continue dans $J$. La fonction $g\circ f$ est alors continue dans $I$.
\end{prop}
\begin{prop}
 Soit $f$ une fonction continue dans un intervalle $I$ et $J$ un intervalle inclus dans $I$. La restriction $f_{|J}$ est alors continue dans $J$.
\end{prop}
\index{prolongement par continuité}
\begin{prop}[prolongement par continuité]
 Soit $I$ un intervalle ouvert en une de ses extrémités $a$ et $f$ une fonction continue dans $I$. Soit $J = I\cup\{a\}$. Il existe une fonction $g$ continue dans $J$ dont la restriction à $I$ est égale à $f$ si et seulement si $f$ admet une limite finie en $a$ (notée $l$).\newline
Lorsqu'une telle fonction $g$ existe, elle est unique et appelée le prolongement par continuité de $f$. Elle est définie par :
\begin{displaymath}
 \forall x\in J : g(x)=\left\lbrace 
\begin{aligned}
 f(x) &\text{ si } x\in I \\
 l &\text{ si } x=a
\end{aligned}
\right. 
\end{displaymath}
\end{prop}
\begin{demo}
 à compléter
\end{demo}


\section{Image d'un intervalle}
\begin{propn} \label{proptvi}
Soit $\varphi$ une fonction continue sur un intervalle $I$ et $a<b$ deux éléments de $I$ tels que $\varphi(a)<\varphi(b)$. Soit $\lambda \in [\varphi(a),\varphi(b)]$. Il existe alors $c\in]a,b[$ tel que $\varphi(c)=\lambda$.
\end{propn}
\begin{demo}
\index{question de cours!théorème de la valeur intermédiaire (avec démonstration)} \index{théorème de la valeur intermédiaire}
 Notons $A=\left\lbrace x\in[a,b] \text{ tels que } \varphi(x)\leq \lambda \right\rbrace$.\newline
Comme $A$ est une partie de $[a,b]$, elle est évidemment bornée. Elle est non vide car l'hypothèse $\varphi(a)<\lambda$ entraine $a\in A$. Cette partie admet donc une borne supérieure\index{borne supérieure} notée $c$.
\begin{displaymath}
 c = \sup\left\lbrace x\in[a,b] \text{ tels que } \varphi(x)\leq \lambda \right\rbrace
\end{displaymath}
Comme la fonction $\varphi$ est continue en $a$ avec $\varphi(a)<\lambda$, il existe un $\alpha >0$ tel que $\varphi(x)<\lambda$ pour tout $x$ dans $[a,a+\alpha[$. On en déduit que $a<c$.\newline
Considérons ce qui se passe à gauche de $c$. Pour tout $n\in\N^*$, $c-\frac{1}{n}$ n'est pas un majorant de $A$ (car la borne supérieure est le plus petit des majorants). Il existe donc un élément $a_n\in A$ tel que
\begin{displaymath}
 c-\frac{1}{n} < a_n \leq c
\end{displaymath}
On en déduit d'après le théorème d'encadrement\index{théorème d'encadrement} que $(a_n)_{n\in \N*}$ converge vers $c$.\newline
Comme $\varphi$ est continue en $c$, $(\varphi(a_n))_{n\in \N*}$ converge vers $\varphi(c)$. D'autre part, par définition de $A$, 
\begin{displaymath}
 \forall n\in \N^* : \varphi(a_n) \leq \lambda
\end{displaymath}
Le théorème de passage à la limite dans une inégalité entraine alors $\varphi(c)\leq \lambda$.\newline
Examinons maintenant ce qui se passe à droite de $c$.\newline
Remarquons d'abord que l'inégalité $\varphi(c)\leq \lambda$ que l'on vient de montrer implique $c<b$ car, par hypothèse, $\lambda < \varphi(b)$. L'intervalle $]c,b]$ est donc non vide. Pour tout $x$ dans cet intervalle, comme $x\leq c$ est faux et que $c$ est un majorant de $A$, on est certain que $x\not \in A$ c'est à dire $\varphi(x)>\lambda$. Autrement dit
\begin{displaymath}
 \forall x \in ]c,b] : \varphi_{|]c,b]}>\lambda
\end{displaymath}
Par le théorème de passage à la limite dans une inégalité, ceci entraine que \emph{la limite à droite stricte} de $\varphi$ en $c$ est supérieure ou égale à $\lambda$. Or cette limite est $\varphi(c)$ car $\varphi$ est continue en $c$. On obtient donc bien la deuxième inégalité
\begin{displaymath}
 \lambda \leq \varphi(c)
\end{displaymath}
\end{demo}

\begin{thm}[Théorème de la valeur intermédiaire]
Les trois formulations suivantes sont équivalentes.
 \begin{description}
  \item[formulation 1] Soit $I$ un intervalle, $f$ une fonction continue sur $I$, $a$ et $b$ deux éléments de $I$ tels que $f(a)f(b)<0$. Il existe $c\in \left] \min(a,b), \max(a,b)\right[$ tel que $f(c)=0$.
  \item[formulation 2] Soit $I$ un intervalle, $a$ et $b$ deux éléments de $I$, $f$ continue sur $I$ et $\lambda \in \left[ m, M\right] $ avec $m = \min(f(a),f(b))$ et $M = \max(f(a),f(b))$. Il existe $c\in \left[ \min(a,b),\max(a,b)\right] $ tel que $f(c)=\lambda$.
  \item[formulation 3] Soit $I$ un intervalle et $f$ une fonction continue sur $I$. Alors $f(I)$ est un intervalle.
 \end{description}
\end{thm}
\begin{demo}
 \begin{itemize}
  \item La propriété \ref{proptvi} entraine les formulations 1 et 2. Il suffit de considérer une fonction auxiliaire adéquate.
  \item La formulation 2 entraine que $f(I)$ convexe donc c'est un intervalle. (voir \href{\baseurl C2192.pdf}{Axiomatique du corps des réels})
 \end{itemize}
\end{demo}
\begin{rem}
 Parmi les conséquences classiques, pour les fonctions continues sur un intervalle on peut citer : si la fonction ne s'annule pas elle garde un signe constant ou encore si elle est à valeurs entières elle est constante.
\end{rem}

\begin{thm}
  Toute fonction continue et injective sur un intervalle est strictement monotone.
\end{thm}
\begin{demo}
La démonstration n'est pas exigible, elle est traitée sous forme d'\href{http://back.maquisdoc.net/v-1/index.php?act=chelt&id_elt=5207}{exercice}.  
\end{demo}

\section{Image d'un segment}
\index{question de cours!image d'un segment (avec démonstration)}
\begin{thm}
 Une fonction continue sur un segment est bornée et elle atteint ses bornes (autrement dit elle admet un maximum et un minimum global).
\end{thm}
\begin{demo}
 Considérons une fonction $f$ continue sur un intervalle $[a,b]$ avec $a<b$.
\begin{enumerate}
 \item On va montrer par l'absurde que $f$ est majorée. \newline
Si $f$ est non majorée, alors, aucun entier naturel $n$ n'est un majorant de $f$. Donc pour tout $n\in \N$, il existe un $x_n\in[a,b]$ tel que $n<f(x_n)$. On a formé ainsi une suite $\left( x_n\right) _{n\in \N}$ d'éléments de $[a,b]$ telle que $f(x_n)>n$ pour tous les $n$.\newline
Cette suite est bornée car elle est à valeurs dans $[a,b]$. D'après le théorème de Bolzano Weirstrass\index{théorème de Bolzano Weirstrass}, on peut en extraire une suite convergente. Il existe donc une partie infinie $\mathcal I$ de $\N$ et un réel $x$ tels que la suite extraite $\left( x_n\right) _{n\in \mathcal I}$ converge vers $x$.\newline
De $a\leq x_n \leq b$ pour tout $n$ de $\mathcal{I}$, on tire $x\in[a,b]$ par le théorème de passage à la limite dans une inégalité\index{le théorème de passage à la limite dans une inégalité}.\newline
Comme $f$ est continue en $x$, on peut déduire que $\left( f(x_n)\right) _{n\in \mathcal I}$ converge vers $f(x)$.\newline
Ceci est en contradiction avec le fait que $n\leq f(x_n)$ pour tous les $n$ de $\mathcal I$ ce qui entraine la divergence vers $+\infty$. 

\item On peut montrer de la même manière que la fonction est minorée. Elle est donc bornée.

\item Comme la fonction est bornée, elle admet une borne inférieure $m$. Montrons qu'il existe un $x\in[a,b]$ tel que $f(x)=m$.\newline
Pour tout $n$ non nul, $m+\frac{1}{n}$ n'est pas un minorant de $f$. Il existe donc une suite $\left( x_n\right) _{n\in \N^*}$ d'éléments de $[a,b]$ tels que $m \leq f(x_n)\leq m+\frac{1}{n}$.\newline
Comme cette suite est à valeurs dans $[a,b]$, elle est bornée. D'après le théorème de Bozano Weirstrass, on peut en extraire une suite $\left( x_n\right) _{n\in \mathcal I}$ qui converge vers un réel $x$.\newline
Comme plus haut, d'après le théorème de passage à la limite dans une inégalité, $x$ est dans $[a,b]$. La fonction $f$ étant continue en $a$, la suite $\left( f(x_n)\right) _{n\in \mathcal I}$ converge vers $f(x)$.\newline
D'autre part, encore une fois par passage à la limite dans une inégaité, de $m \leq f(x_n)\leq m+\frac{1}{n}$, on tire $f(x)=m$ ce qu'on voulait démontrer.

\item On démontre de manière analogue que la fonction atteint sa borne supérieure. 
\end{enumerate}
\end{demo}
\begin{rem}
 Comme on sait déjà que l'image continue d'un intervalle est un intervalle, on peut en déduire que si $f$ est continue sur un segment $[a,b]$, il existe des réels $m$ et $M$, des éléments $x_{min}$ et $x_{max}$ de $[a,b]$ tels que $f([a,b])=[m,M]$ avec $m=f(x_{min})$, $M=f(x_{max})$.
\end{rem}

\section{Continuité d'une bijection réciproque}
L'étude locale des fonction strictement monotones montre le théorème suivant:
\begin{thm}[continuité d'une application monotone sur un intervalle]
 Soit $f$ une fonction monotone définie dans un intervalle $I$. Si $f(I)$ est un intervalle, la fonction $f$ est continue dans $I$.
\end{thm}
\begin{demo}
  à rédiger à partir de l'étude locale des fonctions croissantes. Si une fonction strictement monotone est discontinue en un point alors l'image de l'intervalle \emph{n'est pas un intervalle}.
\end{demo}

\begin{thm}
 Soit $I$ un intervalle et $f$ une fonction continue et strictement monotone sur $I$. Alors $f(I)=J$ est un intervalle et la corestriction de $f$ à $J$ est une bijection de $I$ vers $J$. Sa bijection réciproque est strictement monotone de même sens que $f$ et elle est continue.
\end{thm}
\begin{demo}
 Toute application strictement monotone est injective. La corestriction de $f$ à l'image de $f$ est évidemment surjective, elle est donc bijective. Il est clair que sa bijection réciproque est strictement monotone de même sens. Notons la $g$. La fonction $g$ est strictement monotone et définie dans $I$. Rappelons le résultat sur la \href{\baseurl C2064.pdf}{continuité d'une application monotone définie dans un intervalle} $J$. Une telle application est continue lorsque $g(J)$ est un intervalle. C'est évidemment le cas ici puisque $g(J)$ est l'intervalle $I$ de départ.
\end{demo}

D'après l'étude locale des fonctions monotones, une fonction monotone sur un intervalle est continue si et seulement si l'image de l'intervalle est un intervalle.

\end{document}

