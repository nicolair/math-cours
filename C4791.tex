%<dscrpt>Fichier de déclarations Latex à inclure au début d'un élément de cours.</dscrpt>

\documentclass[a4paper]{article}
\usepackage[hmargin={1.8cm,1.8cm},vmargin={2.4cm,2.4cm},headheight=13.1pt]{geometry}

%includeheadfoot,scale=1.1,centering,hoffset=-0.5cm,
\usepackage[pdftex]{graphicx,color}
\usepackage[french]{babel}
%\selectlanguage{french}
\addto\captionsfrench{
  \def\contentsname{Plan}
}
\usepackage{fancyhdr}
\usepackage{floatflt}
\usepackage{amsmath}
\usepackage{amssymb}
\usepackage{amsthm}
\usepackage{stmaryrd}
%\usepackage{ucs}
\usepackage[utf8]{inputenc}
%\usepackage[latin1]{inputenc}
\usepackage[T1]{fontenc}


\usepackage{titletoc}
%\contentsmargin{2.55em}
\dottedcontents{section}[2.5em]{}{1.8em}{1pc}
\dottedcontents{subsection}[3.5em]{}{1.2em}{1pc}
\dottedcontents{subsubsection}[5em]{}{1em}{1pc}

\usepackage[pdftex,colorlinks={true},urlcolor={blue},pdfauthor={remy Nicolai},bookmarks={true}]{hyperref}
\usepackage{makeidx}

\usepackage{multicol}
\usepackage{multirow}
\usepackage{wrapfig}
\usepackage{array}
\usepackage{subfig}


%\usepackage{tikz}
%\usetikzlibrary{calc, shapes, backgrounds}
%pour la présentation du pseudo-code
% !!!!!!!!!!!!!!      le package n'est pas présent sur le serveur sous fedora 16 !!!!!!!!!!!!!!!!!!!!!!!!
%\usepackage[french,ruled,vlined]{algorithm2e}

%pr{\'e}sentation du compteur de niveau 2 dans les listes
\makeatletter
\renewcommand{\labelenumii}{\theenumii.}
\renewcommand{\thesection}{\Roman{section}.}
\renewcommand{\thesubsection}{\arabic{subsection}.}
\renewcommand{\thesubsubsection}{\arabic{subsubsection}.}
\makeatother


%dimension des pages, en-t{\^e}te et bas de page
%\pdfpagewidth=20cm
%\pdfpageheight=14cm
%   \setlength{\oddsidemargin}{-2cm}
%   \setlength{\voffset}{-1.5cm}
%   \setlength{\textheight}{12cm}
%   \setlength{\textwidth}{25.2cm}
   \columnsep=1cm
   \columnseprule=0.5pt

%En tete et pied de page
\pagestyle{fancy}
\lhead{MPSI-\'Eléments de cours}
\rhead{\today}
%\rhead{25/11/05}
\lfoot{\tiny{Cette création est mise à disposition selon le Contrat\\ Paternité-Pas d'utilisations commerciale-Partage des Conditions Initiales à l'Identique 2.0 France\\ disponible en ligne http://creativecommons.org/licenses/by-nc-sa/2.0/fr/
} }
\rfoot{\tiny{Rémy Nicolai \jobname}}


\newcommand{\baseurl}{http://back.maquisdoc.net/data/cours\_nicolair/}
\newcommand{\urlexo}{http://back.maquisdoc.net/data/exos_nicolair/}
\newcommand{\urlcours}{https://maquisdoc-math.fra1.digitaloceanspaces.com/}

\newcommand{\N}{\mathbb{N}}
\newcommand{\Z}{\mathbb{Z}}
\newcommand{\C}{\mathbb{C}}
\newcommand{\R}{\mathbb{R}}
\newcommand{\D}{\mathbb{D}}
\newcommand{\K}{\mathbf{K}}
\newcommand{\Q}{\mathbb{Q}}
\newcommand{\F}{\mathbf{F}}
\newcommand{\U}{\mathbb{U}}
\newcommand{\p}{\mathbb{P}}


\newcommand{\card}{\mathop{\mathrm{Card}}}
\newcommand{\Id}{\mathop{\mathrm{Id}}}
\newcommand{\Ker}{\mathop{\mathrm{Ker}}}
\newcommand{\Vect}{\mathop{\mathrm{Vect}}}
\newcommand{\cotg}{\mathop{\mathrm{cotan}}}
\newcommand{\sh}{\mathop{\mathrm{sh}}}
\newcommand{\ch}{\mathop{\mathrm{ch}}}
\newcommand{\argsh}{\mathop{\mathrm{argsh}}}
\newcommand{\argch}{\mathop{\mathrm{argch}}}
\newcommand{\tr}{\mathop{\mathrm{tr}}}
\newcommand{\rg}{\mathop{\mathrm{rg}}}
\newcommand{\rang}{\mathop{\mathrm{rg}}}
\newcommand{\Mat}{\mathop{\mathrm{Mat}}}
\newcommand{\MatB}[2]{\mathop{\mathrm{Mat}}_{\mathcal{#1}}\left( #2\right) }
\newcommand{\MatBB}[3]{\mathop{\mathrm{Mat}}_{\mathcal{#1} \mathcal{#2}}\left( #3\right) }
\renewcommand{\Re}{\mathop{\mathrm{Re}}}
\renewcommand{\Im}{\mathop{\mathrm{Im}}}
\renewcommand{\th}{\mathop{\mathrm{th}}}
\newcommand{\repere}{$(O,\overrightarrow{i},\overrightarrow{j},\overrightarrow{k})$}
\newcommand{\cov}{\mathop{\mathrm{Cov}}}

\newcommand{\absolue}[1]{\left| #1 \right|}
\newcommand{\fonc}[5]{#1 : \begin{cases}#2 \rightarrow #3 \\ #4 \mapsto #5 \end{cases}}
\newcommand{\depar}[2]{\dfrac{\partial #1}{\partial #2}}
\newcommand{\norme}[1]{\left\| #1 \right\|}
\newcommand{\se}{\geq}
\newcommand{\ie}{\leq}
\newcommand{\trans}{\mathstrut^t\!}
\newcommand{\val}{\mathop{\mathrm{val}}}
\newcommand{\grad}{\mathop{\overrightarrow{\mathrm{grad}}}}

\newtheorem*{thm}{Théorème}
\newtheorem{thmn}{Théorème}
\newtheorem*{prop}{Proposition}
\newtheorem{propn}{Proposition}
\newtheorem*{pa}{Présentation axiomatique}
\newtheorem*{propdef}{Proposition - Définition}
\newtheorem*{lem}{Lemme}
\newtheorem{lemn}{Lemme}

\theoremstyle{definition}
\newtheorem*{defi}{Définition}
\newtheorem*{nota}{Notation}
\newtheorem*{exple}{Exemple}
\newtheorem*{exples}{Exemples}


\newenvironment{demo}{\renewcommand{\proofname}{Preuve}\begin{proof}}{\end{proof}}
%\renewcommand{\proofname}{Preuve} doit etre après le begin{document} pour fonctionner

\theoremstyle{remark}
\newtheorem*{rem}{Remarque}
\newtheorem*{rems}{Remarques}

\renewcommand{\indexspace}{}
\renewenvironment{theindex}
  {\section*{Index} %\addcontentsline{toc}{section}{\protect\numberline{0.}{Index}}
   \begin{multicols}{2}
    \begin{itemize}}
  {\end{itemize} \end{multicols}}


%pour annuler les commandes beamer
\renewenvironment{frame}{}{}
\newcommand{\frametitle}[1]{}
\newcommand{\framesubtitle}[1]{}

\newcommand{\debutcours}[2]{
  \chead{#1}
  \begin{center}
     \begin{huge}\textbf{#1}\end{huge}
     \begin{Large}\begin{center}Rédaction incomplète. Version #2\end{center}\end{Large}
  \end{center}
  %\section*{Plan et Index}
  %\begin{frame}  commande beamer
  \tableofcontents
  %\end{frame}   commande beamer
  \printindex
}


\makeindex
\begin{document}
\noindent

\debutcours{Suites et fonctions à valeurs complexes}{alpha}

Dans cet exposé, on cherche autant que possible à utiliser les propositions et théorèmes usuels relatifs aux suites et fonctions à valeurs réelles. L'utilisation des $\varepsilon$, $\alpha$ est la plupart du temps inutile.

\section{Suites à valeurs complexes}
Dans toute cette section, le symbole $\mathcal I$ désigne une partie infinie de $\N$.
\begin{defi}
 On dira qu'une partie $\Omega$ de $\C$ est bornée si et seulement si il existe un réel $R>0$ tel que 
\begin{displaymath}
 \forall z\in \Omega : |z|\leq R
\end{displaymath}
Une suite à valeurs complexes est dite bornée lorsque l'ensemble de ses valeurs est une partie bornée de $\C$.
\end{defi}
\begin{rems}
\begin{enumerate}
 \item Une partie de $\C$ est bornée lorsqu'elle est contenue dans un certain disque centré à l'origine.
\item Une suite à valeurs complexes $(z_n)_{n\in \mathcal I}$ est bornée si et seulement si il existe un réel $R$ tel que :
\begin{displaymath}
 \forall n\in \mathcal I : |z_n|\leq R
\end{displaymath}
\end{enumerate}
\end{rems}

\begin{defi}
 Une suite à valeurs complexes $(z_n)_{n\in \mathcal I}$ converge vers un nombre complexe $z$ si et seulement si la suite réelle $(|z_n-z|)_{n\in \mathcal I}$ converge vers $0$. On notera 
\begin{displaymath}
 (z_n)_{n\in \mathcal I} \rightarrow z 
\end{displaymath}
\end{defi}

\begin{prop}
\begin{displaymath}
 (z_n)_{n\in \mathcal I} \rightarrow z \Rightarrow (|z_n|)_{n\in \mathcal I} \rightarrow |z|
\end{displaymath}
Toute suite convergente à valeurs complexes est bornée.
\end{prop}

\begin{prop}
 Soit $(z_n)_{n\in \mathcal I}$ et $(z'_n)_{n\in \mathcal I}$ deux suites à valeurs complexes qui convergent respectivement vers $z$ et $z'$. Soit $\lambda\in \C$  Alors :
\begin{align*}
 (\overline{z_n})_{n\in \mathcal I}\rightarrow \overline{z} & &
(z_n+z'_n)_{n\in \mathcal I} \rightarrow z+z'  & &
\lambda(z_n)_{n\in \mathcal I} \rightarrow \lambda z & &
(z_nz'_n)_{n\in \mathcal I} \rightarrow zz'
\end{align*}
\end{prop}
\begin{demo}
 Les résultats se déduisent des propriétés des suites réelles et des relations :
\begin{align*}
 |\overline{z_n} - \overline{z}| &= |z_n -z | \\
|\lambda z_n -\lambda z | &=|z||z_n -z |\\
|(z_n+z'_n)-(z+z')| &\leq |z_n -z | + |z'_n -z '|\\
|(z_nz'_n)-(zz')| &\leq |z'_n||z_n -z | + |z||z'_n -z '|
\end{align*}
\end{demo}
\begin{prop}
 La suite à valeurs complexes $(z_n)_{n\in \mathcal I}$ converge vers $z$ si et seulement si :
\begin{displaymath}
 \left\lbrace 
\begin{aligned}
 (\Re(z_n))_{n\in \mathcal I} &\rightarrow \Re z \\
(\Im(z_n))_{n\in \mathcal I} &\rightarrow \Im z
\end{aligned}
\right. 
\end{displaymath}
\end{prop}
\begin{demo}
 Si on suppose la convergence de la suite complexe, on utilise la convergence de la suite conjuguée puis les opérations pour obtenir les convergence des suites de parties réelles et imaginaires. 
\begin{align*}
 (\Re(z_n))_{n\in \mathcal I} &= \dfrac{1}{2}(z_n)_{n\in \mathcal I} + \dfrac{1}{2}(\overline{z_n})_{n\in \mathcal I}\\
(\Im(z_n))_{n\in \mathcal I} &= \dfrac{1}{2i}(z_n)_{n\in \mathcal I} - \dfrac{1}{2i}(\overline{z_n})_{n\in \mathcal I}
\end{align*}
Dans l'autre sens, si on suppose les convergences des suites de parties réelles et imaginaires, on obtient la suite complexe par combinaison :
\begin{displaymath}
 (z_n)_{n\in \mathcal I} = (\Re(z_n))_{n\in \mathcal I} + i(\Im z_n)_{n\in \mathcal I}
\end{displaymath}
\end{demo}
\index{théorème de Bolzano Weirstrass pour les suites complexes}
\begin{thm}[Théorème de Bolzano-Weirstass pour les suites complexes]
 De toute suite bornée à valeur complexe, on peut extraire une suite convergente.
\end{thm}
\begin{demo}
 Soit $(z_n)_{n\in \mathcal I}$ une suite bornée à valeurs complexes. Comme, pour tous les $n$, $|\Re z_n |\leq |z_n|$ et $|\Re z_n |\leq |z_n|$, les suites de parties réelles et imaginaires sont des suites \emph{bornées de nombres réels}.\newline
Appliquons le théorème de Bolzano-Weirstrass à la suite des parties réelles. Il existe donc une partie infinie $\mathcal J_1$ de $\mathcal I$ telle que  $(\Re(z_n))_{n\in \mathcal J_1}$ converge.\newline
La suite $(\Im(z_n))_{n\in \mathcal J_1}$ est encore une suite bornée de nombre réel. On peut appliquer une deuxième fois le théorème de Bolzano-Weirstrass. Il existe une partie infinie $\mathcal J_2$ de $\mathcal J_1$ telle que  $(\Im(z_n))_{n\in \mathcal J_2}$ converge.\newline
La suite $(\Re(z_n))_{n\in \mathcal J_2}$ est convergente car elle est extraite de $(\Re(z_n))_{n\in \mathcal J_1}$. Ainsi les deux suites 
\begin{align*}
 (\Re(z_n))_{n\in \mathcal J_2} & & (\Im(z_n))_{n\in \mathcal J_2}
\end{align*}
convergent ce qui assure la convergence de
\begin{displaymath}
 (z_n)_{n\in \mathcal J_2}
\end{displaymath}
\end{demo}
Pour le théorème de Bolzano-Weirstrass, voir la section sur les \href{\baseurl C2069.pdf}{suites de réels}.\newline
Le théorème de Bolzano-Weirstrass pour les suites à valeurs complexes joue un rôle capital dans la \href{\baseurl C5213.pdf}{démonstration du théorème de d'Alembert}. 

\section{Fonctions à valeurs complexes.}
Les fonctions d'une variable réelle et à valeurs complexes sont des cas particuliers de \href{\baseurl C1618.pdf}{courbes paramétrées}. Voir aussi l'entrée \emph{fonctions d'une variable réelle à valeurs vectorielles} du \href{\baseurl C4199.pdf}{Glossaire de début d'année}
\subsection{Fonctions continues}
\subsection{Fonctions dérivables}
Attention, le théorème des accroissements finis n'est pas valable pour une fonction à valeurs complexes. Il est très facile d'imaginer le mouvement qui revient à son point de départ sans jamais s'arrêter. En revanche, l'inégalité des accroissements finis reste valable sous la forme suivante.\index{inégalité des accroissements finis pour une fonction à valeurs complexes}
\begin{prop}[Inégalité des accroissements finis pour une fonction à valeurs complexes]
 Soit $f$ une fonction à valeurs complexes de classe $\mathcal C^1$ sur un segment $[a,b]$, soit $M$ un réel tel que $|f(t)\leq M$ pour tous les $t\in[a,b]$. Alors:
\begin{displaymath}
 |f(b)-f(a)|\leq M(b-a)
\end{displaymath}
\end{prop}
\begin{demo}
 On ne donne pas ici de démonstration de ce résultat. Une démonstration très simple sera obtenue en utilisant l'\href{\baseurl C2189.pdf}{intégration des fonctions à valeurs complexes} et la relation fondamentale liant \href{\baseurl C2190.pdf}{intégration et dérivation}.
\end{demo}


\end{document}
