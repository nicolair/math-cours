\subsection{Arithmétique dans l'ensemble des entiers relatifs}
\begin{itshape}L'objectif de ce chapitre est d'étudier les propriétés de la divisibilité des entiers et des congruences.
\end{itshape}

\subsubsubsection{a) Divisibilité et division euclidienne}
\begin{parcolumns}[rulebetween,distance=\parcoldist]{2}
  \colchunk{Divisibilité dans $\Z$, diviseurs, multiples.}
  \colchunk{Caractérisation des couples d'entiers associés.}
  \colplacechunks

  \colchunk{Théorème de la division euclidienne.}
  \colchunk{}
  \colplacechunks

 \end{parcolumns}

\subsubsubsection{b) PGCD et algorithme d'Euclide}
\begin{parcolumns}[rulebetween,distance=\parcoldist]{2}
  \colchunk{PGCD de deux entiers naturels dont l'un au moins est non nul. }
  \colchunk{Le PGCD de $a$ et $b$ est défini comme étant le plus grand élément (pour l'ordre naturel dans $\N$) de l'ensemble des diviseurs communs à $a$ et $b$.\newline
  Notation $a\wedge b$.}
  \colplacechunks

  \colchunk{Algorithme d'Euclide.}
  \colchunk{L'ensemble des diviseurs communs à $a$ et $b$ est égal à l'ensemble des diviseurs de $a\wedge b$.\newline
$a\wedge b$ est le plus grand élément (au sens de la divisibilité) de l'ensemble des diviseurs communs à $a$ et $b$.}
  \colplacechunks

  \colchunk{Extension au cas de deux entiers relatifs.}
  \colchunk{}
  \colplacechunks
  
  \colchunk{Relation de Bézout.}
  \colchunk{L'algorithme d'Euclide fournit une relation de Bézout.\newline
$\dbf$ I : algorithme d'Euclide étendu.\newline
 L'étude des idéaux de $\Z$ est hors programme.}
  \colplacechunks

  \colchunk{PPCM. }
  \colchunk{Notation $a \vee b$.\newline
Lien avec le PGCD.}
  \colplacechunks

 \end{parcolumns}

\subsubsubsection{c) Entiers premiers entre eux}
\begin{parcolumns}[rulebetween,distance=\parcoldist]{2}
  \colchunk{Couple d'entiers premiers entre eux. }
  \colchunk{}
  \colplacechunks

  \colchunk{Théorème de Bézout. }
  \colchunk{Forme irréductible d'un rationnel.}
  \colplacechunks

  \colchunk{Lemme de Gauss. }
  \colchunk{}
  \colplacechunks

  \colchunk{PGCD d'un nombre fini d'entiers, relation de Bézout. Entiers premiers entre eux dans leur ensemble, premiers entre eux deux à deux. }
  \colchunk{}
  \colplacechunks
 \end{parcolumns}

\subsubsubsection{d) Nombres premiers}
\begin{parcolumns}[rulebetween,distance=\parcoldist]{2}

  \colchunk{Nombre premier. }
  \colchunk{$\dbf$ I : crible d'Eratosthène.}
  \colplacechunks

  \colchunk{L'ensemble des nombres premiers est infini. }
  \colchunk{}
  \colplacechunks

  \colchunk{Existence et unicité de la décomposition d'un entier naturel non nul en produit de nombres premiers. }
  \colchunk{}
  \colplacechunks

  \colchunk{Pour $p$ premier, valuation $p$-adique. }
  \colchunk{Notation $v_p (n)$.

  Caractérisation de la divisibilité en termes de valuations $p$-adiques.

  Expressions du PGCD et du PPCM à l'aide des valuations $p$-adiques.}
  \colplacechunks

 \end{parcolumns}

\subsubsubsection{e) Congruences}
\begin{parcolumns}[rulebetween,distance=\parcoldist]{2}
  \colchunk{Relation de congruence modulo un entier sur $\Z$. }
  \colchunk{Notation $a \equiv b \ [n]$.}
  \colplacechunks

  \colchunk{Opérations sur les congruences : somme, produit.}
  \colchunk{Les anneaux $\Z / n \Z$ sont hors programme.}
  \colplacechunks

  \colchunk{Petit théorème de Fermat.}
  \colchunk{}
  \colplacechunks
 \end{parcolumns}
