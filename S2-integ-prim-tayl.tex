
\subsubsubsection{f) Calcul de primitives}
\begin{parcolumns}[rulebetween,distance=\parcoldist]{2}
  \colchunk{Primitives usuelles.}
  \colchunk{Sont exigibles les seules primitives mentionnées dans le chapitre \og Techniques fondamentales de calcul en analyse\fg.}
  \colplacechunks

  \colchunk{Calcul de primitives par intégration par parties, par changement de variable.}
  \colchunk{}
  \colplacechunks

  \colchunk{Utilisation de la décomposition en éléments simples pour calculer les primitives d'une fraction rationnelle.}
  \colchunk{On évitera tout excès de technicité.}
  \colplacechunks
\end{parcolumns}

\subsubsubsection{g) Formules de Taylor}
\begin{parcolumns}[rulebetween,distance=\parcoldist]{2}
  \colchunk{Pour une fonction $f$ de classe $\mathcal{C}^{n+1}$, formule de Taylor avec reste intégral au point $a$ à l'ordre $n$.}
  \colchunk{}
  \colplacechunks

  \colchunk{Inégalité de Taylor-Lagrange pour une fonction de classe $\mathcal{C}^{n+1}$.}
  \colchunk{L'égalité de Taylor-Lagrange est hors programme.}
  \colplacechunks

  \colchunk{}
  \colchunk{On soulignera la différence de nature entre la formule de Taylor-Young (locale) et les formules de Taylor globales (reste intégral et inégalité de Taylor-Lagrange).}
  \colplacechunks
\end{parcolumns}
