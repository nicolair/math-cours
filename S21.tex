%!  pour pdfLatex
\documentclass[a4paper]{article}
\usepackage[hmargin={1.5cm,1.5cm},vmargin={2.4cm,2.4cm},headheight=13.1pt]{geometry}

\usepackage[pdftex]{graphicx,color}
%\usepackage{hyperref}

\usepackage[utf8]{inputenc}
\usepackage[T1]{fontenc}
\usepackage{lmodern}
%\usepackage[frenchb]{babel}
\usepackage[french]{babel}

\usepackage{fancyhdr}
\pagestyle{fancy}

%\usepackage{floatflt}

\usepackage{parcolumns}
\setlength{\parindent}{0pt}
\usepackage{xcolor}

%pr{\'e}sentation des compteurs de section, ...
\makeatletter
%\renewcommand{\labelenumii}{\theenumii.}
\renewcommand{\thepart}{}
\renewcommand{\thesection}{}
\renewcommand{\thesubsection}{}
\renewcommand{\thesubsubsection}{}
\makeatother

\newcommand{\subsubsubsection}[1]{\bigskip \rule[5pt]{\linewidth}{2pt} \textbf{ \color{red}{#1} } \newline \rule{\linewidth}{.1pt}}
\newlength{\parcoldist}
\setlength{\parcoldist}{1cm}

\usepackage{maths}
\newcommand{\dbf}{\leftrightarrows}
% remplace les commandes suivantes 
%\usepackage{amsmath}
%\usepackage{amssymb}
%\usepackage{amsthm}
%\usepackage{stmaryrd}

%\newcommand{\N}{\mathbb{N}}
%\newcommand{\Z}{\mathbb{Z}}
%\newcommand{\C}{\mathbb{C}}
%\newcommand{\R}{\mathbb{R}}
%\newcommand{\K}{\mathbf{K}}
%\newcommand{\Q}{\mathbb{Q}}
%\newcommand{\F}{\mathbf{F}}
%\newcommand{\U}{\mathbb{U}}

%\newcommand{\card}{\mathop{\mathrm{Card}}}
%\newcommand{\Id}{\mathop{\mathrm{Id}}}
%\newcommand{\Ker}{\mathop{\mathrm{Ker}}}
%\newcommand{\Vect}{\mathop{\mathrm{Vect}}}
%\newcommand{\cotg}{\mathop{\mathrm{cotan}}}
%\newcommand{\sh}{\mathop{\mathrm{sh}}}
%\newcommand{\ch}{\mathop{\mathrm{ch}}}
%\newcommand{\argsh}{\mathop{\mathrm{argsh}}}
%\newcommand{\argch}{\mathop{\mathrm{argch}}}
%\newcommand{\tr}{\mathop{\mathrm{tr}}}
%\newcommand{\rg}{\mathop{\mathrm{rg}}}
%\newcommand{\rang}{\mathop{\mathrm{rg}}}
%\newcommand{\Mat}{\mathop{\mathrm{Mat}}}
%\renewcommand{\Re}{\mathop{\mathrm{Re}}}
%\renewcommand{\Im}{\mathop{\mathrm{Im}}}
%\renewcommand{\th}{\mathop{\mathrm{th}}}


%En tete et pied de page
\lhead{Programme colle math}
\chead{Semaine 21 du 16/03/20 au 21/03/20}
\rhead{MPSI B Hoche}

\lfoot{\tiny{Cette création est mise à disposition selon le Contrat\\ Paternité-Partage des Conditions Initiales à l'Identique 2.0 France\\ disponible en ligne http://creativecommons.org/licenses/by-sa/2.0/fr/
} }
\rfoot{\tiny{Rémy Nicolai \jobname}}


\begin{document}
\subsection{Intégration (fin)}

\subsubsubsection{f) Calcul de primitives}
\begin{parcolumns}[rulebetween,distance=\parcoldist]{2}
  \colchunk{Primitives usuelles.}
  \colchunk{Sont exigibles les seules primitives mentionnées dans le chapitre \og Techniques fondamentales de calcul en analyse\fg.}
  \colplacechunks

  \colchunk{Calcul de primitives par intégration par parties, par changement de variable.}
  \colchunk{}
  \colplacechunks

  \colchunk{Utilisation de la décomposition en éléments simples pour calculer les primitives d'une fraction rationnelle.}
  \colchunk{On évitera tout excès de technicité.}
  \colplacechunks
\end{parcolumns}

\subsubsubsection{g) Formules de Taylor}
\begin{parcolumns}[rulebetween,distance=\parcoldist]{2}
  \colchunk{Pour une fonction $f$ de classe $\mathcal{C}^{n+1}$, formule de Taylor avec reste intégral au point $a$ à l'ordre $n$.}
  \colchunk{}
  \colplacechunks

  \colchunk{Inégalité de Taylor-Lagrange pour une fonction de classe $\mathcal{C}^{n+1}$.}
  \colchunk{L'égalité de Taylor-Lagrange est hors programme.}
  \colplacechunks

  \colchunk{}
  \colchunk{On soulignera la différence de nature entre la formule de Taylor-Young (locale) et les formules de Taylor globales (reste intégral et inégalité de Taylor-Lagrange).}
  \colplacechunks
\end{parcolumns}

\subsection{Matrices}
\subsubsection{A - Calcul Matriciel}

\subsubsubsection{a) Espaces de matrices}
\begin{parcolumns}[rulebetween,distance=2.5cm]{2}
  \colchunk{Espace vectoriel $\mat np\K$ des matrices à $n$ lignes et $p$ colonnes à coefficients dans $\K$.}
  \colchunk{}
  \colplacechunks
   \colchunk{Base canonique de $\mat np\K$.}
  \colchunk{Dimension de $\mat np\K$.}
  \colplacechunks
\end{parcolumns}

\subsubsubsection{b) Produit matriciel}
\begin{parcolumns}[rulebetween,distance=2.5cm]{2}

   \colchunk{Bilinéarité, associativité.}
  \colchunk{}
  \colplacechunks
   \colchunk{Produit d'une matrice de la base canonique de $\mat np\K$ par une matrice de la base canonique de $\mat pq\K$.}
  \colchunk{}
  \colplacechunks
   \colchunk{Anneau $\matc n\K$.}
  \colchunk{Non commutativité si $n\ge 2$. Exemples de diviseurs de zéro et de matrices nilpotentes.}
  \colplacechunks
   \colchunk{Formule du binôme.}
  \colchunk{Application au calcul de puissances.}
  \colplacechunks
   \colchunk{Matrice inversible, inverse. Groupe linéaire.}
  \colchunk{Notation $\Gl_n(\K)$.}
  \colplacechunks
   \colchunk{Produit de matrices diagonales, de matrices triangulaires supérieures, inférieures.}
  \colchunk{}
  \colplacechunks
\end{parcolumns}

\subsubsubsection{c) Transposition}
\begin{parcolumns}[rulebetween,distance=2.5cm]{2}

   \colchunk{Transposée d'une matrice.}
  \colchunk{Notations $^tA$, $A^T$}
  \colplacechunks
   \colchunk{Opérations sur les transposées~: combinaison linéaire, produit, inverse.}
  \colchunk{}
  \colplacechunks
\end{parcolumns}


\bigskip
\begin{center}
 \textbf{Questions de cours}
\end{center}
Intégration par parties.\newline
Changement de variable.
Lemme d'intégration de la négligeabilité et formule de Taylor-Young.\newline
Formule de Taylor avec reste intégral.\newline
Associativité du produit matriciel.\newline
Transposée d'une matrice produit, inverse.

\begin{center}
 \textbf{Prochain programme}
\end{center}
Matrices et applications linéaires. Changements de base, équivalence, similitude.
\end{document}
