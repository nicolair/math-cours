\subsection{Séries numériques}
\begin{itshape}
L'étude des séries prolonge celle des suites. Elle permet d'illustrer le chapitre \og Analyse  asymptotique\fg
$\;$ et, à travers la notion de développement décimal  de mieux appréhender les nombres réels.

L'objectif majeur est la maîtrise de la convergence absolue ; tout excès de technicité est exclu.
\end{itshape}

\subsubsubsection{a) Généralités}
\begin{parcolumns}[rulebetween,distance=\parcoldist]{2}
  \colchunk{Sommes partielles. Convergence, divergence. Somme et restes d'une série convergente.}
  \colchunk{La série est notée $\displaystyle{\sum u_n}$. En cas de convergence, sa somme est notée $\displaystyle\sum_{n=0}^{+\infty}u_n$.}
  \colplacechunks

  \colchunk{Linéarité de la somme.}
  \colchunk{}
  \colplacechunks

  \colchunk{Le terme général d'une série convergente tend vers 0.}
  \colchunk{Divergence grossière.}
  \colplacechunks

  \colchunk{Séries géométriques : condition nécessaire et suffisante de convergence, somme.}
  \colchunk{}
  \colplacechunks

  \colchunk{Lien suite-série.}
  \colchunk{La suite $(u_n)$ et la série $\displaystyle \displaystyle\sum_{}^{}(u_{n+1}~-~u_n)$ ont même nature.}
  \colplacechunks
\end{parcolumns}

\subsubsubsection{b) Séries à termes positifs}
\begin{parcolumns}[rulebetween,distance=\parcoldist]{2}
  \colchunk{Une série à termes positifs converge si et seulement si la suite de ses sommes partielles est majorée.}
  \colchunk{}
  \colplacechunks

  \colchunk{Si $0 \leqslant u_n \leqslant v_n$ pour tout $n$, la convergence de $\displaystyle \sum v_n$ implique celle de $\displaystyle \sum u_n$.}
  \colchunk{}
  \colplacechunks

  \colchunk{Si $(u_n)_{n \in \N}$ et $(v_n)_{n \in \N}$ sont positives et si $u_n \sim v_n$, les séries $\displaystyle \sum u_n$ et $\displaystyle \sum v_n$ ont même nature.}
  \colchunk{}
  \colplacechunks
\end{parcolumns}

\subsubsubsection{c) Comparaison série-intégrale dans le cas monotone}
\begin{parcolumns}[rulebetween,distance=\parcoldist]{2}
  \colchunk{Si $f$ est monotone, encadrement des sommes partielles de $\displaystyle \sum f (n)$ à l'aide de la méthode des rectangles.}
  \colchunk{Application à l'étude de sommes partielles et de restes.}
  \colplacechunks

  \colchunk{Séries de Riemann.}
  \colchunk{}
  \colplacechunks
\end{parcolumns}

\subsubsubsection{d) Séries absolument convergentes}
\begin{parcolumns}[rulebetween,distance=\parcoldist]{2}
  \colchunk{Convergence absolue.}
  \colchunk{}
  \colplacechunks

  \colchunk{La convergence absolue implique la convergence.}
  \colchunk{Le critère de Cauchy est hors programme. La convergence de la série absolument convergente $\sum u_n$ est établie à partir de celles de $\sum {u_n}^+$ et $\sum {u_n}^-$.}
  \colplacechunks

  \colchunk{Si $(u_n)$ est une suite complexe, si $(v_n)$ est une suite d'éléments de $\R^+$, si $u_n=O(v_n)$ et si $\displaystyle{\sum v_n}$ converge, alors $\displaystyle{\sum u_n}$ est absolument convergente donc convergente.}
  \colchunk{}
  \colplacechunks
\end{parcolumns}

\subsubsubsection{e) Représentation décimale des réels}
\begin{parcolumns}[rulebetween,distance=\parcoldist]{2}
  \colchunk{Existence et unicité du développement décimal propre d'un réel.}
  \colchunk{La démonstration n'est pas exigible.}
  \colplacechunks

  \colchunk{}
  \colchunk{}
  \colplacechunks

  \colchunk{}
  \colchunk{}
  \colplacechunks

  \colchunk{}
  \colchunk{}
  \colplacechunks
\end{parcolumns}
