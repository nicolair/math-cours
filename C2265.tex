\input{courspdf.tex}
\debutcours{\'Etude métrique des courbes}{alpha}

\section{Introduction. Définitions}
\subsection{Changement de paramètre admissible}
\begin{figure}[ht]
 \centering
\input{C2265_5.pdf_t}
\caption{changement de paramètre admissible}
\label{fig:C2265_5}
\end{figure}

\subsection{Repère de Frénet}

\subsection{Abscisse curviligne}
Paramétrisation normale

\subsection{Exemples , vocabulaire, conventions}
exemples de "mesures" : coordonnées rectangulaires, longueur (figure \ref{fig:C2265_1}), angle de la tangente avec une direction fixe (figure \ref{fig:C2265_3}), aire (figure \ref{fig:C2265_2})
\begin{figure}[ht]
 \centering
\input{C2265_1.pdf_t}
\caption{$s(M)$ : longueur de l'arc}
\label{fig:C2265_1}
\end{figure}
\begin{figure}[ht]
 \centering
\input{C2265_2.pdf_t}
\caption{$\mathcal A (M)$ : aire balayée}
\label{fig:C2265_2}
\end{figure}
\begin{figure}[ht]
 \centering
\input{C2265_3.pdf_t}
\label{fig:C2265_3}
\caption{$\alpha (M)$ : angle de la tangente avec une direction fixe}
\end{figure}


Dans quelles situations géométriques ces mesures sont-elles des paramètres admissibles ?
\section{Théorème de relèvement}
\begin{prop}
 Soit $I$ un intervalle de $\R$, soit $z\in \mathcal C^k(I,\C)$ où $k\geq 1$, on suppose de plus que $z(t)$ est de module $1$ pour tout $t\in I$. Il existe alors une fonction $\alpha \in \mathcal C^{k}(I,\R)$ telle que :
\begin{displaymath}
 f(t)=e^{i\alpha(t)}
\end{displaymath}
\end{prop}
\begin{demo}
 Comme la fonction $t\mapsto z(t)\overline{z(t)}$ est constante, sa dérivée est nulle. On en déduit que $z'(t)\overline{z(t)}$ est imaginaire pur. Définissons une fonction $\theta$ par :
\begin{displaymath}
 \theta(t) = \dfrac{1}{i}z'(t)\overline{z(t)} = i\frac{z't)}{z(t)}
\end{displaymath}
car $z(t)$ est de module $1$. Cette fonction est clairement $\mathcal C^{k-1}$. Fixons un $t_0$ dans $I$ et notons $\alpha_0$ un argument de $z(t_0)$. Définissons une fonction $\alpha$ par :
\begin{displaymath}
 \forall t\in I : \alpha(t) = \alpha_0 + \int_{t_0}^{t}\theta(u)du
\end{displaymath}
Cette fonction est $\mathcal C^k$ car $\theta$ est $\mathcal C^{k-1}$. Considérons alors $w(t)=z(t)e^{i\alpha(t)}$ et calculons sa dérivée :
\begin{displaymath}
 w'(t) = (z'(t)+i\alpha'(t)z(t))e^{i\alpha(t)}=(z'(t)+i\theta(t)z(t))e^{i\alpha(t)}=0
\end{displaymath}
On en déduit que $w$ est constante. Comme elle prend la valeur $1$ et $t_0$, on en déduit que
\begin{displaymath}
 \forall t\in I : z(t) = e^{i\alpha(t)}
\end{displaymath}

\end{demo}

Fonctions à valeur dans $\U$. Application au théorème du programme.

\section{Courbure}
\subsection{Définition par une paramétrisation normale}
définition de la courbure, utilisation de $\tan$,  du rayon de courbure, du centre de courbure, du cercle osculateur, de la développée.
\subsection{Diverses expressions}
\subsubsection{Expression avec l'angle de la tangente et une paramétrisation normale}
\subsubsection{Expression avec un déterminant et une paramétrisation quelconque}
\subsubsection{Expression locale avec une paramétrisation quelconque}
\begin{prop}
 Soit $f\in \mathcal{C}^2(I)$ une paramétrisation quelconque d'une courbe $\Gamma$. Soit $t_0\in I$ et $m_0=f(t_0)$. On note $\overrightarrow{\tau}_0=\overrightarrow{\tau}(m_0)$ et $\overrightarrow{n}_0= \overrightarrow{n}(m_0)$ et $X$, $Y$ les fonctions coordonnées dans le repère de Frénet $(m_0,(\overrightarrow{\tau}_0,\overrightarrow{n}_0))$.\newline
On peut alors exprimer localement la courbure $\gamma(m_0)$:
\begin{displaymath}
 \gamma(m_0)=\lim_{t_0}\frac{2Y(f(t))}{X^2(f(t))}
\end{displaymath}
\end{prop}
\begin{demo}
 à rédiger
\end{demo}


\section{Exemples}
\subsection{Cardioïde}
\subsection{Développante}
La développante d'une courbe $\Gamma$ est une courbe dont la développée est $\Gamma$.
\subsection{Ellipse}
Calcul de la courbure par la formule locale aux points d'intersection avec les axes pour une ellipse d'équation cartésienne
\begin{displaymath}
 \dfrac{x^2}{a^2} + \dfrac{y^2}{b^2}=1
\end{displaymath}
Calcul de la courbure avec la formule utilisant le déterminant pour la paramétrisation trigonométrique. La développée est une astroïde.
\subsection{Intégrales de fonctions à valeurs complexes comme courbes paramétrées}
\begin{displaymath}
 t \rightarrow z_0 + \int_{0}^{t}r(u)e^{i\theta(u)}du
\end{displaymath}
équation intrinsèque.
\end{document}
