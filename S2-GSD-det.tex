\subsubsection{B - Déterminants}
\begin{itshape}
Les objectifs de ce chapitre sont les suivants :
\begin{itemize}
\item  introduire la notion de déterminant d'une famille de vecteurs, en
motivant sa construction par la géométrie;
\item  établir les principales propriétés des déterminants des matrices carrées et des endomorphismes;
\item  indiquer quelques méthodes simples de calcul de déterminants.
\end{itemize}

Dans tout ce chapitre, $E$ désigne un espace vectoriel de dimension finie
$n\geqslant 1$.
\end{itshape}

\subsubsubsection{a) Formes $n$-linéaires alternées}
\begin{parcolumns}[rulebetween,distance=2.5cm]{2}
  \colchunk{Forme $n$-linéaire alternée.}
  \colchunk{La définition est motivée par les notions intuitives d'aire et de volume algébriques, en s'appuyant sur des figures.}
  \colplacechunks

  \colchunk{Antisymétrie, effet d'une permutation.}
  \colchunk{Si $f$ est une forme $n$-linéaire alternée et si $(x_1 , \ldots , x_n)$ est une famille liée, alors $f (x_1 , \ldots , x_n) = 0$.}
  \colplacechunks
\end{parcolumns}


\subsubsubsection{b) Déterminant d'une famille de vecteurs dans une base}
\begin{parcolumns}[rulebetween,distance=2.5cm]{2}
  \colchunk{Si $e$ est une base, il existe une et une seule forme $n$-linéaire alternée $f$ pour laquelle $f (e) = 1$. Toute forme $n$-linéaire alternée est un multiple de $\det_e$.}
  \colchunk{Notation $\det_e$.

  La démonstration de l'existence n'est pas exigible.}
  \colplacechunks

  \colchunk{Expression du déterminant dans une base en fonction des coordonnées.}
  \colchunk{Dans $\R^2$ (resp. $\R^3$), interprétation du déterminant dans la base canonique comme aire orientée (resp. volume orienté) d'un parallélogramme (resp. parallélépipède).}
  \colplacechunks

  \colchunk{Comparaison, si $e$ et $e'$ sont deux bases, de $\det_e$ et $\det_{e'}$.}
  \colchunk{}
  \colplacechunks

  \colchunk{La famille $(x_1 , \ldots , x_n)$ est une base si et seulement si $\det_e (x_1 , \ldots , x_n) \ne 0$.}
  \colchunk{}
  \colplacechunks

  \colchunk{Orientation d'un espace vectoriel réel de dimension finie.}
  \colchunk{$\dbf$ PC : orientation d'un espace de dimension $3$.}
  \colplacechunks
\end{parcolumns}

\subsubsubsection{c) Déterminant d'un endomorphisme}
\begin{parcolumns}[rulebetween,distance=2.5cm]{2}
  \colchunk{Déterminant d'un endomorphisme.}
  \colchunk{}
  \colplacechunks

  \colchunk{Déterminant d'une composée.}
  \colchunk{Caractérisation des automorphismes.}
  \colplacechunks
\end{parcolumns}

\subsubsubsection{d) Déterminant d'une matrice carrée}
\begin{parcolumns}[rulebetween,distance=2.5cm]{2}
  \colchunk{Déterminant d'une matrice carrée.}
  \colchunk{}
  \colplacechunks

  \colchunk{Déterminant d'un produit.}
  \colchunk{Relation $\det (\lambda A) = \lambda^n \det (A)$.

  Caractérisation des matrices inversibles.}
  \colplacechunks

  \colchunk{Déterminant d'une transposée.}
  \colchunk{}
  \colplacechunks
\end{parcolumns}

\subsubsubsection{e) Calcul des déterminants}
\begin{parcolumns}[rulebetween,distance=2.5cm]{2}
  \colchunk{Effet des opérations élémentaires.}
  \colchunk{}
  \colplacechunks

  \colchunk{Cofacteur. Développement par rapport à une ligne ou une colonne.}
  \colchunk{}
  \colplacechunks

  \colchunk{Déterminant d'une matrice triangulaire par blocs, d'une matrice triangulaire.}
  \colchunk{}
  \colplacechunks

  \colchunk{Déterminant de Vandermonde.}
  \colchunk{}
  \colplacechunks
\end{parcolumns}

\subsubsubsection{f) Comatrice}
\begin{parcolumns}[rulebetween,distance=2.5cm]{2}
  \colchunk{Comatrice.}
  \colchunk{Notation $\mbox{Com} (A)$.}
  \colplacechunks

  \colchunk{Relation  $A\,  {}^t \mbox{Com} (A) = {}^t \mbox{Com} (A) A = \det (A) I_n$.}
  \colchunk{Expression de l'inverse d'une matrice inversible.}
  \colplacechunks
\end{parcolumns}
