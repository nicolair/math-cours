\subsection{Calculs algébriques}
Ce chapitre a pour but de présenter quelques notations et techniques fondamentales de calcul algébrique.

\subsubsubsection{a) Sommes et produits}
\begin{parcolumns}[rulebetween,distance=\parcoldist]{2}
  \colchunk{Somme et produit d'une famille finie de nombres complexes.}
  \colchunk{Notations $\sum_{i\in I}a_i$, $\sum_{i=1}^na_i$, $\prod_{i\in I}a_i$, $\prod_{i=1}^na_i$.\newline
  Sommes et produits télescopiques, exemple de changements d'indice et de regroupements de termes.}  
  \colplacechunks
  
  \colchunk{Expressions simplifiées de $\sum_{k=1}^nk$, $\sum_{k=1}^nk^2$, $\sum_{k=1}^nx^k$.}
  \colchunk{}
  \colplacechunks
  
  \colchunk{Factorisation de $a^n-b^n$ pour $n\in \N^*$.}
  \colchunk{}
  \colplacechunks
  
  \colchunk{Sommes doubles. Produit de deux sommes finies, sommes triangulaires.}
  \colchunk{}
  \colplacechunks

\end{parcolumns}

\subsubsubsection{b) Coefficients binomiaux et formule du binôme}
\begin{parcolumns}[rulebetween,distance=\parcoldist]{2}
  \colchunk{Factorielle. Coefficients binomiaux.}
  \colchunk{Notation $\binom{n}{p}$.}
  \colplacechunks
  \colchunk{Relation $\binom{n}{p}=\binom{n}{n-p}$}
  \colchunk{}
  \colplacechunks
  \colchunk{Formule et triangle de Pascal.}
  \colchunk{Lien avec la méthode d'obtention des coefficients binomiaux utilisée en Première (dénombrement de chemins).}
  \colplacechunks
  \colchunk{Formule du binôme dans $\C$.}
  \colchunk{}
  \colplacechunks

  \end{parcolumns}

\subsubsubsection{c) Systèmes linéaires}
\begin{parcolumns}[rulebetween,distance=\parcoldist]{2}
  \colchunk{Système linéaire de $n$ équations à $p$ inconnues à coefficients dans $\R$ ou $\C$.}
  \colchunk{$\leftrightarrows$ PC et SI dans le cas $n=p=2$. \newline
  Interprétation géométrique: intersection de droites dans $\R^2$, de plans dans $\R^3$.}
  \colplacechunks
  
  \colchunk{Système homogène associé. Structure de l'ensemble des solutions.}
  \colchunk{}
  \colplacechunks
  
  \colchunk{Opérations élémentaires.}
  \colchunk{Notations $L_i\leftrightarrow L_j$, $L_i\leftarrow \lambda L_i$ ($\lambda \neq 0$), $L_i\leftarrow \lambda L_i +\lambda L_j$.}
  \colplacechunks

  \colchunk{Algorithme du pivot.}
  \colchunk{$\rightleftarrows$ I: pour des systèmes de taille $n>3$ ou $p>3$, on utilise l'outil informatique.}
  \colplacechunks

\end{parcolumns}
