%!  pour pdfLatex
\documentclass[a4paper]{article}
\usepackage[hmargin={1.5cm,1.5cm},vmargin={2.4cm,2.4cm},headheight=13.1pt]{geometry}

\usepackage[pdftex]{graphicx,color}
%\usepackage{hyperref}

\usepackage[utf8]{inputenc}
\usepackage[T1]{fontenc}
\usepackage{lmodern}
%\usepackage[frenchb]{babel}
\usepackage[french]{babel}

\usepackage{fancyhdr}
\pagestyle{fancy}

%\usepackage{floatflt}

\usepackage{parcolumns}
\setlength{\parindent}{0pt}
\usepackage{xcolor}

%pr{\'e}sentation des compteurs de section, ...
\makeatletter
%\renewcommand{\labelenumii}{\theenumii.}
\renewcommand{\thepart}{}
\renewcommand{\thesection}{}
\renewcommand{\thesubsection}{}
\renewcommand{\thesubsubsection}{}
\makeatother

\newcommand{\subsubsubsection}[1]{\bigskip \rule[5pt]{\linewidth}{2pt} \textbf{ \color{red}{#1} } \newline \rule{\linewidth}{.1pt}}
\newlength{\parcoldist}
\setlength{\parcoldist}{1cm}

\usepackage{maths}
\newcommand{\dbf}{\leftrightarrows}
% remplace les commandes suivantes 
%\usepackage{amsmath}
%\usepackage{amssymb}
%\usepackage{amsthm}
%\usepackage{stmaryrd}

%\newcommand{\N}{\mathbb{N}}
%\newcommand{\Z}{\mathbb{Z}}
%\newcommand{\C}{\mathbb{C}}
%\newcommand{\R}{\mathbb{R}}
%\newcommand{\K}{\mathbf{K}}
%\newcommand{\Q}{\mathbb{Q}}
%\newcommand{\F}{\mathbf{F}}
%\newcommand{\U}{\mathbb{U}}

%\newcommand{\card}{\mathop{\mathrm{Card}}}
%\newcommand{\Id}{\mathop{\mathrm{Id}}}
%\newcommand{\Ker}{\mathop{\mathrm{Ker}}}
%\newcommand{\Vect}{\mathop{\mathrm{Vect}}}
%\newcommand{\cotg}{\mathop{\mathrm{cotan}}}
%\newcommand{\sh}{\mathop{\mathrm{sh}}}
%\newcommand{\ch}{\mathop{\mathrm{ch}}}
%\newcommand{\argsh}{\mathop{\mathrm{argsh}}}
%\newcommand{\argch}{\mathop{\mathrm{argch}}}
%\newcommand{\tr}{\mathop{\mathrm{tr}}}
%\newcommand{\rg}{\mathop{\mathrm{rg}}}
%\newcommand{\rang}{\mathop{\mathrm{rg}}}
%\newcommand{\Mat}{\mathop{\mathrm{Mat}}}
%\renewcommand{\Re}{\mathop{\mathrm{Re}}}
%\renewcommand{\Im}{\mathop{\mathrm{Im}}}
%\renewcommand{\th}{\mathop{\mathrm{th}}}

%\lhead{MPSI B}
% \chead{PLAN DU COURS DE MATH{\'E}MATIQUES }
%\rfoot{\small{\today}}


\begin{document}
  Le programme de premi\`ere ann\'ee MPSI est organis\'e en
trois parties. Dans une premi\`ere partie figurent les notions et
les objets qui doivent \^etre \'etudi\'es d\`es le d\'ebut de
l'ann\'ee scolaire. Il s'agit essentiellement, en partant du
programme de la classe de Terminale S et en s'appuyant sur les
connaissances pr\'ealables des \'etudiants, d'introduire des
notions de base n{\'e}cessaires tant en math{\'e}matiques que dans
les autres disciplines scientifiques (physique, chimie, sciences
industrielles\dots). Certains de ces objets seront consid\'er\'es
comme d\'efinitivement acquis (nombres complexes, coniques, \dots)
et il n'y aura pas lieu de reprendre ensuite leur \'etude dans le
cours de math\'ematiques; d'autres, au contraire, seront revus
plus tard dans un cadre plus g\'en\'eral ou dans une
pr\'esentation plus th\'eorique (groupes, produit scalaire,
\'equations diff\'erentielles, \dots).

Les deuxi\`eme et troisi\`eme parties correspondent \`a un
d\'ecoupage classique entre l'analyse et ses applications
g\'eom\'etriques d'une part, l'alg\`ebre et la g\'eom\'etrie
euclidienne d'autre part.  \vfill\eject

\part{PROGRAMME DE D\'EBUT D'ANN\'EE}\label{part-debann}

\section{NOMBRES COMPLEXES ET G\'EOM\'ETRIE
\'EL\'EMENTAIRE}\label{nbcomp}


\subsection{Nombres complexes}
Colles : \href{S1.pdf}{Semaine 1}\newline
 Exercices : \href{../Exercices/NExo/nexo1.pdf}{Feuille 1}
\begin{itemize}
  \item Corps $\Bbb{C}$ des nombres complexes

\item Groupe $\Bbb{U}$ des nombres complexes de module 1

\item \'Equations du second degr\'e

\item Exponentielle complexe

\item Nombres complexes et g\'eom\'etrie plane
\end{itemize}

\subsection{G\'eom\'etrie \'el\'ementaire du plan}
Colles : \href{S4.pdf}{Semaine 4}
\begin{itemize}
\item Modes de rep\'erage dans le plan

\item Produit scalaire

\item D\'eterminant

\item Droites

\item Cercles
\end{itemize}

\subsection{G\'eom\'etrie \'el\'ementaire de l'espace}
Colles : \href{S5.pdf}{Semaine 5}
\begin{itemize}
\item Modes de rep\'erage dans l'espace
\item Produit scalaire
\item Produit vectoriel
\item Droites et plans
\item Sph\`eres
\end{itemize}

\section{FONCTIONS USUELLES ET \'EQUATIONS DIFF\'ERENTIELLES
LIN\'EAIRES}

 \subsection{Fonctions usuelles}
Colles : \href{S2.pdf}{Semaine 2}

\begin{itemize}
 \item Fonctions exponentielles, logarithmes, puissances
\item Fonctions circulaires
\item Fonction exponentielle complexe
\end{itemize}

\subsection{\'Equations diff\'erentielles lin\'eaires}
Colles : \href{S3.pdf}{Semaine 3}
\begin{itemize}
\item \'Equations lin\'eaires du premier ordre
\item M\'ethode d'Euler
\item \'Equations lin\'eaires du second ordre \`a coefficients
constants
\end{itemize}

\subsection{Courbes param\'etr\'ees. Coniques}
Colles : \href{S6.pdf}{Semaine 6}
\begin{itemize}
\item Courbes planes param\'etr\'ees
\item Coniques
\end{itemize}

 \vfill \eject

\part{ANALYSE ET G\'EOM\'ETRIE DIFF\'ERENTIELLE}

\section{NOMBRES R\'EELS ET COMPLEXES, SUITES ET FONCTIONS}

\subsection{Suites de nombres r\'eels}
Colles : \href{S9.pdf}{Semaine 9}
\begin{itemize}
\item Corps $\Bbb{R}$ des nombres r\'eels
\item Suites de nombres r\'eels
\item Limite d'une suite
\item Relations de comparaison
\item Th\'eor\`emes d'existence de limites
\item Br{\`e}ve extension aux suites complexes
\end{itemize}

\subsection{Fonctions d'une variable r\'eelle \`a valeurs r\'eelles}
Colles : \href{S10.pdf}{Semaine 10}
\begin{itemize}
\item Fonctions d'une variable r\'eelle \`a valeurs r\'eelles
\item \'Etude locale d'une fonction
\item Relations de comparaison
\item Fonctions continues sur un intervalle
\item Br{\`e}ve extension aux fonctions \`a valeurs complexes
\end{itemize}

\section{CALCUL DIFF\'ERENTIEL ET INT\'EGRAL}

\subsection{D\'erivation des fonctions \`a valeurs r\'eelles}
Colles : \href{S11.pdf}{Semaine 11}
\begin{itemize}
\item D\'eriv\'ee en un point, fonction d\'eriv\'ee
\item \'Etude globale des fonctions d\'erivables
\item Fonctions convexes
\item Br\`eve extension aux fonctions \`a valeurs complexes
\end{itemize}

\subsection{Int\'egration sur un segment des fonctions \`a valeurs
r\'eelles} Colles : \href{S17.pdf}{Semaine 17}
\begin{itemize}
\item Fonctions continues par morceaux
\item Int\'egrale d'une fonction continue par morceaux
\item Br{\`e}ve extension aux fonctions {\`a} valeurs
complexes
\end{itemize}

\subsection{Int\'egration et d\'erivation}
Colles : \href{S18.pdf}{Semaine 18}
\begin{itemize}
\item Primitives et int\'egrale d'une fonction continue
\item  Calcul des primitives
\item Formules de Taylor
\item D{\'e}veloppement limit{\'e}s
\end{itemize}

\subsection{Approximation}
Colles : \href{S19.pdf}{Semaine 19}
\begin{itemize}
\item Calcul approch\'e d'une int\'egrale
\item Valeursapproch\'ees de r\'eels
\end{itemize}

\section{NOTIONS SUR LES FONCTIONS DE DEUX VARIABLES
R\'EELLES}

\subsection{Analyse dans un plan r{\'e}el, fonctions continues}
Colles : \href{S27.pdf}{Semaine 27}

\subsection{Calcul diff\'erentiel}
Colles : \href{S28.pdf}{Semaine 28}
\begin{itemize}
\item D\'eriv\'ees partielles premi\`eres
\item D\'eriv\'ees partielles d'ordre $2$
\end{itemize}

\subsection{Calcul int\'egral}
Colles : \href{S29.pdf}Semaine 29


\section{G\'EOM\'ETRIE DIFF\'ERENTIELLE}

\subsection{\'Etude m{\'e}trique des courbes planes}
Colles : \href{S26.pdf}{Semaine 26}

\subsection{Champs de vecteurs du plan et de l'espace}
Colles : \href{S29.pdf}{Semaine 29}

\vfill\eject

\part{ALG\`EBRE ET G\'EOM\'ETRIE}

\section{NOMBRES ET STRUCTURES ALG\'EBRIQUES USUELLES}

\subsection{Vocabulaire relatif aux ensembles et aux applications}
Colles : \href{S7.pdf}{Semaine 7}

\subsection{Nombres entiers naturels, ensembles finis,
d\'enombrements}

Colles : \href{S7.pdf}{Semaine 7}
\begin{itemize}
\item Nombres entiers naturels
\item Ensembles finis
\item Op\'erations sur les ensembles finis, d\'enombrements
\end{itemize}

\subsection{Structures alg\'ebriques usuelles}
Colles : \href{S12.pdf}{Semaine 12}
\begin{itemize}
\item Vocabulaire relatif aux groupes et aux anneaux
\item  Arithm\'etique dans $\Bbb{Z}$. Calculs dans $\Bbb{R}$ ou $\Bbb{C}$.
\end{itemize}

\section{ALG\`EBRE LIN\'EAIRE ET POLYN\^OMES}

\subsection{Espaces vectoriels}
Colles : \href{S12.pdf}{Semaine 12}
\begin{itemize}
\item Espaces vectoriels
\item Translations, sous-espaces affines
\item Applications lin\'eaires
\end{itemize}

\subsection{Dimension des espaces vectoriels}
Colles : \href{S13.pdf}{Semaine 13}
\begin{itemize}
\item Familles de vecteurs
\item Dimension d'un espace vectoriel
\item Dimension d'un sous-espace vectoriel
\item Rang d'une application lin\'eaire
\end{itemize}

\subsection{Polyn\^omes}
Colles : \href{S14.pdf}{Semaine 14}, \href{S15.pdf}{Semaine 15}
\begin{itemize}
\item Polyn\^omes \`a une ind\'etermin\'ee et corps $KK(X)$
\item Fonctions polynomiales et rationnelles
\item Polyn\^omes scind\'es
\item Divisibilit\'e dans l'anneau $K[X]$
\item \'Etude locale d'une fraction rationnelle
\end{itemize}

\subsection{Calcul matriciel}
Colles : \href{S20.pdf}{Semaine 20}, \href{S21.pdf}{Semaine 21}
\begin{itemize}
\item Op\'erations sur les matrices
\item Matrices et applications lin\'eaires
\item Op\'erations \'el\'ementaires sur les matrices
\item Rang d'une matrice
\item Syst\`emes d'\'equations lin\'eaires
\end{itemize}

\subsection{D\'eterminants}
Colles : \href{S22.pdf}{Semaine 22}
\begin{itemize}
\item Groupe sym\'etrique
\item Applications multilin\'eaires
\item D\'eterminant de $n$ vecteurs
\item D\'eterminant d'un endomorphisme
\item D\'eterminant d'une matrice carr\'ee
\end{itemize}

\section{ESPACES VECTORIELS EUCLIDIENS ET G\'EOM\'ETRIE
EUCLIDIENNE}

\subsection{Produit scalaire, espaces vectoriels euclidiens}
Colles : \href{S23.pdf}{Semaine 23}, \href{S24.pdf}{Semaine 24}
\begin{itemize}
\item Produit scalaire
\item Orthogonalit\'e
\item Isom\'etries affines du plan et de l'espace
\item Automorphismes orthogonaux du plan vectoriel euclidien
\item Automorphismes orthogonaux de l'espace
\item D{\'e}placements
\item Similitudes directes du plan
\end{itemize}

\end{document}
