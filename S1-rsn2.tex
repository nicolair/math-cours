\subsection{Nombres réels et suites numériques (fin)}

\subsubsubsection{e) Suites monotones}
\begin{parcolumns}[rulebetween,distance=\parcoldist]{2}
  \colchunk{Théorème de la limite monotone: toute suite monotone possède une limite.}
  \colchunk{Toute suite croissante majorée converge, toute suite croissante non majorée tend vers $+\infty$.}
  \colplacechunks

  \colchunk{Théorème des suites adjacentes.}
  \colchunk{}
  \colplacechunks
\end{parcolumns}

\subsubsubsection{f) Suites extraites}
\begin{parcolumns}[rulebetween,distance=\parcoldist]{2}
  
  \colchunk{Suite extraite}
  \colchunk{}
  \colplacechunks

  \colchunk{Si une suite possède une limite, toutes ses suites extraites possèdent la même limite.}
  \colchunk{Utiisation pour montrer la divergence d'une suite. Si $(u_{2n})$ et $(u_{2n+1})$ tendent vers $l$ alors $(u_n)$ tend vers $l$.}
  \colplacechunks
  
  \colchunk{Théorème de Bolzano-Weierstrass.}
  \colchunk{Les étudiants doivent connaître le principe de démonstration par dichotomie, mais la formalisation précise n'est pas exigible.\newline
  La notion de valeur d'adhérence est hors programme.}
  \colplacechunks
\end{parcolumns}

\subsubsubsection{g) Traduction séquentielle de certaines propriétés}
\begin{parcolumns}[rulebetween,distance=\parcoldist]{2}
  
  \colchunk{Partie dense de $\R$.}
  \colchunk{Une partie est dense dans $\R$ si elle rencontre tout intervalle ouvert non vide.\newline
  Densité de l'ensemble des décimaux, des rationnels, des irrationnels.}
  \colplacechunks

  \colchunk{Caractérisation séquentielle de la densité.}
  \colchunk{}
  \colplacechunks

  \colchunk{Si $X$ est une partie non vide majorée (resp. non majorée) de $\R$, il existe une suite d'éléments de $X$ dont la limite est $\sup X$ (resp. $+\infty$).}
  \colchunk{}
  \colplacechunks

\end{parcolumns}

\subsubsubsection{h) Suites complexes}
\begin{parcolumns}[rulebetween,distance=\parcoldist]{2}
  
  \colchunk{Brève extension des définitions et résultats précédents. Par définition, une suite complexe $\left( z_n \right)_{n \in \N}$ converge vers un nombre complexe $z$ si et seulement si la suite réelle $\left( |z_n - z| \right)_{n \in \N}$ converge vers $0$. Opérations sur les suites convergentes.}
  \colchunk{Caractérisation de la limite en termes de parties réelle et imaginaire.}
  \colplacechunks

  \colchunk{Théorème de Bolzano-Weierstrass.}
  \colchunk{La démonstration n'est pas exigible par le programme de mpsi mais figure comme question de cours pour la classe.}
  \colplacechunks

\end{parcolumns}

\subsubsubsection{i) Suites particulières}
\begin{parcolumns}[rulebetween,distance=\parcoldist]{2}
  
  \colchunk{Suite arithmétique, géométrique.\newline Suite arithmético-géométrique. Suite récurrente linéaire homogène d'ordre 2 à coefficients constants.}
  \colchunk{Les étudiants doivent savoir déterminer une expression du terme général de ces suites.}
  \colplacechunks

  \colchunk{Exemples de suites définies par une relation de récurrence $u_{n+1}=f(u_n)$.}
  \colchunk{Seul résultat exigible: si $(u_n)_{n\in\N}$ converge vers $l$ et si $f$ est continue en $l$, alors $f(l)=l$.}
  \colplacechunks

\end{parcolumns}
