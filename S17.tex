%!  pour pdfLatex
\documentclass[a4paper]{article}
\usepackage[hmargin={1.5cm,1.5cm},vmargin={2.4cm,2.4cm},headheight=13.1pt]{geometry}

\usepackage[pdftex]{graphicx,color}
%\usepackage{hyperref}

\usepackage[utf8]{inputenc}
\usepackage[T1]{fontenc}
\usepackage{lmodern}
%\usepackage[frenchb]{babel}
\usepackage[french]{babel}

\usepackage{fancyhdr}
\pagestyle{fancy}

%\usepackage{floatflt}

\usepackage{parcolumns}
\setlength{\parindent}{0pt}
\usepackage{xcolor}

%pr{\'e}sentation des compteurs de section, ...
\makeatletter
%\renewcommand{\labelenumii}{\theenumii.}
\renewcommand{\thepart}{}
\renewcommand{\thesection}{}
\renewcommand{\thesubsection}{}
\renewcommand{\thesubsubsection}{}
\makeatother

\newcommand{\subsubsubsection}[1]{\bigskip \rule[5pt]{\linewidth}{2pt} \textbf{ \color{red}{#1} } \newline \rule{\linewidth}{.1pt}}
\newlength{\parcoldist}
\setlength{\parcoldist}{1cm}

\usepackage{maths}
\newcommand{\dbf}{\leftrightarrows}
% remplace les commandes suivantes 
%\usepackage{amsmath}
%\usepackage{amssymb}
%\usepackage{amsthm}
%\usepackage{stmaryrd}

%\newcommand{\N}{\mathbb{N}}
%\newcommand{\Z}{\mathbb{Z}}
%\newcommand{\C}{\mathbb{C}}
%\newcommand{\R}{\mathbb{R}}
%\newcommand{\K}{\mathbf{K}}
%\newcommand{\Q}{\mathbb{Q}}
%\newcommand{\F}{\mathbf{F}}
%\newcommand{\U}{\mathbb{U}}

%\newcommand{\card}{\mathop{\mathrm{Card}}}
%\newcommand{\Id}{\mathop{\mathrm{Id}}}
%\newcommand{\Ker}{\mathop{\mathrm{Ker}}}
%\newcommand{\Vect}{\mathop{\mathrm{Vect}}}
%\newcommand{\cotg}{\mathop{\mathrm{cotan}}}
%\newcommand{\sh}{\mathop{\mathrm{sh}}}
%\newcommand{\ch}{\mathop{\mathrm{ch}}}
%\newcommand{\argsh}{\mathop{\mathrm{argsh}}}
%\newcommand{\argch}{\mathop{\mathrm{argch}}}
%\newcommand{\tr}{\mathop{\mathrm{tr}}}
%\newcommand{\rg}{\mathop{\mathrm{rg}}}
%\newcommand{\rang}{\mathop{\mathrm{rg}}}
%\newcommand{\Mat}{\mathop{\mathrm{Mat}}}
%\renewcommand{\Re}{\mathop{\mathrm{Re}}}
%\renewcommand{\Im}{\mathop{\mathrm{Im}}}
%\renewcommand{\th}{\mathop{\mathrm{th}}}


%En tete et pied de page
\lhead{Programme colle math}
\chead{Semaine 17 du 03/02/20 au 08/02/20}
\rhead{MPSI B Hoche}

\lfoot{\tiny{Cette création est mise à disposition selon le Contrat\\ Paternité-Partage des Conditions Initiales à l'Identique 2.0 France\\ disponible en ligne http://creativecommons.org/licenses/by-sa/2.0/fr/
} }
\rfoot{\tiny{Rémy Nicolai \jobname}}


\begin{document}
\subsection{Espaces vectoriels et applications linéaires (2)}

\subsubsection{B - Espaces de dimension finie}
\subsubsubsection{Existence de bases}
\begin{parcolumns}[rulebetween,distance=\parcoldist]{2}

 \colchunk{Un espace vectoriel est dit de dimension finie s'il possède une famille génératrice finie.}
 \colchunk{}
 \colplacechunks

 \colchunk{Si ${(x_i)}_{1\le i\le n}$ engendre $E$ et si ${(x_i)}_{i\in I}$ est libre pour une certaine partie $I$ de $\{1,\dots,n\}$, alors il existe une partie $J$ de $\{1,\dots,n\}$ contenant $I$ pour laquelle ${(x_j)}_{j\in J}$ est une base de $E$.}
  \colchunk{Existence de bases en dimension finie. \\ Théorème de la base extraite~: de toute famille génératrice on peut extraire une base.\\Théorème de la base incomplète~: toute famille libre peut être complétée en une base.}
  \colplacechunks
\end{parcolumns}

\subsubsubsection{Dimension d'un espace de dimension finie}
\begin{parcolumns}[rulebetween,distance=\parcoldist]{2}

 \colchunk{Dans un espace engendré par $n$ vecteurs, toute famille de $n+1$ vecteurs est liée.}
  \colchunk{}
  \colplacechunks
 \colchunk{Dimension d'un espace de dimension finie. }
  \colchunk{Dimensions de $\K^n$, de $\K_n[X]$, de l'espace des solutions d'une équation différentielle linéaire homogène d'ordre $1$, de l'espace des solutions d'une équation différentielle linéaire homogène d'ordre $2$ à coefficients constants, de l'espace des suites vérifiant une relation de récurrence linéaire homogène d'ordre $2$ à coefficients constants.}
  \colplacechunks
 \colchunk{En dimension $n$, une famille de $n$ vecteurs est une base si et seulement si elle est libre, si et seulement si elle est génératrice.}
  \colchunk{}
  \colplacechunks
 \colchunk{Dimension d'un produit fini d'espaces vectoriels de dimension finie.}
  \colchunk{}
  \colplacechunks
 \colchunk{Rang d'une famille finie de vecteurs.}
  \colchunk{Notation $\rg(x_1,\dots,x_n)$.}
  \colplacechunks

\end{parcolumns}

\subsubsubsection{Sous-espaces et dimension}
\begin{parcolumns}[rulebetween,distance=\parcoldist]{2}


 \colchunk{Dimension d'un sous-espace d'un espace de dimension finie, cas d'égalité.}
  \colchunk{Sous-espace de $\R^2$ et $\R^3$.}
  \colplacechunks
 \colchunk{Tout sous-espace d'un espace de dimension finie possède un supplémentaire.}
  \colchunk{Dimension commune des supplémentaires.}
  \colplacechunks
 \colchunk{Base adaptée à un sous-espace, à une décomposition en somme directe d'un nombre fini de sous-espaces.}
  \colchunk{}
  \colplacechunks
 \colchunk{Dimension d'une somme de deux sous-espaces~; formule de Grassmann. Caractérisation des couples de sous-espaces supplémentaires.}
  \colchunk{}
  \colplacechunks
 \colchunk{Si $F_1,\dots,F_p$ sont des sous-espaces de dimension finie, alors~: $\dim\displaystyle\sum_{i=1}^pF_i\le \displaystyle \sum_{i=1}^p\dim F_i$, avec égalité si et seulement si la somme est directe.}
  \colchunk{}
  \colplacechunks
\end{parcolumns}



\bigskip
\begin{center}
 \textbf{Questions de cours}
\end{center}
\begin{itemize}
 \item Une famille est liée si et seulement si un de ses vecteurs est combinaison linéaire des autres.
 \item Soit $(a_1,\cdots,a_p)$ libre et $x$ un vecteur. La famille $(a_1,\cdots,a_p,x)$ est liée si et seulement si $x\in \Vect(a_1,\cdots,a_p)$.
 \item Condition suffisante de dépendance.
 \item Définition d'un espace de dimension finie. Un ev de dimension finie admet des bases.
 \item Définition de la dimension.
 \item Thm de la base incomplète.
 \item Dimension d'un sous-espace. Formule de Grasmmann pour la somme de deux sev.
 \item Dimension d'une somme et caractérisation de la somme directe.
\end{itemize}


\begin{center}
 \textbf{Prochain programme}
\end{center}
Applications linéaires.
\end{document}
