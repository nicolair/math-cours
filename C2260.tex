%<dscrpt>Fichier de déclarations Latex à inclure au début d'un élément de cours.</dscrpt>

\documentclass[a4paper]{article}
\usepackage[hmargin={1.8cm,1.8cm},vmargin={2.4cm,2.4cm},headheight=13.1pt]{geometry}

%includeheadfoot,scale=1.1,centering,hoffset=-0.5cm,
\usepackage[pdftex]{graphicx,color}
\usepackage[french]{babel}
%\selectlanguage{french}
\addto\captionsfrench{
  \def\contentsname{Plan}
}
\usepackage{fancyhdr}
\usepackage{floatflt}
\usepackage{amsmath}
\usepackage{amssymb}
\usepackage{amsthm}
\usepackage{stmaryrd}
%\usepackage{ucs}
\usepackage[utf8]{inputenc}
%\usepackage[latin1]{inputenc}
\usepackage[T1]{fontenc}


\usepackage{titletoc}
%\contentsmargin{2.55em}
\dottedcontents{section}[2.5em]{}{1.8em}{1pc}
\dottedcontents{subsection}[3.5em]{}{1.2em}{1pc}
\dottedcontents{subsubsection}[5em]{}{1em}{1pc}

\usepackage[pdftex,colorlinks={true},urlcolor={blue},pdfauthor={remy Nicolai},bookmarks={true}]{hyperref}
\usepackage{makeidx}

\usepackage{multicol}
\usepackage{multirow}
\usepackage{wrapfig}
\usepackage{array}
\usepackage{subfig}


%\usepackage{tikz}
%\usetikzlibrary{calc, shapes, backgrounds}
%pour la présentation du pseudo-code
% !!!!!!!!!!!!!!      le package n'est pas présent sur le serveur sous fedora 16 !!!!!!!!!!!!!!!!!!!!!!!!
%\usepackage[french,ruled,vlined]{algorithm2e}

%pr{\'e}sentation du compteur de niveau 2 dans les listes
\makeatletter
\renewcommand{\labelenumii}{\theenumii.}
\renewcommand{\thesection}{\Roman{section}.}
\renewcommand{\thesubsection}{\arabic{subsection}.}
\renewcommand{\thesubsubsection}{\arabic{subsubsection}.}
\makeatother


%dimension des pages, en-t{\^e}te et bas de page
%\pdfpagewidth=20cm
%\pdfpageheight=14cm
%   \setlength{\oddsidemargin}{-2cm}
%   \setlength{\voffset}{-1.5cm}
%   \setlength{\textheight}{12cm}
%   \setlength{\textwidth}{25.2cm}
   \columnsep=1cm
   \columnseprule=0.5pt

%En tete et pied de page
\pagestyle{fancy}
\lhead{MPSI-\'Eléments de cours}
\rhead{\today}
%\rhead{25/11/05}
\lfoot{\tiny{Cette création est mise à disposition selon le Contrat\\ Paternité-Pas d'utilisations commerciale-Partage des Conditions Initiales à l'Identique 2.0 France\\ disponible en ligne http://creativecommons.org/licenses/by-nc-sa/2.0/fr/
} }
\rfoot{\tiny{Rémy Nicolai \jobname}}


\newcommand{\baseurl}{http://back.maquisdoc.net/data/cours\_nicolair/}
\newcommand{\urlexo}{http://back.maquisdoc.net/data/exos_nicolair/}
\newcommand{\urlcours}{https://maquisdoc-math.fra1.digitaloceanspaces.com/}

\newcommand{\N}{\mathbb{N}}
\newcommand{\Z}{\mathbb{Z}}
\newcommand{\C}{\mathbb{C}}
\newcommand{\R}{\mathbb{R}}
\newcommand{\D}{\mathbb{D}}
\newcommand{\K}{\mathbf{K}}
\newcommand{\Q}{\mathbb{Q}}
\newcommand{\F}{\mathbf{F}}
\newcommand{\U}{\mathbb{U}}
\newcommand{\p}{\mathbb{P}}


\newcommand{\card}{\mathop{\mathrm{Card}}}
\newcommand{\Id}{\mathop{\mathrm{Id}}}
\newcommand{\Ker}{\mathop{\mathrm{Ker}}}
\newcommand{\Vect}{\mathop{\mathrm{Vect}}}
\newcommand{\cotg}{\mathop{\mathrm{cotan}}}
\newcommand{\sh}{\mathop{\mathrm{sh}}}
\newcommand{\ch}{\mathop{\mathrm{ch}}}
\newcommand{\argsh}{\mathop{\mathrm{argsh}}}
\newcommand{\argch}{\mathop{\mathrm{argch}}}
\newcommand{\tr}{\mathop{\mathrm{tr}}}
\newcommand{\rg}{\mathop{\mathrm{rg}}}
\newcommand{\rang}{\mathop{\mathrm{rg}}}
\newcommand{\Mat}{\mathop{\mathrm{Mat}}}
\newcommand{\MatB}[2]{\mathop{\mathrm{Mat}}_{\mathcal{#1}}\left( #2\right) }
\newcommand{\MatBB}[3]{\mathop{\mathrm{Mat}}_{\mathcal{#1} \mathcal{#2}}\left( #3\right) }
\renewcommand{\Re}{\mathop{\mathrm{Re}}}
\renewcommand{\Im}{\mathop{\mathrm{Im}}}
\renewcommand{\th}{\mathop{\mathrm{th}}}
\newcommand{\repere}{$(O,\overrightarrow{i},\overrightarrow{j},\overrightarrow{k})$}
\newcommand{\cov}{\mathop{\mathrm{Cov}}}

\newcommand{\absolue}[1]{\left| #1 \right|}
\newcommand{\fonc}[5]{#1 : \begin{cases}#2 \rightarrow #3 \\ #4 \mapsto #5 \end{cases}}
\newcommand{\depar}[2]{\dfrac{\partial #1}{\partial #2}}
\newcommand{\norme}[1]{\left\| #1 \right\|}
\newcommand{\se}{\geq}
\newcommand{\ie}{\leq}
\newcommand{\trans}{\mathstrut^t\!}
\newcommand{\val}{\mathop{\mathrm{val}}}
\newcommand{\grad}{\mathop{\overrightarrow{\mathrm{grad}}}}

\newtheorem*{thm}{Théorème}
\newtheorem{thmn}{Théorème}
\newtheorem*{prop}{Proposition}
\newtheorem{propn}{Proposition}
\newtheorem*{pa}{Présentation axiomatique}
\newtheorem*{propdef}{Proposition - Définition}
\newtheorem*{lem}{Lemme}
\newtheorem{lemn}{Lemme}

\theoremstyle{definition}
\newtheorem*{defi}{Définition}
\newtheorem*{nota}{Notation}
\newtheorem*{exple}{Exemple}
\newtheorem*{exples}{Exemples}


\newenvironment{demo}{\renewcommand{\proofname}{Preuve}\begin{proof}}{\end{proof}}
%\renewcommand{\proofname}{Preuve} doit etre après le begin{document} pour fonctionner

\theoremstyle{remark}
\newtheorem*{rem}{Remarque}
\newtheorem*{rems}{Remarques}

\renewcommand{\indexspace}{}
\renewenvironment{theindex}
  {\section*{Index} %\addcontentsline{toc}{section}{\protect\numberline{0.}{Index}}
   \begin{multicols}{2}
    \begin{itemize}}
  {\end{itemize} \end{multicols}}


%pour annuler les commandes beamer
\renewenvironment{frame}{}{}
\newcommand{\frametitle}[1]{}
\newcommand{\framesubtitle}[1]{}

\newcommand{\debutcours}[2]{
  \chead{#1}
  \begin{center}
     \begin{huge}\textbf{#1}\end{huge}
     \begin{Large}\begin{center}Rédaction incomplète. Version #2\end{center}\end{Large}
  \end{center}
  %\section*{Plan et Index}
  %\begin{frame}  commande beamer
  \tableofcontents
  %\end{frame}   commande beamer
  \printindex
}


\makeindex
\begin{document}
\noindent

\debutcours{Groupe symétrique}{1.1 \tiny{le \today}}

\section{Définitions}
\index{groupe symétrique}\index{permutation}
\begin{defi}
 Le groupe des bijections de $\llbracket 1,n \rrbracket$ dans $\llbracket 1,n \rrbracket$ muni de l'opération de composition est appelé groupe symétrique. Il est noté $\mathfrak S_n$. Un élément de ce groupe est appelé une \emph{permutation}.
\end{defi}
On appelera aussi permutation toute bijection d'un ensemble fini dans lui même. Si $\Omega$ est un ensemble fini de cardinal $n$, il existe une bijection $N$ (numérotation) de $\llbracket 1,n \rrbracket$ dans $\Omega$. L'application de $(\mathfrak S_n,\circ)$ dans le groupe des bijections de $\Omega$ muni de $\circ$ qui à $\sigma\in \mathfrak{S}_n$ associe $N\circ \sigma \circ N^{-1}$ est un isomorphisme de groupe.\newline
On considèrera donc toujours des permutations dans des ensembles de nombres entre $1$ et $n$.

Plusieurs notations sont possibles pour les permutations. On peut par exemple utiliser une notation matricielle à deux lignes. La première ligne contient les entiers de $1$ à $n$ et la deuxième contient les images de ces entiers. Par exemple, avec $n=7$,
\begin{displaymath}
 \begin{pmatrix}
  1 & 2 & 3 & 4 & 5 & 6 & 7 \\
  3 & 7 & 1 & 2 & 6 & 4 & 5
 \end{pmatrix}
\circ
 \begin{pmatrix}
  1 & 2 & 3 & 4 & 5 & 6 & 7 \\
  6 & 4 & 1 & 3 & 2 & 7 & 5
 \end{pmatrix}
=
 \begin{pmatrix}
  1 & 2 & 3 & 4 & 5 & 6 & 7 \\
  4 & 2 & 3 & 1 & 7 & 5 & 6
 \end{pmatrix}
\end{displaymath}
\begin{propn}
 Le groupe symétrique $\mathfrak{S}_n$ est de cardinal $n!$. Pour $n\geq3$, $\mathfrak{S}_n$ n'est pas commutatif.
\end{propn}
\begin{demo}
 Le cardinal de l'ensemble des bijections a été calculé dans la section sur les dénombrements. Le groupe n'est pas commutatif car 
\begin{displaymath}
 \begin{pmatrix}
  1 & 2 & 3 \\
  1 & 3 & 2
 \end{pmatrix}
\circ
 \begin{pmatrix}
  1 & 2 & 3 \\
  2 & 1 & 3
 \end{pmatrix}
 =
 \begin{pmatrix}
  1 & 2 & 3 \\
  3 & 1 & 2
 \end{pmatrix}
\neq
 \begin{pmatrix}
  1 & 2 & 3 \\
  2 & 1 & 3
 \end{pmatrix}
\circ
 \begin{pmatrix}
  1 & 2 & 3 \\
  1 & 3 & 2
 \end{pmatrix}
=
 \begin{pmatrix}
  1 & 2 & 3 \\
  2 & 3 & 1
 \end{pmatrix}
\end{displaymath}
\end{demo}
\index{cycle} \index{cycle: longeur} \index{cycle: support} \index{permutation circulaire}
\begin{defi}
 Soit $k$ entier entre $1$ et $n$ et $a_1,\cdots,a_k$ des entiers deux à deux distincts entre $1$ et $n$. On note
\[
 \begin{pmatrix}
  a_1 & a_2 & \cdots & a_k
 \end{pmatrix}: \hspace{0.5cm}
 \left\lbrace
 \begin{aligned}
   \llbracket 1,n \rrbracket &\rightarrow \llbracket 1,n \rrbracket \\
   x &\mapsto 
    \left\lbrace 
       \begin{aligned}
           x&\text{ si }x\not\in\left\lbrace a_1,\cdots,a_n\right\rbrace\\
           a_{i+1} &\text{ si }x=a_i \text{ avec } i\in\left\lbrace,\cdots,k-1\right\rbrace \\
           a_1 &\text{ si }x=a_k
       \end{aligned}
    \right. 
 \end{aligned}
\right.
\]
Cette application est une permutation appelée \emph{cycle} (ou permutation circulaire) de longueur $k$ et de support $\left\lbrace a_1,\cdots,a_k \right\rbrace$.
\end{defi}
\begin{rems}
  \begin{enumerate}
    \item Malgré sa notation matricielle 
     $\begin{pmatrix}
        a_1 & a_2 & \cdots & a_k
     \end{pmatrix}$ désigne une permutation de $\llbracket 1,n \rrbracket$.
     \item Un même cycle se note de plusieurs manières:
\[
  \begin{pmatrix}a_1 & a_2 & a_3 & a_4 \end{pmatrix} = \begin{pmatrix}a_2 & a_3 & a_4 & a_1 \end{pmatrix} = \cdots 
\]

  \end{enumerate}

\end{rems}

 \index{cycles disjoints}
\begin{defi}
 On dira que deux cycles sont \emph{disjoints} si et seulement si leurs supports sont disjoints.
\end{defi}
\begin{propn}
 Deux cycles disjoints commutent. C'est à dire que $\sigma_1 \circ \sigma_2 = \sigma_2 \circ \sigma_1 $ lorsque $\sigma_1$ et $\sigma_2$ sont des cycles disjoints.
\end{propn}
\begin{demo}
 \'Evident avec la définition.
\end{demo}
\index{transposition}
\begin{defi}
Une \emph{transposition} est un cycle de longueur $2$. 
\end{defi}
\index{orbites d'une permutation}
\begin{defi}[orbites d'une permutation]
Soit $\sigma \in \mathfrak{S}_n$ et $a\in \llbracket 1,n \rrbracket$. L'orbite de $a$ pour $\sigma$ est l'ensemble des $\sigma^k(a)$ pour $k\in \Z$.  
\end{defi}
\begin{propn}
  Les différentes orbites d'une même permutation $\sigma$ constituent une partition de $\llbracket 1,n \rrbracket$.
\end{propn}


\section{Décompositions}
\begin{propn}[décomposition en cycles disjoints] \label{prop:decompcycl}
 Toute permutation est la composition de cycles disjoints qui commutent.
\end{propn}
\begin{propn}[décomposition en transpositions]\label{prop:decomptranspo}
 Toute permutation est la composition d'un certain nombre de transpositions.
\end{propn}
\begin{demo}
 Comme toute permutation est une composée de cycles, il suffit de démontrer que tout cycle est la composée de transpositions. Cela peut se faire de plusieurs manières.\newline
Soit $p$ éléments distincts $a_1,\cdots,a_p$ de $\llbracket 1, n\rrbracket$.
\[
 \begin{pmatrix}
  a_1 & a_2 & \cdots & a_p
 \end{pmatrix}
 = (a_1\; a_2)\circ (a_2\; a_3)\circ \cdots \circ (a_{p-1}\; a_p) \\
 = (a_1\;a_p)\circ(a_1\;a_{p-1})\circ\cdots\circ (a_1\;a_2).
\]
On vérifie ces formules en examinant l'image d'un élément quelconque.
\end{demo}

\begin{rems}
 \begin{itemize}
\item On peut convenir que l'identité est la composée de $0$ transposition (ou de deux identiques).
\item Toute permutation $\sigma$ admet \emph{plusieurs} décomposition en transpositions. Le point important est que les nombres de transpositions intervenant dans chaque décomposition d'une permutation ont \emph{tous la même parité}. C'est l'objet de la section suivante relative à la signature d'une transposition.
\item On peut donner une deuxième démonstration par récurrence sur le nombre de points fixes d'une permutation.
 \end{itemize}
\end{rems}

\section{Signature}
\begin{defi}
 Soit $\sigma\in \mathfrak S_n$, la signature de $\sigma$ (notée $\varepsilon(\sigma)$) est définie par :
\begin{displaymath}
 \varepsilon(\sigma) = \prod_{i<j}\dfrac{\sigma(i)-\sigma(j)}{i-j}
\end{displaymath}
la somme étant étendue à tous les couples $(i,j)$ d'entiers entre $1$ et $n$ tels que $i<j$.
\end{defi}
En fait $\varepsilon$ est un morphisme dans $(\Q^*, .)$
\begin{propn}
 \begin{displaymath}
  \forall (\sigma_1,\sigma_2)\in \mathfrak S_n ^2 : \varepsilon(\sigma_1 \circ \sigma_2)
= \varepsilon(\sigma_1) \varepsilon(\sigma_2)
 \end{displaymath}
\end{propn}
\begin{demo}
 \begin{multline*}
  \varepsilon(\sigma_1 \circ \sigma_2)
= \prod_{i<j}\dfrac{\sigma_1(\sigma_2(i))-\sigma_1(\sigma_2(j))}{i-j}
= \prod_{i<j}\dfrac{\sigma_1(\sigma_2(i))-\sigma_1(\sigma_2(j))}{\sigma_2(i)-\sigma_2(j)}
  \prod_{i<j}\dfrac{\sigma_2(i)-\sigma_2(j)}{i-j}\\
=  \prod_{i_2<j_2}\dfrac{\sigma_1(i_2)-\sigma_1(j_2)}{i_2-j_2}
  \prod_{i<j}\dfrac{\sigma_2(i)-\sigma_2(j)}{i-j} 
= \varepsilon(\sigma_1)\varepsilon(\sigma_2)
 \end{multline*}
\end{demo}
En notant $\mathcal P_2$ l'ensemble des paires d'éléments de $\llbracket 1,n \rrbracket$, on a utilisé le fait que l'application
\begin{displaymath}
 \left\lbrace 
\begin{aligned}
 \mathcal P_2 &\rightarrow \mathcal P_2 \\
 \{i,j\} &\rightarrow \{\sigma(i),\sigma(j)\}
\end{aligned}
\right. 
\end{displaymath}
est une bijection pour toute permutation $\sigma$.
\begin{propn}
 Pour toute transposition $\tau = (i_0\;i_1)$, $\varepsilon(\tau)=-1$.
\end{propn}
\begin{demo}
 On suppose que $i_0<j_0$. On classe les éléments  $\mathcal P_2$ en trois catégories
\begin{description}
 \item [type 1] : ceux dont l'intersection avec $\{i_0,j_0\}$ est vide.
\item [type 2] : ceux dont l'intersection avec $\{i_0,j_0\}$ est de cardinal 1.
\item [type 3] : ceux dont l'intersection avec $\{i_0,j_0\}$ est de cardinal 2.
\end{description}
et on évalue la contribution de chaque type au produit.\newline
Les paires de type 1 ont évidemment une contribution égale à $1$.\newline
Les paires de type 2 (il y en a $2(n-2)$) sont les $\{k,i_0\}$ et les $\{k,j_0\}$ avec $k\not\in\{i_0,j_0\}$. Leur contribution est
\begin{displaymath}
 \prod_{k\not\in\{i_0,j_0\}}\dfrac{(\tau(k)-\tau(i_0))(\tau(k)-\tau(j_0)}{(k-i_0)(k-j_0)}
=
 \prod_{k\not\in\{i_0,j_0\}}\dfrac{(k-j_0)(k-i_0}{(k-i_0)(k-j_0)} =1
\end{displaymath}
Il existe une seule paire de type 2. c'est $\{i_0,j_0\}$ elle même, sa contribution est évidemment égale à $-1$.
\end{demo}
Conclusion $\varepsilon$ est un morphisme entre les groupes $(\mathfrak S_n,\circ)$ et $(\{-1,+1\},.)$.
\index{groupe alterné}
\begin{defi}
 Une permutation est dite paire lorsque sa signature est $+1$. L'ensemble des permutations paires forme le groupe alterné (noté $\mathcal A_n$). C'est un sous-groupe de $\mathfrak S_n$ (le noyau de la signature).
\end{defi}
\begin{rem}
 Soit $\tau$ une transposition et $\sigma$ une permutation. Alors $\sigma$ est paire si et seulement si $\tau \circ \sigma$ est impaire. On en déduit qu'il y a autant de permutations paires que d'impaires donc le groupe alterné $\mathcal A_n$ contient $\frac{n!}{2}$ éléments.
\end{rem}

\begin{propn}
 La signature d'un cycle de longueur $p$ est $(-1)^{p-1}$. En particulier les cycles de longueur $3$ sont des permutations paires.
\end{propn}
\begin{demo}
 Cela résulte des décompositions d'un cycle en transpositions (prop \ref{prop:decomptranspo})
\end{demo}

\end{document}
