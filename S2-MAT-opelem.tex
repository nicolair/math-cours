\subsubsection{D - Opérations élémentaires et systèmes linéaires.}
\subsubsubsection{a) Opérations élémentaires}
\begin{parcolumns}[rulebetween,distance=2.5cm]{2}

   \colchunk{Interprétation en termes de produit matriciel. }
  \colchunk{Les opérations élémentaires sont décrites dans le paragraphe \og Systèmes linéaires \fg du chapitre \og Calculs algébriques \fg.}
  \colplacechunks
   \colchunk{Les opérations élémentaires sur les colonnes (resp. lignes) conservent l'image (resp. noyau). Les opérations élémentaires conservent le rang.}
  \colchunk{Application au calcul du rang et à l'inversion de matrices.}
  \colplacechunks
\end{parcolumns}

\subsubsubsection{b) Systèmes linéaires}
\begin{parcolumns}[rulebetween,distance=2.5cm]{2}
   \colchunk{\'Ecriture matricielle d'un système linéaire.}
  \colchunk{Interprétation géométrique~: intersection d'hyperplans affines.}
  \colplacechunks
   \colchunk{Systèmes homogène associé. Rang, dimension de l'espace des solutions.}
  \colchunk{}
  \colplacechunks
   \colchunk{Compatibilité d'un système linéaire. Structure affine de l'espace des solutions.}
  \colchunk{}
  \colplacechunks
   \colchunk{Le système carré $Ax=b$ d'inconnue $x$ possède une et une seule solution si et seulement si $A$ est inversible. Système de Cramer.}
  \colchunk{Le théorème de Rouché-Fontené et les matrices bordantes sont hors programme.}
  \colplacechunks
   \colchunk{\S Algorithme du pivot de Gauss}
  \colchunk{}
 \end{parcolumns}
