\subsubsection{A - Inégalités dans $\R$}
\begin{parcolumns}[rulebetween,distance=\parcoldist]{2}
  \colchunk{Relation d'ordre sur $\R$. Compatibilité avec les opérations}
  \colchunk{Exemple de majoration et de minoration de sommes, de produit et de quotient.}
  \colplacechunks
  
  \colchunk{Parties positive et négative d'un réel. Valeur absolue. Inégalité triangulaire.}
  \colchunk{Notations $x^+$, $x^-$.}
  \colplacechunks
  
  \colchunk{Intervalles de $\R$.}
  \colchunk{Interprétation sur la droite réelle d'inégalités du type $|x-a|\leq b$.}
  \colplacechunks

  \colchunk{Parties majorées, minorées, bornées.\newline Majorant, minorant, maximum (ou plus grand élément), minimum (ou plus petit élémént).}
  \colchunk{Le \og plus simple des encadrements\fg (terminologie locale) :
\begin{displaymath}
n \min(x_1, \cdots, x_n) \leq \sum_{i=1}^n x_i \leq n \max(x_1, \cdots, x_n)
\end{displaymath}
Les notions de borne supérieure ou inférieure ne seront introduites que lors du cours de présentation axiomatique de $\R$.}

  \colplacechunks
  
  \colchunk{Exemples.}
  \colchunk{Inégalité de Cauchy-Schwarz. Preuve de la divergence de la série harmonique avec des puissances de $2$ et le plus simple des encadrements.}
    
\end{parcolumns}

