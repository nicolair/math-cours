%<dscrpt>Fichier de déclarations Latex à inclure au début d'un élément de cours.</dscrpt>

\documentclass[a4paper,landscape,twocolumn]{article}
\usepackage[hmargin={1.8cm,1.8cm},vmargin={2.4cm,2.4cm},headheight=13.1pt]{geometry}

%includeheadfoot,scale=1.1,centering,hoffset=-0.5cm,
\usepackage[pdftex]{graphicx,color}
\usepackage[french]{babel}
%\selectlanguage{french}
\addto\captionsfrench{
  \def\contentsname{Plan}
}
\usepackage{fancyhdr}
\usepackage{floatflt}
\usepackage{amsmath}
\usepackage{amssymb}
\usepackage{amsthm}
\usepackage{stmaryrd}
%\usepackage{ucs}
\usepackage[utf8]{inputenc}
%\usepackage[latin1]{inputenc}
\usepackage[T1]{fontenc}


\usepackage{titletoc}
%\contentsmargin{2.55em}
\dottedcontents{section}[2.5em]{}{1.8em}{1pc}
\dottedcontents{subsection}[3.5em]{}{1.2em}{1pc}
\dottedcontents{subsubsection}[5em]{}{1em}{1pc}

\usepackage[pdftex,colorlinks={true},urlcolor={blue},pdfauthor={remy Nicolai},bookmarks={true}]{hyperref}
\usepackage{makeidx}

\usepackage{multicol}
\usepackage{multirow}
\usepackage{wrapfig}
\usepackage{array}
\usepackage{subfig}


%\usepackage{tikz}
%\usetikzlibrary{calc, shapes, backgrounds}
%pour la présentation du pseudo-code
% !!!!!!!!!!!!!!      le package n'est pas présent sur le serveur sous fedora 16 !!!!!!!!!!!!!!!!!!!!!!!!
%\usepackage[french,ruled,vlined]{algorithm2e}

%pr{\'e}sentation du compteur de niveau 2 dans les listes
\makeatletter
\renewcommand{\labelenumii}{\theenumii.}
\renewcommand{\thesection}{\Roman{section}.}
\renewcommand{\thesubsection}{\arabic{subsection}.}
\renewcommand{\thesubsubsection}{\arabic{subsubsection}.}
\makeatother


%dimension des pages, en-t{\^e}te et bas de page
%\pdfpagewidth=20cm
%\pdfpageheight=14cm
%   \setlength{\oddsidemargin}{-2cm}
%   \setlength{\voffset}{-1.5cm}
%   \setlength{\textheight}{12cm}
%   \setlength{\textwidth}{25.2cm}
   \columnsep=1cm
   \columnseprule=0.5pt

%En tete et pied de page
\pagestyle{fancy}
\lhead{MPSI-\'Eléments de cours}
\rhead{\today}
%\rhead{25/11/05}
\lfoot{\tiny{Cette création est mise à disposition selon le Contrat\\ Paternité-Pas d'utilisations commerciale-Partage des Conditions Initiales à l'Identique 2.0 France\\ disponible en ligne http://creativecommons.org/licenses/by-nc-sa/2.0/fr/
} }
\rfoot{\tiny{Rémy Nicolai \jobname}}


\newcommand{\baseurl}{http://back.maquisdoc.net/data/cours\_nicolair/}
\newcommand{\urlexo}{http://back.maquisdoc.net/data/exos_nicolair/}
\newcommand{\urlcours}{https://maquisdoc-math.fra1.digitaloceanspaces.com/}

\newcommand{\N}{\mathbb{N}}
\newcommand{\Z}{\mathbb{Z}}
\newcommand{\C}{\mathbb{C}}
\newcommand{\R}{\mathbb{R}}
\newcommand{\D}{\mathbb{D}}
\newcommand{\K}{\mathbf{K}}
\newcommand{\Q}{\mathbb{Q}}
\newcommand{\F}{\mathbf{F}}
\newcommand{\U}{\mathbb{U}}
\newcommand{\p}{\mathbb{P}}


\newcommand{\card}{\mathop{\mathrm{Card}}}
\newcommand{\Id}{\mathop{\mathrm{Id}}}
\newcommand{\Ker}{\mathop{\mathrm{Ker}}}
\newcommand{\Vect}{\mathop{\mathrm{Vect}}}
\newcommand{\cotg}{\mathop{\mathrm{cotan}}}
\newcommand{\sh}{\mathop{\mathrm{sh}}}
\newcommand{\ch}{\mathop{\mathrm{ch}}}
\newcommand{\argsh}{\mathop{\mathrm{argsh}}}
\newcommand{\argch}{\mathop{\mathrm{argch}}}
\newcommand{\tr}{\mathop{\mathrm{tr}}}
\newcommand{\rg}{\mathop{\mathrm{rg}}}
\newcommand{\rang}{\mathop{\mathrm{rg}}}
\newcommand{\Mat}{\mathop{\mathrm{Mat}}}
\newcommand{\MatB}[2]{\mathop{\mathrm{Mat}}_{\mathcal{#1}}\left( #2\right) }
\newcommand{\MatBB}[3]{\mathop{\mathrm{Mat}}_{\mathcal{#1} \mathcal{#2}}\left( #3\right) }
\renewcommand{\Re}{\mathop{\mathrm{Re}}}
\renewcommand{\Im}{\mathop{\mathrm{Im}}}
\renewcommand{\th}{\mathop{\mathrm{th}}}
\newcommand{\repere}{$(O,\overrightarrow{i},\overrightarrow{j},\overrightarrow{k})$}
\newcommand{\cov}{\mathop{\mathrm{Cov}}}

\newcommand{\absolue}[1]{\left| #1 \right|}
\newcommand{\fonc}[5]{#1 : \begin{cases}#2 \rightarrow #3 \\ #4 \mapsto #5 \end{cases}}
\newcommand{\depar}[2]{\dfrac{\partial #1}{\partial #2}}
\newcommand{\norme}[1]{\left\| #1 \right\|}
\newcommand{\se}{\geq}
\newcommand{\ie}{\leq}
\newcommand{\trans}{\mathstrut^t\!}
\newcommand{\val}{\mathop{\mathrm{val}}}
\newcommand{\grad}{\mathop{\overrightarrow{\mathrm{grad}}}}

\newtheorem*{thm}{Théorème}
\newtheorem{thmn}{Théorème}
\newtheorem*{prop}{Proposition}
\newtheorem{propn}{Proposition}
\newtheorem*{pa}{Présentation axiomatique}
\newtheorem*{propdef}{Proposition - Définition}
\newtheorem*{lem}{Lemme}
\newtheorem{lemn}{Lemme}

\theoremstyle{definition}
\newtheorem*{defi}{Définition}
\newtheorem*{nota}{Notation}
\newtheorem*{exple}{Exemple}
\newtheorem*{exples}{Exemples}


\newenvironment{demo}{\renewcommand{\proofname}{Preuve}\begin{proof}}{\end{proof}}
%\renewcommand{\proofname}{Preuve} doit etre après le begin{document} pour fonctionner

\theoremstyle{remark}
\newtheorem*{rem}{Remarque}
\newtheorem*{rems}{Remarques}

\renewcommand{\indexspace}{}
\renewenvironment{theindex}
  {\section*{Index} %\addcontentsline{toc}{section}{\protect\numberline{0.}{Index}}
   \begin{multicols}{2}
    \begin{itemize}}
  {\end{itemize} \end{multicols}}


%pour annuler les commandes beamer
\renewenvironment{frame}{}{}
\newcommand{\frametitle}[1]{}
\newcommand{\framesubtitle}[1]{}

\newcommand{\debutcours}[2]{
  \chead{#1}
  \begin{center}
     \begin{huge}\textbf{#1}\end{huge}
     \begin{Large}\begin{center}Rédaction incomplète. Version #2\end{center}\end{Large}
  \end{center}
  %\section*{Plan et Index}
  %\begin{frame}  commande beamer
  \tableofcontents
  %\end{frame}   commande beamer
  \printindex
}


\makeindex
\begin{document}
\noindent

\debutcours{Espaces préhilbertiens réels}{1.2 \tiny{le \today}}

Cette section est complétée par \href{\baseurl C6172.pdf}{Espace euclidien en dimension 2 ou 3} qui introduit en particulier les notions de produits vectoriels et de produit mixte.\footnote{Merci à Yann Duplouy (MPSI B 2009-2010) pour les premiers éléments de la saisie \LaTeX de cette partie.}

\section{Produits scalaires}
\subsection{Définition}
\begin{defi}
Une \emph{forme bilinéaire} d'un $\K$-espace vectoriel $E$ est une application de $E\times E$ dans $\K$ qui est linéaire par rapport à chacune de ses variables. C'est à dire que pour tout $x\in E$, les applications
\begin{displaymath}
 y\rightarrow B(x,y) \text{ et } y\rightarrow B(y,x)
\end{displaymath}
sont des formes linéaires.
\end{defi}
\index{forme bilinéaire symétrique}
\begin{defi}
On dira que la forme bilinéaire est \emph{symétrique} lorsque 
\begin{displaymath}
  \forall (x,y)\in E^2,\; B(x,y) = B(y,x)
\end{displaymath}
\end{defi}
\index{produit scalaire}
\begin{defi}[Produit scalaire]
Un produit scalaire sur un espace vectoriel \emph{réel} $E$ est une forme bilinéaire symétrique vérifiant
\begin{displaymath}
\forall x \in E, \; B(x,x) \geq 0 \text{ et } B(x,x) = 0 \Rightarrow x = 0_E .
\end{displaymath}
\end{defi}
\begin{exples}
\begin{itemize}
\item Produit scalaire canonique sur $\R^n$ ou sur les matrices colonnes.
\begin{itemize}
  \item[Sur $\R^n$].
\[
  \forall x = (x_1, \cdots, x_n) \in \R^n,\;  \forall y = (y_1, \cdots, y_n) \in \R^n, \hspace{0.5cm}
  B(x,y) = \sum_{i=1}^{n}x_iy_i.
\]
  \item[Sur $\mathcal{M}_{n,1}(\R)$].
\begin{multline*}
  \forall X = 
  \begin{pmatrix}
    x_1\\ \vdots \\ x_n
  \end{pmatrix}
 \in \mathcal{M}_{n,1}(\R),\;
   \forall Y = 
  \begin{pmatrix}
    y_1\\ \vdots \\ y_n
  \end{pmatrix}
 \in \mathcal{M}_{n,1}(\R), \\
 B(X,Y) = \sum_{i=1}^{n}x_iy_i.
\end{multline*}
\end{itemize}

\item $E = \mathcal{C}^{0}( [a,b] )$ avec $a < b$. Pour $(f,g) \in E^2$,
\begin{displaymath}
 B(f,g) = \int_{a}^{b} f(x) g(x) \,dx
\end{displaymath}
\item $E = \mathcal{M}_{p}(\K)$, $M$ et $N \in E$:
\begin{center}
$B(M,N) = \mathrm{tr}( \trans M N )$
\end{center}
\end{itemize}
\end{exples}
\newpage
Dans la suite, on va considérer des $\R$-espaces vectoriels munis de produits scalaires. On désigne de telles structures par des noms particuliers.
\index{espace euclidien}\index{espace pré-hilbertien réel}
\begin{itemize}
\item \og Espace euclidien \fg~ si le $\R$-espace vectoriel $E$ est de dimension \emph{finie}.
\item \og Espace pré-hilbertien réel\fg~  si le $\R$-espace vectoriel $E$ est de dimension \emph{quelconque}.
\end{itemize}
\begin{nota}
Tous les produits scalaires seront notés $(~/~)$, de manière infixe.
\end{nota}

\subsection{Formules}
Dans la suite du cours, $E$ est un $\R$-espace vectoriel (pré-hilbertien réel), muni d'un produit scalaire $(~/~)$.
\begin{nota}
Pour tout vecteur $x$ de $E$, $ \Vert x \Vert  = \sqrt{ (x/x) }$ est la \emph{norme} (associée au produit scalaire) de $x$.\index{norme d'un vecteur}
\end{nota}
\begin{prop}
\begin{align*}
\forall (x,y) \in E^2,\hspace{0.5cm}     &\Vert x+y \Vert ^2 =  \Vert x \Vert ^2 +  \Vert y \Vert ^2 + 2(x/y). \\
\forall (x_1,\cdots,x_n)\in E^n,\hspace{0.5cm} &\Vert \sum_{i=1}^n x_i \Vert^2 = \sum_{i=1}^n \Vert x_i \Vert^2 + 2 \sum_{\stackrel{(i,j)\in\llbracket 1,n \rrbracket^2}{i < j}}(x_i / x_j).
\end{align*}
\end{prop}
\begin{proof}
Il suffit de développer $(x+y/x+y)$ ou $(\sum_{i=1}^n x_i/ \sum_{j=1}^n x_j)$ par bilinéarité en utilisant le caractère symétrique.
\end{proof}
\index{identités de polarisation}
\newpage
\begin{prop}[Identité de \og polarisation\fg]
On peut exprimer le produit scalaire à partir de la norme. Pour tous vecteurs $x$ et $y$ de $E^2$:
\begin{align*}
 \Vert x+ y\Vert^2 = \Vert x \Vert ^2 + \Vert y \Vert^2 + 2 (x/y) \Rightarrow& 
 (x/y) = \frac{1}{2} \left(  \Vert x+y \Vert ^2 -  \Vert x \Vert ^2 -  \Vert y \Vert ^2 \right).\\
 (x+y/x-y) = \Vert x \Vert^2 - \Vert y \Vert^2 \Rightarrow& (x/y)= \frac{1}{4} \left(  \Vert x+y \Vert ^2 -  \Vert x-y \Vert ^2 \right). 
\end{align*}
\end{prop}
\newpage
\index{inégalité de Cauchy-Schwarz}
\begin{prop}[Inégalité de Cauchy-Schwarz]
Soit $(~/~)$ un produit scalaire sur $E$.
\begin{displaymath}
\forall(x,y) \in E^2,\;| (x/y) | \leq  \Vert  x  \Vert   \Vert  y  \Vert .
\end{displaymath}
Il y a égalité si et seulement si $(x,y)$ est liée.
\end{prop}
\begin{proof}
Considérons l'application  $\varphi$ :
\begin{displaymath}
 \left\lbrace 
\begin{aligned}
 \R & \rightarrow  \R \\
 t & \mapsto   \Vert x + ty \Vert ^2 =  \Vert y^2 \Vert  t^2 + 2 (x/y) t +  \Vert  x  \Vert ^2
\end{aligned}
\right. .
\end{displaymath}
Cette expression ne peut prendre que des valeurs positives. Le discriminant $4(  \Vert (x/y) \Vert ^2 -  \Vert y \Vert ^2 \Vert x \Vert ^2 )$ de ce polynôme en $t$ du second degré est alors inférieur ou égal à 0.\newline
Si le discriminant est nul, il existe un $t$ tel que $x+ty = 0_E$, ce qui signifie que $(x,y)$ est liée.
\end{proof}
Cette inégalité a deux conséquences:
\begin{itemize} \index{inégalité triangulaire}
\item \emph{l'inégalité triangulaire}:
\begin{center}
$ \Vert  x + y \Vert ^2 =  \Vert x \Vert ^2 + 2(x/y) +  \Vert y \Vert ^2 \leq  \Vert x \Vert ^2 + 2  \Vert x \Vert   \Vert y \Vert  +  \Vert y \Vert ^2$ \\
$ \Vert x+y \Vert ^2 \leq ( \Vert x \Vert  +  \Vert y \Vert )^2$
\end{center}
\item et la définition de \emph{l'écart angulaire} :\index{écart angulaire}
\end{itemize}
\newpage
\begin{defi}
L'écart angulaire entre deux vecteurs $x$ et $y$ est défini par:
\begin{center}
$\arccos \displaystyle \frac{ (x/y) }{  \Vert x \Vert   \Vert y \Vert  } \in [0, \pi]$.
\end{center}
\end{defi}
\begin{proof}[Existence]
L'écart angulaire est bien défini car, d'après l'inégalité de Cauchy-Schwarz, on a $\frac{ (x/y) }{  \Vert x \Vert   \Vert y \Vert  } \in [0,1]$.
\end{proof}
\newpage
\begin{exple}
Quel est l'écart angulaire entre la fonction exponentielle et la fonction $x \mapsto 1$?
\begin{itemize}
\item \emph{Dans quel espace ?} L'espace des fonctions continues sur $[0,1]$.
\item \emph{Pour quel produit scalaire ?} On choisit
\begin{displaymath}
(f/g) = \int_{0}^{1} f(t) g(t) \,dt.
\end{displaymath}
\end{itemize}
Notons $1$ la fonction constante égale à $1$ sur $[0,1]$ et $e$ la restriction à $[0,1]$ de la fonction exponentielle. On peut alors calculer:
\begin{multline*}
 \Vert 1 \Vert  = \sqrt{ \int_{0}^{1} 1 \,dt } = 1, \\
 \Vert e \Vert  = \sqrt{ \int_{0}^{1} (e^{t})^2 \,dt } = \sqrt{ \int_{0}^{1} e^{2t} \,dt } = \sqrt{ \left[ \frac{e^{2t}}{2} \right]_{0}^{1} } = \sqrt{ \frac{e^2 - 1}{2} }\\
 (1 / e) = \int_{0}^{1} e^{t} \, dt = e - 1
\end{multline*}
L'écart angulaire est alors égal à :
\begin{displaymath}
\displaystyle \arccos \frac{ (e - 1) \sqrt{2} }{ \sqrt{ e^2 - 1 } } = \arccos \sqrt{2 \, \frac{e - 1}{e + 1}}.
\end{displaymath}
On remarque que si on introduit une fonction continue positive et fixée dans l'intégrale, entre $g(t)$ et $\, dt$, on change le produit scalaire donc aussi l'écart angulaire.
\end{exple}
\clearpage
\subsection{Aspect matriciel}
La possibilité d'une représentation matricielle d'un produit scalaire vient de la bilinéarité.
\begin{defi}\index{matrice d'un produit scalaire}
La matrice dans une base $\mathcal{B} = (b_1, ..., b_p)$ d'un produit scalaire $(~/~)$ de $E$ est la matrice de $\mathcal{M}_{p}(\K)$ dont le terme $i,j$ est $(b_i/b_j)$ pour tout $(i,j)$ de $\llbracket 1,p \rrbracket^2$.  On peut la noter $\Mat_\mathcal{B} (~/~)$.
\end{defi}
\begin{prop}
La matrice d'un produit scalaire dans une base est symétrique.
\end{prop}
\begin{exple}
$E = \R_{3}[X]$, $\mathcal{C} = (1, X, X^2, X^3)$:
\begin{center}
$(P/Q) = \displaystyle \int_{0}^{1} P(t)Q(t) \, dt$.
\end{center}
La matrice dans $\mathcal{C}$ de ce produit scalaire est:
\begin{displaymath}
\renewcommand{\arraystretch}{1.8}
\begin{pmatrix}
1 & \frac{1}{2} & \frac{1}{3} & \frac{1}{4} \\
\frac{1}{2} & \frac{1}{3} & \frac{1}{4} & \frac{1}{5} \\
\frac{1}{3} & \frac{1}{4} & \frac{1}{5} & \frac{1}{6} \\
\frac{1}{4} & \frac{1}{5} & \frac{1}{6} & \frac{1}{7}
\end{pmatrix}.
\end{displaymath}
\end{exple}
\newpage
\begin{prop}[Expression matricielle du produit scalaire]
Soit $E$ un $\R$ espace vectoriel muni d'une base $\mathcal{B}$ et d'un produit scalaire  $(~/~)$. Soit $S$ la matrice de $(~/~)$ dans $\mathcal{B}$. Alors, pour tout $x$ et tout $y$ de $E$,
\begin{displaymath}
(x/y) = \trans X S\,Y \text{ avec } X = \Mat_\mathcal{B} x \text{ et } Y = \Mat_\mathcal{B} y .
\end{displaymath}
\end{prop}
\begin{proof}
Notons $X = \begin{pmatrix} x_1 \\ \vdots \\ x_p \end{pmatrix}$ et $Y = \begin{pmatrix} y_1 \\ \vdots \\ y_p \end{pmatrix}$. Alors :
\begin{multline*}
(x/y) = \left( \sum_{i} x_i b_i \left/ \right. \sum_{i} y_i b_i \right)
 = \sum_{i=1}^{p} \left( \sum_{j=1}^{p} x_i y_i (b_i/b_j) \right)
 = \sum_{i} x_i \left( \sum_{j=1}^{p} b_{i,j} y_j \right)\\
 = \sum_{i} x_i (\text{ terme $i$ de la matrice }S\,Y) = \trans X\,S\,Y .
\end{multline*}
\end{proof}
\begin{prop}[Effet d'un changement de base]
Soit $E$ un $\R$ espace vectoriel muni d'une base $\mathcal{B}$ et d'un produit scalaire  $(~/~)$. Soit $S$ la matrice de $(~/~)$ dans $\mathcal{B}$. Soit $\mathcal{C}$ une autre base de $E$. On note $P = P_{\mathcal{BC}}$ la matrice de passage de $\mathcal{B}$ dans $\mathcal{C}$.\newline
La matrice de $(~/~)$ dans $\mathcal{C}$ est alors égale à $\trans P SP$.
\end{prop}
\begin{proof}
On cherche le terme $i,j$ de la matrice de $(~/~)$ dans $\mathcal{C}$.
\begin{displaymath}
(c_i/c_j) = \underset{c_i(P)}{\underbrace{\trans {\Mat_{\mathcal{B}} c_i}}}~S~\underset{c_j(P)}{\underbrace{\trans{\Mat_{\mathcal{B}} c_j}}} = \trans{ c_i(P) ) S c_j(P) } = L_i( \trans P ) S C_j(P) = \text{terme }i,j\text{ de }\trans P S P .
\end{displaymath}
\end{proof}
\begin{rem}
$\trans{(\trans PSP)} = \trans P\,\trans{S}\,\trans{\trans P} = \trans P\,S\,P$.
\end{rem}
\newpage
\section{Orthogonalité}
\subsection{Introduction}
Un produit scalaire sur $E$ induit une application linéaire de $E$ dans le dual $E^*$. Soit $E$ un espace euclidien avec un produit scalaire $(/)$. Pour tout $a\in E$, notons $(a/.)$ l'application 
\begin{displaymath}
 \left\lbrace 
\begin{aligned}
 E &\rightarrow K \\ x &\mapsto (a/x)
\end{aligned}
\right. .
\end{displaymath}
Elle est clairement linéaire, c'est donc une forme linéaire élément de $E^*$ dual de $E$. Ceci définit une application de $E$ dans le dual $E^*$.
Cette application est un isomorphisme. 
\begin{prop}
 Soit $E$ un espace euclidien avec un produit scalaire $(/)$. L'application
\begin{displaymath}
 \left\lbrace 
\begin{aligned}
  E &\rightarrow E^* \\ a &\mapsto (a/.)
\end{aligned}
\right. .
\end{displaymath}
est un isomorphisme. Pour toute forme linéaire $\alpha\in E^*$, il existe un unique vecteur $u\in E$ tel que
\begin{displaymath}
 \alpha = (u/ .)\Leftrightarrow
\forall x\in E : \alpha(x) = (u/x) .
\end{displaymath}
\end{prop}
\begin{demo}
 La linéarité de la fonction $(a/.)$ de $E$ dans $\R$ est facile à vérifier de même que celle de la fonction $a \mapsto (a/.)$ de $E$ dans $E^*$ (on la note $\varphi$). On sait d'autre part que $E$ et $E^*$ sont de même dimension. Il suffit donc de vérifier que $\ker \varphi$ se réduit au vecteur nul. Or $a\in\ker \varphi$ si et seulement si $(a/.)$ est la forme linéaire nulle c'est à dire que, pour tout $x\in E$, on doit avoir $(a/x)=0$. En particulier, ceci doit être vrai pour $x=a$ ce qui entraîne $\Vert a\Vert^2=0$ d'où $x=0_E$.
\end{demo}
\clearpage
\subsection{Vecteurs orthogonaux. Orthogonal d'une partie}
\begin{defi} \index{orthogonal d'une partie}
 Soit $E$ un espace euclidien avec un produit scalaire $(/)$. Deux vecteurs $x$ et $y$ de $E$ sont dits \emph{orthogonaux} si et seulement si $(x/y)=0$.
\end{defi}
\begin{defi}
 Soit $E$ un espace euclidien avec un produit scalaire $(/)$ et $A$ une partie de $E$. L'orthogonal de $A$ (noté $A^\bot$) est formé par les éléments de $E$ orthogonaux à tous les éléments de $A$.
\begin{displaymath}
 \forall x\in E : \left( x\in A^\bot \Leftrightarrow
\forall a\in A : (a/x)=0 \right) .
\end{displaymath}
\end{defi}
\begin{prop}
 Soit $E$ un espace euclidien avec un produit scalaire $(/)$ et $A$, $B$ deux parties de $E$ alors 
\begin{displaymath}
 A \subset B \Rightarrow B^\bot \subset A^\bot .
\end{displaymath}
\end{prop}
\begin{demo}
On suppose $A \subset B$, soit $x\in B^\bot$. Pour tout $a\in A$, $a\in B$ donc $(x/a)=0$. On en déduit $x \in A^\bot$.
\end{demo}
\begin{rem}
 Attention à la proposition suivante qui caractérise seulement une inclusion et non une égalité.
\[
 \left(  \forall (a,b) \in A\times B,\; (a/b) = 0\right) \Leftrightarrow A \subset B^\bot \Leftrightarrow B \subset A^\bot. 
\]
\end{rem}

\index{hyperplan orthogonal}
\begin{defi}
Hyperplan orthogonal à un vecteur non nul. Pour un vecteur $a$ fixé, $\{a\}^{\bot}=\Vect(a)^\bot$ est le noyau de la forme linéaire $x\rightarrow (a/x)$. C'est donc un hyperplan. Lorsque la dimension est finie (espace euclidien), sa dimension est $\dim E -1$.  
\end{defi}

\begin{prop}
 Soit $E$ un espace euclidien avec un produit scalaire $(/)$ et $A$ une partie de $E$ alors $A^\bot$ est un sous espace vectoriel. De plus, $(\Vect(A))^\bot = A^\bot$.
\end{prop}
\begin{demo}
Pour tout $a\in A$, l'ensemble des $x$ de $E$ orthogonaux à $a$ est un hyperplan, $A^\bot$ est l'intersection de tous ces hyperplans donc encore un sous-espace.
\end{demo}

\clearpage
\subsection{Familles orthogonales}
 \begin{defi}
  Une famille est \emph{orthogonale} lorsque ses vecteurs sont deux à deux orthogonaux. Une famille est dite \emph{orthonormale} ou \emph{orthonormée} lorsqu'elle orthogonale et que ses vecteurs sont de norme 1 (unitaires).
 \end{defi}
\begin{rem}
  Si $(a_1,\cdots a_p)$ est une famille orthogonale de vecteurs non nuls alors 
\begin{displaymath}
 (\frac{1}{\Vert a_1\Vert}a_1,\cdots, \frac{1}{\Vert a_p\Vert}a_p)
\end{displaymath}
est une famille orthonormée.
\end{rem}

\begin{prop}
 Une famille orthogonale de vecteurs non nuls est libre. C'est en particulier vérifié pour une famille orthonormée.
\end{prop}
\begin{demo}
 En effet soit $(a_1,\cdots a_p)$ une famille orthogonale de vecteurs non nuls et $\lambda_1, \cdots, \lambda_p$  des scalaires vérifiant:
\begin{displaymath}
 \lambda_1 a_1 +\cdots +\lambda_p a_p = 0_E .
\end{displaymath}
Alors, pour tout $i$ entre $1$ et $p$ :
\begin{displaymath}
 \left( \bigl(\lambda_1 a_1 +\cdots +\lambda_p a_p\bigr)/a_i\right)  = 0_K\Rightarrow
\lambda_i \Vert a_i\Vert^2=0_K\Rightarrow\lambda_i=0 .
\end{displaymath}
\end{demo}
\begin{prop}[Expression des coordonnées d'un vecteur dans une base orthogonale, orthonormale]
 Soit $(a_1,\cdots a_p)$ une famille orthogonale de vecteurs non nuls et $x\in \Vect(a_1,\cdots,a_p)$. Alors
\begin{displaymath}
 x = \sum_{i=1}^p\frac{(x/a_i)}{\Vert a_i\Vert^2}a_i.
\end{displaymath}
 Soit $(a_1,\cdots a_p)$ est une famille orthonormée de vecteurs et $x\in \Vect(a_1,\cdots,a_p)$. Alors
\begin{displaymath}
 x = \sum_{i=1}^p (x/a_i) a_i .
\end{displaymath}
Lorsque la famille est une base, ces formules expriment les coordonnées de $x$.
\end{prop}

\begin{prop}
 La restriction d'un produit scalaire à un sous-espace est encore une produit scalaire.
\end{prop}
\begin{demo}
 Toutes les propriétés bilinéarité, symétrie et positivité découlent des propriétés du produit scalaire dans l'espace de départ.
\end{demo}
\clearpage
\index{existence d'une base orthogonale}
\begin{thm}[Existence d'une base orthogonale]
Dans tout espace euclidien, il existe des bases orthogonales.
\end{thm}
\begin{demo}
 On démontre ce théorème par récurrence sur la dimension de l'espace. Introduisons la propriété
\begin{quote}
 $\mathcal{P}_n$ : Tout espace euclidien de dimension $n$ admet des bases orthogonales.
\end{quote}
Il est évident que cette proposition est vraie pour la dimension $1$ car toute base (formée d'un seul vecteur) est orthogonale.\newline
Montrons maintenant que $\mathcal{P}_{n-1}$ entraîne $\mathcal{P}_{n}$.\newline
On considère un espace euclidien de dimension $n$ dans lequel on fixe un vecteur non nul $u_1$. On note $H$ l'hyperplan orthogonal à $u_1$. Il est de dimension $n-1$ et il est euclidien pour la restriction du produit scalaire. La proposition $\mathcal{P}_{n-1}$ entraîne alors l'existence d'une base orthogonale $(u_2,\cdots,u_n)$ de $H$. Il est immédiat que $(u_1,u_2,\cdots,u_n)$ est encore orthogonale et forme une base de $E$.
\end{demo}
\clearpage
\subsection{Aspect matriciel}
\begin{prop}
 Soit $E$ un espace euclidien avec un produit scalaire $(/ )$. Une base $\mathcal B$ de $E$ est orthogonale si et seulement si la matrice du produit scalaire dans $\mathcal B$ est diagonale. La base $\mathcal B$ est orthonormée si et seulement si la matrice du produit scalaire dans $\mathcal B$ est la matrice identité.
\end{prop}
\begin{demo}
  C'est une simple traduction des définitions.
\end{demo}

\begin{prop}
 Soit $p$ entier naturel non nul et $S\in \mathcal M_p(\R)$. La matrice $S$ est la matrice d'un produit scalaire dans une base si et seulement si il existe $P\in GL_p(\R)$ telle que $S = \trans P\, P$.
\end{prop}
\begin{demo}
 Si $S = \trans P\, P$ alors $S$ est symétrique. Considérons un espace $E$ avec une base $\mathcal B$. Définissons une forme bilinéaire symétrique dans $E$ par sa matrice dans $\mathcal{B}$ que l'on impose égale à $S$. Soit $x$ un vecteur quelconque dans $E$ et $X$ la matrice colonne de ses coordonnées dans $\mathcal B$. Alors
\begin{multline*}
 (x/x) = \trans X S X = \left( \trans X \trans P\right) \left(  P X\right) = \trans \left(  P X\right)\left(  P X\right)
= y_1^2+\cdots+y_p^2 \\
\text{ avec }
\left(  P X\right)
=
\begin{pmatrix}
 y_1\\ \vdots \\ y_p
\end{pmatrix} .
\end{multline*}
On en déduit la positivité nécessaire à un produit scalaire.\newline
Réciproquement, si $S$ est la matrice d'un produit scalaire d'un espace $E$, dans une base $\mathcal{B}$, il existe une base $\mathcal{A}$ de $E$ orthonormée pour ce produit scalaire. La matrice du produit scalaire dans $\mathcal{A}$ est $I_p$ et la formule de changement de base s'écrit
\begin{displaymath}
 I_p = \trans P_{\mathcal{B}\mathcal{A}} S P_{\mathcal{B}\mathcal{A}} \Rightarrow
S = \trans P_{\mathcal{A}\mathcal{B}} P_{\mathcal{A}\mathcal{B}}.
\end{displaymath}
\end{demo}
\begin{rem}
 On en déduit que le déterminant de la matrice d'un produit scalaire est strictement positif.
\end{rem}
\clearpage
\section{Supplémentaire orthogonal - Projections orthogonales.}
\index{supplémentaire orthogonal}
\subsection{Supplémentaire orthogonal}
\begin{thm} Soit $A$ un sous-espace vectoriel d'un espace euclidien $E$ : 
 \begin{displaymath}
 \dim A^{\perp} = \dim E - \dim A .
\end{displaymath}
De plus $A$ et $A^\perp$ sont supplémentaires. On dira que $A^\bot$ est \emph{le} supplémentaire \emph{orthogonal} de $A$.
\end{thm}
\begin{demo}
On note $n$ la dimension de $E$ et $p$ celle de $A$. On considère une base orthonormée $(a_1,\cdots,a_p)$ de $A$ et l'application
\begin{displaymath}
\varphi : \left\lbrace 
 \begin{aligned}
  E &\rightarrow \R^p \\
  x &\rightarrow \left( (x/a_1),(x/a_2),\cdots,(x/a_p)\right) 
 \end{aligned}
\right. .
\end{displaymath}
Cette application est clairement linéaire de noyau $\Vect(a_1,\cdots,a_p)^\bot= A^\bot$. Le théorème du rang conduit à :
\begin{displaymath}
 \dim E = \dim A^\bot +\rg \varphi .
\end{displaymath}
Or $\varphi$ est surjective car les vecteurs de la base canonique de $\R^p$ sont les images des $a_i$. On en tire $\dim A^\bot = \dim E - \dim A$.\newline
L'intersection de $A\cap A^\bot$ est réduite au vecteur nul car tout vecteur dans cette intersection est orthogonal à lui même. Avec les dimensions, cela entraîne que $A$ et $A^\bot$ sont supplémentaires.
\end{demo}
\begin{rem}
 On déduit du théorème précédent que $\dim A^{\perp\perp} = \dim A$ dans un espace euclidien.
\end{rem}
\clearpage
\index{base orthogonale incomplète}
\begin{prop}[Base orthogonale incomplète]
 Soit $(a_1,\cdots,a_p)$ une famille orthogonale dans un espace euclidien $E$ de dimension $n$. Il existe $b_{p+1},\cdots,b_n$ tels que $(a_1,\cdots,a_p,b_{p+1},\cdots,b_n)$ soit une base orthogonae de $E$. (idem en remplaçant orthogonale par orthonormée)
\end{prop}
\begin{demo}
 Soit $A=\Vect(a_1,\cdots,a_p)$, on peut considérer une base orthogonale $(b_{p+1},\cdots,b_n)$ du supplémentaire orthogonal.
\end{demo}
\begin{prop}
 Soit $A$, $B$ des sous-espaces vectoriels d'un espace vectoriel $E$. Alors :
\begin{displaymath}
 A^{\bot \bot}=A ,\hspace{1cm} (A+B)^\bot = A^\bot \cap B^\bot, \hspace{1cm} (A\cap B)^\bot = A^\bot + B^\bot .
\end{displaymath}
\end{prop}
\begin{proof}
\begin{itemize}
\item Pour tout $a \in A$, on veut montrer que $a \in A^{\perp\perp}$. On va donc le tester face aux éléments de $A^{\perp}$. Pour tout $x \in A^{\perp}$, 
\begin{displaymath}
 (\underset{\in A}{a} / \underset{\in A^{\perp}}{x}) = 0
\end{displaymath}
donc $a \in A^{\perp\perp}$, ce qui signifie que $A \subset A^{\perp\perp}$.\newline
On obtient l'égalité des sous-espaces en adjoignant l'égalité des dimensions à cette inclusion. En effet, d'après le théorème, $\dim A^{\perp\perp} = \dim A$.
\begin{displaymath}
 \left.
\begin{aligned}
&A \subset A^{\perp\perp} \\
&\dim A = \dim A^{\perp \perp}
\end{aligned}
 \right\rbrace \Rightarrow A = A^{\perp \perp} .
\end{displaymath}
\item Pour l'orthogonal de la somme:
\begin{displaymath}
 \left. 
\begin{aligned}
&A \subset A + B & (A+B)^{\perp} \subset A^{\perp} \\
&B \subset A + B & (A+B)^{\perp} \subset B^{\perp}
\end{aligned}
\right\rbrace 
\Rightarrow (A+B)^{\perp} \subset A^{\perp} \cap B^{\perp} .
\end{displaymath}
Montrons maintenant que $( A^{\perp} \cap B^{\perp} ) \subset (A+B)^{\perp}$. On doit vérifier:
\[
\forall x \in A^{\perp} \cap B^{\perp}, \forall y \in A+B : (x/y) = 0 .
\]
Or, pour tout $y$ de $A+B$ il existe $a \in A$ et $b \in B$ tels que: $y = a + b$ et on peut conclure avec
\[
(x/y) = (\underset{A^{\perp}}{x}/\underset{\in A}{a}) + (\underset{\in B^{\perp}}{x} / \underset{\in B}{b}) = 0 .
\]
\item On obtient la troisième égalité à partir de la deuxième. Appliquons la deuxième propriété à $A^{\perp}$ et $B^{\perp}$ :
\[
(A^{\perp} + B^{\perp})^{\perp} = (A^{\perp})^{\perp} \cap (B^{\perp})^{\perp} = A \cap B
\]
puis la première:
\begin{displaymath}
(A^{\perp} + B^{\perp})^{\perp} = A \cap B 
 \Rightarrow A^{\perp} + B^{\perp} = (A \cap B)^{\perp} .
\end{displaymath}
\end{itemize}
\end{proof}
\newpage
\subsection{Projections orthogonales}
\index{projection orthogonale}
\begin{prop}
 Soit $A$ un sous-espace de $E$ et $p_A$ la projection orthogonale sur $A$ (c'est à dire la projection sur $A$ parallèlement à $A^\bot$) alors pour toute base orthonormée $(a_1,\cdots,a_p)$ de $A$, on a :
\begin{displaymath}
 \forall x\in E : p_A(x)=\sum_{i=1}^p(x/a_i)a_i .
\end{displaymath}
\end{prop}
\begin{prop}
 Soit $A$ un sous-espace de $E$ et $p_A$ la projection orthogonale sur $A$ alors :
\begin{displaymath}
 \begin{aligned}
  &\forall x \in E :& \Vert p_A(x) \Vert \leq \Vert x \Vert, \\
  &\forall (x,y) \in E^2 :& (p_A(x)/y) = (x/p_A(y)).
 \end{aligned}
\end{displaymath}
\end{prop}
\begin{demo}
 L'orthogonalité se traduit par le théorème de Pythagore
 \begin{displaymath}
\left\|x \right\|^2 = \left\| p_A(x) + p_{A^\perp}(x) \right\|^2 =
\left\| p_A(x) \right\|^2 + \left\| p_{A^\perp}(x) \right\|^2
\geq \left\| p_A(x) \right\|^2 .
 \end{displaymath}
D'autre part,
\begin{multline*}
(p_A(x)/y) = (p_A(x)/p_A(y) + p_{A^\perp}(y)) = (p_A(x)/p_A(y))\\
= (p_A(x)+p_{A^\perp}(x)/p_A(y)) = (x/p_A(y)) .
\end{multline*}
\end{demo}
\newpage
\index{distance à un sous-espace}
\begin{prop}[Distance à un sous-espace]
 Soit $A$ un sous-espace de $E$ et $x\in E$. alors :
\begin{displaymath}
 \Vert x - p_A(x)\Vert = \Vert p_{A^\bot}(x)\Vert
=\min\left\lbrace \Vert x-a\Vert, a\in A\right\rbrace .
\end{displaymath}
Ce nombre est appelé distance de $x$ à $A$ et noté $d(x,A)$.
\end{prop}
\begin{demo}
Pour tout $a\in A$, décomposons $x-a$.
\begin{displaymath}
\Vert x - a\Vert^2=\Vert p_{A\perp}(x) + \underset{\in A}{\underbrace{p_A(x) -a}}\Vert^2  
= \Vert p_{A\perp}(x) \Vert^2 + \Vert p_A(x) -a\Vert^2 \geq \Vert p_{A\perp}(x) \Vert^2 .
\end{displaymath}
Donc $\Vert p_{A\perp}(x) \Vert$ est un minorant de $\left\lbrace \Vert x-a\Vert, a\in A\right\rbrace$ qui est atteint pour $a=p_A(x)$ c'est donc le plus petit élément de cet ensemble.
\end{demo}
\newpage
Rappelons qu'un endomophisme $s$ est une symétrie si et seulement si $s\circ s = \Id_E$. Les sous espaces $A = \ker(s-\Id_E)$ et $B = \ker(s + \Id_E)$ sont alors supplémentaires et 
\[
 s = P_{A,B} - P_{B,A}.
\]
\index{symétrie orthogonale}\index{affinité orthogonale}\index{réflexion}
On dit que $s$ est la symétrie par  rapport à $A$ dans la direction $B$. La symétrie $s$ est dite \emph{orthogonale} si et seulement si $A = \ker(s-\Id_E)$ et $B = \ker(s + \Id_E)$ sont orthogonaux. Une symétrie orthogonale par rapport à un sous-espace $A$ s'écrit donc
\[
 s_A = P_{A} - P_{A^\bot}.
\]
Une réflexion est une symétrie orthogonale par rapport à un hyperplan.\newline
La reflexion par rapport à l'hyperplan orthogonal à un vecteur non nul $a$ est la fonction
\begin{displaymath}
 x\rightarrow x - 2\frac{(a/x)}{\Vert a\Vert^2}a.
\end{displaymath}
Une affinité orthogonale par rapport à un sous-espace $A$ est un endomorphisme de la forme $p_A - \lambda p_{A^\bot}$ avec $\lambda$ réel. Pour $\lambda = -1$, on retrouve une symétrie orthogonale. 
\clearpage
\section{Orthogonalisation de Gram-Schmidt}
\index{orthogonalisation de Gram-Schmidt}
\subsection{Aspect vectoriel}
Dans $E$ préhilbertien avec un produit scalaire $(/)$, on considère une famille libre quelconque $(e_1,\cdots,e_n)$. La famille $(a_1,\cdots,a_n)$ (dite orthogonalisée par la méthode de Schmidt) est définie par :
\begin{displaymath}
 a_1 = e_1 \text{ et }
\forall i \in\{1,\cdots, n-1\} : a_{i+1} = e_{i+1}-\sum_{k=1}^i\frac{(e_{i+1}/a_k)}{\Vert a_k\Vert^2}a_k.
\end{displaymath}
Cette famille vérifie les propriétés suivantes
\begin{displaymath}
 (\mathcal {GS})\hspace{1cm}
\left\lbrace
\begin{aligned}
 &- (a_1,\cdots,a_n) \text{ est orthogonale} \\
 &- \forall i\in\{1,\cdots,n\} : \Vect(a_1,\cdots,a_i)=\Vect(e_1,\cdots,e_i)
\end{aligned}
 \right. .
\end{displaymath}
De plus, si $(c_1,\cdots,c_n)$ est une autre suite vérifiant les propriétés $(\mathcal{GS})$, il existe un unique $n$-uplet $(\lambda_1,\cdots,\lambda_n)$ de scalaires non nuls tels que $b_i=\lambda_i a_i$ pour tous les $i$ entre $1$ et $n$.

\begin{rem}
Cette définition de la famille orthogonalisée est récursive puisque le calcul de $a_{k+1}$ nécessite d'avoir préalablement calculé tous les termes $a_1$ à $a_k$.
\end{rem}
\clearpage
\begin{proof}
Montrons que $(a_1, ..., a_n)$ vérifie $(\mathcal{GS})$. Considérons la propriété $\mathcal{P}_{k}$ : $(a_1, ..., a_k)$ est une base orthogonale de $\Vect(e_1, ..., e_k)$.\newline
La propriété $\mathcal{P}_{1}$ est clairement vérifiée. Montrons maintenant que: $\mathcal{P}_{k} \Rightarrow \mathcal{P}_{k+1}$ :\\
D'après $\mathcal{P}_{k} : (a_1, ..., a_k)$ est une base orthonormée de $\Vect(e_1, ..., e_k) = V_k$ donc:
\begin{multline*}
 \sum_{i=1}^{k} \frac{ (e_{k+1}/a_i) }{ ||a_i||^2 } = P_{V_k} (e_{k+1})  
 \Rightarrow 
a_{k+1} = e_{k+1} - p_{V_{k}}(e_{k+1}) = p_{V_{k}^{\perp}}( e_{k+1} ) \in V_{k}^{\perp} \\
\Rightarrow
a_{k+1} \in V_{k}^{\perp} .
\end{multline*}
Le vecteur $a_{k+1}$ est donc orthogonal à tous les vecteurs $a_i$ pour $i$ de $1$ à $k$. La famille $(a_1, ..., a_{k+1})$ est orthogonale.\\
D'autre part, 
\begin{displaymath}
a_{k+1} = e_{k+1} - \underset{\in V_k = \Vect(e_1, ..., e_k)}{\underbrace{p_{V_{k}}(e_{k+1})}} \in V_{k+1} .
\end{displaymath}
La famille $(a_1, ..., a_{k+1})$ est donc une famille orthogonale de $\Vect(e_1, ..., e_{k+1})$. De plus $a_{k+1} \neq 0$, sinon $e_{k+1} \in V_{k}$, ce qui est en contradiction avec le fait que $(e_1, ..., e_n)$ est libre. On en déduit que $\mathcal{P}_{k+1}$ est vraie.

Montrons que $(b_1, ..., b_n)$ vérifie $(\mathcal{GS})$ implique qu'il existe $\lambda_1,\cdots, \lambda_n$ tels que $b_i = \lambda_i a_i$. $V_i = \Vect(e_1, ..., e_i) = \Vect(a_1, ..., a_i)$.\\
Le point important est que
\begin{displaymath}
\dim \left( V_{i+1} \cap V_{i}^{\perp}\right) = 1 .
\end{displaymath}
En effet, comme $V_i \subset V_{i+1}$, on en déduit que $V_{i}$ est un hyperplan de $V_{i+1}$. On restreint alors le produit scalaire à $V_{i+1}$. L'orthogonal de $V_i$ comme partie de $V_{i+1}$ est de dimension 1. Cet orthogonal est constitué des $x \in V_{i+1}$ tels que: $x \perp V_i$, c'est-à-dire $V_{i+1} \cap V_{i}^{\perp}$.
On a donc montré que: $(b_1, ..., b_n)$ vérifie $(\mathcal{GS})$ est équivalent à:
\begin{displaymath}
\forall i \in \rrbracket 2, n \rrbracket, b_i \in V_{i} \cap V_{i-1}^{\perp} \text{ et } b_1 \in V_1.
\end{displaymath}
\end{proof}
\begin{rem}
 La formule $a_{k+1} = e_{k+1} - p_{V_{k}}(e_{k+1})$ s'écrit encore $a_{k+1} = p_{V_{k}\perp}(e_{k+1})$. Le fait que les vecteurs de la famille orthogonalisée soient chacun des projections orthogonales des vecteurs de la famille initiale entraîne que $\Vert a_k\Vert \leq \Vert e_k\Vert$.
\end{rem}

\clearpage
\begin{exple}\index{polynômes orthogonaux}
\textbf{Polynômes orthogonaux}
On se place dans $\mathbb{R}[X]$ et on considère le produit scalaire:

\begin{center}
$(P/Q) = \displaystyle \int_{0}^{1} P(t)Q(t) \, dt$.
\end{center}

On veut orthogonaliser $(1, X, ..., X^{n})$ pour former les polynômes de Legendre $(L_0, L_1, ..., L_n)$.

On va ici se limiter à $(1, X, X^2, X^3)$ (les $e_i$) donc à $(L_0, L_1, ... L_n)$ (les $a_i$).
\begin{itemize}
\item $L_0 = 1$.\medskip
\item $L_1 = \displaystyle X - \frac{ (X/1) }{ ||1||^2 } 1 = X - \frac{1}{2}$
\item $L_2 = \displaystyle X^2 - \frac{ (X^2/1) }{ ||1||^2 } 1 - \frac{ (X^2/L_1) }{ ||L_1||^2 } L_1 = X^2 - \frac{1}{3} - \frac{\frac{1}{12}}{\frac{1}{12}} \left( X - \frac{1}{2} \right) = X^2 - X + \frac{1}{6}$.\medskip
\item $L_3 = \displaystyle X^3 - \frac{(X^3/L_0)}{||L_0||^2} L_0 - \frac{ (X^3/L_1) }{ ||L_1||^2 } L_1 - \frac{ (X^3/L_2) }{ ||L_2||^2 } L_2 = X^3 - \frac{1}{4} - \frac{\frac{3}{40}}{\frac{1}{12}} \left( X - \frac{1}{2} \right) - \frac{\frac{1}{120}}{\frac{1}{180}} \left( X^2 - X + \frac{1}{6} \right)$ \\
$L_3 = X^3 - \frac{1}{4} - \frac{9}{10} \left( X - \frac{1}{2} \right) - \frac{3}{2} \left( X^2 - X + \frac{1}{6} \right)$ \medskip\\
$L_3 = X^3 - \frac{3}{2} X^2 + \frac{3}{5} X - \frac{1}{20}$.\medskip
\end{itemize}
\begin{rem}
 Une propriété des polynômes orthogonaux. 
Soit $L_n$ de degré $n$. 
\[
\left(  \forall k \in \{0, ..., n-1\} : \displaystyle \int_{0}^{1} L_{n}(t) t^{k} \, dt = 0 \right) \Rightarrow
L_n \text{ admet $n$ racines dans $]0,1[$.}
\]
En effet, $a_1, ..., a_s$ sont les racines de multiplicité impaire de $L_n$, alors $s < r$ d'où 
\[
\left( (x-a_1) ... (x-a_n) / L_n \right) = \displaystyle \underset{\textrm{ne change pas de signe}}{\underbrace{\int_{0}^{1} L_n(t-a_1)...(t-a_s) \, dt}} \neq 0.
\]
\end{rem}

\end{exple}

\clearpage
\subsection{Aspect matriciel}
Dans un espace euclidien $E$ avec un produit scalaire $(/)$, on considère une base quelconque $\mathcal E = (e_1,\cdots,e_n)$ et une famille orthogonalisée $\mathcal A =(a_1,\cdots,a_n)$. Les matrices de passage sont triangulaires supérieures avec des $1$ sur la diagonale.\newline
En particulier, $S$ est la matrice d'un produit scalaire si et seulement si il existe une matrice triangulaire supérieure $T$ avec des $1$ sur la diagonale telle que  $\trans T\,S\, T$ soit diagonale.
\clearpage
\section{Hyperplans affines}
Vecteur normal à un hyperplan affine.

Ligne de niveau de $M\mapsto (\overrightarrow{AM}/\overrightarrow{n})$.

Si l'espace est orienté, orientation d'un hyperplan autour d'un vecteur normal.

\'Equations d'un hyperplan affine. Cas particulier de $\R^2$ et $\R^3$.

Distance à un hyperplan affine défini par un point $A$ et un vecteur normal unitaire $\overrightarrow{n}$:
\begin{displaymath}
  \left| (\overrightarrow{AM}/\overrightarrow{n})\right|
\end{displaymath}


\end{document}
