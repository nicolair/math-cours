%!  pour pdfLatex
\documentclass[a4paper]{article}
\usepackage[hmargin={1.5cm,1.5cm},vmargin={2.4cm,2.4cm},headheight=13.1pt]{geometry}

\usepackage[pdftex]{graphicx,color}
%\usepackage{hyperref}

\usepackage[utf8]{inputenc}
\usepackage[T1]{fontenc}
\usepackage{lmodern}
%\usepackage[frenchb]{babel}
\usepackage[french]{babel}

\usepackage{fancyhdr}
\pagestyle{fancy}

%\usepackage{floatflt}

\usepackage{parcolumns}
\setlength{\parindent}{0pt}
\usepackage{xcolor}

%pr{\'e}sentation des compteurs de section, ...
\makeatletter
%\renewcommand{\labelenumii}{\theenumii.}
\renewcommand{\thepart}{}
\renewcommand{\thesection}{}
\renewcommand{\thesubsection}{}
\renewcommand{\thesubsubsection}{}
\makeatother

\newcommand{\subsubsubsection}[1]{\bigskip \rule[5pt]{\linewidth}{2pt} \textbf{ \color{red}{#1} } \newline \rule{\linewidth}{.1pt}}
\newlength{\parcoldist}
\setlength{\parcoldist}{1cm}

\usepackage{maths}
\newcommand{\dbf}{\leftrightarrows}
% remplace les commandes suivantes 
%\usepackage{amsmath}
%\usepackage{amssymb}
%\usepackage{amsthm}
%\usepackage{stmaryrd}

%\newcommand{\N}{\mathbb{N}}
%\newcommand{\Z}{\mathbb{Z}}
%\newcommand{\C}{\mathbb{C}}
%\newcommand{\R}{\mathbb{R}}
%\newcommand{\K}{\mathbf{K}}
%\newcommand{\Q}{\mathbb{Q}}
%\newcommand{\F}{\mathbf{F}}
%\newcommand{\U}{\mathbb{U}}

%\newcommand{\card}{\mathop{\mathrm{Card}}}
%\newcommand{\Id}{\mathop{\mathrm{Id}}}
%\newcommand{\Ker}{\mathop{\mathrm{Ker}}}
%\newcommand{\Vect}{\mathop{\mathrm{Vect}}}
%\newcommand{\cotg}{\mathop{\mathrm{cotan}}}
%\newcommand{\sh}{\mathop{\mathrm{sh}}}
%\newcommand{\ch}{\mathop{\mathrm{ch}}}
%\newcommand{\argsh}{\mathop{\mathrm{argsh}}}
%\newcommand{\argch}{\mathop{\mathrm{argch}}}
%\newcommand{\tr}{\mathop{\mathrm{tr}}}
%\newcommand{\rg}{\mathop{\mathrm{rg}}}
%\newcommand{\rang}{\mathop{\mathrm{rg}}}
%\newcommand{\Mat}{\mathop{\mathrm{Mat}}}
%\renewcommand{\Re}{\mathop{\mathrm{Re}}}
%\renewcommand{\Im}{\mathop{\mathrm{Im}}}
%\renewcommand{\th}{\mathop{\mathrm{th}}}


%En tete et pied de page
\lhead{Programme colle math}
\chead{Semaine 13 du 06/01/20 au 11/01/20}
\rhead{MPSI B Hoche}

\lfoot{\tiny{Cette création est mise à disposition selon le Contrat\\ Paternité-Partage des Conditions Initiales à l'Identique 2.0 France\\ disponible en ligne http://creativecommons.org/licenses/by-sa/2.0/fr/
} }
\rfoot{\tiny{Rémy Nicolai \jobname}}


\begin{document}
\subsection{Exercices sur le calcul local.}
Développements limités et asymptotiques.

\subsection{Structures algébriques usuelles}
\begin{itshape}Le programme, strictement limité au vocabulaire décrit ci-dessous, a pour objectif de permettre une présentation unifiée
des exemples usuels. En particulier, l'étude de lois artificielles est exclue.

La notion de sous-groupe figure dans ce chapitre par commodité. Le professeur a la liberté de l'introduire plus tard.
\end{itshape}

\subsubsubsection{a) Lois de composition internes}
\begin{parcolumns}[rulebetween,distance=\parcoldist]{2}
  \colchunk{Loi de composition interne. }
  \colchunk{}
  \colplacechunks

  \colchunk{Associativité, commutativité, élément neutre, inversibilité, distributivité.}
  \colchunk{ Inversibilité et inverse du produit de deux éléments inversibles.}
  \colplacechunks

  \colchunk{Partie stable.}
  \colchunk{}
  \colplacechunks

 \end{parcolumns}

\subsubsubsection{b) Structure de groupe}
\begin{parcolumns}[rulebetween,distance=\parcoldist]{2}

  \colchunk{Groupe.}
  \colchunk{Notation $x^n$ dans un groupe multiplicatif, $nx$ dans un groupe additif.

  Exemples usuels : groupes additifs $\Z$, $\Q$, $\R$, $\C$, groupes multiplicatifs $\Q^*$, $\Q_+^*$, $\R^*$, $\R_+^*$, $\C^*$, $\U$, $\U_n$.}
  \colplacechunks

  \colchunk{Groupe des permutations d'un ensemble.}
  \colchunk{Notation $\mathfrak{S}_X$.}
  \colplacechunks

  \colchunk{Sous-groupe : définition, caractérisation.}
  \colchunk{}
  \colplacechunks
 \end{parcolumns}

\subsubsubsection{c) Structures d'anneau et de corps}
\begin{parcolumns}[rulebetween,distance=\parcoldist]{2}
  \colchunk{Anneau, corps. }
  \colchunk{Tout anneau est unitaire, tout corps est commutatif.

  Exemples usuels : $\Z$, $\Q$, $\R$, $\C$.}
  \colplacechunks

  \colchunk{Calcul dans un anneau.}
  \colchunk{Relation $a^n-b^n$ et formule du binôme si $a$ et $b$ commutent.}
  \colplacechunks

  \colchunk{Groupe des inversibles d'un anneau.}
  \colchunk{}
  \colplacechunks
 \end{parcolumns}




\subsection{Polynômes et fractions rationnelles (1)}
\begin{itshape}L'objectif de ce chapitre est d'étudier les propriétés de base de ces objets formels et de les exploiter pour la résolution de problèmes portant sur les équations algébriques et les fonctions numériques. Le programme se limite au cas où le corps de base $\K$ est $\R$ ou $\C$.
\end{itshape}

\subsubsubsection{a) Anneau des polynômes à une indéterminée}
\begin{parcolumns}[rulebetween,distance=\parcoldist]{2}
  \colchunk{Anneau $\K [X]$.}
  \colchunk{La contruction de $\K [X]$ n'est pas exigible.

  Notations $\displaystyle \sum_{i=0}^d a_i X^i, \sum_{i=0}^{+ \infty} a_i X^i$.}
  \colplacechunks

  \colchunk{Degré, coefficient dominant, polynôme unitaire.}
  \colchunk{Le degré du polynôme nul est $- \infty$.

  Ensemble $\K_n [X]$ des polynômes de degré au plus $n$.}
  \colplacechunks

  \colchunk{Degré d'une somme, d'un produit.}
  \colchunk{Le produit de deux polynômes non nuls est non nul.}
  \colplacechunks

  \colchunk{Composition.}
  \colchunk{$\dbf$ I : représentation informatique d'un polynôme ; somme, produit.}
  \colplacechunks
\end{parcolumns}


\bigskip
\begin{center}
 \textbf{Questions de cours}
 \end{center}
\textbf{Algèbre générale} :
 Unicité de l'élément neutre dans un groupe.\newline
\textbf{Polynômes}:\newline
Pratique de la division euclidienne. Racine et divisibilité. 


\begin{center}
 \textbf{Prochain programme}
\end{center}

Arithmétique dans $\Z$. Polynômes (2).
\end{document}
