\input{courspdf.tex}
\debutcours{Périodicité du développement dans une base}{alpha}

En particulier on s'attache à expliquer pourquoi l'écriture décimale d'un rationnel est périodique à partir d'un certain rang ?
Nouvelle démonstration. \`A vérifier.
On considère un nombre rationnel $x=\frac{p}{q}$ avec $q>1$. D'après un \href{http://localhost/v-1/index.php?act=chelt&id_elt=5639}{exercice}, la suite des restes modulo $q$ de $p10^n$ est périodique à partir d'un certain rang. Il existe donc des suites $\left(a_n \right)_{n\in\N}$ et $\left(r_n\right)_{n\in\N}$ périodiques à partir d'un certain rang et vérifiant
\begin{displaymath}
 p10^n = a_nq + r_n \text{ avec } r_n\in\{0,\cdots,q-1\}
\end{displaymath}
car la périodicité des restes entraîne celle des quotients. On en déduit :
\begin{displaymath}
 \dfrac{p}{q} = a_n10^{-n} + \dfrac{r_n}{q}10^{-n} \text{ avec } 0\leq \dfrac{r_n}{q}<1
\end{displaymath}
Ceci signifie que $a_n10^{-n}=m_n(x)$ est l'approximation à l'ordre $n$ par défaut de $x$. Or la décimale d'ordre $n$ de $x$ est le reste de la division euclidienne par $10$ de $10^nm-n(x)$ qui est égal à $a_n$. La périodicité de la suite des $a_n$ entraine celle de ce reste donc celle de la suite des décimales.


Ancienne démonstration.
Ce résultat est valable pour n'importe quelle base $d$ de numération. On pose 
\begin{displaymath}
 d =10
\end{displaymath}
pour des raisons psychologiques mais on peut le remplacer par n'inporte quel entier supérieur ou égal à 2. On considère la suite des
\begin{displaymath}
 x_n = d^n -1
\end{displaymath}
L'écriture décimale de $x_n$ ne contient que des $9$ (en fait $n$ exactement). Cette suite peut aussi se définir par récurrence par $x_0=0$ et $x_{n+1}=d x_n + d-1$. C'est à dire $x_{n+1}=10 x_n + 9$ dans le cas de la base $10$.\newline
Considérons maintenant un rationnel $\frac{p}{q}$ quelconque. \newline
Le point important est que la suite des $d^n$ ne prend q'un nombre fini de valeurs  modulo $q$.\newline
Il existe donc des entiers $n<m$ tels que
\begin{align*}
 d^n -1 &\equiv  d^m -1 &(q) \\
 d^{n}(d^{m-n} -1) &\equiv 0 &(q) 
\end{align*}
Autrement dit $q$ divise un nombre $d^{n}(d^{m-n} -1)$. Dans le cas de la base $10$ le développement décimal d'un tel nombre est de la forme
\begin{displaymath}
 \underbrace{99\cdots 9}_{m-n}\underbrace{00\cdots0}_n
\end{displaymath}
En multipliant en haut et en bas par ce qu'il faut, on obtient
\begin{displaymath}
 \dfrac{p}{q} = \dfrac{x}{d^{n}(d^{m-n} -1)}
\end{displaymath}
On peut écrire explicitement le développement de ce nombre. En effet :
\begin{displaymath}
 \dfrac{1}{d^\alpha -1}= \dfrac{1}{d^\alpha} + \dfrac{1}{d^{2\alpha}} + \cdots
\end{displaymath}
donc 
\begin{displaymath}
 \dfrac{1}{d^{n}(d^{m-n} -1)}= \dfrac{1}{d^m} + \dfrac{1}{d^{m+(m-n)}} + \dfrac{1}{d^{m+2(m-n)}}+ \cdots
\end{displaymath}
En écrivant le développement en base $d$ du numérateur on obtient un développement qui est périodique de période $m-n$ à partir d'un certain rang.

\end{document}
