%<dscrpt>Fichier de déclarations Latex à inclure au début d'un élément de cours.</dscrpt>

\documentclass[a4paper]{article}
\usepackage[hmargin={1.8cm,1.8cm},vmargin={2.4cm,2.4cm},headheight=13.1pt]{geometry}

%includeheadfoot,scale=1.1,centering,hoffset=-0.5cm,
\usepackage[pdftex]{graphicx,color}
\usepackage[french]{babel}
%\selectlanguage{french}
\addto\captionsfrench{
  \def\contentsname{Plan}
}
\usepackage{fancyhdr}
\usepackage{floatflt}
\usepackage{amsmath}
\usepackage{amssymb}
\usepackage{amsthm}
\usepackage{stmaryrd}
%\usepackage{ucs}
\usepackage[utf8]{inputenc}
%\usepackage[latin1]{inputenc}
\usepackage[T1]{fontenc}


\usepackage{titletoc}
%\contentsmargin{2.55em}
\dottedcontents{section}[2.5em]{}{1.8em}{1pc}
\dottedcontents{subsection}[3.5em]{}{1.2em}{1pc}
\dottedcontents{subsubsection}[5em]{}{1em}{1pc}

\usepackage[pdftex,colorlinks={true},urlcolor={blue},pdfauthor={remy Nicolai},bookmarks={true}]{hyperref}
\usepackage{makeidx}

\usepackage{multicol}
\usepackage{multirow}
\usepackage{wrapfig}
\usepackage{array}
\usepackage{subfig}


%\usepackage{tikz}
%\usetikzlibrary{calc, shapes, backgrounds}
%pour la présentation du pseudo-code
% !!!!!!!!!!!!!!      le package n'est pas présent sur le serveur sous fedora 16 !!!!!!!!!!!!!!!!!!!!!!!!
%\usepackage[french,ruled,vlined]{algorithm2e}

%pr{\'e}sentation du compteur de niveau 2 dans les listes
\makeatletter
\renewcommand{\labelenumii}{\theenumii.}
\renewcommand{\thesection}{\Roman{section}.}
\renewcommand{\thesubsection}{\arabic{subsection}.}
\renewcommand{\thesubsubsection}{\arabic{subsubsection}.}
\makeatother


%dimension des pages, en-t{\^e}te et bas de page
%\pdfpagewidth=20cm
%\pdfpageheight=14cm
%   \setlength{\oddsidemargin}{-2cm}
%   \setlength{\voffset}{-1.5cm}
%   \setlength{\textheight}{12cm}
%   \setlength{\textwidth}{25.2cm}
   \columnsep=1cm
   \columnseprule=0.5pt

%En tete et pied de page
\pagestyle{fancy}
\lhead{MPSI-\'Eléments de cours}
\rhead{\today}
%\rhead{25/11/05}
\lfoot{\tiny{Cette création est mise à disposition selon le Contrat\\ Paternité-Pas d'utilisations commerciale-Partage des Conditions Initiales à l'Identique 2.0 France\\ disponible en ligne http://creativecommons.org/licenses/by-nc-sa/2.0/fr/
} }
\rfoot{\tiny{Rémy Nicolai \jobname}}


\newcommand{\baseurl}{http://back.maquisdoc.net/data/cours\_nicolair/}
\newcommand{\urlexo}{http://back.maquisdoc.net/data/exos_nicolair/}
\newcommand{\urlcours}{https://maquisdoc-math.fra1.digitaloceanspaces.com/}

\newcommand{\N}{\mathbb{N}}
\newcommand{\Z}{\mathbb{Z}}
\newcommand{\C}{\mathbb{C}}
\newcommand{\R}{\mathbb{R}}
\newcommand{\D}{\mathbb{D}}
\newcommand{\K}{\mathbf{K}}
\newcommand{\Q}{\mathbb{Q}}
\newcommand{\F}{\mathbf{F}}
\newcommand{\U}{\mathbb{U}}
\newcommand{\p}{\mathbb{P}}


\newcommand{\card}{\mathop{\mathrm{Card}}}
\newcommand{\Id}{\mathop{\mathrm{Id}}}
\newcommand{\Ker}{\mathop{\mathrm{Ker}}}
\newcommand{\Vect}{\mathop{\mathrm{Vect}}}
\newcommand{\cotg}{\mathop{\mathrm{cotan}}}
\newcommand{\sh}{\mathop{\mathrm{sh}}}
\newcommand{\ch}{\mathop{\mathrm{ch}}}
\newcommand{\argsh}{\mathop{\mathrm{argsh}}}
\newcommand{\argch}{\mathop{\mathrm{argch}}}
\newcommand{\tr}{\mathop{\mathrm{tr}}}
\newcommand{\rg}{\mathop{\mathrm{rg}}}
\newcommand{\rang}{\mathop{\mathrm{rg}}}
\newcommand{\Mat}{\mathop{\mathrm{Mat}}}
\newcommand{\MatB}[2]{\mathop{\mathrm{Mat}}_{\mathcal{#1}}\left( #2\right) }
\newcommand{\MatBB}[3]{\mathop{\mathrm{Mat}}_{\mathcal{#1} \mathcal{#2}}\left( #3\right) }
\renewcommand{\Re}{\mathop{\mathrm{Re}}}
\renewcommand{\Im}{\mathop{\mathrm{Im}}}
\renewcommand{\th}{\mathop{\mathrm{th}}}
\newcommand{\repere}{$(O,\overrightarrow{i},\overrightarrow{j},\overrightarrow{k})$}
\newcommand{\cov}{\mathop{\mathrm{Cov}}}

\newcommand{\absolue}[1]{\left| #1 \right|}
\newcommand{\fonc}[5]{#1 : \begin{cases}#2 \rightarrow #3 \\ #4 \mapsto #5 \end{cases}}
\newcommand{\depar}[2]{\dfrac{\partial #1}{\partial #2}}
\newcommand{\norme}[1]{\left\| #1 \right\|}
\newcommand{\se}{\geq}
\newcommand{\ie}{\leq}
\newcommand{\trans}{\mathstrut^t\!}
\newcommand{\val}{\mathop{\mathrm{val}}}
\newcommand{\grad}{\mathop{\overrightarrow{\mathrm{grad}}}}

\newtheorem*{thm}{Théorème}
\newtheorem{thmn}{Théorème}
\newtheorem*{prop}{Proposition}
\newtheorem{propn}{Proposition}
\newtheorem*{pa}{Présentation axiomatique}
\newtheorem*{propdef}{Proposition - Définition}
\newtheorem*{lem}{Lemme}
\newtheorem{lemn}{Lemme}

\theoremstyle{definition}
\newtheorem*{defi}{Définition}
\newtheorem*{nota}{Notation}
\newtheorem*{exple}{Exemple}
\newtheorem*{exples}{Exemples}


\newenvironment{demo}{\renewcommand{\proofname}{Preuve}\begin{proof}}{\end{proof}}
%\renewcommand{\proofname}{Preuve} doit etre après le begin{document} pour fonctionner

\theoremstyle{remark}
\newtheorem*{rem}{Remarque}
\newtheorem*{rems}{Remarques}

\renewcommand{\indexspace}{}
\renewenvironment{theindex}
  {\section*{Index} %\addcontentsline{toc}{section}{\protect\numberline{0.}{Index}}
   \begin{multicols}{2}
    \begin{itemize}}
  {\end{itemize} \end{multicols}}


%pour annuler les commandes beamer
\renewenvironment{frame}{}{}
\newcommand{\frametitle}[1]{}
\newcommand{\framesubtitle}[1]{}

\newcommand{\debutcours}[2]{
  \chead{#1}
  \begin{center}
     \begin{huge}\textbf{#1}\end{huge}
     \begin{Large}\begin{center}Rédaction incomplète. Version #2\end{center}\end{Large}
  \end{center}
  %\section*{Plan et Index}
  %\begin{frame}  commande beamer
  \tableofcontents
  %\end{frame}   commande beamer
  \printindex
}


\makeindex
\begin{document}
\noindent

\debutcours{\'Ecriture dans une base}{alpha}

En particulier on s'attache à expliquer pourquoi l'écriture décimale d'un rationnel est périodique à partir d'un certain rang ?
Nouvelle démonstration. \`A vérifier.
On considère un nombre rationnel $x=\frac{p}{q}$ avec $q>1$. D'après un \href{http://localhost/v-1/index.php?act=chelt&id_elt=5639}{exercice}, la suite des restes modulo $q$ de $p10^n$ est périodique à partir d'un certain rang. Il existe donc des suites $\left(a_n \right)_{n\in\N}$ et $\left(r_n\right)_{n\in\N}$ périodiques à partir d'un certain rang et vérifiant
\begin{displaymath}
 p10^n = a_nq + r_n \text{ avec } r_n\in\{0,\cdots,q-1\}
\end{displaymath}
car la périodicité des restes entraîne celle des quotients. On en déduit :
\begin{displaymath}
 \dfrac{p}{q} = a_n10^{-n} + \dfrac{r_n}{q}10^{-n} \text{ avec } 0\leq \dfrac{r_n}{q}<1
\end{displaymath}
Ceci signifie que $a_n10^{-n}=m_n(x)$ est l'approximation à l'ordre $n$ par défaut de $x$. Or la décimale d'ordre $n$ de $x$ est le reste de la division euclidienne par $10$ de $10^nm-n(x)$ qui est égal à $a_n$. La périodicité de la suite des $a_n$ entraine celle de ce reste donc celle de la suite des décimales.


Ancienne démonstration.
Ce résultat est valable pour n'importe quelle base $d$ de numération. On pose 
\begin{displaymath}
 d =10
\end{displaymath}
pour des raisons psychologiques mais on peut le remplacer par n'inporte quel entier supérieur ou égal à 2. On considère la suite des
\begin{displaymath}
 x_n = d^n -1
\end{displaymath}
L'écriture décimale de $x_n$ ne contient que des $9$ (en fait $n$ exactement). Cette suite peut aussi se définir par récurrence par $x_0=0$ et $x_{n+1}=d x_n + d-1$. C'est à dire $x_{n+1}=10 x_n + 9$ dans le cas de la base $10$.\newline
Considérons maintenant un rationnel $\frac{p}{q}$ quelconque. \newline
Le point important est que la suite des $d^n$ ne prend q'un nombre fini de valeurs  modulo $q$.\newline
Il existe donc des entiers $n<m$ tels que
\begin{align*}
 d^n -1 &\equiv  d^m -1 &(q) \\
 d^{n}(d^{m-n} -1) &\equiv 0 &(q) 
\end{align*}
Autrement dit $q$ divise un nombre $d^{n}(d^{m-n} -1)$. Dans le cas de la base $10$ le développement décimal d'un tel nombre est de la forme
\begin{displaymath}
 \underbrace{99\cdots 9}_{m-n}\underbrace{00\cdots0}_n
\end{displaymath}
En multipliant en haut et en bas par ce qu'il faut, on obtient
\begin{displaymath}
 \dfrac{p}{q} = \dfrac{x}{d^{n}(d^{m-n} -1)}
\end{displaymath}
On peut écrire explicitement le développement de ce nombre. En effet :
\begin{displaymath}
 \dfrac{1}{d^\alpha -1}= \dfrac{1}{d^\alpha} + \dfrac{1}{d^{2\alpha}} + \cdots
\end{displaymath}
donc 
\begin{displaymath}
 \dfrac{1}{d^{n}(d^{m-n} -1)}= \dfrac{1}{d^m} + \dfrac{1}{d^{m+(m-n)}} + \dfrac{1}{d^{m+2(m-n)}}+ \cdots
\end{displaymath}
En écrivant le développement en base $d$ du numérateur on obtient un développement qui est périodique de période $m-n$ à partir d'un certain rang.

\end{document}