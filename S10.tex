%!  pour pdfLatex
\documentclass[a4paper]{article}
\usepackage[hmargin={1.5cm,1.5cm},vmargin={2.4cm,2.4cm},headheight=13.1pt]{geometry}

\usepackage[pdftex]{graphicx,color}
%\usepackage{hyperref}

\usepackage[utf8]{inputenc}
\usepackage[T1]{fontenc}
\usepackage{lmodern}
%\usepackage[frenchb]{babel}
\usepackage[french]{babel}

\usepackage{fancyhdr}
\pagestyle{fancy}

%\usepackage{floatflt}

\usepackage{parcolumns}
\setlength{\parindent}{0pt}
\usepackage{xcolor}

%pr{\'e}sentation des compteurs de section, ...
\makeatletter
%\renewcommand{\labelenumii}{\theenumii.}
\renewcommand{\thepart}{}
\renewcommand{\thesection}{}
\renewcommand{\thesubsection}{}
\renewcommand{\thesubsubsection}{}
\makeatother

\newcommand{\subsubsubsection}[1]{\bigskip \rule[5pt]{\linewidth}{2pt} \textbf{ \color{red}{#1} } \newline \rule{\linewidth}{.1pt}}
\newlength{\parcoldist}
\setlength{\parcoldist}{1cm}

\usepackage{maths}
\newcommand{\dbf}{\leftrightarrows}
% remplace les commandes suivantes 
%\usepackage{amsmath}
%\usepackage{amssymb}
%\usepackage{amsthm}
%\usepackage{stmaryrd}

%\newcommand{\N}{\mathbb{N}}
%\newcommand{\Z}{\mathbb{Z}}
%\newcommand{\C}{\mathbb{C}}
%\newcommand{\R}{\mathbb{R}}
%\newcommand{\K}{\mathbf{K}}
%\newcommand{\Q}{\mathbb{Q}}
%\newcommand{\F}{\mathbf{F}}
%\newcommand{\U}{\mathbb{U}}

%\newcommand{\card}{\mathop{\mathrm{Card}}}
%\newcommand{\Id}{\mathop{\mathrm{Id}}}
%\newcommand{\Ker}{\mathop{\mathrm{Ker}}}
%\newcommand{\Vect}{\mathop{\mathrm{Vect}}}
%\newcommand{\cotg}{\mathop{\mathrm{cotan}}}
%\newcommand{\sh}{\mathop{\mathrm{sh}}}
%\newcommand{\ch}{\mathop{\mathrm{ch}}}
%\newcommand{\argsh}{\mathop{\mathrm{argsh}}}
%\newcommand{\argch}{\mathop{\mathrm{argch}}}
%\newcommand{\tr}{\mathop{\mathrm{tr}}}
%\newcommand{\rg}{\mathop{\mathrm{rg}}}
%\newcommand{\rang}{\mathop{\mathrm{rg}}}
%\newcommand{\Mat}{\mathop{\mathrm{Mat}}}
%\renewcommand{\Re}{\mathop{\mathrm{Re}}}
%\renewcommand{\Im}{\mathop{\mathrm{Im}}}
%\renewcommand{\th}{\mathop{\mathrm{th}}}


%En tete et pied de page
\lhead{Programme colle math}
\chead{Semaine 10 du 02/12/19 au 07/12/19}
\rhead{MPSI B Hoche}

\lfoot{\tiny{Cette création est mise à disposition selon le Contrat\\ Paternité-Partage des Conditions Initiales à l'Identique 2.0 France\\ disponible en ligne http://creativecommons.org/licenses/by-sa/2.0/fr/
} }
\rfoot{\tiny{Rémy Nicolai \jobname}}


\begin{document}

\subsection{Limites, continuité, dérivabilité}
\subsubsection{C - Dérivabilité}

\subsubsubsection{a) Nombre dérivé, fonction dérivée}
\begin{parcolumns}[rulebetween,distance=\parcoldist]{2}
  \colchunk{Dérivabilité en un point, nombre dérivé.}
  \colchunk{Développement limité à l'orde 1.\newline
  Interprétation géométrique. $\leftrightarrows$ SI: identification d'un modèle de comportement au voisinage d'un point de comportement.\newline
  $\leftrightarrows$ SI: représentation de la fonction sinus cardinal au voisinage de $0$.
  $\leftrightarrows$ I: méthode de Newton.}
  \colplacechunks
  
  \colchunk{La dérivabilité entraîne la continuité.}
  \colchunk{}
  \colplacechunks

  \colchunk{Dérivabilité à gauche, à droite.}
  \colchunk{}
  \colplacechunks
  
  \colchunk{Dérivabilité et dérivée sur un intervalle.}
  \colchunk{}
  \colplacechunks

  \colchunk{Opérations sur les fonctions dérivables et les dérivées: combinaison linéaire, produit, quotient, composition, réciproque.}
  \colchunk{Tangente au graphe d'une réciproque.}
  \colplacechunks
\end{parcolumns}

\subsubsubsection{b) Extremum local et point critique}
\begin{parcolumns}[rulebetween,distance=\parcoldist]{2}
  \colchunk{Extremum local}
  \colchunk{}
  \colplacechunks

  \colchunk{Condition nécessaire en un point intérieur.}
  \colchunk{Un point critique est un zéro de la dérivée.}
  \colplacechunks
\end{parcolumns}

\subsubsubsection{c) Théorème de Rolle et des accroissements finis.}
\begin{parcolumns}[rulebetween,distance=\parcoldist]{2}
  \colchunk{Théorème de Rolle.}
  \colchunk{Utilisation pour établir l'existence de zéros d'une fonction.}
  \colplacechunks

  \colchunk{\'Egalité des accroissements finis.}
  \colchunk{Interprétations géométriques et cinématiques.}
  \colplacechunks

  \colchunk{Inégalité des accroissements finis: si $f$ est dérivable et si $|f'|$ est majorée par $K$, alors $f$ est $K$-lipschitzienne.}
  \colchunk{La notion de fonction lipschitzienne est introduite à cette occasion.\newline
  Application à l'étude des des suites définies par une relation de récurrence $u_{n+1}=f(u_n)$.}
  \colplacechunks

  \colchunk{Caractérisation des fonctions dérivables constantes, monotones, strictement monotones sur un intervalle.}
  \colchunk{}
  \colplacechunks

  \colchunk{Théorème de la limite de la dérivée: si $f$ est continue sur $I$, dérivable sur $I\setminus\{a\}$ et si $\underset{\underset{x\neq a}{x\rightarrow a}}{\lim}f'(x)=l\in\overline{\R}$, alors $\underset{x\rightarrow a}{\lim}\frac{f(x)-f(a)}{x-a}=l$.}
  \colchunk{Interprétation géométrique.\newline
  Si $l\in\R$, alors $f$ est dérivable en $a$ et $f'$ continue en $a$.}
  \colplacechunks

\end{parcolumns}

\subsubsubsection{d) Fonctions de classe $\mathcal{C}^k$}
\begin{parcolumns}[rulebetween,distance=\parcoldist]{2}
  
  \colchunk{Pour $k\in\N \cup \{\infty\}$, fonction de classe $\mathcal{C}^k$.}
  \colchunk{}
  \colplacechunks

  \colchunk{Opérations sur les fonctions de classe $\mathcal{C}^k$: combinaison linéaire, produit (formule de Leibniz), quotient, composition, réciproque.}
  \colchunk{}
  \colplacechunks

  \colchunk{Théorème de classe $\mathcal{C}^k$ par prolongement: si $f$ est de classe $\mathcal{C}^k$ sur $I\setminus\{a\}$ et si $f^{(i)}$ possède une limite finie lorsque $x$ tend vers $a$ pour tout $i\in\{0,\cdots,k\}$, alors $f$ admet un prolongement de classe $\mathcal{C}^k$ sur $I$. }
  \colchunk{}
  \colplacechunks
\end{parcolumns}

\subsubsubsection{e) Fonctions complexes}
\begin{parcolumns}[rulebetween,distance=\parcoldist]{2}
  
  \colchunk{Brève extension des définitions et résultats précédents.}
  \colchunk{Caractérisation de la dérivabilité en termes de parties réelle et imaginaire.}
  \colplacechunks

  \colchunk{Inégalité des accroissements finis pour une fonction de classe $\mathcal{C}^1$.}
  \colchunk{Le résultat, admis à ce stade sera justifé dans le chapitre \og Intégration\fg.}
  \colplacechunks
\end{parcolumns}


\bigskip
\begin{center}
 \textbf{Questions de cours}
\end{center}
Ce chapitre sert aussi d'introduction aux développements locaux qui sont systématiquement préférés aux taux dans les démonstrations.

\textbf{Démonstrations}

Dérivabilité de $(x\mapsto x)$, $(x\mapsto x^n)$ (pour $n\in \N$), $(x\mapsto \frac{1}{x})$ à l'aide de développements.\newline
Dérivation : produit, composée.\newline
Fonction dérivable en un extrémum local. Théorème de Rolle.\newline
\'Egalité et inégalité des accroissements finis. Théorème du tableau de variations. Théorème de la limite de la dérivée.\newline
Exemple de fonction dérivable dont la dérivée est non continue.\newline
Formule de Leibniz.

\begin{center}
 \textbf{Prochain programme}
\end{center}
Développements.
\end{document}
