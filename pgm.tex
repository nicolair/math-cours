%! original siTeXMac(project): pdftex

\input fr

\def\date{}

%\input macros


\catcode`\@=11

\font\goth=eufm10
\font\ineg=msam8
\font\star=msam10
\font\vid=msbm10
\font\bsl=cmbxsl10 at 10pt % gras-pench{\'e}
\font\large=cmr10 at 12pt
\font\Large=cmr10 at 14pt
\font\largeb=cmbx10 at 12pt
\font\Largeb=cmbx10 at 14pt
\font\pcar=cmr8 at 8pt % pour {\'e}crire les si\`ecles
\font\tenbb=cmssbx10 at 10pt % police provisoire pour R,N,Q,Z
\font\sevenbb=cmbx10 at 7pt
\font\fivebb=cmbx10 at 5pt

\everymath{\displaystyle}
\newfam\bbfam
\textfont\bbfam=\tenbb
\scriptfont\bbfam=\sevenbb
\scriptscriptfont\bbfam=\fivebb
\def\bb{\fam\bbfam\tenbb}

\catcode`\;=\active
\def;{\relax\ifhmode\ifdim\lastskip>\z@
\unskip\fi\kern.2em\fi\string;}

\catcode`\:=\active
\def:{\relax\ifhmode\ifdim\lastskip>\z@\unskip\fi
\penalty\@M\ \fi\string:}

\catcode`\!=\active
\def!{\relax\ifhmode\ifdim\lastskip>\z@
\unskip\fi\kern.2em\fi\string!}

\catcode`\?=\active
\def?{\relax\ifhmode\ifdim\lastskip>\z@
\unskip\fi\kern.2em\fi\string?}

%\def\^#1{\if#1i{\accent"5E\i}\else{\accent"5E #1}\fi}
%\def\"#1{\if#1i{\accent"7F\i}\else{\accent"7F #1}\fi}

\newif\ifpagetitre \pagetitretrue
\newtoks\hautpagetitre
\hautpagetitre={\tenrm\hfil\the\premiertitre\hfil}
\newtoks\baspagetitre \baspagetitre={\hfil}

\newtoks\partiecourante \partiecourante={\hfil}
\newtoks\titrecourant \titrecourant={\hfil}
\newtoks\premiertitre \premiertitre={\hfil}

\newtoks\hautpagegauche \newtoks\hautpagedroite
\hautpagegauche={\tenrm\folio\hfill{\the\partiecourante}}
\hautpagedroite={\tenrm{\the\titrecourant}\hfill\folio}

\newtoks\baspagegauche \baspagegauche={\hfil}
\newtoks\baspagedroite \baspagedroite={\hfil}

% \headline={\ifnum\pageno=1\the\hautpagetitre\else\the\hautpagedroite \fi}

% \footline={\hfil}

\def\nopagenumbers{\def\folio{\hfil}}

\catcode`\@=12

\let\optionkeymacros\null
\let\dis=\displaystyle
\let\scr=\scriptstyle
\let\so=\medskip
\let\eps=\varepsilon % Le "bon" epsilon
\let\vphi=\varphi % Le phi usuel
\let\tend=\rightarrow
\let\Tend=\longrightarrow
\let\ssi=\Longleftrightarrow

\def\op{{\star F}}
\def\frac#1#2{{#1\over#2}}
\def\text#1{\hbox{\rm #1}}
\def\d{\,\hbox{\rm d}\,}
\def\trait{\par\centerline{\hbox{\vrule height .4pt depth 0pt width 12cm}}}
\def\ie{\mathrel{\hbox{\ineg 6}}} % <= fran{\c c}ais
\def\le{\mathrel{\hbox{\ineg 6}}} % <= fran{\c c}ais
\def\leq{\mathrel{\hbox{\ineg 6}}} % <= fran{\c c}ais
\def\se{\mathrel{\hbox{\ineg >}}} % >= fran{\c c}ais
\def\ge{\mathrel{\hbox{\ineg >}}} % >= fran{\c c}ais
\def\geq{\mathrel{\hbox{\ineg >}}} % >= fran{\c c}ais
\def\vide{\hbox{\vid~?}}
\def\Z{{\bb Z}}
\def\R{{\bb R}}
\def\C{{\bb C}}
\def\N{{\bb N}}
\def\Q{{\bb Q}}
\def\K{{\bb K}}
\def\U{{\bb U}}
\def\dim{{\rm dim}\,}
\def\sev{{\rm sous-espace vectoriel}}
\def\ker{{\rm Ker}\,}
\def\Ker{{\rm Ker}\,}
\def\re{{\rm Re}\,}
\def\im{{\rm Im}\,}
\def\gav{{\rm GA}(E)}
\def\gle{{\rm GL}(E)}
\def\mnpk{{\cal M}_{n,p}(\K)}
\def\mnk{{\cal M}_n(\K)}
\def\glnk{{\rm GL}_n(\K)}
\def\det{{\rm Det}\,}
\def\card{{\rm Card}}
\def\tr{{\rm Tr}\,}
\def\e{{\rm e}}
\def\ch{\mathop{\rm ch}\nolimits}
\def\sh{\mathop{\rm sh}\nolimits}
\def\th{\mathop{\rm th}\nolimits}
\def\argch{\mathop{\rm Arg\,ch}\nolimits}
\def\argsh{\mathop{\rm Arg\,sh}\nolimits}
\def\argth{\mathop{\rm Arg\,th}\nolimits}
\def\arccos{\mathop{\rm Arc\,cos}\nolimits}
\def\arcsin{\mathop{\rm Arc\,sin}\nolimits}
\def\arctan{\mathop{\rm Arc\,tan}\nolimits}
\def\adh#1{\overline{\!#1}}
\def\rond#1{\buildrel\;\circ\over #1}
\def\cnp#1#2{{\displaystyle\Big({{\textstyle #1}\atop%
{\textstyle #2}}\Big)}}
\def\hfl#1#2{\smash{\mathop{\hbox to 4mm{\rightarrowfill}}
\limits^{#1}_{#2}}}
\def\vect#1{\overrightarrow{#1}} % vecteur
\def\Frac#1#2{{\displaystyle#1\over\displaystyle#2}}
\def\frac#1#2{{\scriptstyle#1\over\scriptstyle#2}}
\def\Der#1#2{\Frac{\hbox{d}#1}{\hbox{d}#2}} % Ex:\Der{y}{x}
\def\Derr#1#2{\Frac{\hbox{d}^2#1}{\hbox{d}#2^2}}
\def\Dron#1#2{\Frac{\partial#1}{\partial#2}}
\def\dron#1#2{\frac{\partial#1}{\partial#2}}
\def\<<{\leavevmode\raise.3ex\hbox{$\scriptscriptstyle\langle\!\langle$}}
\def\>>{\leavevmode\raise.3ex\hbox{$\scriptscriptstyle\rangle\!\rangle$}}
\def\implique{\ \Longrightarrow\ }
\def\vabs#1{\vert#1\vert}
\def\norme#1{\Vert#1\Vert}
\def\Norme#1{\vert\vert\vert#1\vert\vert\vert}
\def\[{[\![}
\def\]{]\!]}

% SUPERPOSITION
\def\up#1{\raise 1ex\hbox{\septtm#1}}
% op{\'e}rateur avec dessous: build {op{\'e}rateur} {dessous}
\def\build#1#2{\mathrel{\mathop{\kern 0pt#1}\limits_{#2}}}
% op{\'e}rateur avec deux dessous: Build {op{\'e}rateur} {dessous1}{dessous2}
\def\Build#1#2#3{\build{{#1}}{\scriptstyle{#2}\atop\scriptstyle{#3}}}
% fl\`eche double : fleche {variable} {valeur}
\def\fleche#1#2{\build{\hbox to 9mm{\rightarrowfill}}{{#1}\rightarrow{#2}}}
% fl\`eche triple : Fleche {variable} {valeur} {3i\`eme ligne}
\def\Fleche#1#2#3{\build{\hbox to 9mm{\rightarrowfill}}
{\scriptstyle{#1}\rightarrow{#2}\atop\scriptstyle{#3}}}
% encadrement d'une bo{\^i}te: cadre {largeur blanc} {bo{\^i}te}
\long\def\cadre#1#2{\vbox{\hrule\hbox{\vrule%
\vbox spread#1{\vfil\hbox spread#1{\hfil#2\hfil}\vfil}\vrule}\hrule}\par}

\def\tp{\centerline{\bsl Travaux pratiques}\so}
\def\tit#1{{\parindent=-2mm{\bf#1}\smallskip}}
\long\def\TITRE#1{\bigskip\bigskip

\centerline{\Large#1}

\bigskip}
\long\def\TIT#1{\bigskip\centerline{\largeb#1}\bigskip}
\long\def\Titre#1{\bigskip{\large#1}\bigskip}
\long\def\titre#1{\bigskip\centerline{\hfill{\bf #1}\hfill}
\medskip}
\long\def\tx#1#2{\hbox{\hbox to 94mm{\vtop{\hsize=90mm#1\vfill}\hfill}
\hfill\hbox to 74mm{\vtop{\hsize=70mm#2\vfill}\hfill}}}%\filbreak}


\hsize=170mm \vsize=250mm
\hoffset=-4mm \voffset=-1mm
\pretolerance=500 \tolerance=1000 \brokenpenalty=5000

%\fhyph
\frenchspacing
\overfullrule=0cm %\emergencystretch=10pt

\null
\vskip 0.5cm

\parindent=0mm
\abovedisplayskip=6pt plus 2pt minus 4pt
\abovedisplayshortskip=0pt plus 2pt
\belowdisplayskip=6pt plus 2pt minus 4pt
\belowdisplayshortskip=0pt plus 2pt

\def\bg{\bigskip}
\def\md{\medskip}
\def\cl{\centerline}
\def\info{informatique}

\long\def\tx#1#2{\hbox{\hbox to 94mm{\vtop{\hsize=90mm#1\vfill}\hfill}
\hfill\hbox to 74mm{\vtop{\hsize=70mm#2\vfill}\hfill}}\filbreak}

\long\def\txv#1#2{\hbox{\vrule\hskip2mm\hbox to
94mm{\vtop{\hsize=90mm#1\vfill}\hfill} \hfill\hbox to
74mm{\vtop{\hsize=70mm#2\vfill}\hfill}}\filbreak}

\long\def\txz#1#2{\hbox{\hbox to
94mm{\vtop{\hsize=90mm#1\vfill}\hfill} \hfill\hbox to
74mm{\vtop{\hsize=70mm#2\vfill}\hfill}\hskip2mm\vrule width 1mm}\filbreak}

\def\vrg{\raise.3mm\hbox{$\hskip.2mm,\ $}}





\titrecourant={\bf MPSI\ }
\partiecourante={\bf MPSI\ }
\premiertitre={\bf CLASSE DE PREMI\`ERE ANN\'EE MPSI}
