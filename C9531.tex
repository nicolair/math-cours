%<dscrpt>Fichier de déclarations Latex à inclure au début d'un élément de cours.</dscrpt>

\documentclass[a4paper]{article}
\usepackage[hmargin={1.8cm,1.8cm},vmargin={2.4cm,2.4cm},headheight=13.1pt]{geometry}

%includeheadfoot,scale=1.1,centering,hoffset=-0.5cm,
\usepackage[pdftex]{graphicx,color}
\usepackage[french]{babel}
%\selectlanguage{french}
\addto\captionsfrench{
  \def\contentsname{Plan}
}
\usepackage{fancyhdr}
\usepackage{floatflt}
\usepackage{amsmath}
\usepackage{amssymb}
\usepackage{amsthm}
\usepackage{stmaryrd}
%\usepackage{ucs}
\usepackage[utf8]{inputenc}
%\usepackage[latin1]{inputenc}
\usepackage[T1]{fontenc}


\usepackage{titletoc}
%\contentsmargin{2.55em}
\dottedcontents{section}[2.5em]{}{1.8em}{1pc}
\dottedcontents{subsection}[3.5em]{}{1.2em}{1pc}
\dottedcontents{subsubsection}[5em]{}{1em}{1pc}

\usepackage[pdftex,colorlinks={true},urlcolor={blue},pdfauthor={remy Nicolai},bookmarks={true}]{hyperref}
\usepackage{makeidx}

\usepackage{multicol}
\usepackage{multirow}
\usepackage{wrapfig}
\usepackage{array}
\usepackage{subfig}


%\usepackage{tikz}
%\usetikzlibrary{calc, shapes, backgrounds}
%pour la présentation du pseudo-code
% !!!!!!!!!!!!!!      le package n'est pas présent sur le serveur sous fedora 16 !!!!!!!!!!!!!!!!!!!!!!!!
%\usepackage[french,ruled,vlined]{algorithm2e}

%pr{\'e}sentation du compteur de niveau 2 dans les listes
\makeatletter
\renewcommand{\labelenumii}{\theenumii.}
\renewcommand{\thesection}{\Roman{section}.}
\renewcommand{\thesubsection}{\arabic{subsection}.}
\renewcommand{\thesubsubsection}{\arabic{subsubsection}.}
\makeatother


%dimension des pages, en-t{\^e}te et bas de page
%\pdfpagewidth=20cm
%\pdfpageheight=14cm
%   \setlength{\oddsidemargin}{-2cm}
%   \setlength{\voffset}{-1.5cm}
%   \setlength{\textheight}{12cm}
%   \setlength{\textwidth}{25.2cm}
   \columnsep=1cm
   \columnseprule=0.5pt

%En tete et pied de page
\pagestyle{fancy}
\lhead{MPSI-\'Eléments de cours}
\rhead{\today}
%\rhead{25/11/05}
\lfoot{\tiny{Cette création est mise à disposition selon le Contrat\\ Paternité-Pas d'utilisations commerciale-Partage des Conditions Initiales à l'Identique 2.0 France\\ disponible en ligne http://creativecommons.org/licenses/by-nc-sa/2.0/fr/
} }
\rfoot{\tiny{Rémy Nicolai \jobname}}


\newcommand{\baseurl}{http://back.maquisdoc.net/data/cours\_nicolair/}
\newcommand{\urlexo}{http://back.maquisdoc.net/data/exos_nicolair/}
\newcommand{\urlcours}{https://maquisdoc-math.fra1.digitaloceanspaces.com/}

\newcommand{\N}{\mathbb{N}}
\newcommand{\Z}{\mathbb{Z}}
\newcommand{\C}{\mathbb{C}}
\newcommand{\R}{\mathbb{R}}
\newcommand{\D}{\mathbb{D}}
\newcommand{\K}{\mathbf{K}}
\newcommand{\Q}{\mathbb{Q}}
\newcommand{\F}{\mathbf{F}}
\newcommand{\U}{\mathbb{U}}
\newcommand{\p}{\mathbb{P}}


\newcommand{\card}{\mathop{\mathrm{Card}}}
\newcommand{\Id}{\mathop{\mathrm{Id}}}
\newcommand{\Ker}{\mathop{\mathrm{Ker}}}
\newcommand{\Vect}{\mathop{\mathrm{Vect}}}
\newcommand{\cotg}{\mathop{\mathrm{cotan}}}
\newcommand{\sh}{\mathop{\mathrm{sh}}}
\newcommand{\ch}{\mathop{\mathrm{ch}}}
\newcommand{\argsh}{\mathop{\mathrm{argsh}}}
\newcommand{\argch}{\mathop{\mathrm{argch}}}
\newcommand{\tr}{\mathop{\mathrm{tr}}}
\newcommand{\rg}{\mathop{\mathrm{rg}}}
\newcommand{\rang}{\mathop{\mathrm{rg}}}
\newcommand{\Mat}{\mathop{\mathrm{Mat}}}
\newcommand{\MatB}[2]{\mathop{\mathrm{Mat}}_{\mathcal{#1}}\left( #2\right) }
\newcommand{\MatBB}[3]{\mathop{\mathrm{Mat}}_{\mathcal{#1} \mathcal{#2}}\left( #3\right) }
\renewcommand{\Re}{\mathop{\mathrm{Re}}}
\renewcommand{\Im}{\mathop{\mathrm{Im}}}
\renewcommand{\th}{\mathop{\mathrm{th}}}
\newcommand{\repere}{$(O,\overrightarrow{i},\overrightarrow{j},\overrightarrow{k})$}
\newcommand{\cov}{\mathop{\mathrm{Cov}}}

\newcommand{\absolue}[1]{\left| #1 \right|}
\newcommand{\fonc}[5]{#1 : \begin{cases}#2 \rightarrow #3 \\ #4 \mapsto #5 \end{cases}}
\newcommand{\depar}[2]{\dfrac{\partial #1}{\partial #2}}
\newcommand{\norme}[1]{\left\| #1 \right\|}
\newcommand{\se}{\geq}
\newcommand{\ie}{\leq}
\newcommand{\trans}{\mathstrut^t\!}
\newcommand{\val}{\mathop{\mathrm{val}}}
\newcommand{\grad}{\mathop{\overrightarrow{\mathrm{grad}}}}

\newtheorem*{thm}{Théorème}
\newtheorem{thmn}{Théorème}
\newtheorem*{prop}{Proposition}
\newtheorem{propn}{Proposition}
\newtheorem*{pa}{Présentation axiomatique}
\newtheorem*{propdef}{Proposition - Définition}
\newtheorem*{lem}{Lemme}
\newtheorem{lemn}{Lemme}

\theoremstyle{definition}
\newtheorem*{defi}{Définition}
\newtheorem*{nota}{Notation}
\newtheorem*{exple}{Exemple}
\newtheorem*{exples}{Exemples}


\newenvironment{demo}{\renewcommand{\proofname}{Preuve}\begin{proof}}{\end{proof}}
%\renewcommand{\proofname}{Preuve} doit etre après le begin{document} pour fonctionner

\theoremstyle{remark}
\newtheorem*{rem}{Remarque}
\newtheorem*{rems}{Remarques}

\renewcommand{\indexspace}{}
\renewenvironment{theindex}
  {\section*{Index} %\addcontentsline{toc}{section}{\protect\numberline{0.}{Index}}
   \begin{multicols}{2}
    \begin{itemize}}
  {\end{itemize} \end{multicols}}


%pour annuler les commandes beamer
\renewenvironment{frame}{}{}
\newcommand{\frametitle}[1]{}
\newcommand{\framesubtitle}[1]{}

\newcommand{\debutcours}[2]{
  \chead{#1}
  \begin{center}
     \begin{huge}\textbf{#1}\end{huge}
     \begin{Large}\begin{center}Rédaction incomplète. Version #2\end{center}\end{Large}
  \end{center}
  %\section*{Plan et Index}
  %\begin{frame}  commande beamer
  \tableofcontents
  %\end{frame}   commande beamer
  \printindex
}


\makeindex
\begin{document}
\noindent

\debutcours{Inégalité de Chebychev}{alpha}
\section{Présentation - linéarité}
La forme discrète de l'inégalité de Chebychev porte sur des suites croissantes de nombres réels et fournit un encadrement du produit des valeurs moyennes.
\begin{displaymath}
\left.
\begin{aligned}
  &a_1\leq a_2\leq \cdots \leq a_p\\ &b_1\leq b_2\leq \cdots \leq b_p 
\end{aligned}
\right\rbrace 
\Rightarrow
\frac{1}{p}\sum_{k=1}^p a_kb_{n-k+1}
\leq \left(\frac{1}{p}\sum_{k=1}^p a_k \right) \left(\frac{1}{p}\sum_{k=1}^p b_k \right)
\leq \frac{1}{p}\sum_{k=1}^p a_kb_k 
\end{displaymath}
La forme continue porte sur des fonctions $f$ et $g$ continues et croissantes sur un segment $[a,b]$:
\begin{displaymath}
\frac{1}{b-a}\int_{a}^b f(t)g(b-t)\, dt
\leq \left(\frac{1}{b-a}\int_{a}^b f(t)\, dt \right) \left(\frac{1}{b-a}\int_{a}^b g(t)\, dt \right)
\leq \frac{1}{b-a}\int_{a}^b f(t)g(t)\, dt 
\end{displaymath}
Ces formules seront démontrées en deux temps. On prouve d'abord dans cette section que l'inégalité à droite dans un cas particulier entraîne l'encadrement dans le cas général.\newline
Admettons ici que
\begin{equation}
\left.
\begin{aligned}
  &a_1\leq a_2\leq \cdots \leq a_p\\ &b_1\leq b_2\leq \cdots \leq b_p \\ &\sum_{k=1}^p a_k = 0
\end{aligned}
\right\rbrace 
\Rightarrow
0 \leq \frac{1}{p}\sum_{k=1}^p a_kb_k
\label{cheb1}
\end{equation}
Dans le cas où la somme des $a_k$ n'est pas nulle, on peut noter $\overline{a}$ la valeur moyenne, définir $a'_k$ par $a'_k = a_k -\overline{a}$ et appliquer la proposition \ref{cheb1} aux familles (toujours croissantes) des $a'_k$ et $b_k$. On en déduit par linéarité 
\begin{equation}
\left.
\begin{aligned}
  &a_1\leq a_2\leq \cdots \leq a_p\\ &b_1\leq b_2\leq \cdots \leq b_p
\end{aligned}
\right\rbrace 
\Rightarrow
\left(\frac{1}{p}\sum_{k=1}^p a_k \right) \left(\frac{1}{p}\sum_{k=1}^p b_k \right)
\leq \frac{1}{p}\sum_{k=1}^p a_kb_k
\label{cheb2}
\end{equation}
Pour obtenir la partie gauche de l'inégalité, considérons les familles $(a_1,\cdots,a_p)$ et $(-b_p, -b_{p-1}, \cdots, -b_1)$. Elles sont croissantes. On peut leur appliquer la proposition \ref{cheb2}. On en tire l'inégalité cherchée par linéarité.\newline
Les raisonnements sont identiques dans le cas continu.\newline
La preuve de l'inégalité de droite repose sur des identités remarquables. On peut aussi la montrer par récurrence dans le cas discret.
\begin{prop}
Soit $(a_1,\cdots,a_p)\in \R^p$ et $(b_1,\cdots,b_p)\in \R^p$:
\begin{displaymath}
\frac{1}{p}\sum_{k=1}^p a_kb_k = \left(\frac{1}{p}\sum_{k=1}^p a_k \right) \left(\frac{1}{p}\sum_{k=1}^p b_k \right)
+ \frac{1}{2p^2}\sum_{(i,j)\in \{1,\cdots,p^2\}}^p (a_i-a_j)(b_i-b_j)
\end{displaymath}
Soit $f$ et $g$ dans $\mathcal{C}[a,b]$:
\begin{multline*}
  \frac{1}{b-a}\int_a^bf(t)g(t)\,dt =
\left(\frac{1}{b-a}\int_a^bf(t)\,dt \right) \left(\frac{1}{b-a}\int_a^bg(t)\,dt \right)\\
+ \frac{1}{2(b-a)^2}\int_a^b \int_a^b(f(x)-f(y))(g(x)-g(y))\,dx\,dy
\end{multline*}
\end{prop}
\begin{demo}
  Ne pas se laisser impressionner et développer la somme ou l'intégrale double par linéarité. On aboutit à la relation facilement.
\end{demo}
On obtient l'inégalité dans le cas croissant en découpant en deux morceaux. Par exemple, pour le cas continu avec des fonctions $f$ et $g$ croissantes:
\begin{multline*}
\int_a^b \int_a^b(f(x)-f(y))(g(x)-g(y))\,dx\,dy
= \int_a^b \left( \int_a^b(f(x)-f(y))(g(x)-g(y))\,dx\right) \,dy \\
= \int_a^b \left( 
\int_a^y\underset{\leq 0}{(f(x)-f(y))}\underset{\leq 0}{(g(x)-g(y))}\,dx 
+ \int_y^b\underset{\geq 0}{(f(x)-f(y))}\underset{\geq 0}{(g(x)-g(y))}\,dx
\right) \,dy \geq 0
\end{multline*}
Le raisonnement est analogue dans le cas discret. On peut remarquer que, dans les cas considérés, l'inégalité est stricte sauf si une des deux familles/ fonctions est constante.

\section{Applications}
\subsection{Inégalité de Nesbitt}
\index{Inégalité de Nesbitt}
Soit $a$, $b$, $c$ des réels tels que $a+b$, $b+c$, $c+a$ strictement positifs. Alors
\begin{displaymath}
  \frac{a}{b+c} + \frac{b}{c+a} + \frac{c}{a+b} \geq \frac{3}{2}
\end{displaymath}
Comme la relation est inchangée par permutation des lettres, on peut supposer $a\leq b \leq c$ ce qui entraîne $a+b\leq a+c \leq b+c$ et appliquer l'inégalité de Chebychev à
\begin{displaymath}
 \frac{a}{b+c} + \frac{b}{c+a} + \frac{c}{a+b} + 3 = (a+b+c)(\frac{1}{b+c} + \frac{1}{c+a} + \frac{1}{a+b})  
 \leq 3\left(\frac{a}{b+c} + \frac{b}{c+a} + \frac{c}{a+b} \right) 
\end{displaymath}
\subsection{Avec des dominos}
Soit $a_1\geq a_2 \geq \cdots \geq a_p >0$, alors
\begin{displaymath}
  \sum_{k=1}^p\frac{a_k}{(2k-1)(2k+1)}\geq \frac{1}{2p}\sum_{k=1}^p a_k
\end{displaymath}
\subsection{Une inégalité trigonométrique}
Soit $a$, $b$, $c$ tels que $0< a < b < c <\frac{\pi}{2}$. Alors
\begin{displaymath}
  \frac{\sin a + \sin b + \sin c}{(\tan b + \tan c)\cos a + (\tan c + \tan a)\cos b + (\tan a + \tan b)\cos c} \leq \frac{1}{2}
\end{displaymath}

\section{Autres inégalités}
Pour terminer citons deux autres inégalités classiques qui se placent dans le même contexte.
\subsection{Inégalité de Cauchy-Scharz}
Cette fois c'est la somme des produits qui est encadrée.
Soit $a_1, a_2, \cdots , a_p$ et $b_1, b_2, \cdots , b_p$ deux familles quelconques de nombres réels. 
\begin{displaymath}
\left|\sum_{k=1}^p a_k b_k\right| \leq \sqrt{\sum_{k=1}^p a_k^2}\sqrt{\sum_{k=1}^p b_k^2}  
\end{displaymath}
La démonstration repose sur la fonction du second degré
\begin{displaymath}
  t \mapsto \left(\sum_{k=1}^p(a_k + tb_k)^2 \right) 
\end{displaymath}
qui ne prend que des valeurs positives. L'inégalité revient à la négativité du discriminant.

\subsection{Inégalité de réordonnement}
\index{Inégalité de réordonnement}
Soit $a_1\leq a_2\leq \cdots \leq a_p$ et $b_1\leq b_2\leq \cdots \leq b_p$ deux familles croissantes de nombres réels. Soit $\sigma$ une permutation des entiers de $1$ à $p$:
\begin{displaymath}
\sum_{k=1}^p a_k b_{p-k+1} \leq \sum_{k=1}^p a_k b_{\sigma(k)} \leq \sum_{k=1}^p a_k b_k  
\end{displaymath}
L'inégalité de droite se prouve par récurrence sur $p$. L'inégalité de gauche se déduit de la première par linéarité.\newline
On peut déduire l'inégalité de Chebychev de celle du réordonnement.\newline
Dans les conditions indiquées, on peut écrire $p$ inégalités en permutant circulairement les $b_i$ de gauche
\begin{align*}
  a_1 b_1 + a_2b_2 + \cdots + a_nb_n &\leq a_1 b_1 + a_2b_2 + \cdots + a_nb_n \\
  a_1 b_2 + a_2b_3 + \cdots + a_nb_1 &\leq a_1 b_1 + a_2b_2 + \cdots + a_nb_n \\
  a_1 b_3 + a_2b_4 + \cdots + a_nb_2 &\leq a_1 b_1 + a_2b_2 + \cdots + a_nb_n \\
  \vdots & \\
  a_1 b_n + a_2b_1 + \cdots + a_nb_n &\leq a_1 b_1 + a_2b_2 + \cdots + a_nb_n \\
\end{align*}
En sommant, on obtient
\begin{displaymath}
  (a_1+\cdots + a_n)(b_1+b_2+\cdots b_n) \leq n(a_1 b_1 + a_2b_2 + \cdots + a_nb_n) 
\end{displaymath}
qui permet de conclure.\newline
Exemple. Soit $a_1, \cdots, a_p$ des nombres naturels non nuls deux à deux distincts. Alors
\begin{displaymath}
  \frac{a_1}{1^2} +  \frac{a_2}{2^2} + \cdots +  \frac{a_p}{p^2} > \frac{1}{1} + \frac{1}{2} + \cdots +\frac{1}{p}
\end{displaymath}
Notons $b_1, \cdots, b_p$ les valeurs des $a_k$ dans l'ordre croissant. Comme les $a_k$ sont deux à deux distincts, on a de plus $b_k\geq k$. Par réordonnement:
\begin{displaymath}
  \left. 
\begin{aligned}
b_1 < b_2 < \cdots <b_p \\ \frac{1}{1^2}< \frac{1}{2^2}< \cdots <\frac{1}{p^2}  
\end{aligned}
\right\rbrace \Rightarrow
\frac{b_1}{1^2}+ \frac{b_2}{2^2}+ \cdots +\frac{b_p}{p^2} < \frac{a_1}{1^2}+ \frac{a_2}{2^2}+ \cdots +\frac{a_p}{p^2}
\end{displaymath}
et
\begin{displaymath}
\frac{b_1}{1^2}+ \frac{b_2}{2^2}+ \cdots +\frac{b_p}{p^2} \geq  \frac{1}{1^2}+ \frac{2}{2^2}+ \cdots +\frac{p}{p^2}
=\frac{1}{1} + \frac{1}{2} + \cdots + \frac{1}{p} 
\end{displaymath}

\section{Un exercice résolu}
Soit $a, b, c$ réels strictement positifs. Montrer que 
\begin{displaymath}
  \frac{ab}{a+b} + \frac{ac}{a+c}+\frac{bc}{b+c} < \frac{3}{2}\frac{ab+bc+ac}{a+b+c}
\end{displaymath}
On s'inspire de la démonstration de l'inégalité de Nesbitt. Notons $S$ la somme à majorer:
\begin{displaymath}
\left. 
\begin{aligned}
  \frac{ab}{a+b} +c = \frac{ab + ac + bc}{a+b}\\
  \frac{bc}{b+c} +a = \frac{bc + ba + ca}{b+c}\\
  \frac{ca}{c+a} +b = \frac{ca + bc + ba}{c+a}  
\end{aligned}
\right\rbrace \Rightarrow
S + (a+b+c) = \left(ab+bc+ca \right) \left(\frac{1}{a+b}+\frac{1}{b+c}+\frac{1}{c+a} \right) 
\end{displaymath}
Comme l'inégalité à démontrer est invariante par permutation des lettres, on peut supposer $a<b<c$ ce qui entraîne $ab < ac < bc$ et $a+b < a+c < b+c$ d'où, par l'inégalité de Chebychev,
\begin{displaymath}
\left. 
\begin{aligned}
  ab < ac < bc\\
  \frac{1}{b+c} < \frac{1}{a+c} < \frac{1}{a+b}  
\end{aligned}
\right\rbrace \Rightarrow
\left(ab+bc+ca \right) \left(\frac{1}{a+b}+\frac{1}{b+c}+\frac{1}{c+a} \right) > 3S 
\end{displaymath}
On en tire $S < \frac{1}{2}(a+b+c)$. Malheureusement cette inégalité est plus faible que celle qui est demandée car
\begin{multline*}
  \frac{3}{2}\frac{ab+bc+ac}{a+b+c} - \frac{1}{2}(a+b+c)
  =\frac{1}{2(a+b+c)}\left( 3(ab+bc+ac)-(a+b+c)^2\right)\\
  =\frac{1}{2(a+b+c)}\left( ab+bc+ac-a^2 - b^2 -c^2\right) < 0
\end{multline*}
Par l'inégalité de réordonnement lorsque $a<b<c$.\newline
Curieusement, on peut améliorer l'inégalité en appliquant
\begin{displaymath}
\left(ab+bc+ca \right) \left(\frac{1}{a+b}+\frac{1}{b+c}+\frac{1}{c+a} \right) = S +(a+b+c) <\frac{3}{2}(a+b+c)  
\end{displaymath}
aux inverses c'est à dire avec $\frac{1}{a}$, $\frac{1}{b}$, $\frac{1}{c}$. Il vient:
\begin{displaymath}
\left(\frac{1}{ab}+\frac{1}{bc}+\frac{1}{ca} \right)S< \frac{3}{2}\left(  \frac{1}{a}+\frac{1}{b}+\frac{1}{c} \right) 
\Leftrightarrow
\frac{a+b+c}{abc} S < \frac{3}{2}\frac{ab+bc+ac}{abc}
\Leftrightarrow
S < \frac{3}{2}\frac{ab+bc+ac}{a+b+c}
\end{displaymath}

\end{document}
