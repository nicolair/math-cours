%<dscrpt>Fichier de déclarations Latex à inclure au début d'un élément de cours.</dscrpt>

\documentclass[a4paper]{article}
\usepackage[hmargin={1.8cm,1.8cm},vmargin={2.4cm,2.4cm},headheight=13.1pt]{geometry}

%includeheadfoot,scale=1.1,centering,hoffset=-0.5cm,
\usepackage[pdftex]{graphicx,color}
\usepackage[french]{babel}
%\selectlanguage{french}
\addto\captionsfrench{
  \def\contentsname{Plan}
}
\usepackage{fancyhdr}
\usepackage{floatflt}
\usepackage{amsmath}
\usepackage{amssymb}
\usepackage{amsthm}
\usepackage{stmaryrd}
%\usepackage{ucs}
\usepackage[utf8]{inputenc}
%\usepackage[latin1]{inputenc}
\usepackage[T1]{fontenc}


\usepackage{titletoc}
%\contentsmargin{2.55em}
\dottedcontents{section}[2.5em]{}{1.8em}{1pc}
\dottedcontents{subsection}[3.5em]{}{1.2em}{1pc}
\dottedcontents{subsubsection}[5em]{}{1em}{1pc}

\usepackage[pdftex,colorlinks={true},urlcolor={blue},pdfauthor={remy Nicolai},bookmarks={true}]{hyperref}
\usepackage{makeidx}

\usepackage{multicol}
\usepackage{multirow}
\usepackage{wrapfig}
\usepackage{array}
\usepackage{subfig}


%\usepackage{tikz}
%\usetikzlibrary{calc, shapes, backgrounds}
%pour la présentation du pseudo-code
% !!!!!!!!!!!!!!      le package n'est pas présent sur le serveur sous fedora 16 !!!!!!!!!!!!!!!!!!!!!!!!
%\usepackage[french,ruled,vlined]{algorithm2e}

%pr{\'e}sentation du compteur de niveau 2 dans les listes
\makeatletter
\renewcommand{\labelenumii}{\theenumii.}
\renewcommand{\thesection}{\Roman{section}.}
\renewcommand{\thesubsection}{\arabic{subsection}.}
\renewcommand{\thesubsubsection}{\arabic{subsubsection}.}
\makeatother


%dimension des pages, en-t{\^e}te et bas de page
%\pdfpagewidth=20cm
%\pdfpageheight=14cm
%   \setlength{\oddsidemargin}{-2cm}
%   \setlength{\voffset}{-1.5cm}
%   \setlength{\textheight}{12cm}
%   \setlength{\textwidth}{25.2cm}
   \columnsep=1cm
   \columnseprule=0.5pt

%En tete et pied de page
\pagestyle{fancy}
\lhead{MPSI-\'Eléments de cours}
\rhead{\today}
%\rhead{25/11/05}
\lfoot{\tiny{Cette création est mise à disposition selon le Contrat\\ Paternité-Pas d'utilisations commerciale-Partage des Conditions Initiales à l'Identique 2.0 France\\ disponible en ligne http://creativecommons.org/licenses/by-nc-sa/2.0/fr/
} }
\rfoot{\tiny{Rémy Nicolai \jobname}}


\newcommand{\baseurl}{http://back.maquisdoc.net/data/cours\_nicolair/}
\newcommand{\urlexo}{http://back.maquisdoc.net/data/exos_nicolair/}
\newcommand{\urlcours}{https://maquisdoc-math.fra1.digitaloceanspaces.com/}

\newcommand{\N}{\mathbb{N}}
\newcommand{\Z}{\mathbb{Z}}
\newcommand{\C}{\mathbb{C}}
\newcommand{\R}{\mathbb{R}}
\newcommand{\D}{\mathbb{D}}
\newcommand{\K}{\mathbf{K}}
\newcommand{\Q}{\mathbb{Q}}
\newcommand{\F}{\mathbf{F}}
\newcommand{\U}{\mathbb{U}}
\newcommand{\p}{\mathbb{P}}


\newcommand{\card}{\mathop{\mathrm{Card}}}
\newcommand{\Id}{\mathop{\mathrm{Id}}}
\newcommand{\Ker}{\mathop{\mathrm{Ker}}}
\newcommand{\Vect}{\mathop{\mathrm{Vect}}}
\newcommand{\cotg}{\mathop{\mathrm{cotan}}}
\newcommand{\sh}{\mathop{\mathrm{sh}}}
\newcommand{\ch}{\mathop{\mathrm{ch}}}
\newcommand{\argsh}{\mathop{\mathrm{argsh}}}
\newcommand{\argch}{\mathop{\mathrm{argch}}}
\newcommand{\tr}{\mathop{\mathrm{tr}}}
\newcommand{\rg}{\mathop{\mathrm{rg}}}
\newcommand{\rang}{\mathop{\mathrm{rg}}}
\newcommand{\Mat}{\mathop{\mathrm{Mat}}}
\newcommand{\MatB}[2]{\mathop{\mathrm{Mat}}_{\mathcal{#1}}\left( #2\right) }
\newcommand{\MatBB}[3]{\mathop{\mathrm{Mat}}_{\mathcal{#1} \mathcal{#2}}\left( #3\right) }
\renewcommand{\Re}{\mathop{\mathrm{Re}}}
\renewcommand{\Im}{\mathop{\mathrm{Im}}}
\renewcommand{\th}{\mathop{\mathrm{th}}}
\newcommand{\repere}{$(O,\overrightarrow{i},\overrightarrow{j},\overrightarrow{k})$}
\newcommand{\cov}{\mathop{\mathrm{Cov}}}

\newcommand{\absolue}[1]{\left| #1 \right|}
\newcommand{\fonc}[5]{#1 : \begin{cases}#2 \rightarrow #3 \\ #4 \mapsto #5 \end{cases}}
\newcommand{\depar}[2]{\dfrac{\partial #1}{\partial #2}}
\newcommand{\norme}[1]{\left\| #1 \right\|}
\newcommand{\se}{\geq}
\newcommand{\ie}{\leq}
\newcommand{\trans}{\mathstrut^t\!}
\newcommand{\val}{\mathop{\mathrm{val}}}
\newcommand{\grad}{\mathop{\overrightarrow{\mathrm{grad}}}}

\newtheorem*{thm}{Théorème}
\newtheorem{thmn}{Théorème}
\newtheorem*{prop}{Proposition}
\newtheorem{propn}{Proposition}
\newtheorem*{pa}{Présentation axiomatique}
\newtheorem*{propdef}{Proposition - Définition}
\newtheorem*{lem}{Lemme}
\newtheorem{lemn}{Lemme}

\theoremstyle{definition}
\newtheorem*{defi}{Définition}
\newtheorem*{nota}{Notation}
\newtheorem*{exple}{Exemple}
\newtheorem*{exples}{Exemples}


\newenvironment{demo}{\renewcommand{\proofname}{Preuve}\begin{proof}}{\end{proof}}
%\renewcommand{\proofname}{Preuve} doit etre après le begin{document} pour fonctionner

\theoremstyle{remark}
\newtheorem*{rem}{Remarque}
\newtheorem*{rems}{Remarques}

\renewcommand{\indexspace}{}
\renewenvironment{theindex}
  {\section*{Index} %\addcontentsline{toc}{section}{\protect\numberline{0.}{Index}}
   \begin{multicols}{2}
    \begin{itemize}}
  {\end{itemize} \end{multicols}}


%pour annuler les commandes beamer
\renewenvironment{frame}{}{}
\newcommand{\frametitle}[1]{}
\newcommand{\framesubtitle}[1]{}

\newcommand{\debutcours}[2]{
  \chead{#1}
  \begin{center}
     \begin{huge}\textbf{#1}\end{huge}
     \begin{Large}\begin{center}Rédaction incomplète. Version #2\end{center}\end{Large}
  \end{center}
  %\section*{Plan et Index}
  %\begin{frame}  commande beamer
  \tableofcontents
  %\end{frame}   commande beamer
  \printindex
}


\makeindex
\begin{document}
\noindent

\debutcours{Fonctions d'une variable réelle : vocabulaire}{alpha}

L'étude des \href{\baseurl C2063.pdf}{fonctions d'une variable réelle} est réparti entre divers documents.\newline
Toutes les fonctions considérées ici sont à valeurs réelles. Un point important dans le vocabulaire relatif aux fonctions est de bien faire la différence entre ce qui se rapporte à l'espace de départ et ce qui se rapporte à l'espace d'arrivée. Cela est d'autant plus difficile dans le cas des fonctions d'une variable réelle à valeurs réelles qu'il s'agit chaque fois de $\R$ ou d'une partie de $\R$. Il est commode de s'aider de schémas en convenant de dessiner la droite réelle \emph{horizontalement} quand elle est associée à l'espace de départ et \emph{verticalement} quand elle est associée à l'espace d'arrivée.  
\section{Opérations}
\begin{defi}
 Soit $f$ et $g$ deux fonctions définies sur un intervalle $I$. Somme, produit de deux fonctions. Multiplication d'une fonction par un réel.\newline
Les fonctions $\sup(f,g)$ et $\inf(f,g)$ sont définies par :
\begin{displaymath}
 \forall x\in I \left\lbrace
\begin{aligned}
 \sup(f,g)(x) &= \max(f(x),g(x)) \\
 \inf(f,g)(x) &= \min(f(x),g(x))
\end{aligned}
\right. 
\end{displaymath}
\end{defi}
$f^+$, $f^-$, $|f|$
\begin{figure}[h!t]
 \centering
\input{C3901_1.pdf_t}
\caption{Expression de $\sup$ et $\inf$ avec une valeur absolue}
\label{fig:C3901_1}
\end{figure}

\begin{prop}
 \begin{align*}
  \sup(f,g)=\frac{1}{2}\left(f + g + |f-g| \right)
& &  
 \inf(f,g)=\frac{1}{2}\left(f + g - |f-g| \right)
 \end{align*}

\end{prop}

\section{Inégalités}
\subsection{Fonctions majorées, minorées, bornées}
\index{fonctions majorées, minorées, bornées}
\begin{defi}[fonctions majorées, minorées, bornées]
 Une fonction $f$ est dite majorée (respectivement minorée, bornée) si et seulement si l'ensemble de ses valeurs est majoré (resp minoré, borné)
\end{defi}

L'ensemble des fonctions définies sur un ensemble $I$ et bornées et noté $\mathcal{B(I,\R)}$, il est stable pour les cinq opérations.
\index{borne supérieure ou inférieure d'une fonction}
\begin{defi}[borne supérieure ou inférieure d'une fonction]
 Pour une fonction $f$ définie dans $I$.\newline
Lorsque $f$ est majorée, on appelle  borne supérieure de $f$ la borne supérieure de l'ensemble de ses valeurs. Lorsque $f$ est minorée, on appelle  borne inférieure de $f$ la borne inférieure de l'ensemble de ses valeurs. Notations
\begin{align*}
 \sup_I f = \sup \left\lbrace f(t), t\in I\right\rbrace = \sup f(I)
& &
  \inf_I f = \inf \left\lbrace f(t), t\in I\right\rbrace = \inf f(I)
\end{align*}
\end{defi}
\begin{defi}[Inégalités entre fonctions.]
 Soit $f$ et $g$ deux fonctions définies dans $I$, on dira que $f\leq g$ si et seulement si $f(t)\leq g(t)$ pour tous les $t$ dans $I$.
\end{defi}
 

\index{extrémum, extémum local}
\subsection{Extrémum, extrémum local}
\begin{defi}[extrémum global, extémum local]
 Soit $I$ un intervalle de $\R$ et $f$ une fonction définie dans $I$ ...
\end{defi}
\begin{rems}
 \begin{itemize}
  \item Tout extrémum global est un extrémum local.
  \item Une fonction $f$ admet un maximum global si et seulement si l'ensemble de ses valeurs admet un plus grand élément.
  \item Une fonction $f$ admet un minimum global si et seulement si l'ensemble de ses valeurs admet un plus petit élément.
 \end{itemize}

\end{rems}

Cette notion d'extrémum local joue un rôle très important dans le \index{théorème de Rolle} \href{\baseurl C2070.pdf}{théorème de Rolle}.

\subsection{Fonctions monotones}
Fonctions croissantes, strictement croissantes, décroissantes, strictement décroissante.
\begin{rem}
 Une fonction strictement monotone est injective.
\end{rem}

\begin{prop}[Composition des fonctions monotones.]
 La composée de deux fonctions monotones de même sens est croissante, la composée de deux fonctions monotones de sens contraire est décroissante.
\end{prop}
 
\section{Fonctions périodiques}
\index{fonctions périodiques}
\begin{defi}
 Une fonction $f$ définie dans $\R$ est \emph{périodique} si et seulement si il existe un $T$ réel non nul tel que $f(x+T)=f(x)$ pour tous les $x\in\R$. On dit alors qu'elle est $T$-périodique.
\end{defi}
\index{groupe additif des périodes}
\begin{prop}
 L'ensemble des $T$ réels tels que $f(x+T)=f(x)$ pour tous les $x\in\R$ forme un sous-groupe additif de $\R$.
\end{prop}
\begin{demo}
 à rédiger
\end{demo}
Pour un $T\neq 0$ fixé, l'ensemble des fonctions $T$-périodiques est stable pour les cinq opérations.
\section{Fonctions lipschitziennes}
\index{fonctions lipschitziennes}
\begin{defi}
 Une fonction $f$ définie sur un intervalle $I$ de $\R$ est \emph{lipschitzienne} de rapport $k$ si et seulement si 
\begin{displaymath}
 \forall (x,y)\in I^2 : |f(y)-f(x)|\leq k |y-x|
\end{displaymath}
Une fonction est lipschitzienne si et seulement si il existe un réel $k$ pour lequel elle est lipschitzienne de rapport $k$.
\end{defi}
\begin{exples}
 \begin{itemize}
  \item La fonction valeur absolue définie dans $\R$ est lipschitzienne de rapport 1 .
  \item La fonction $t\rightarrow t$ définie dans $\R$ est lipschitzienne de rapport 1.
  \item La fonction $t\rightarrow t^2$ définie dans $\R$ n'est pas est lipschitzienne.
  \item La fonction $t\rightarrow t^2$ définie  dans un segment est lipschitzienne.
 \end{itemize}
\end{exples}
 La restriction d'une fonction lipschitzienne de rapport $k$ est lipschitzienne de rapport $k$.

\section{Propriété locale}
\index{propriété locale d'une fonction}
Soit $a$ un nommbre réel, on dira qu'une fonction définie dans un intervalle $I$ vérifie une propriété \emph{localement en } $a$ si et seulement si il existe un $\alpha >0$ tel que \emph{la restriction} de la fonction à $I\cap]a-\alpha,a+\alpha[$ vérifie la propriété.
\begin{exples}
 \begin{itemize}
  \item Une fonction $f$ est localement bornée en $a$ si et seulement si il existe un $\alpha>0$ et un $M$ tels que $|f(t)|\leq M$ pour tous les $t$ de $I\cap]a-\alpha,a+\alpha[$.
 \end{itemize}

\end{exples}

\end{document}
