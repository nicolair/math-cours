\subsubsection{B - Fonction de la variable réelle à valeurs réelles ou complexes (fin)}

\subsubsubsection{e) Dérivation d'une fonction complexe d'une variable réelle.}
\begin{parcolumns}[rulebetween,distance=\parcoldist]{2}
  \colchunk{Dérivée d'une fonction à valeurs complexes.}
  \colchunk{La dérivée est définie par ses parties réelle et imaginaire.}
  \colplacechunks
  
  \colchunk{Dérivée d'une combinaison linéaire, d'un produit, d'un quotient.}
  \colchunk{Brève extension des résultats sur les fonctions à valeurs réelles.}
  \colplacechunks
  
  \colchunk{Dérivée de $\exp(\varphi)$ où $\varphi$ est une fonction dérivable à valeurs complexes.}
  \colchunk{$\leftrightarrows$ PC et SI: électrocinétique.}
  \colplacechunks
  
  \colchunk{Exemple de dérivation d'une fonction à valeurs complexes.}
  \colchunk{Pour $z\in\C\setminus \R$, fonction à valeurs complexes dont la dérivée est $t\mapsto \frac{1}{t+z}$.}
  \colplacechunks
  
\end{parcolumns}
