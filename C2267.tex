%<dscrpt>Fichier de déclarations Latex à inclure au début d'un élément de cours.</dscrpt>

\documentclass[a4paper]{article}
\usepackage[hmargin={1.8cm,1.8cm},vmargin={2.4cm,2.4cm},headheight=13.1pt]{geometry}

%includeheadfoot,scale=1.1,centering,hoffset=-0.5cm,
\usepackage[pdftex]{graphicx,color}
\usepackage[french]{babel}
%\selectlanguage{french}
\addto\captionsfrench{
  \def\contentsname{Plan}
}
\usepackage{fancyhdr}
\usepackage{floatflt}
\usepackage{amsmath}
\usepackage{amssymb}
\usepackage{amsthm}
\usepackage{stmaryrd}
%\usepackage{ucs}
\usepackage[utf8]{inputenc}
%\usepackage[latin1]{inputenc}
\usepackage[T1]{fontenc}


\usepackage{titletoc}
%\contentsmargin{2.55em}
\dottedcontents{section}[2.5em]{}{1.8em}{1pc}
\dottedcontents{subsection}[3.5em]{}{1.2em}{1pc}
\dottedcontents{subsubsection}[5em]{}{1em}{1pc}

\usepackage[pdftex,colorlinks={true},urlcolor={blue},pdfauthor={remy Nicolai},bookmarks={true}]{hyperref}
\usepackage{makeidx}

\usepackage{multicol}
\usepackage{multirow}
\usepackage{wrapfig}
\usepackage{array}
\usepackage{subfig}


%\usepackage{tikz}
%\usetikzlibrary{calc, shapes, backgrounds}
%pour la présentation du pseudo-code
% !!!!!!!!!!!!!!      le package n'est pas présent sur le serveur sous fedora 16 !!!!!!!!!!!!!!!!!!!!!!!!
%\usepackage[french,ruled,vlined]{algorithm2e}

%pr{\'e}sentation du compteur de niveau 2 dans les listes
\makeatletter
\renewcommand{\labelenumii}{\theenumii.}
\renewcommand{\thesection}{\Roman{section}.}
\renewcommand{\thesubsection}{\arabic{subsection}.}
\renewcommand{\thesubsubsection}{\arabic{subsubsection}.}
\makeatother


%dimension des pages, en-t{\^e}te et bas de page
%\pdfpagewidth=20cm
%\pdfpageheight=14cm
%   \setlength{\oddsidemargin}{-2cm}
%   \setlength{\voffset}{-1.5cm}
%   \setlength{\textheight}{12cm}
%   \setlength{\textwidth}{25.2cm}
   \columnsep=1cm
   \columnseprule=0.5pt

%En tete et pied de page
\pagestyle{fancy}
\lhead{MPSI-\'Eléments de cours}
\rhead{\today}
%\rhead{25/11/05}
\lfoot{\tiny{Cette création est mise à disposition selon le Contrat\\ Paternité-Pas d'utilisations commerciale-Partage des Conditions Initiales à l'Identique 2.0 France\\ disponible en ligne http://creativecommons.org/licenses/by-nc-sa/2.0/fr/
} }
\rfoot{\tiny{Rémy Nicolai \jobname}}


\newcommand{\baseurl}{http://back.maquisdoc.net/data/cours\_nicolair/}
\newcommand{\urlexo}{http://back.maquisdoc.net/data/exos_nicolair/}
\newcommand{\urlcours}{https://maquisdoc-math.fra1.digitaloceanspaces.com/}

\newcommand{\N}{\mathbb{N}}
\newcommand{\Z}{\mathbb{Z}}
\newcommand{\C}{\mathbb{C}}
\newcommand{\R}{\mathbb{R}}
\newcommand{\D}{\mathbb{D}}
\newcommand{\K}{\mathbf{K}}
\newcommand{\Q}{\mathbb{Q}}
\newcommand{\F}{\mathbf{F}}
\newcommand{\U}{\mathbb{U}}
\newcommand{\p}{\mathbb{P}}


\newcommand{\card}{\mathop{\mathrm{Card}}}
\newcommand{\Id}{\mathop{\mathrm{Id}}}
\newcommand{\Ker}{\mathop{\mathrm{Ker}}}
\newcommand{\Vect}{\mathop{\mathrm{Vect}}}
\newcommand{\cotg}{\mathop{\mathrm{cotan}}}
\newcommand{\sh}{\mathop{\mathrm{sh}}}
\newcommand{\ch}{\mathop{\mathrm{ch}}}
\newcommand{\argsh}{\mathop{\mathrm{argsh}}}
\newcommand{\argch}{\mathop{\mathrm{argch}}}
\newcommand{\tr}{\mathop{\mathrm{tr}}}
\newcommand{\rg}{\mathop{\mathrm{rg}}}
\newcommand{\rang}{\mathop{\mathrm{rg}}}
\newcommand{\Mat}{\mathop{\mathrm{Mat}}}
\newcommand{\MatB}[2]{\mathop{\mathrm{Mat}}_{\mathcal{#1}}\left( #2\right) }
\newcommand{\MatBB}[3]{\mathop{\mathrm{Mat}}_{\mathcal{#1} \mathcal{#2}}\left( #3\right) }
\renewcommand{\Re}{\mathop{\mathrm{Re}}}
\renewcommand{\Im}{\mathop{\mathrm{Im}}}
\renewcommand{\th}{\mathop{\mathrm{th}}}
\newcommand{\repere}{$(O,\overrightarrow{i},\overrightarrow{j},\overrightarrow{k})$}
\newcommand{\cov}{\mathop{\mathrm{Cov}}}

\newcommand{\absolue}[1]{\left| #1 \right|}
\newcommand{\fonc}[5]{#1 : \begin{cases}#2 \rightarrow #3 \\ #4 \mapsto #5 \end{cases}}
\newcommand{\depar}[2]{\dfrac{\partial #1}{\partial #2}}
\newcommand{\norme}[1]{\left\| #1 \right\|}
\newcommand{\se}{\geq}
\newcommand{\ie}{\leq}
\newcommand{\trans}{\mathstrut^t\!}
\newcommand{\val}{\mathop{\mathrm{val}}}
\newcommand{\grad}{\mathop{\overrightarrow{\mathrm{grad}}}}

\newtheorem*{thm}{Théorème}
\newtheorem{thmn}{Théorème}
\newtheorem*{prop}{Proposition}
\newtheorem{propn}{Proposition}
\newtheorem*{pa}{Présentation axiomatique}
\newtheorem*{propdef}{Proposition - Définition}
\newtheorem*{lem}{Lemme}
\newtheorem{lemn}{Lemme}

\theoremstyle{definition}
\newtheorem*{defi}{Définition}
\newtheorem*{nota}{Notation}
\newtheorem*{exple}{Exemple}
\newtheorem*{exples}{Exemples}


\newenvironment{demo}{\renewcommand{\proofname}{Preuve}\begin{proof}}{\end{proof}}
%\renewcommand{\proofname}{Preuve} doit etre après le begin{document} pour fonctionner

\theoremstyle{remark}
\newtheorem*{rem}{Remarque}
\newtheorem*{rems}{Remarques}

\renewcommand{\indexspace}{}
\renewenvironment{theindex}
  {\section*{Index} %\addcontentsline{toc}{section}{\protect\numberline{0.}{Index}}
   \begin{multicols}{2}
    \begin{itemize}}
  {\end{itemize} \end{multicols}}


%pour annuler les commandes beamer
\renewenvironment{frame}{}{}
\newcommand{\frametitle}[1]{}
\newcommand{\framesubtitle}[1]{}

\newcommand{\debutcours}[2]{
  \chead{#1}
  \begin{center}
     \begin{huge}\textbf{#1}\end{huge}
     \begin{Large}\begin{center}Rédaction incomplète. Version #2\end{center}\end{Large}
  \end{center}
  %\section*{Plan et Index}
  %\begin{frame}  commande beamer
  \tableofcontents
  %\end{frame}   commande beamer
  \printindex
}


\makeindex
\begin{document}
\noindent

\debutcours{Fonctions d'une variable géométrique : continuité}{alpha}

\section{Topologie}
\subsection{Normes}
\index{norme}\index{norme euclidienne}
Définition. Exemples $N_1$, $N_2$, $N_\infty$. Distance associée à une norme.
\index{normes équivalentes}\index{théorème d'équivalence des normes en dimension finie}
Normes équivalentes. On admet que toutes les normes sont équivalentes.
\index{boules}
Disques (boules). \newline

ici des dessins de boules carrés manquent

Traduction avec des boules de l'équivalence des normes.

\subsection{Parties ouvertes}
\index{partie ouverte}
Définition. Un disque ouvert est une partie ouverte.

\subsection{Suites}
Convergence d'une suite. Unicité de la limite. Indépendance vis à vis de la norme.
\index{théorème de Bolzano-Weirstrass}
Théorème de Bolzano-Weirstrass

Exercice: une partie est dite fermée si et seulement si son complémentaire est une partie ouverte. Une partie $F$ est fermée si et seulement si, pour toute suite convergente d'éléments de $F$, la limite est dans $F$.

\section{Fonctions à valeurs numériques}
\subsection{Exemples - Opérations}

Les fonctions coordonnées $x$ et $y$. Les fonctions qui s'expriment algébriquement à partir des fonctions coordonnées.
Si $f$ est une fonction dont les valeurs réelles sont dans un intervalle $I$ et si $\varphi$ est une fonction définie dans $I$ et à valeurs réelles alors $\varphi \circ f$ est une fonction à valeurs numériques d'une variable géométrique.

Le graphe d'une fonction est une surface. Les lignes de niveau : représentation "IGN".

\subsection{Limite - continuité}
\subsubsection{Définitions}
Soit $\Omega$ une partie de $E$ définition de $\overline{\Omega}$.

Limite et continuité en un point $a$.

La convergence est indépendante de la norme choisie.

Les normes sont continues, les fonctions coordonnées sont continues.

\subsubsection{Opérations}
\begin{prop}
 Soit $f$ et $g$ deux fonctions définies dans $\Omega$, soit $\lambda\in \R$, $a\in \overline{\Omega}$ ,$l\in R$:
\begin{displaymath}
 \left. 
\begin{aligned}
 f\xrightarrow{a}l_f\\
 g\xrightarrow{a}l_g
\end{aligned}
\right\rbrace 
\Rightarrow 
\left\lbrace 
\begin{aligned}
 f+g &\xrightarrow{a} l_f+l_g\\
 \lambda f &\xrightarrow{a} \lambda l_f\\
 fg &\xrightarrow{a} l_fl_g\\
 \sup(f,g) &\xrightarrow{a} \max(l_fl_g)\\
 \inf(fg) &\xrightarrow{a} \min(l_f,l_g)\\
\end{aligned}
\right. 
\end{displaymath}
Soit $\varphi$ une fonction définie dans un intervalle $I$ de $\R$ et à valeurs dans $\R$ tel que $f(\Omega)\subset I$.
\begin{displaymath}
 \left. 
\begin{aligned}
 f\xrightarrow{a}l_f\\
 \varphi \xrightarrow{l_f}\lambda_\varphi
\end{aligned}
\right\rbrace 
\Rightarrow
\varphi \circ f \xrightarrow{a}\lambda_\varphi
\end{displaymath}
\end{prop}
\begin{demo}
 à rédiger ... ou pas !
\end{demo}

\begin{rems}
 \begin{itemize}
  \item On peut étendre facilement certaines implications pour des limites dans $\overline{\R}$.
  \item Ces implications sont valables pour les fonctions continues avec $a\in \Omega$ et $l_f=f(a)$. En particulier, $\mathcal C^0(\Omega,\R)$ est une sous-algèbre de $\mathcal F(\Omega, \R)$.
 \end{itemize}

\end{rems}

\subsubsection{Restrictions}
\begin{prop}
 Soit $f$ une fonction définie dans $\Omega$ et à valeur dans $\R$. Soit $a\in \overline{\Omega}$. Soit $\gamma$ une courbe paramétrée définie dans un intervalle $I$ et à valeurs dans $\Omega$, soit $t_0\in \overline{I}$.
\begin{displaymath}
 \left. 
\begin{aligned}
 f\xrightarrow{a} l_f\\
\gamma \xrightarrow{t_0} a
\end{aligned}
\right\rbrace 
\Rightarrow f\circ \gamma \xrightarrow{t_0} l_f
\end{displaymath}
\end{prop}
\begin{demo}
 à rédiger ... ou pas !
\end{demo}
\begin{prop}
 Soit $f$ une fonction définie dans $\Omega$ et à valeur dans $\R$, soit $a\in \overline{\Omega}$. Soit $\left( a_n\right) _{n\in \N}$ une suite de points de $\Omega$.
\begin{displaymath}
 \left. 
\begin{aligned}
 f &\xrightarrow{a} l_f\\
 \left( a_n\right) _{n\in \N} &\rightarrow a
\end{aligned}
\right\rbrace 
\Rightarrow \left( f(a_n)\right) _{n\in \N} \rightarrow l_f
\end{displaymath}
\end{prop}
\begin{demo}
 à rédiger ... ou pas !
\end{demo}
ici deux petits dessins manquent : un chemin continu et un en pointillé
\begin{rem}
 Le terme \emph{application partielle} qui apparait dans le programme est largement généralisé par les notions de restrictions présentées ici. Une application partielle ou sens du programme est simplement la restriction de la fonction à une droite parallèle à l'un des axes. Une telle droite est le support d'une courbe paramétrée évidente.
\end{rem}

\subsubsection{Encadrement}
\begin{prop}
 Soit $f$, $g$, $h$ des fonctions définies dans $\Omega$ et à valeurs réelles, soit $a\in \overline{\Omega}$ et $l\in \R$.
\begin{displaymath}
 \left. 
\begin{aligned}
 &f\leq g \leq h\\
 &f\xrightarrow{a}l\\
 &h\xrightarrow{a}l\\
\end{aligned}
\right\rbrace 
\Rightarrow g\xrightarrow{a}l
\end{displaymath}
\end{prop}
\begin{demo}
 à rédiger ... ou pas !
\end{demo}

\subsubsection{Exemples-Pratique}

On définit une fonction $f$ dans $\R^2$ privé des deux axes par:
\begin{displaymath}
 \forall (x,y)\in \R^2\text{ tels que } x\neq 0 \text{ et } y\neq 0 :
f((x,y))=\frac{\ch(xy)-\cos(xy)}{x^2y^2}
\end{displaymath}
On peut définir $\varphi$ dans $\R^*$ par :
\begin{displaymath}
 \forall t\in \R^* : \varphi(t)=\frac{\ch(t)-\cos(t)}{t^2}
\end{displaymath}
On change alors la signification des lettres $x$ et $y$. Elles désignent maintenant \emph{les FONCTIONS coordonnées} et on peut exprimer $f$ à l'aide des opérations introduites plus haut :
\begin{displaymath}
 f = \varphi \circ xy
\end{displaymath}
On vérifie à l'aide d'un développement limité usuel que $\varphi\xrightarrow{0}1$ et on en déduit avec les résultats relatifs aux opérations que $f\xrightarrow{O}1$ où $O$ désigne l'origine du repère. La convergence est la même $f\xrightarrow{a}1$ pour tout point $a$ situé sur les axes.

Fonctions homogènes de degré $0$.\index{fonction homogène}
Une fonction homogène de degré $0$ est continue en $O$ si et seulement si elle est constante.
\end{document}