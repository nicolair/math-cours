%!  pour pdfLatex
\documentclass[a4paper]{article}
\usepackage[hmargin={1.5cm,1.5cm},vmargin={2.4cm,2.4cm},headheight=13.1pt]{geometry}

\usepackage[pdftex]{graphicx,color}
%\usepackage{hyperref}

\usepackage[utf8]{inputenc}
\usepackage[T1]{fontenc}
\usepackage{lmodern}
%\usepackage[frenchb]{babel}
\usepackage[french]{babel}

\usepackage{fancyhdr}
\pagestyle{fancy}

%\usepackage{floatflt}

\usepackage{parcolumns}
\setlength{\parindent}{0pt}
\usepackage{xcolor}

%pr{\'e}sentation des compteurs de section, ...
\makeatletter
%\renewcommand{\labelenumii}{\theenumii.}
\renewcommand{\thepart}{}
\renewcommand{\thesection}{}
\renewcommand{\thesubsection}{}
\renewcommand{\thesubsubsection}{}
\makeatother

\newcommand{\subsubsubsection}[1]{\bigskip \rule[5pt]{\linewidth}{2pt} \textbf{ \color{red}{#1} } \newline \rule{\linewidth}{.1pt}}
\newlength{\parcoldist}
\setlength{\parcoldist}{1cm}

\usepackage{maths}
\newcommand{\dbf}{\leftrightarrows}
% remplace les commandes suivantes 
%\usepackage{amsmath}
%\usepackage{amssymb}
%\usepackage{amsthm}
%\usepackage{stmaryrd}

%\newcommand{\N}{\mathbb{N}}
%\newcommand{\Z}{\mathbb{Z}}
%\newcommand{\C}{\mathbb{C}}
%\newcommand{\R}{\mathbb{R}}
%\newcommand{\K}{\mathbf{K}}
%\newcommand{\Q}{\mathbb{Q}}
%\newcommand{\F}{\mathbf{F}}
%\newcommand{\U}{\mathbb{U}}

%\newcommand{\card}{\mathop{\mathrm{Card}}}
%\newcommand{\Id}{\mathop{\mathrm{Id}}}
%\newcommand{\Ker}{\mathop{\mathrm{Ker}}}
%\newcommand{\Vect}{\mathop{\mathrm{Vect}}}
%\newcommand{\cotg}{\mathop{\mathrm{cotan}}}
%\newcommand{\sh}{\mathop{\mathrm{sh}}}
%\newcommand{\ch}{\mathop{\mathrm{ch}}}
%\newcommand{\argsh}{\mathop{\mathrm{argsh}}}
%\newcommand{\argch}{\mathop{\mathrm{argch}}}
%\newcommand{\tr}{\mathop{\mathrm{tr}}}
%\newcommand{\rg}{\mathop{\mathrm{rg}}}
%\newcommand{\rang}{\mathop{\mathrm{rg}}}
%\newcommand{\Mat}{\mathop{\mathrm{Mat}}}
%\renewcommand{\Re}{\mathop{\mathrm{Re}}}
%\renewcommand{\Im}{\mathop{\mathrm{Im}}}
%\renewcommand{\th}{\mathop{\mathrm{th}}}


%En tete et pied de page
\lhead{Programme colle math}
\chead{Semaine 18 du 24/02/20 au 29/02/20}
\rhead{MPSI B Hoche}

\lfoot{\tiny{Cette création est mise à disposition selon le Contrat\\ Paternité-Partage des Conditions Initiales à l'Identique 2.0 France\\ disponible en ligne http://creativecommons.org/licenses/by-sa/2.0/fr/
} }
\rfoot{\tiny{Rémy Nicolai \jobname}}


\begin{document}
\subsection{Espaces vectoriels et applications linéaires (3)}

\subsubsection{C - Applications linéaires}
\subsubsubsection{Généralités}
\begin{parcolumns}[rulebetween,distance=\parcoldist]{2}
 \colchunk{Application linéaire.}
  \colchunk{}
  \colplacechunks
 \colchunk{Opérations sur les applications linéaires~: combinaison linéaire, composition, réciproque. Isomorphismes.}
  \colchunk{L'ensemble $\mathcal L(E,F)$ est un espace vectoriel.\\ Bilinéarité de la composition.}
  \colplacechunks
 \colchunk{Image et image réciproque d'un sous-espace par une application linéaire. Image d'une application linéaire.}
  \colchunk{}
  \colplacechunks
 \colchunk{Noyau d'un application linéaire. Caractérisation de l'injectivité.}
  \colchunk{}
  \colplacechunks
 \colchunk{Si ${(x_i)}_{i\in I}$ est une famille génératrice de $E$ et si $u\in\mathcal L(E,F)$, alors $\Im g=\Vect(u(x_i),\,i\in I)$.}
  \colchunk{}
  \colplacechunks
 \colchunk{Image d'une base par un endomorphisme.}
  \colchunk{}
  \colplacechunks
 \colchunk{Application linéaire de rang fini, rang. Invariance par composition par un isomorphisme.}
  \colchunk{Notation $\rg(u)$.}
  \colplacechunks
\end{parcolumns}

\subsubsubsection{Endomorphismes}
\begin{parcolumns}[rulebetween,distance=\parcoldist]{2}
 \colchunk{Identité, homothétie.}
  \colchunk{notation $\Id_E$.}
  \colplacechunks
 \colchunk{Anneau $(\mathcal L(E),+,\circ)$.}
  \colchunk{Non commutativité si $\dim E\ge 2$\\ Notation $uv$ pour la composée $u\circ v$.}
  \colplacechunks
 \colchunk{projection ou projecteur, symétrie~: définition géométrique, caractérisation des endomorphismes vérifiant $p^2=p$ et $s^2=\Id$. }
  \colchunk{}
  \colplacechunks
 \colchunk{Automorphismes. Groupe linéaire.}
  \colchunk{Notation $GL(E)$.}
  \colplacechunks
\end{parcolumns}

\subsubsubsection{Détermination d'une application linéaire}
\begin{parcolumns}[rulebetween,distance=\parcoldist]{2}
 \colchunk{Si ${(e_i)}_{i\in I}$ est une base de $E$ et ${(f_i)}_{i\in I}$ une famille de vecteurs de $F$, alors il existe une et une seule application $u\in\mathcal L(E,F)$ telle que pour tout $i\in i:\,u(e_i)=f_i$.}
  \colchunk{En MPSI B, ce résultat est connu sous le nom de \og Théorème de prolongement linéaire\fg. Caractérisation de l'injectivité, de la surjectivité, de la bijectivité de $u$.}
  \colplacechunks
 \colchunk{Classification, à isomorphisme près, des espaces de dimension finie par leur dimension.}
  \colchunk{}
  \colplacechunks
 \colchunk{Une application linéaire entre deux espaces de même dimension finie est bijective si et seulement si elle est injective, si et seulement si elle est surjective.}
  \colchunk{}
  \colplacechunks
 \colchunk{Un endomorphisme d'un espace de dimension finie est inversible à gauche si et seulement si il est inversible à droite.}
  \colchunk{}
  \colplacechunks
 \colchunk{Dimension de $\mathcal L(E,F)$ si $E$ et $F$ sont de dimension finie.}
  \colchunk{}
  \colplacechunks
 \colchunk{Si $E_1,\dots,E_p$ sont des sous-espaces de $E$ tels que $E=\bigoplus_{i=1}^pE_i$ et si $u_i\in \mathcal L(E_i,F)$ pour tout $i$, alors il existe une et une seule application $u\in\mathcal L(E,F)$ telle que $u_{|E_i}=u_i$ pour tout $i$.}
  \colchunk{}
  \colplacechunks
\end{parcolumns}

\subsubsubsection{Théorème du rang}
\begin{parcolumns}[rulebetween,distance=\parcoldist]{2}
 \colchunk{Si $u\in\mathcal L(E,F)$ et si $S$ est un supplémentaire de $\Ker u$ dans $E$, alors $u$ induit un isomorphisme de $S$ sur $\Im u$.}
  \colchunk{}
  \colplacechunks
 \colchunk{Théorème du rang~: $\dim E=\dim \Ker u+\rg(u)$.}
  \colchunk{}
  \colplacechunks
\end{parcolumns}

\subsubsubsection{Formes linéaires et hyperplans.}
\begin{parcolumns}[rulebetween,distance=\parcoldist]{2}
 \colchunk{Forme linéaire.}
  \colchunk{Formes coordonnées relativement à une base.}
  \colplacechunks
 \colchunk{Hyperplan.}
  \colchunk{Un hyperplan est le noyau d'une forme linéaire non nulle. \'Equations d'un hyperplan dans une base en dimension finie.}
  \colplacechunks
 \colchunk{Si $H$ est un hyperplan de $E$, alors pour toute droite $D$ non contenue dans $H$, $E=H\oplus D$. Réciproquement, tout supplémentaire d'une droite est une hyperplan.}
  \colchunk{En dimension $n$, les hyperplans sont exactement les sous-espaces de dimension $n-1$.}
  \colplacechunks
 \colchunk{Comparaison de deux équations d'un même hyperplan.}
  \colchunk{}
  \colplacechunks
 \colchunk{Si $E$ est un espace de dimension finie $n$, l'intersection de $m$ hyperplans est de dimension au moins $n-m$. Réciproquement, tout sous-espace de $E$ de dimension $n-m$ est l'intersection de $m$ hyperplans.}
  \colchunk{droites vectorielles de $\R^2$, droites et plans vectoriels de $\R^3$.\\L'étude de la dualité est hors programme.}
  \colplacechunks
\end{parcolumns}


\bigskip

\begin{center}
 \textbf{Questions de cours}
\end{center}
Caractérisations des projecteurs et des symétries.\newline
Théorème du rang.\newline
Si $H$ est un hyperplan de $E$, alors pour toute droite $D$ non contenue dans $H$, $E=H\oplus D$. Réciproquement, tout supplémentaire d'une droite est une hyperplan.\newline
Si $E$ est un espace de dimension finie $n$, l'intersection de $m$ hyperplans est de dimension au moins $n-m$. Réciproquement, tout sous-espace de $E$ de dimension $n-m$ est l'intersection de $m$ hyperplans.

\begin{center}
 \textbf{Prochain programme}
\end{center}

Révision des semaines 13 à 18 (algèbre)

\end{document}
