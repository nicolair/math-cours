\subsection{Nombres complexes et trigonométrie}
L’objectif de ce chapitre est de consolider et d’approfondir les notions sur les nombres complexes acquises en classe de 
Terminale. Le programme combine les aspects suivants :
\begin{itemize}
 \item l’étude algébrique du corps $\C$, équations algébriques (équations du second degré, racines $n$-ièmes d'un nombre complexe) ;  
 \item l’interprétation géométrique des nombres complexes et l’utilisation des nombres complexes en géométrie plane;
 \item l’exponentielle complexe et ses applications à la trigonométrie.
\end{itemize}
Il est recommandé d’illustrer le cours par de nombreuses figures.

\subsubsubsection{a) Nombres complexes}
\begin{parcolumns}[rulebetween,distance=\parcoldist]{2}
  \colchunk{Parties réelle et imaginaire.}
  \colchunk{La construction de $\C$ n'est pas exigible.}
  \colplacechunks

  \colchunk{Opérations sur les nombres complexes.}
  \colchunk{}
  \colplacechunks

  \colchunk{Conjugaison, compatibilité avec les opérations.}
  \colchunk{}
  \colplacechunks

  \colchunk{Point du plan associé à un nombre complexe, affixe d'un point, affixe d'un vecteur.}
  \colchunk{On identifie $\C$ au plan usuel muni d'un repère orthonormé direct.}
  \colplacechunks
\end{parcolumns}

\subsubsubsection{b) Module}
\begin{parcolumns}[rulebetween,distance=\parcoldist]{2}
  \colchunk{Module}
  \colchunk{Interprétation géométrique de $|z-z'|$, cercles et disques.}
  \colplacechunks

  \colchunk{Relation $|z|^2=z\bar{z}$, module d'un produit, d'un quotient.}
  \colchunk{}
  \colplacechunks

  \colchunk{Inégalité triangulaire, cas d'égalité.}
  \colchunk{}
  \colplacechunks
\end{parcolumns}

\subsubsubsection{c) Nombres complexes de module 1 et trigonométrie}
\begin{parcolumns}[rulebetween,distance=\parcoldist]{2}
  \colchunk{Cercle trigonométrique. Paramétrisation par les fonctions circulaires.}
  \colchunk{Notation $\U$.\newline
  Les étudiants doivent savoir retrouver les formules du type $\cos(\pi-x)=-\cos x$ et résoudre des équations et inéquations trigonométriques en s'aidant du cercle trigonométrique.}
  \colplacechunks

  \colchunk{Définition de $e^{it}$ pour $t\in\R$. Exponentielle d'une somme. Formules de trigonométrie exigibles: $\cos(a\pm b)$, $\sin(a\pm b)$, $\cos(2a)$, $\sin(2a)$, $\cos a \cos b$, $\sin a \cos b$, $\sin a \sin b$.}
  \colchunk{Les étudiants doivent savoir factoriser des expressions du type $\cos p + \cos q$.}
  \colplacechunks

  \colchunk{Fonction tangente.}
  \colchunk{La fonction tangente n'a pas été introduite au lycée. Notation $\tan$.}
  \colplacechunks

  \colchunk{Formule exigible: $\tan(a\pm b)$.}
  \colchunk{}
  \colplacechunks

  \colchunk{Formules d'Euler.}
  \colchunk{Linéarisation,\newline calcul de $\sum_{k=0}^n\cos(kt)$, de $\sum_{k=0}^n\sin(kt)$.}
  \colplacechunks

  \colchunk{Formule de Moivre.}
  \colchunk{Les étudiants doivent savoir retrouver les expressions de $\cos(nt)$ et de $\sin(nt)$ en fonction de $\cos t$ et $\sin t$.}
  \colplacechunks
  \end{parcolumns}

\subsubsubsection{d) Formes trigonométriques}
\begin{parcolumns}[rulebetween,distance=\parcoldist]{2}
  \colchunk{Forme trigonométrique $re^{i\theta}$ avec $r>0$ d'un nombre complexe non nul. Arguments. Arguments d'un produit, d'un quotient.}
  \colchunk{Relation de congruence modulo $2\pi$ sur $\R$.}
  \colplacechunks

  \colchunk{Factorisation de $1\pm e^{it}$.}
  \colchunk{}
  \colplacechunks

  \colchunk{Transformation de $a\cos t + b\sin t$ en $A\cos(t-\varphi)$.}
  \colchunk{$\leftrightarrows$ PC et SI: amplitude et phase.}
  \colplacechunks
\end{parcolumns}

\subsubsubsection{e) \'Equations du second degré}
\begin{parcolumns}[rulebetween,distance=\parcoldist]{2}
  \colchunk{Résolution des équations du second degré dans $\C$.}
  \colchunk{Calcul des racines carrées d'un nombre donné sous forme algébrique.}
  \colplacechunks

  \colchunk{Somme et produit des racines.}
  \colchunk{}
  \colplacechunks

\end{parcolumns}

\subsubsubsection{f) Racines $n$-ièmes}
\begin{parcolumns}[rulebetween,distance=\parcoldist]{2}
  \colchunk{Description des racines $n$-ièmes de l'unité, d'un nombre complexe non nul donné sous forme trigonométrique.}
  \colchunk{Notation $\U_n$. Représentation géométrique.}
  \colplacechunks

\end{parcolumns}

\subsubsubsection{g) Exponentielle complexe}
\begin{parcolumns}[rulebetween,distance=\parcoldist]{2}
  \colchunk{Définition de $e^z$ pour $z$ complexe: $e^z = e^{\Re(z)}e^{i\Im(z)}$.}
  \colchunk{Notation $\exp(z)$, $e^z$.\newline
  $\leftrightarrows$ PC et SI: définition d'une impédance complexe en régime sinusoïdal.}
  \colplacechunks

  \colchunk{Exponentielle d'une somme.}
  \colchunk{}
  \colplacechunks

  \colchunk{Pour tous $z$ et $z'$ dans $\C$, $\exp(z)=\exp(z')$ si et seulement si $z-z'\in 2i\pi \Z$.}
  \colchunk{}
  \colplacechunks

  \colchunk{Résolution de l'équation $exp(z)=a$.}
  \colchunk{}
  \colplacechunks
\end{parcolumns}

\subsubsubsection{h) Interprétation géométrique des nombres complexes}
\begin{parcolumns}[rulebetween,distance=\parcoldist]{2}
  \colchunk{Interprétation géométrique du module et d'un argument de $\frac{c-b}{c-a}$.}
  \colchunk{Traduction de l'alignement, de l'orthogonalité.}
  \colplacechunks

  \colchunk{Interprétation géométrique desapplications $z\mapsto az+b$.}
  \colchunk{Similitudes directes. Cas particuliers: translation, homothéties, rotations.}
  \colplacechunks

  \colchunk{Interprétation géométrique de la conjugaison.}
  \colchunk{L'étude générale des similitudes indirectes est hors programme.}
  \colplacechunks
\end{parcolumns}
