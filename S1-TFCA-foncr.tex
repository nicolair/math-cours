\subsubsection{B - Fonction de la variable réelle à valeurs réelles ou complexes}
\subsubsubsection{a) Généralités sur les fonctions}
\begin{parcolumns}[rulebetween,distance=\parcoldist]{2}
  \colchunk{Ensemble de définition}
  \colchunk{}
  \colplacechunks
    
  \colchunk{Représentation graphique d'une fonction $f$ à valeurs réelles.}
  \colchunk{Graphes des fonctions $x\mapsto f(x)+a$, $x\mapsto f(x+a)$, $x\mapsto f(a-x)$, $x\mapsto f(ax)$, $x\mapsto af(x)$.\newline
  Résolution graphique d'équations et d'inéquations du type $f(x)=\lambda$ et $f(x)\geq \lambda$.}
  \colplacechunks
    
  \colchunk{Parité, imparité, périodicité.}
  \colchunk{Interprétation géométrique de ces propriétés.}
  \colplacechunks
    
  \colchunk{Somme, produit, composée.}
  \colchunk{}
  \colplacechunks
    
  \colchunk{Monotonie (large et stricte).}
  \colchunk{}
  \colplacechunks
    
  \colchunk{Fonctions majorées, minorées, bornées.}
  \colchunk{Traduction géométrique de ces propriétés. \newline
  Une fonction est bornée si et seulement si $|f|$ est majorée.}
  \colplacechunks    
\end{parcolumns}

\subsubsubsection{b) Dérivation}
\begin{parcolumns}[rulebetween,distance=\parcoldist]{2}
  \colchunk{\'Equation de la tangente en un point.}
  \colchunk{}
  \colplacechunks

  \colchunk{Dérivée d'une combinaison linéaire, d'un produit, d'un quotient, d'une composée.}
  \colchunk{Ces résultats sont admis à ce stade.\newline
  $\rightleftarrows$ SI: étude cinématique.\newline
  $\rightleftarrows$ PC: exemples de calculs de dérivées partielles.\newline
  \`A ce stade, toute théorie sur les fonctions de plusieurs variables est hors programme.}
  \colplacechunks

  \colchunk{Caractérisation des fonctions dérivables constantes, monotones, strictement monotones sur un intervalle.}
  \colchunk{Résultat admis à ce stade. Les étudiants doivent savoir introduire des fonctions pour établir des inégalités.}
  \colplacechunks

  \colchunk{Tableau de variation.}
  \colchunk{}
  \colplacechunks

  \colchunk{Graphe d'une réciproque.}
  \colchunk{}
  \colplacechunks

  \colchunk{Dérivée d'une réciproque.}
  \colchunk{Interprétation géométrique de la dérivabilité et du calcul de la dérivée d'une bijection réciproque.}
  \colplacechunks

  \colchunk{Dérivées d'ordre supérieur.}
  \colchunk{}
  \colplacechunks
  \end{parcolumns}
  
\subsubsubsection{c) \'Etude d'une fonction}
\begin{parcolumns}[rulebetween,distance=\parcoldist]{2}
  \colchunk{Détermination des symétries et des périodicités afin de réduire le domaine d'étude, tableau de variations, asymptotes verticales et horizontales, tracé du graphe.}
  \colchunk{Application à la recherche d'extrémums et à l'obtention d'inégalités.}
  \colplacechunks
  \end{parcolumns}
  
\subsubsubsection{d) Fonctions usuelles}
\begin{parcolumns}[rulebetween,distance=\parcoldist]{2}
  \colchunk{Fonctions exponentielle, logarithme népérien, puissances.}
  \colchunk{Dérivée, variation et graphe.\newline
  Les fonctions puissances sont définies sur $\R_+^*$ et prolongées en $0$ le cas échéant. Seules les fonctions puissances entières sont en outre défines sur $\R_-^*$.\newline
  $\leftrightarrows$ SI: logarithme décimal pour la représentation des diagrammes de Bode.}
  \colplacechunks

  \colchunk{Relations $(xy)^\alpha=x^\alpha y^\alpha$, $x^{\alpha+\beta}=x^\alpha x^\beta$, $(x^\alpha)^\beta=x^{\alpha \beta}$.\newline
  Croissances comparées des fonctions logarithme, puissances et exponentielle.}
  \colchunk{}
  \colplacechunks

  \colchunk{Fonction sinus, cosinus, tangente.}
  \colchunk{$\leftrightarrows$ PC et SI}
  \colplacechunks

  \colchunk{Fonctions circulaires réciproques.}
  \colchunk{Notation $\arcsin$, $\arccos$, $\arctan$.}
  \colplacechunks

  \colchunk{Fonctions hyperboliques.}
  \colchunk{Notations $\sh$, $\ch$, $\th$. \newline
  Seule relation de trigonométrie hyperbolique exigible: $\ch^2x - \sh^2 x =1$.\newline
  Les fonctions hyperboliques réciproques sont hors programmes.}
  \colplacechunks
\end{parcolumns}

