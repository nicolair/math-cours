\subsubsection{B - Matrices et applications linéaires}
\subsubsubsection{a) Matrice d'une application linéaire dans des bases.}
\begin{parcolumns}[rulebetween,distance=2.5cm]{2}
  \colchunk{Matrice d'une famille de vecteurs dans une base, d'une application linéaire dans un couple de bases.}
  \colchunk{Notation $\Mat _{e,f}(u)$.\\Isomorphisme $u\mapsto \Mat_{e,f}(u)$.}
  \colplacechunks
   \colchunk{Coordonnées de l'image d'un vecteur par une application linéaire.}
  \colchunk{}
  \colplacechunks
   \colchunk{Matrice d'une composée d'applications linéaires. Lien entre matrices inversibles et isomorphismes.}
  \colchunk{Cas particulier des endomorphismes.}
  \colplacechunks
\end{parcolumns}
\subsubsubsection{b) Application linéaire canoniquement associée à une matrice}
\begin{parcolumns}[rulebetween,distance=2.5cm]{2}

   \colchunk{Noyau, image et rang d'une matrice.}
  \colchunk{Les colonnes engendrent l'image, les lignes donnent un système d'équations du noyau. Une matrice carrée est inversible si et seulement si son noyau est réduit au sous-espace nul.}
  \colplacechunks
   \colchunk{Condition d'inversibilité d'une matrice triangulaire. L'inverse d'une matrice triangulaire est une matrice triangulaire.}
  \colchunk{}
  \colplacechunks
\end{parcolumns}

\subsubsubsection{c) Blocs}
\begin{parcolumns}[rulebetween,distance=2.5cm]{2}

   \colchunk{Matrice par blocs.}
  \colchunk{Interprétation géométrique.}
  \colplacechunks
   \colchunk{Théorème du produit par blocs.}
  \colchunk{La démonstration n'est pas exigible.}
  \colplacechunks
\end{parcolumns}
