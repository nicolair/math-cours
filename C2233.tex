%<dscrpt>Fichier de déclarations Latex à inclure au début d'un élément de cours.</dscrpt>

\documentclass[a4paper]{article}
\usepackage[hmargin={1.8cm,1.8cm},vmargin={2.4cm,2.4cm},headheight=13.1pt]{geometry}

%includeheadfoot,scale=1.1,centering,hoffset=-0.5cm,
\usepackage[pdftex]{graphicx,color}
\usepackage[french]{babel}
%\selectlanguage{french}
\addto\captionsfrench{
  \def\contentsname{Plan}
}
\usepackage{fancyhdr}
\usepackage{floatflt}
\usepackage{amsmath}
\usepackage{amssymb}
\usepackage{amsthm}
\usepackage{stmaryrd}
%\usepackage{ucs}
\usepackage[utf8]{inputenc}
%\usepackage[latin1]{inputenc}
\usepackage[T1]{fontenc}


\usepackage{titletoc}
%\contentsmargin{2.55em}
\dottedcontents{section}[2.5em]{}{1.8em}{1pc}
\dottedcontents{subsection}[3.5em]{}{1.2em}{1pc}
\dottedcontents{subsubsection}[5em]{}{1em}{1pc}

\usepackage[pdftex,colorlinks={true},urlcolor={blue},pdfauthor={remy Nicolai},bookmarks={true}]{hyperref}
\usepackage{makeidx}

\usepackage{multicol}
\usepackage{multirow}
\usepackage{wrapfig}
\usepackage{array}
\usepackage{subfig}


%\usepackage{tikz}
%\usetikzlibrary{calc, shapes, backgrounds}
%pour la présentation du pseudo-code
% !!!!!!!!!!!!!!      le package n'est pas présent sur le serveur sous fedora 16 !!!!!!!!!!!!!!!!!!!!!!!!
%\usepackage[french,ruled,vlined]{algorithm2e}

%pr{\'e}sentation du compteur de niveau 2 dans les listes
\makeatletter
\renewcommand{\labelenumii}{\theenumii.}
\renewcommand{\thesection}{\Roman{section}.}
\renewcommand{\thesubsection}{\arabic{subsection}.}
\renewcommand{\thesubsubsection}{\arabic{subsubsection}.}
\makeatother


%dimension des pages, en-t{\^e}te et bas de page
%\pdfpagewidth=20cm
%\pdfpageheight=14cm
%   \setlength{\oddsidemargin}{-2cm}
%   \setlength{\voffset}{-1.5cm}
%   \setlength{\textheight}{12cm}
%   \setlength{\textwidth}{25.2cm}
   \columnsep=1cm
   \columnseprule=0.5pt

%En tete et pied de page
\pagestyle{fancy}
\lhead{MPSI-\'Eléments de cours}
\rhead{\today}
%\rhead{25/11/05}
\lfoot{\tiny{Cette création est mise à disposition selon le Contrat\\ Paternité-Pas d'utilisations commerciale-Partage des Conditions Initiales à l'Identique 2.0 France\\ disponible en ligne http://creativecommons.org/licenses/by-nc-sa/2.0/fr/
} }
\rfoot{\tiny{Rémy Nicolai \jobname}}


\newcommand{\baseurl}{http://back.maquisdoc.net/data/cours\_nicolair/}
\newcommand{\urlexo}{http://back.maquisdoc.net/data/exos_nicolair/}
\newcommand{\urlcours}{https://maquisdoc-math.fra1.digitaloceanspaces.com/}

\newcommand{\N}{\mathbb{N}}
\newcommand{\Z}{\mathbb{Z}}
\newcommand{\C}{\mathbb{C}}
\newcommand{\R}{\mathbb{R}}
\newcommand{\D}{\mathbb{D}}
\newcommand{\K}{\mathbf{K}}
\newcommand{\Q}{\mathbb{Q}}
\newcommand{\F}{\mathbf{F}}
\newcommand{\U}{\mathbb{U}}
\newcommand{\p}{\mathbb{P}}


\newcommand{\card}{\mathop{\mathrm{Card}}}
\newcommand{\Id}{\mathop{\mathrm{Id}}}
\newcommand{\Ker}{\mathop{\mathrm{Ker}}}
\newcommand{\Vect}{\mathop{\mathrm{Vect}}}
\newcommand{\cotg}{\mathop{\mathrm{cotan}}}
\newcommand{\sh}{\mathop{\mathrm{sh}}}
\newcommand{\ch}{\mathop{\mathrm{ch}}}
\newcommand{\argsh}{\mathop{\mathrm{argsh}}}
\newcommand{\argch}{\mathop{\mathrm{argch}}}
\newcommand{\tr}{\mathop{\mathrm{tr}}}
\newcommand{\rg}{\mathop{\mathrm{rg}}}
\newcommand{\rang}{\mathop{\mathrm{rg}}}
\newcommand{\Mat}{\mathop{\mathrm{Mat}}}
\newcommand{\MatB}[2]{\mathop{\mathrm{Mat}}_{\mathcal{#1}}\left( #2\right) }
\newcommand{\MatBB}[3]{\mathop{\mathrm{Mat}}_{\mathcal{#1} \mathcal{#2}}\left( #3\right) }
\renewcommand{\Re}{\mathop{\mathrm{Re}}}
\renewcommand{\Im}{\mathop{\mathrm{Im}}}
\renewcommand{\th}{\mathop{\mathrm{th}}}
\newcommand{\repere}{$(O,\overrightarrow{i},\overrightarrow{j},\overrightarrow{k})$}
\newcommand{\cov}{\mathop{\mathrm{Cov}}}

\newcommand{\absolue}[1]{\left| #1 \right|}
\newcommand{\fonc}[5]{#1 : \begin{cases}#2 \rightarrow #3 \\ #4 \mapsto #5 \end{cases}}
\newcommand{\depar}[2]{\dfrac{\partial #1}{\partial #2}}
\newcommand{\norme}[1]{\left\| #1 \right\|}
\newcommand{\se}{\geq}
\newcommand{\ie}{\leq}
\newcommand{\trans}{\mathstrut^t\!}
\newcommand{\val}{\mathop{\mathrm{val}}}
\newcommand{\grad}{\mathop{\overrightarrow{\mathrm{grad}}}}

\newtheorem*{thm}{Théorème}
\newtheorem{thmn}{Théorème}
\newtheorem*{prop}{Proposition}
\newtheorem{propn}{Proposition}
\newtheorem*{pa}{Présentation axiomatique}
\newtheorem*{propdef}{Proposition - Définition}
\newtheorem*{lem}{Lemme}
\newtheorem{lemn}{Lemme}

\theoremstyle{definition}
\newtheorem*{defi}{Définition}
\newtheorem*{nota}{Notation}
\newtheorem*{exple}{Exemple}
\newtheorem*{exples}{Exemples}


\newenvironment{demo}{\renewcommand{\proofname}{Preuve}\begin{proof}}{\end{proof}}
%\renewcommand{\proofname}{Preuve} doit etre après le begin{document} pour fonctionner

\theoremstyle{remark}
\newtheorem*{rem}{Remarque}
\newtheorem*{rems}{Remarques}

\renewcommand{\indexspace}{}
\renewenvironment{theindex}
  {\section*{Index} %\addcontentsline{toc}{section}{\protect\numberline{0.}{Index}}
   \begin{multicols}{2}
    \begin{itemize}}
  {\end{itemize} \end{multicols}}


%pour annuler les commandes beamer
\renewenvironment{frame}{}{}
\newcommand{\frametitle}[1]{}
\newcommand{\framesubtitle}[1]{}

\newcommand{\debutcours}[2]{
  \chead{#1}
  \begin{center}
     \begin{huge}\textbf{#1}\end{huge}
     \begin{Large}\begin{center}Rédaction incomplète. Version #2\end{center}\end{Large}
  \end{center}
  %\section*{Plan et Index}
  %\begin{frame}  commande beamer
  \tableofcontents
  %\end{frame}   commande beamer
  \printindex
}


\makeindex
\begin{document}
\noindent

\debutcours{Matrices: familles de vecteurs, applications linéaires}{1.6 \tiny{le \today}}

\section{Matrices d'une famille de vecteurs}
\index{matrice d'un vecteur dans une base}
\begin{defi}[Matrice d'un vecteur dans une base]
 Soit $E$ un $\K$-espace vectoriel de dimension $p$ dont $\mathcal E = (e_1,\cdots,e_p)$ est une base. Soit $u\in E$.  La matrice de $u$ dans $\mathcal E$ est la matrice colonne des coordonnées de $u$ dans $\mathcal E$. Elle est notée $\Mat_{\mathcal{E}}(u)$.
\[
 \Mat_{\mathcal E}(u)=\begin{bmatrix}
  \lambda_1\\
  \vdots\\
 \lambda_p
                      \end{bmatrix}
\Leftrightarrow
u = \lambda_1e_1+\cdots+\lambda_pe_p.
\]
\end{defi}

\begin{exples}
 \begin{enumerate}
  \item Soit $\mathcal{C}=(e_1,e_2,e_3)$ la base canonique de $\R^3$ et $x=(x_1,x_2,x_3)$. Alors
\begin{displaymath}
\Mat_{\mathcal{C}}(x)=
\begin{bmatrix}
 x_1 \\ x_2 \\ x_3 
\end{bmatrix} .
\end{displaymath}

\item Polynômes d'interpolation. Soit $(a_0,a_1, \cdots, a_n)$ des éléments distincts de $\K$ et $\mathcal{L}=(L_0,\cdots , L_n)$ la base (de $\K_n[X]$) des \href{https://maquisdoc-math.fra1.digitaloceanspaces.com/C2233_1.pdf}{polynômes d'interpolation de Lagrange} attachée à ces valeurs.
\[
  \forall P \in \K_n[X], \hspace{0.5cm} 
\Mat_{\mathcal{L}}(P) =
\begin{bmatrix}
  \widetilde{P}(a_0)\\ \vdots \\ \widetilde{P}(a_n)
\end{bmatrix}
\text{ car } P = \widetilde{P}(a_0)L_0 + \cdots + \widetilde{P}(a_n)L_n.
\]

\item Les espaces de matrices disposent de la base canonique formée par les matrices élémentaires. La matrice d'une matrice est la colonne des coordonnées dans la base canonique.\newline
Par exemple $\mathcal{E} = \left( E(1,1), E(1,2), E(2,1), E(2,2) \right)$ est une base de $\mathcal{M}_2(\K)$ et 
\begin{displaymath}
  A= \begin{pmatrix}
       a & b \\ c & d
     \end{pmatrix}
     = aE(1,1) + bE(1,2) + cE(2,1) + dE(2,2) 
\Rightarrow
\Mat_\mathcal{E}(A) =
\begin{pmatrix}
  a \\ b \\ c \\ d
\end{pmatrix}.
\end{displaymath}
En particulier dans la base canonique, la matrice d'une matrice colonne est elle même alors que la matrice d'une ligne est sa \href{https://maquisdoc-math.fra1.digitaloceanspaces.com/C2233_2.pdf}{transposée}.
 \end{enumerate}
\end{exples}

\begin{propn}
 Soit $\mathcal{B}$ une base d'un $\K$-espace vectoriel $E$ de dimension $p$. L'application
\begin{displaymath}
 \left\lbrace 
\begin{aligned}
 E &\rightarrow \mathcal{M}_{p,1}(\K)\\
 x&\mapsto \mathop{\mathrm{Mat}}_{\mathcal B}(x)
\end{aligned}
\right. 
\end{displaymath}
est un isomorphisme.
\end{propn}
\begin{demo}
Ceci est justifié par le fait qu'un vecteur est caractérisé par ses coordonnées dans une base donnée et que les fonctions coordonnées dans une base sont linéaires.  
\end{demo}
\begin{rem}
 La linéarité de cette application se traduit par:
\begin{displaymath}
 \mathop{\mathrm{Mat}}_{\mathcal B}(\lambda_1 x_1+\lambda_1 x_1+\cdots \lambda_q x_q)
= \lambda_1 \mathop{\mathrm{Mat}}_{\mathcal B}(x_1) + \lambda_2 \mathop{\mathrm{Mat}}_{\mathcal B}(x_2) +\cdots +
\lambda_q \mathop{\mathrm{Mat}}_{\mathcal B}(x_q).
\end{displaymath}
\end{rem}

\index{matrice d'une famille de vecteurs dans une base}
\begin{defi}[Matrice d'une famille de vecteurs dans une base]
 Soit $E$ un $\K$-espace vectoriel de dimension $p$, $\mathcal B = (e_1,\cdots,e_p)$ une base de $E$ et $\mathcal U=(u_1,\cdots,u_q)$ une famille de vecteurs de $E$.  La matrice de $\mathcal U$ dans $\mathcal B$ est la matrice dont la colonne $j$ (quelconque entre $1$ et $q$) est $\Mat_\mathcal B(u_j)$. 
Elle est notée
\[
 \Mat_\mathcal B(\mathcal U) = \Mat_\mathcal B(u_1,\cdots,u_q)\;
\text{ et vérifie }\;
 \forall j\in\{1,\cdots,q\} : C_j\left( \Mat_\mathcal B(\mathcal U)\right) = \Mat_\mathcal B(u_j).
\]
\end{defi}
\begin{rem}
 Dans les conditions de la définition précédente, $\Mat_\mathcal B(\mathcal U)$ est une matrice à $p$ lignes et $q$ colonnes.
\end{rem}

\begin{propn}
Soit $\mathcal B $ une base d'un $\K$-espace vectoriel $E$, $(u_1,\cdots,u_q)\in E^q$ et $(\lambda_1,\cdots,\lambda_q) \in \K^q$.
 \begin{displaymath}
  \Mat_\mathcal B(u_1,\cdots,u_q)
\begin{bmatrix}
  \lambda_1\\
  \vdots\\
 \lambda_q
\end{bmatrix} 
= \Mat_\mathcal B(\lambda_1u_1 + \cdots + \lambda_qu_q).
 \end{displaymath}
\end{propn}
\begin{demo}
 D'après une des propriétés du produit matriciel, par linéarité,
\begin{multline*}
   \Mat_\mathcal B(u_1,\cdots,u_q)
\begin{bmatrix}
  \lambda_1\\
  \vdots\\
 \lambda_q
\end{bmatrix} 
= C_1(\Mat_\mathcal B(u_1,\cdots,u_q))\lambda_1+C_2(\Mat_\mathcal B(u_1,\cdots,u_q))\lambda_2+\cdots+ C_q(\Mat_\mathcal B(u_1,\cdots,u_q))\lambda_q \\ 
= \lambda_1 \Mat_\mathcal B(u_1) + \lambda_2 \Mat_\mathcal B(u_2) + \cdots + \lambda_q \Mat_\mathcal B(u_q) 
= \Mat_\mathcal B(\lambda_1u_1 + \cdots + \lambda_qu_q).
\end{multline*}
\end{demo}
\index{matrice de passage}
\begin{defi}[Matrice de passage]
Soit $\mathcal B$ et $\mathcal B'$ deux bases d'un $\K$-espace vectoriel $E$ de dimension finie. La matrice de passage de  $\mathcal B$ vers $\mathcal B'$ est la matrice de la famille $\mathcal B'$ dans la base $\mathcal B$. Elle est notée $P_{\mathcal B \mathcal B'}$.
 \begin{displaymath}
  P_{\mathcal B \mathcal B'}= \Mat_\mathcal B (\mathcal B')
 \end{displaymath}
\end{defi}
\begin{exples}
 \begin{enumerate}
  \item Soit $\mathcal{A} = (a_1,a_2,a_3)$ une base d'un $\K$-espace vectoriel $E$ et $\mathcal{B}=(b_1,b_2,b_3)$ une famille de vecteurs définie par 
\[
 \left\lbrace 
 \begin{aligned}
  b_1 &= a_1 + a_2 \\
  b_2 &= a_1 + a_2 + a_3 \\
  b_3 &= a_1 -a_2 + a_3
 \end{aligned}
\right. \Rightarrow
\Mat_{\mathcal{A}}(\mathcal{B})
= \begin{pmatrix}
 1 & 1 & 1 \\
 1 & 1 & -1 \\
 0 & 1 & 1
  \end{pmatrix}
.
\]
Ici, les $a$ s'expriment en fonction des $b$, la famille $\mathcal{B}$ est aussi une base (car génératrice à 3 vecteurs) et on peut former les deux matrices de passage.
\[
\left\lbrace 
\begin{aligned}
 a_1 &= b_1 -\frac{1}{2} b_2 + \frac{1}{2}b_3\\
 a_2 &= \frac{1}{2}b_2 - \frac{1}{2}b_3\\
 a_3 &= b_2 - b_1
\end{aligned}
\right. \Rightarrow
P_{\mathcal{B} \mathcal{A}}=
\begin{pmatrix}
 1 & 0 & -1 \\
 -\frac{1}{2} & \frac{1}{2} &1 \\
 \frac{1}{2} & -\frac{1}{2} & 0
\end{pmatrix}, \hspace{0.5cm}
P_{\mathcal{B} \mathcal{A}} = \Mat_{\mathcal{A}}(\mathcal{B})
= \begin{pmatrix}
 1 & 1 & 1 \\
 1 & 1 & -1 \\
 0 & 1 & 1
  \end{pmatrix}
.
\]
On remarque que le produit des deux matrices de passage est $I_3$ (voir proposition \ref{prop: passinv}).
  \item Par définition : $P_{\mathcal B \mathcal B}= \Mat_\mathcal B (\mathcal B)=I_p$.
 \end{enumerate}
\end{exples}

\index{formule du changement de base}
\begin{propn}[Formules de changement de base]
 Soit $\mathcal B$ et $\mathcal B'$ deux bases d'un $\K$-espace vectoriel $E$ de dimension finie, soit $u$ un vecteur de $E$ et $(u_1,\cdots,u_q)$ une famille de $E$:
\[
 \Mat_{\mathcal B}(u) = P_{\mathcal B \mathcal B'}\Mat_{\mathcal B'}(u), \hspace{0.5cm}
\Mat_{\mathcal B}(u_1,\cdots,u_q) = P_{\mathcal B \mathcal B'}\Mat_{\mathcal B'}(u_1,\cdots,u_q).
\]
\end{propn}
\begin{demo}
 Notons $p$ la dimension de $E$ avec $\mathcal{B}'=(e'_1,\cdots,e'_p)$ et 
\begin{displaymath}
 \begin{pmatrix} \lambda_1 \\ \vdots \\ \lambda_p \end{pmatrix}= \Mat_{\mathcal B '}(u)
\end{displaymath}
les coordonnées de $u$. On peut alors écrire
\begin{displaymath}
 P_{\mathcal B \mathcal B'}\Mat_{\mathcal B'}(u)
= \Mat_{\mathcal B}(e'_1,\cdots,e'_q)\begin{pmatrix} \lambda_1 \\ \vdots \\ \lambda_p \end{pmatrix}
= \Mat_{\mathcal B}(\lambda_1 e'_1 + \cdots + \lambda_q e'_q)
= \Mat_{\mathcal B}(u).
\end{displaymath}
Pour une famille de vecteurs, il suffit d'appliquer le premier résultat colonne par colonne.
\end{demo}

\begin{propn} \label{prop: passinv}
Soit $\mathcal B$ et $\mathcal B'$ deux bases d'un $\K$-espace vectoriel $E$ de dimension finie. La matrice de passage $P_{\mathcal B \mathcal B'}$ est inversible d'inverse $P_{\mathcal B' \mathcal B}$
\end{propn}
\begin{demo}
 On utilise la formule de changement de base en prenant $\mathcal{B}$ dans le rôle de la famille $(u_1,u_2,\cdots,u_p)$. On en déduit
\begin{displaymath}
 P_{\mathcal B \mathcal B'}P_{\mathcal B' \mathcal B}=P_{\mathcal B \mathcal B'}\Mat_{\mathcal B'}(\mathcal B)= \Mat_{\mathcal B}(\mathcal B) = I_p.
\end{displaymath}
On obtient l'autre produit en échangeant les rôles de $\mathcal{B}$ et $\mathcal{B}'$.
\end{demo}
\begin{rem}
 Si on connait les coordonnées d'un vecteur $u$ dans une base $\mathcal B$ et les coordonnées des vecteurs de $\mathcal B$ dans une base $\mathcal B'$ (c'est à dire $P_{\mathcal B' \mathcal B}$), il est facile \href{https://maquisdoc-math.fra1.digitaloceanspaces.com/C2233_3.pdf}{(en remplaçant, développant, regroupant)} d'obtenir les coordonnées de $u$ dans $\mathcal B'$. Cela se traduit par la relation matricielle
\begin{displaymath}
 \Mat_{\mathcal B'} (u) = P_{\mathcal B' \mathcal B}\Mat_\mathcal B (u).
\end{displaymath}
En revanche si on connait seulement la matrice de passage $P_{\mathcal B \mathcal B'}$, un calcul supplémentaire est nécessaire.
\end{rem}
\begin{rems}
On peut traduire matriciellement le caractère libre ou lié d'une famille de vecteurs. Soit $E$ un $\K$-espace vectoriel de dimension $p$, soit $\mathcal B$ une base de $E$.
 \begin{enumerate}
  \item Soit $u_1,\cdots,u_p$ une famille de $p$ vecteurs de $E$.
\begin{displaymath}
(u_1,\cdots,u_p)\text{ base de } E \Leftrightarrow (u_1,\cdots,u_p)\text{ libre }
\Leftrightarrow \Mat_\mathcal B(u_1,\cdots,u_p) \text{ inversible}.
\end{displaymath}
\item Soit $u_1,\cdots,u_q$ une famille de $q$ vecteurs de $E$.
\begin{displaymath}
(u_1,\cdots,u_q)\text{ liée }
\Leftrightarrow \exists X\in \mathcal M_{q,1}, 
X\neq
\begin{bmatrix}
  0\\
  \vdots\\
 0
\end{bmatrix} 
\text{ tel que }
\Mat_\mathcal B(u_1,\cdots,u_q)X=
\begin{bmatrix}
  0\\
  \vdots\\
 0
\end{bmatrix} .
\end{displaymath}
\end{enumerate}
\end{rems}

\section{Matrices d'une application linéaire}
\subsection{Définition. Image d'un vecteur.}
\index{matrice d'une application linéaire dans des bases}
\begin{defi}[matrice d'une application linéaire dans des bases]
 Soit $E$ et $F$ deux $\K$-espaces vectoriels, soit $\mathcal U=(u_1,\cdots,u_q)$ une base de $E$ et $\mathcal V=(v_1,\cdots,v_p)$ une base de $F$, soit $f$ une application linéaire de $E$ dans $F$. La matrice de $f$ dans les bases $\mathcal U$ et $\mathcal V$ est la matrice à $p$ lignes et $q$ colonnes définie par
\begin{displaymath}
 \Mat_{\mathcal U \mathcal V}f = \Mat_{\mathcal V}(f(\mathcal U))= \Mat_{\mathcal V}(f(u_1),\cdots,f(u_q))
\end{displaymath}
Dans le cas d'un endomorphisme $f\in\mathcal L(E)$, lorsque la base est la même au départ et à l'arrivée, on note
\begin{displaymath}
 \Mat_{\mathcal U}f =\Mat_{\mathcal U \mathcal U}f
\end{displaymath}
\end{defi}
\begin{rem}
 Soit $\mathcal U$ et $\mathcal V$ des bases d'un $\K$-espace vectoriel $E$
\[
 \Mat_{\mathcal U} Id_E = I_p , \hspace{1cm}
 \Mat_{\mathcal V \mathcal U}Id_E = P_{\mathcal U \mathcal V} \; \text{ (matrice de passage).}
\]
\end{rem}

\begin{propn} \label{matimagevect}
 Soit $E$ et $F$ deux $\K$-espaces vectoriels, soit $\mathcal U=(u_1,\cdots,u_q)$ une base de $E$ et $\mathcal V=(v_1,\cdots,v_p)$ une base de $F$, soit $f$ une application linéaire de $E$ dans $F$ et $u$ un vecteur de $E$:
\begin{displaymath}
 \Mat_{\mathcal V}(f(u)) = \Mat_{\mathcal U \mathcal V}f \, \Mat_{\mathcal U}u
\end{displaymath}
\end{propn}
\begin{demo}
Notons $A = \Mat_{\mathcal U \mathcal V}f$ et $X = \Mat_{\mathcal U}u$.
\[
 f(x) = f(x_1u_1 + \cdots + x_q u_q) = \sum_{j=1}^q x_jf(u_j)
 = \sum_{j=1}^q x_j \left( \sum_{i=1}^p a_{i j}v_i\right) 
 = \sum_{i=1}^p \left( \sum_{j=1}^q a_{i j}x_j\right)v_i
\]
où $\sum_{i=1}^p a_{i j}x_j$ est à la fois la ligne $i$ de la matrice colonne $AX$ et la coordonnée de $f(x)$ relative à $v_i$.
\end{demo}

\begin{propn} \label{prlgtlin}
 Soit $E$ et $F$ deux $\K$-espaces vectoriels, soit $\mathcal U=(u_1,\cdots,u_q)$ une base de $E$ et $\mathcal V=(v_1,\cdots,v_p)$ une base de $F$. L'application \og$\Mat_{\mathcal U \mathcal V}$\fg~ de $\mathcal L (E,F)$ dans $\mathcal M _{pq}(\K)$ qui, à une application linéaire, associe sa matrice dans les bases considérées est un isomorphisme de $\K$-espace vectoriel.
\end{propn}
\begin{demo}
 Il s'agit en fait d'une reformulation du théorème de prolongement linéaire \href{https://maquisdoc-math.fra1.digitaloceanspaces.com/C2233_5.pdf}{(détail manuscrit)}.\index{théorème de prolongement linéaire}
\end{demo}

\begin{propn} \label{matcomp}
Soit $E$, $F$, $G$ trois $\K$-espaces vectoriels, soit $\mathcal U$ une base de $E$ et $\mathcal V$ une base de $F$ et $\mathcal W$ une base de $G$, soit $f\in\mathcal L(E,F)$ et $g\in\mathcal L(F,G)$:
\begin{displaymath}
 \Mat_{\mathcal U \mathcal W}g\circ f = \Mat_{\mathcal V \mathcal W}g\,\Mat_{\mathcal U \mathcal V}f
\end{displaymath}
\end{propn}
\begin{demo}
On note $A = \Mat_{\mathcal V \mathcal W}g \in \mathcal{M}_{p q}(\K)$ et $B = \Mat_{\mathcal U \mathcal v}f \in \mathcal{M}_{q r}(\K)$ c'est à dire $\dim(E) = r$, $\dim(F) = q$ et $\dim(G) =p$. Alors, d'après la proposition \ref{matimagevect},
\[
 \forall k \in \llbracket 1,r \rrbracket,\;  C_k(AB) = AC_k(B) = \Mat_{\mathcal V \mathcal W}g\, \Mat_{\mathcal{V}}(f(u_k))
 = \Mat_{\mathcal{W}}(g\circ f (u_k))
 = C_k(\Mat_{\mathcal U \mathcal W}g \circ f).
\]
Ce qui prouve le résultat (\href{https://maquisdoc-math.fra1.digitaloceanspaces.com/C2233_4.pdf}{détail manuscrit}).
\end{demo}

\begin{propn}
 Soit $\mathcal U$ une base d'un $\K$-espace vectoriel $E$ de dimension $p$. L'application de $\mathcal L(E)$ dans $\mathcal M_p(\K)$ qui à tout $f$ associe $\Mat_{\mathcal U}f$ est un isomorphisme d'algèbre.
\end{propn}
\begin{demo}
Cette application est bijective (proposition \ref{prlgtlin}). Elle transporte les opérations (proposition \ref{matcomp}) et l'élément unité (neutre multiplicatif):
\begin{multline*}
\forall (f,g) \in \mathcal{L}(E)^2, \forall \lambda \in \K, \hspace{1cm}
 \Mat_{\mathcal U}(f+g) =  \Mat_{\mathcal U}(f) +  \Mat_{\mathcal U}(g),  \; \Mat_{\mathcal U}(\lambda f) =  \lambda \Mat_{\mathcal U}(f), \\
 \Mat_{\mathcal U}(f\circ g) =  \Mat_{\mathcal U}(f)  \Mat_{\mathcal U}(g), \; \Mat_{\mathcal U}(\Id_E) = I_p.
\end{multline*}
\end{demo}

\subsection{Application linéaire canoniquement associée à une matrice}
\index{application linéaire canoniquement associée à une matrice}
\index{noyau d'une matrice} \index{image d'une matrice} \index{rang d'une matrice}
\begin{defi}
  Soit $M\in \mathcal{M}_{p,q}(\K)$. L'application linéaire canoniquement associée à $M$ est la multiplication (notée $\mu_M$ ou $\gamma_M$) d'une colonne par $M$ soit:
\begin{displaymath}
\mu_M:\;  \left\lbrace 
  \begin{aligned}
    \mathcal{M}_{q,1}(\K) \rightarrow& \mathcal{M}_{p,1}(\K) \\
    X \mapsto& MX
  \end{aligned}
  \right. 
\end{displaymath}
Le noyau de $M$ est le noyau de $\mu_M$, l'image de $M$ est l'image de $\mu_M$, le rang de $M$ est le rang de $\mu_M$.
\end{defi}
\begin{rems}
  \begin{itemize}
    \item Les colonnes de $M$ engendrent son image.
    \item Les lignes de $M$ forment un système d'équations de son noyau.
    \item $\forall (M,M') \in \mathcal{M}_{p}(\K)^2$, $\mu_M \circ \mu_{M'} = \mu_{MM'}$. On en déduit la proposition suivante 
  \end{itemize}
\end{rems}
\begin{propn}\label{carctinvers}
  Une matrice carrée $M$ est inversible si et seulement si $\mu_M$ est bijective si et seulement si son noyau est réduit au sous espace nul ou si ses colonnes forment une base de l'espace des matrices colonnes.
\end{propn}

\begin{propn}
  Une matrice triangulaire est inversible si et seulement si ses termes diagonaux sont tous non nuls.
\end{propn}
\begin{demo}
Soit $A$ triangulaire supérieure avec les termes diagonaux $a_{i i}$ tous non nuls. Montrons que la famille des colonnes est libre. Si
\[
\lambda_1 
 \begin{pmatrix}
  a_{1 1} \\ 0 \\ 0 \\ \vdots \\ 0
 \end{pmatrix}
+ \lambda_2 
\begin{pmatrix}
  a_{1 2} \\ a_{2 2} \\ 0 \\ \vdots \\ 0
 \end{pmatrix}
 \cdots 
+ \lambda_p 
\begin{pmatrix}
  a_{1 p} \\ a_{2 p} \\ \vdots \\ 0 \\ a_{p p}
 \end{pmatrix} 
 =
\begin{pmatrix}
  0 \\ 0 \\ \vdots \\ 0 \\ 0
 \end{pmatrix} 
\text{ alors }
\left.
\begin{aligned}
  \lambda_p a_{p p} &= 0 \\ a_{p p} &\neq 0
\end{aligned}
\right\rbrace \Rightarrow \lambda_p = 0
\]
en considérant seulement la dernière ligne de cette relation. On considère ensuite la ligne $p-1$ et on remonte....\newline
D'après la proposition \ref{carctinvers}, la matrice $A$ est inversible car la famille de ses colonnes est libre.\newline
Si $A$ est triangulaire inférieure, on raisonne de la même manière mais en descendant en considérant d'abord la première ligne de la combinaison puis la deuxième ...\newline
Supposons qu'il existe $i$ tel que $a_{i i} = 0$. Alors les $i$ premières colonnes de $A$ sont dans $\Vect(X_1,\cdots, X_{i-1})$ où les $X_k$ forment la base canonique de l'espace des colonnes. D'après la condition suffisante de dépendance, une famille de $i$ vecteurs dans un espace de dimension $i-1$ est toujours liée donc $A$ n'est pas inversible. Ceci montre que $A$ inversible entraine que tous ses termes diagonaux sont non nuls.
\end{demo}

\begin{propn}
  L'inverse d'une matrice inversible triangulaire supérieure (resp inférieure) est triangulaire supérieure (resp inférieure).
\end{propn}
\begin{demo}
  Soit $A$ triangulaire supérieure inversible $p\times p$. Notons $Y_1, \cdots Y_p$ ses colonnes. Elles s'expriment en fonctions des $(X_1, \cdots X_p)$ de la base canonique des colonnes.
\[
  \left\lbrace
  \begin{aligned}
    \underset{\neq 0}{a_{1 1}} X_1&              &                     & &= Y_1 \\
    a_{1 2} X_1& + \underset{\neq 0 }{a_{2 2}}X_2&                     & &= Y_2 \\
               &             &                                         & &\vdots \\
    a_{1 2} X_1& + a_{2 2}X_2& + \cdots + \underset{\neq 0}{a_{p p}}X_p& &= Y_p
  \end{aligned}
  \right.
\]
On peut exprimer les $Y$ avec les $X$ car les $a_{i i}$ sont non nuls. Il existe des $b_{i j}$ formant une matrice $B$ tels que :
\[
  \left\lbrace
  \begin{aligned}
    X_1 &= b_{1 1} Y_1\\
    X_2 &= b_{1 2} Y_1 + b_{2 2}Y_2 \\
        &\vdots \\
    X_p &= b_{1 p} Y_1& + b_{2 p}Y_2& + \cdots + b_{p p}Y_p
  \end{aligned}
  \right.
  \Leftrightarrow
  I_p = B A
  \text{ avec }
  B =
  \begin{pmatrix}
    b_{1 1} & b_{1 2} & \cdots &b_{1 p} \\
    0       & b_{2 2} & \cdots &b_{2 p} \\
            &  0      & \ddots & \vdots \\
    0       & 0       &        &b_{p p} \\
  \end{pmatrix}
\]
qui est triangulaire supérieure. Pour les matrices triangulaires inférieures, on peut raisonner de manière analogue ou utiliser la transposition pour se ramener au premier cas.
\end{demo}


\subsection{Des bases pour une matrice simple}
\begin{propn} \label{prop: equivJr}
 Soit $E$ de dimension $q$ et $F$ de dimension $p$ deux $\K$-espaces vectoriels. Soit $f\in\mathcal L(E,F)$ une application linéaire de rang $r$. Alors $r\leq\min(p,q)$ et il existe des bases $\mathcal U$ de $E$ et $\mathcal V$ de $F$ telles que :
\begin{displaymath}
 \Mat_{\mathcal U \mathcal V}f = 
\begin{bmatrix}
1      & 0      &   &   &        & 0 \\
0      & \ddots & 0 &   &        & \\
       & 0      & 1 & 0 &        & 0 \\
\vdots & \vdots & 0 & 0 &        & \vdots \\
       &        &   &   & \ddots & \\
0      & \cdots & 0 &   & \cdots & 0
\end{bmatrix}\;
\text{ (notée $J_r(p,q)$) }.
\end{displaymath}
\end{propn}
\begin{demo}
 Le rang $r$ de $f$ est la dimension de $f(E)$ qui est un sous-espace vectoriel de $F$ donc $r\leq \dim F = p$. 
 D'autre part, d'après le théorème du rang, $\dim E=\dim(\ker f) + \rg f$ donc $\rg f\leq \dim E = q$.\newline
Considérons une base $(v_1,\cdots,v_r)$ de $f(E)$.\newline
Chacun de ses vecteurs étant une image, il existe $(u_1,\cdots,u_r) \in E^r$ tels que $f(u_i)=v_i$ pour $i \in \llbracket 1, r\rrbracket$. D'après le théorème du rang, $\dim(\ker f) = q-r$, considérons une base $(u_{r+1},\cdots,u_q)$ de $\ker f$ et formons la famille
\begin{displaymath}
 \mathcal U = (u_1,\cdots ,u_r,u_{r+1},\cdots,u_q).
\end{displaymath}
Montrons que c'est une base de $E$.\newline
Comme elle est formée de $q=\dim E$ vecteurs, il suffit de prouver qu'elle est libre.
\[
\forall (\lambda_1,\cdots,\lambda_q) \in \R^q,\hspace{0.5cm} \text{ si } \lambda_1 u_1 +\cdots +\lambda_q u_q = 0_E,                                                                                                                                                    
\]
en composant par $f$, on élimine les vecteurs du noyau d'où
\begin{displaymath}
\lambda_1 f(u_1) + \cdots +\lambda_r f(u_r) = \lambda_1 v_1 +\cdots +\lambda_r v_r = 0_F   .                                                                                                                                                          
\end{displaymath}
La famille $(v_1,\cdots,v_r)$ est libre car c'est une base de $f(E)$ donc $\lambda_i = 0$ pour $i \in \llbracket 1, r\rrbracket$.\newline
Pour les autres $\lambda_j$, on exploite le fait que $(u_{r+1},\cdots,u_q)$ est libre car c'est une base du noyau.\newline
La famille $\mathcal{U}$ est bien libre, c'est une base de $E$.\newline
On complète la famille libre $(v_1,\cdots,v_r)$ en une base $\mathcal{V} = (v_1,\cdots,v_p)$.\newline
On aura alors $f(u_i)=v_i$ pour $i$ entre $1$ et $r$ et $f(u_i)=0_F$ pour les autres $i$. La matrice de $f$ dans les bases $\mathcal{U}$ et $\mathcal{V}$ est bien celle annoncée.
\end{demo}

\subsection{Changements de bases}
\index{formule du changement de base}
\begin{propn}[formule du changement de base] \label{chgtbase}
 Soit $E$ et $F$ deux $\K$-espaces vectoriels et $f\in\mathcal L(E,F)$, soit $\mathcal U$ et $\mathcal U'$ deux bases de $E$, soit $\mathcal V$ et $\mathcal V'$ deux bases de $F$: 
\begin{displaymath}
 \Mat_{\mathcal U' \mathcal V'}f 
= P_{\mathcal V' \mathcal V} \, \Mat_{\mathcal U \mathcal V}f \, P_{\mathcal U \mathcal U'}
= P_{\mathcal V \mathcal V'}^{-1} \, \Mat_{\mathcal U \mathcal V}f \, P_{\mathcal U \mathcal U'}
\end{displaymath}
\end{propn}
\begin{demo}
On utilise l'interprétation d'une matrice de passage comme la matrice d'une identité et la proposition \ref{matcomp} sur la matrice d'une composée d'applications linéaires.
\begin{displaymath}
\left\lbrace 
\begin{aligned}
P_{\mathcal V' \mathcal V} &= \Mat_{\mathcal V \mathcal V'}\Id_F\\ 
P_{\mathcal U \mathcal U'} &= \Mat_{\mathcal U' \mathcal U}\Id_E
\end{aligned}
\right. \Rightarrow 
P_{\mathcal V' \mathcal V} \, \Mat_{\mathcal U \mathcal V}f \, P_{\mathcal U \mathcal U'}
= \Mat_{\mathcal V \mathcal V'}\Id_F \, \Mat_{\mathcal U \mathcal V}f \, \Mat_{\mathcal U' \mathcal U}\Id_E = \Mat_{\mathcal U' \mathcal V'}f.
\end{displaymath}
\end{demo}

\begin{propn}[formule du changement de base pour un endomorphisme]
 Soit $E$ un $\K$-espace vectoriel, soit $\mathcal U$, $\mathcal{U}'$ deux bases de $E$, soit $f\in\mathcal L(E)$. La formule du changement de base pour les matrices de $f$ obtenues avec la même base au départ et à l'arrivée s'écrit:
\begin{displaymath}
 \Mat_{\mathcal{U}'} f = P_{\mathcal{U}' \mathcal U} \Mat_{\mathcal U}( f) P_{\mathcal U \mathcal{U}'}
= P_{\mathcal U \mathcal{U}'}^{-1}\Mat_{\mathcal U}( f) P_{\mathcal U \mathcal{U}'}.
\end{displaymath}
\end{propn}
\begin{demo}
 On applique la formule précédente avec $\mathcal{U} = \mathcal{V}$ et $\mathcal{U}' = \mathcal{V}'$.
\[
 P_{\mathcal{V}\mathcal{V}'} = P_{\mathcal{U}\mathcal{U}'} 
 \Rightarrow 
 \Mat_{\mathcal{U}'}(f) = P_{\mathcal U \mathcal{U}'}^{-1}\Mat_{\mathcal U}( f) P_{\mathcal U \mathcal{U}'}.
\]
\end{demo}


\section{Matrices d'une famille de formes linéaires}
\index{matrice d'une forme linéaire}
\begin{defi}
 Soit $\mathcal U =(u_1,\cdots,u_p)$ une base d'un $\K$-espace vectoriel $E$ et $\varphi\in E^*$ une forme linéaire sur $E$. La matrice de $\varphi$ dans la base $\mathcal U$ est définie par
\begin{displaymath}
 \Mat_{\mathcal U}\varphi = \Mat_{\mathcal U (1)}\varphi
=
\begin{bmatrix}
 \varphi(u_1) & \varphi(u_2) & \cdots & \varphi(u_p)
\end{bmatrix}
\end{displaymath}
où $(1)$ désigne la base canonique de $\K$ considéré comme $\K$-espace vectoriel.
\end{defi}
\begin{rem}
 Attention à ne pas confondre la matrice ligne $\Mat_{\mathcal U}\varphi$ avec la matrice colonne $\Mat_{\mathcal U^*}\varphi$ où $\mathcal U^*$ désigne la base de $E^*$.
\begin{align*}
 \Mat_{\mathcal U}\varphi
=
\begin{bmatrix}
 \varphi(u_1) & \varphi(u_2) & \cdots & \varphi(u_p)
\end{bmatrix}
& &
 \Mat_{\mathcal U^*}\varphi
=
\begin{bmatrix}
 \varphi(u_1) \\ \varphi(u_2) \\ \vdots \\ \varphi(u_p)
\end{bmatrix}
\end{align*}
\end{rem}

\section{Matrices équivalentes et rang}
\begin{defi}[\'Equivalence entre deux matrices]
Soit $A$ et $B$ deux matrices de même taille ($p$ lignes et $q$ colonnes). On dit que $A$ et $B$ sont équivalentes si et seulement si 
\[
 \exists P\in \text{GL}_p(\K),\;\exists Q\in \text{GL}_q(\K)\;\text{ tq } B = P\, A\, Q.
\]
\end{defi}
\index{matrices équivalentes}
\begin{rem}
  D'après la proposition \ref{chgtbase} (formule de changement de base), deux matrices sont équivalentes si et seulement si elles représentent la \emph{même} application linéaire dans des bases différentes.
\end{rem}
Il n'existe pas de notation universellement reconue pour cette notion. Dans ce chapitre, on notera $A \sim B$ lorsque les matrices $A$ et $B$ sont équivalentes.
\begin{propn}
La relation d'équivalence est une relation d'équivalence. 
\end{propn}
\begin{demo}
On doit vérifier les propriétés caractéristiques d'une relation d'équivalence. Soit $A$, $B$, $C$ dans $\mathcal{M}_{p q}(\K)$.
\begin{itemize}
 \item Réflexivité. $A = I_p\, A\, I_q$ entraine $A \sim A$.
 \item Symétrie. $B \sim A$ entraine qu'il existe $P \in \text{GL}_p(\K)$ et $Q \in \text{GL}_q(\K)$ tels que $B = P\, A \, Q$ donc $A = P^{-1}B\, Q^{-1}$ donc $A \sim B$. 
 \item Transitivité. Si $A\sim B$ et $B \sim C$, il existe $P, P' \in \text{GL}_p(\K)$ et $Q,Q' \in \text{GL}_q(\K)$ tels que
\[
 \left. 
 \begin{aligned}
  B &= P A Q \\ C &= P' B Q'
 \end{aligned}
\right\rbrace \Rightarrow C = (P'P) A (QQ') \Rightarrow C \sim A.
\]
\end{itemize}
\end{demo}

\begin{defi}
Le rang (des colonnes) d'une matrice est le rang de la famille de ses colonnes.  
\end{defi}
\begin{rem}
 On a vu dans le chapitre sur les matrices pour elles mêmes qu'une matrice carrée $p \times p$ est inversible si et seulement si ses colonnes engendrent $\mathcal{M}_{p,1}(\K)$ c'est à dire si et seulement si son rang est $p$.
\end{rem}

\begin{propn}
Si une matrice représente une famille de vecteurs dans une base, son rang est égal au rang de la famille de vecteurs.\newline
Si une matrice représente une application linéaire dans des bases, son rang est égal au rang de l'application linéaire. 
\end{propn}
\begin{demo}
Soit $A\in \mathcal{M}_{p q}(\K)$ représentant une famille de vecteurs dans une base. $A = \Mat_{\mathcal{E}}(u_1,\cdots, u_q)$ pour $\mathcal{E}$ une base d'un $\K$-espace vectoriel $E$ de dimension $p$ et $(u_1, \cdots, u_q)$ une famille de vecteurs de $E$.  
\begin{multline*}
 \rg(A) = \rg\left( C_1(A), \cdots , C_q(A)\right) \hspace{0.5cm} \text{ (rang (des colonnes) d'une matrice)} \\
  = \rg\left( \MatB{E}{u_1}, \cdots , \MatB{E}{u_q}\right) \hspace{0.5cm} \text{ (matrice dans une base d'une famille de vecteurs)} \\
  = \rg(u_1, \cdots, u_q) \hspace{0.5cm} \text{ car $\MatB{E}{.}$ isomorphisme de } E \text{ dans } \mathcal{M}_{p,1}(\K).
\end{multline*}
Soit $A\in \mathcal{M}_{p q}(\K)$ représentant une application linéaire dans des bases: $A= \MatBB{U}{V}{f}$ avec $\mathcal{U} = (u_1, \cdots, u_q)$ base de $E$ et $\mathcal{V} = (v_1, \cdots, v_p)$ base de $F$.
\begin{multline*}
 A = \MatB{U}{f(u_1), \cdots, f(u_q) } \hspace{0.5cm} \text{ (matrice d'une application linéaire dans des bases)}\\
 \Rightarrow \rg(A) = \rg(f(u_1), \cdots, f(u_q))\text{ (première partie de la proposition)}\\
  = \dim \left( \Vect(f(u_1), \cdots, f(u_q))\right) = \rg(f) \text{ (rang d'une application linéaire)}.
\end{multline*}
\end{demo}

\begin{propn}
  Deux matrices équivalentes sont de même rang.
\end{propn}
\begin{demo}
  En effet elles représentent une même application linéaire dans des bases différentes.
\end{demo}
\begin{rem}
  On ne change pas le rang d'une matrice en la multipliant d'un côté ou de l'autre par une matrice inversible. En effet la matrice obtenue ainsi est équivalente à la première.
\end{rem}

\begin{propn}
  Soit $M \in \mathcal{M}{p,q}(\K)$ et $r\in \llbracket 0, \min(p,q)\rrbracket$.
\[
  \rg(M) = r \Leftrightarrow M \sim J_r(p,q).
\]
  Deux matrices $p\times q$ sont équivalentes si et seulement si elles ont le même rang.
\end{propn}
\begin{demo}
 La première propriété est une reformulation de la proposition \ref{prop: equivJr} proposant un jeu de \og bonnes bases\fg~ pour une application linéaire en fonction de son rang.\newline
 On a déjà vu que deux matrices équivalentes ont le même rang. Réciproquement, si deux matrices ont le même rang $r$, elles sont toutes les deux équivalentes à $J_r(p,q)$. Elles sont équivalentes par transivité et symétrie de la relation.
\end{demo}

\begin{propn}
  Le rang d'une matrice est égal à celui de sa transposée.
\end{propn}
\begin{demo}
Soit $M \in \mathcal{M}{p,q}(\K)$ de rang $r$. D'après la proposition précédente,
\begin{displaymath}
  \exists P \in \text{GL}_p(\K), \; \exists Q \in \text{GL}_q(\K)\;\text{ telles que } M = P J_r(p,q) Q
  \Rightarrow \trans{M} = \trans{Q}\, \trans{J_r(p,q)}\, \trans{P}
\end{displaymath}
Comme $\trans{P}$ et $\trans{Q}$ sont inversibles, le rang de $\trans{M}$ est celui de $\trans{J_r(p,q)} = J_r(q,p)$ c'est à dire $r = \rg(M)$.
\end{demo}
\begin{rem}
  Une matrice et sa transposée ne sont pas de même taille.
\end{rem}
\begin{propn}
  Le rang des lignes est égal au rang des colonnes.
\end{propn}
\begin{demo}
 Soit $A \in \mathcal{M}_{p, q}(\K)$. Le rang des lignes de $A$ est le rang de la famille $\left( L_1(A),\cdots, L_p(A)\right)$ de $p$ matrices lignes. La transposition définit un isomorphisme entre les matrices lignes et les matrices colonnes donc elle conserve le rang:
\[
 \rg\left( L_1(A),\cdots, L_p(A)\right) = \rg \left( \trans{L_1(A)},\cdots, \trans{L_p(A)}\right)
 = \rg\left( C_1(\trans{A}),\cdots, C_p(\trans{A})\right)
 = \rg(\trans{A})
\]
par définition du rang (des colonnes) de $\trans{A}$. On conclut avec $\rg(\trans{A}) = \rg(A)$.
\end{demo}
\noindent La notion de matrice extraite \index{matrices extraites} a été définie dans le chapitre sur les matrices pour elles mêmes.
\begin{propn}
 Soit $A \in \mathcal{M}_{pq}(\K)$ et $E$ extraite de $A$. Alors $\rg(E) \leq \rg(A)$.
\end{propn}
\begin{demo}
Soit $I\subset\llbracket 1,p \rrbracket$ et $J \subset \llbracket 1,q \rrbracket$ tels que $E = A_{I J}$.\newline
En utilisant le rang des colonnes: 
\[
 \rg(A) = \rg(C_1(A), \cdots, C_q(A)) \geq \rg(C_j(A), j\in J) = \rg(A_{\llbracket 1, p\rrbracket J}).
\]
En utilisant le rang des lignes et en notant $B = A_{\llbracket 1, p\rrbracket J}$
\[
\rg(A_{\llbracket 1, p\rrbracket J}) = \rg(B) = \rg\left( L_1(B), \cdots, L_p\right)
 \geq \rg\left( L_i(B), i \in I\right) = \rg(E). 
\]
\end{demo}

\begin{propn}[Rang et matrices carrées extraites] \label{prop: rgmatext}
  Soit $A$ une matrice non nulle et $\mathcal{I}$ l'ensemble des $s$ tels qu'il existe une matrice carrée inversible $s\times s$ extraite de $A$. Le rang de $A$ est alors le plus grand élément de $\mathcal{I}$.
\end{propn}
\begin{demo}
Le rang d'une matrice nulle est nul, on ne considère que des matrices non nulles. De toute matrice non nulle, on peut extraire des matrices carrées inversibles: au moins les matricettes $1\times 1$ associées aux coefficients non nuls. On en déduit que $1 \in \mathcal{I}$.\newline
Pour tout $i\in \mathcal{I}$, $i$ est la taille d'une matrice carrée extraite inversible donc $i$ est le rang d'une matrice extraite donc $i \leq \rg(A)$ d'après la proposition précédente.\newline
Il reste à montrer que si $r = \rg(A)$, on peut extraire de $A$ une matrice $r\times r$ inversible.\newline
Utilisons d'abord le rang des colonnes. Soit $J = \left\lbrace j_1, \cdots, j_r\right\rbrace $ tel que 
\[
 \Vect\left( C_1(A), \cdots, C_q(A)\right) = \Vect\left( C_{j_1}(A), \cdots, C_{j_r}(A)\right)
\]
Considérons la matrice $p\times r$ extraite de $A$ en conservant toutes les lignes et les colonnes associées à $J$ seulement, son rang est $r$. Le rang de ses lignes est aussi $r$ donc on peut en extraire $r$ lignes formant une famille libre. La matrice extraite $A_{I J}$ est de rang $r$ donc inversible. \href{https://maquisdoc-math.fra1.digitaloceanspaces.com/C2233_8.pdf}{(détail manuscrit)}
\end{demo}

\section{Matrices semblables et trace}
\index{matrices semblables}
\subsection{Définitions}
\begin{defi}
  Deux matrices carrées de même taille $A$ et $B$ dans $\mathcal{M}_p(\K)$ sont semblables si et seulement si il existe $P\in GL_p(\K)$ telle que 
\begin{displaymath}
  B = P^{-1}\, A\, P.
\end{displaymath}
\end{defi}
\begin{rem}
 Comme pour la notion de \emph{matrices équivalentes}, il n'existe pas de notation universelle pour les matrices semblables. On note encore $A \sim B$ pour indiquer que les matrices sont semblables. Cela contribue grandement à la confusion entre ces deux notions distinctes. Cette confusion est augmentée encore par le fait que la relation de similitude est \emph{une relation d'équivalence} qui n'est pas \emph{la} relation d'équivalence entre deux matrices.
\end{rem}
\begin{rem}
  D'après la formule de changement de base pour les matrices d'endomorphismes, deux matrices carrées sont semblables si et seulement si elles représentent \emph{le même endomorphisme} dans des bases différentes.
\end{rem}

On rappelle que la trace \index{trace d'une matrice carrée} d'une matrice carrée est la somme de ses termes diagonaux. La fonction trace notée $\tr$ est une forme linéaire sur $\mathcal{M}_p(\K)$. Elle vérifie
\begin{displaymath}
  \forall (A,B)\in \mathcal{M}_p(\K)^2,\; \tr(AB) = \tr(BA)
\end{displaymath}
\begin{propn}
  Deux matrices semblables ont la même trace.
\end{propn}
\begin{demo}
 Si $A$ et $B$ sont semblables dans $\mathcal{M}_p(\K)$, il existe $P\in \text{GL}_(\K)$ telles que $B = P^{-1}A\,P$. La trace d'un produit est indépendant de l'ordre des facteurs:
\[
 \tr(B) = \tr\left( P^{-1}(AP)\right) = \tr\left( (AP)P^{-1}\right) = \tr\left( A\right) . 
\]
\end{demo}
\begin{rem}
  On dit que la trace est un invariant par similitude.
\end{rem}

\begin{rem}
 \index{trace d'un endomorphisme} On en déduit que les traces des matrices représentant un endomorphisme dans une base sont égales entre elles donc indépendantes de la base choisie. Si $\mathcal U$ et $\mathcal V$ sont deux bases d'un $\K$-espace vectoriel $E$ et $f$ un endomorphisme de $E$:
\begin{displaymath}
 \tr\left( \Mat_{\mathcal V} f\right) 
=\tr \left(P_{\mathcal V \mathcal U}^{-1} \, \Mat_{\mathcal U}f \, P_{\mathcal U \mathcal V} \right) 
=\tr \left(\Mat_{\mathcal U}f \, P_{\mathcal U \mathcal V} P_{\mathcal U \mathcal V}^{-1}\right)
=\tr \left(\Mat_{\mathcal U}f\right) 
\end{displaymath}
Cet élément de $\K$ est appelé \emph{trace} de $f$ et noté $\tr (f)$. C'est la trace de l'\emph{endomorphisme}.
Cette nouvelle fonction trace est une forme linéaire sur l'espace des endomorphismes. 
\end{rem}
\begin{defi}
  La trace d'un endomorphisme est la trace de n'importe quelle matrice rerésentant cet endomorphisme dans une base.
\end{defi}

\begin{propn}
  Si $p$ est un projecteur dans un espace $E$ de dimension finie, sa trace est le dimension de son image. Si $A$ et $B$ sont deux sous-espaces supplementaires et $p$ la projection sur $A$ parallèlement à $B$, alors $\tr(p) = \dim(A)$.
\end{propn}
\begin{demo}
  On considère une base $\mathcal{E} = (a_1,\cdots,a_{\alpha},b_1,\cdots,b_{\beta})$ de $E$ avec $(a_1,\cdots,a_{\alpha})$ base de $A$ et $(b_1,\cdots,b_{\beta})$ base de $B$. Alors
\begin{displaymath}
  \Mat_{\mathcal{E}}(p)=
\begin{pmatrix}
1      & 0 &        &        & 0      &        & 0      \\
0      & 1 & \ddots &        &        &        &        \\
\vdots &   & \ddots & 0      & \vdots &        & \vdots \\
0      &   &        & 1      & 0      &        & 0      \\
0      &   &        & 0      & \vdots &        & \vdots \\
\vdots &   &        & \vdots &        &        &        \\
0      &   & \cdots & 0      & 0      &        & 0
\end{pmatrix}
\end{displaymath}
\end{demo}

\subsection{Introduction à la réduction des endomorphismes}
Cette partie utilise les méthodes de la section de \href{\baseurl C2234.pdf}{cours portant sur les opérations élémentaires} (calcul de rang et résolution de sysèmes d'équations).
Dans toute cette partie, $E$ désigne un $\K$-espace vectoriel de dimension $p > 0$.
\subsubsection{Polynômes}
\begin{defi}
  Soit $f\in \mathcal{L}(E)$, $A \in \mathcal{M}_p(\K)$, $P\in \K[X]$. On peut substituer $f$ ou $A$ à $X$ et définir 
  $P(f)$ et $P(A)$ par:
\[
  P = a_0 + a_1X + \cdots +a_nX^n
  \Rightarrow
  \left\lbrace
  \begin{aligned}
    P(f) &= a_0 \Id_E + a_1 f + \cdots + a_n f^n \in \mathcal{L}(E) \\
    P(A) &= A_0 I_p + a_1 A + \cdots + a_n A^n \in \mathcal{M}_p(\K)
  \end{aligned}
  \right. .
\]
\end{defi}
\begin{rem}
  On renonce aux notations $\widetilde{P}$ et $\widehat{P}$ qui étaient utilisées pour la substitution d'un élément du corps ou d'un autre polynôme.
\end{rem}

Les propriétés suivantes traduisent les règles de calculs usuelles qui sont les mêmes dans les algèbres $\K[X]$, $\mathcal{L}(E)$ et $\mathcal{M}_p(\K)$. Attention tout de même à la commutativité qui n'est vérifiée que pour les polynômes.
\begin{propn}
  Soit $f\in \mathcal{L}(E)$, $A \in \mathcal{M}_p(\K)$. Pour tous $P$ et $Q$ dans $\K[X]$, pour tous $\lambda \in \K$.
\[
\begin{aligned}
  &(P+Q)(f) = P(f) + Q(f) & &(P+Q)(A) = P(A) + Q(A) \\
  &(\lambda p)(f) = \lambda P(f) & &(\lambda p)(A) = \lambda P(A)\\
  &(PQ)(f) = P(f) \circ Q(f) & &(PQ)(A) = P(A) Q(A)
\end{aligned}
\]
\end{propn}
\index{polynôme annulateur}
\begin{defi}[Polynome annulateur] Soit $P \in \K[X]$ non nul.\newline
  $P \in \K[X]$ est un polynôme annulateur de $f\in \mathcal{L}(E)$ $\Leftrightarrow P(f) = 0_{\mathcal{L}(E)}$.\\
  $P \in \K[X]$ est un polynôme annulateur de $A\in \mathcal{M}_p(\K)$ $\Leftrightarrow P(A) = 0_{\mathcal{M}_{p}(\K)}$.
\end{defi}
\begin{propn}
  Soit $f \in \mathcal{L}(E)$ et $A \in \mathcal{M}_p(\K)$. Ils admettent des polynômes annulateurs.
\end{propn}
\begin{demo}
  Les espaces vectoriels $\mathcal{L}(E)$ et $\mathcal{M}_p(\K)$ sont de dimension $p^2$. D'après la condition suffisante de dépendance, \index{condition suffisante de dépendance} les familles
  \[
    \left( \Id_E, f, \cdots , f^{p^2}\right) \hspace{0.5cm} \text{ et} \hspace{0.5cm} \left( I_p, A, \cdots , A^{p^2}\right)
  \]
sont liées. Une relation linéaire donne les coefficients d'un polynôme annulateur.
\end{demo}
\begin{propn}
  Soit $f \in \mathcal{L}(E)$ et $A \in \mathcal{M}_p(\K)$.
  \begin{enumerate}
    \item L'ensemble des degrés des polynômes annulateurs de $f$ (resp $A$) admet un plus petit élément $m$.
    \item Un polynôme annulateur de degré $m$ est appelé un polynôme minimal de $f$ (resp de $A$).
    \item Un polynôme minimal de $f$ divise tout polynôme annulateur de $f$ (resp $A$).
    \item Deux polynômes minimaux de $f$ (resp de $A$) sont égaux à la multiplication près par un élément non nul du corps $\K$.
  \end{enumerate} 
\end{propn}
\begin{demo}
  Pour la proposition 2. Soit $M$ minimal et $P$ annulateur de $f$. Divisons $P$ par $M$.
\[
  \exists (Q,R)\in \K[X]^2, \text{ tq } P = QM + R \text{ avec } \deg(R) < m = \deg(M). 
\]
En substituant $f$ à $X$, on obtient $R(f) = \underset{=0_{\mathcal{L}(E)}}{\underbrace{P(f)}} 
+ Q(f) \circ \underset{=0_{\mathcal{L}(E)}}{\underbrace{M(f)}} = 0_{\mathcal{L}(E)} \Rightarrow \deg(R) \geq m$ (si $R$ non nul) ce qui est absurde. Donc $R$ est nul c'est à dire que $M$ divise $P$.\newline
La dernière propriété vient de ce que deux polynômes minimaux se divisent mutuellement.
\end{demo}
\index{polynôme caractéristique}
\begin{propn}[Polynôme caractéristique]
  Soit $f \in \mathcal{L}(E)$ et $A \in \mathcal{M}_p(\K)$.
\[
\begin{aligned}
  \exists P\in \K[X] \text{ tq } \forall \lambda \in \K, \; \rg(f - \lambda \Id_E) < p \Leftrightarrow P(\lambda) = 0 \\
  \exists P\in \K[X] \text{ tq } \forall \lambda \in \K, \; \rg(A - \lambda I_p) < p \Leftrightarrow P(\lambda) = 0
\end{aligned}
\]
\end{propn}
\begin{demo}
  On transforme la matrice $A - \lambda I_p$ par des opérations élémentaires (algorithme du pivot total) dans un contexte formel (paramètre $\lambda$). On obtient que le rang est $p$ sauf si $\lambda$ est racine d'un certain polynôme $P$.\newline
  On verra plus tard une autre preuve utilisant le \href{\baseurl C2261.pdf}{déterminant} d'une matrice carrée.\newline
  Pour un endomorphisme, on utilise une matrice carrée le représentant dans une base.
\end{demo}
\begin{rem}
  On dispose d'algorithmes élémentaires pour calculer un polynôme caractéristique. Ce n'est pas le caspour un polynôme minimal.
\end{rem}
\index{valeur propre}\index{vecteur propre}\index{spectre}
\begin{defi}
  Soit $f \in \mathcal{L}(E)$ et $A \in \mathcal{M}_p(\K)$. Soit $\lambda \in \K$.\medskip
  \begin{itemize}
    \item $\lambda$ est une valeur propre de $f \Leftrightarrow \exists x\in E,\; x\neq 0_E \text{ tel que } f(x)= \lambda x$.\medskip
    \item $\lambda$ est une valeur propre de $A \Leftrightarrow \exists X\in \mathcal{M}_{p ,1}(\K) \; X\neq 0_{\mathcal{M}_{p ,1}(\K)} \text{ tel que } f(x)= \lambda x$.\medskip
  \end{itemize}
On dit alors que $x$ est un vecteur propre de valeur propre $\lambda$ et que $X$ est une colonne propre de valeur propre $\lambda$.\newline
Le spectre de $f$ (resp $A$) est l'ensemble des valeurs propres de $f$ (resp $A$).
\end{defi}
\begin{propn} \label{polycarac}
  Soit $f \in \mathcal{L}(E)$ et $A \in \mathcal{M}_p(\K)$. Soit $P$ un polynôme caractéristique de $f$ (resp $(A)$.
\[
  \lambda \text{ valeur propre de } f \text{ (resp $A$) } \Leftrightarrow P(\lambda) = 0.
\]
Les valeurs propres sont les racines d'un polynôme caractéristique.
\end{propn}
\begin{demo}
  Il suffit de remarquer que 
\[
  \forall x \in E, \; f(x) = \lambda x \Leftrightarrow x \in \ker(f - \lambda \Id_E).
\]
Donc $\lambda$ est une valeur propre si et seulement si $ker(f - \lambda \Id_E)$ n'est pas réduit à $0$ c'est à dire que l'endomorphisme n'est pas bijectif donc de rang strictement plus petit que $p$.
\end{demo}
\begin{propn} \label{vpracpolmin}
  Soit $f \in \mathcal{L}(E)$ et $A \in \mathcal{M}_p(\K)$. Soit $M$ un polynôme minimal de $f$ (resp $A$):
\[
  \lambda \text{ valeur propre de } f \text{ (resp $A$) } \Leftrightarrow M(\lambda) = 0.
\]
\end{propn}
\begin{demo}
  Soit $\lambda$ une valeur propre. Montrons que $M(\lambda) = 0$.\newline
  En effet, il existe un vecteur propre $x$ (non nul) de valeur propre $\lambda$ donc 
\[
  f(x) = \lambda x \Rightarrow M(f)(x) = M(\lambda) x \Rightarrow M(\lambda) = 0 \text{ car } x\neq 0_E.
\]
Réciproquement, si $\lambda$ est une racine de $M$ alors il existe $M_1 \in \K[X]$ tel que 
\[
  M = (X- \lambda) M_1 \Rightarrow 0_{\mathcal{L}(E)} = (f - \lambda \Id_E) \circ M_1(f)
\] 
Si $f - \lambda \Id_E$ était bijective, en composant par la bijection réciproque, on en déduirait $M_1(f) = 0_{\mathcal{L}(E)}$ en contradiction avec le minimalité de $M$. On peut conclure que $f - \lambda \Id_E$ n'est pas injective donc que $\lambda$ est une valeur propre.
\end{demo}


\subsubsection{Réduction}
\index{diagonalisation} \index{triangularisation}
Réduire un endomorphisme c'est une trouver une base dans laquelle la matrice a une forme simple. On envisage ici deux réductions: la diagonalisation et la triangularisation (ou trigonalisation). Pour une matrice carrée, il s'agit de trouver une matrice diagonale ou triangulaire semblable à la matrice donnée et de préciser les matrices de passage.
ns le paragraphe suivant.
\begin{center}
  \begin{Large}\textbf{IMPORTANT}\end{Large}\newline
  Transformer par opérations élémentaires une matrice en une matrice triangulaire \begin{large}\textbf{ce n'est pas}\end{large} triangulariser.
\end{center}

Exemple de triangularisation  à compléter

Les endomorphismes ou les matrices carrées ne sont pas forcément diagonalisables ou triangularisable. La proposition suivante caractérise le caractère diagonalisable. On présente  d'autres conditions dans le paragraphe suivant. 
\begin{propn}\label{caracdiag}
  Soit $f\in \mathcal{L}(E)$ et $\lambda_1, \cdots ,\lambda_s$ ses valeurs propres.
\[
  f \text{ est diagonalisable } \Leftrightarrow \ker(f-\lambda_1 \Id_E) \oplus \cdots \oplus \ker(f-\lambda_s \Id_E) = E. 
\]
\end{propn}
\begin{defi}
  $\ker(f-\lambda_i \Id_E)$ est le sous-espace propre de valeur propre $\lambda_i$.
\end{defi}

\begin{demo}
  Soit $f\in \mathcal{L}(E)$. Il est diagonalisable si et seulement si il existe une base $\mathcal{U}=(u_1, \cdots, u_p)$ dans laquelle la matrice de $f$ est diagonale. En notant $\lambda_1, \cdots, \lambda_p$ les termes de la diagonale, cela signifie que
\[
  \forall i \in \llbracket 1,p \rrbracket, \; f(u_i) = \lambda_i u_i.
\]
Autrement dit, $f$ est diagonalisable si et seulement si il existe une base de vecteurs propres de $f$.\medskip\newline 
\indent \textbullet\hspace{0.5cm} Plan de la preuve de $f \text{ est diagonalisable } \Rightarrow \ker(f-\lambda_1 \Id_E) \oplus \cdots \oplus \ker(f-\lambda_s \Id_E) = E$.\newline
Supposons $f$ diagonalisable dans une base $\mathcal{U}=(u_1, \cdots, u_p)$ de vecteurs propres pour laquelle
\[
  \MatB{U}{f} = D =
  \begin{pmatrix}
          d_1 & 0      & \cdots & 0        \\
            0 & \ddots & \ddots & \vdots   \\
    \vdots    & \ddots & \ddots &    0     \\
      0       & \cdots &   0    & d_p 
  \end{pmatrix}.
\]
\begin{enumerate}
  \item Les $d_i$ sont les seules valeurs propres. Ils ne sont pas forcément deux à deux distincts. On note $\lambda_1, \cdots, \lambda_s$ avec $s \leq p$ les valeurs distinctes des $d_i$.
  \item $E$ est la somme des sous-espaces propres.
  \item Par récurrence: $m$ sous-espaces propres attachés à des valeurs propres distinctes sont en somme directe.
\end{enumerate} 
Preuve de 1.\newline
Le polynôme caractéristique de $f$ est $(X-d_1)\cdots(X-d_p)$ car
\[
\forall \lambda \in \C,\;
\MatB{U}{f-\lambda\Id_E}=
  \begin{pmatrix}
  d_1 - \lambda & 0      & \cdots & 0        \\
              0 & \ddots & \ddots & \vdots   \\
      \vdots    & \ddots & \ddots &    0     \\
        0       & \cdots &   0    & d_p - \lambda 
  \end{pmatrix}.  
\]
qui est diagonale donc triangulaire supérieure ce qui permet de caractériser $\rg(f-\lambda \Id_E) < p$.\medskip\newline
Preuve de 2.\newline
Pour tout $x\in E$, chaque composante de sa décomposition dans $\mathcal{U}$ est un vecteur propres. La somme des composantes attachées à la même valeur propre $\lambda_i$ appartient à $\ker(f-\lambda_i \Id_E)$. Ceci assure 
\[
  \left( \forall x \in E, \; x\in \ker(f-\lambda_1 \Id_E) + \cdots +\ker(f-\lambda_s \Id_E)\right)
  \Rightarrow E = \ker(f-\lambda_1 \Id_E) + \cdots +\ker(f-\lambda_s \Id_E).
\]
Preuve de 3. \newline
Montrons la propriété pour 2 sous-espaces propres.
\[
  \left. 
  \begin{aligned}
     x &\in \ker(f-\lambda_1 \Id_E)\\
     x &\in \ker(f-\lambda_2 \Id_E)
  \end{aligned}
  \right\rbrace
  \Rightarrow f(x) = \lambda_1 x = \lambda_2 x \Rightarrow \underset{\neq 0}{\underbrace{(\lambda_2 - \lambda_1)}}x = 0_E \Rightarrow x = 0_E.
\]
Montrons que la propriétés pour $m-1$ sous-espaces propres entraine celle pour $m$. Considérons des $x_i \in \ker(f-\lambda_i \Id_E)$ pour $i\in \llbracket 1,m\rrbracket$:
\begin{multline*}
  x_1 + \cdots +x_m = 0_E
  \Rightarrow
  \left\lbrace
  \begin{aligned}
  x_1 + \cdots +x_m &= 0_E &\times& -\lambda_1\\  
  f(x_1 + \cdots +x_m)= \lambda_1 x_1 + \cdots + \lambda_m x_m  &= 0_E &\times&1 
  \end{aligned}
   \right. \\
   \Rightarrow
   (\lambda_2 - \lambda_1)x_2 + \cdots + (\lambda_m - \lambda_1)x_m = 0_E
   \Rightarrow x_2 = \cdots = x_m = 0_E
\end{multline*}
d'après l'hypothèse de récurrence pour $m-1$.\medskip\newline 
\indent \textbullet\hspace{0.5cm} Preuve de  $\ker(f-\lambda_1 \Id_E) \oplus \cdots \oplus \ker(f-\lambda_s \Id_E) = E \Rightarrow f \text{ est diagonalisable }$.\newline
Il suffit de concaténer des bases de chaque sous-espace propre pour obtenir une base de $E$ constituée uniquement de vecteurs propres.
\end{demo}

\subsubsection{Conditions}
\begin{propn}
  Soit $f\in \mathcal{L}(E)$ avec $K = \C$. Alors $f$ est triangularisable.
\end{propn}
\begin{demo}
  On raisonne par récurrence sur la dimension du $\C$-espace vectoriel. Considérons la proposition:
  \begin{quote}
    $\mathcal{P}_p : \hspace{0.5cm} \left( \dim(E) = p \text{ et } f\in \mathcal{L}(E)\right) \Rightarrow f \text{ triangularisable}.$
  \end{quote}
Il n'y a rien à démontrer pour $\mathcal{P}_1$. On propose deux méthodes pour $\mathcal{P}_{p-1} \Rightarrow \mathcal{P}_p$.\newline
Le début est le même. soit $f\in \mathcal{L}(E)$ avec $\dim(E)=p$. Comme le corps de base est $\C$, le polynôme caractéristique admet au moins une racine $\lambda$ qui est une valeur propre.

Méthode 1: avec une projection.\newline
Il existe $u_1\in E$ non nul tel que $f(u_1) = \lambda u_1$. On complète en une base $\mathcal{U} = (u_1,u_2,\cdots,u_p)$ de $E$ et on note $H = \Vect(u_2,\cdots,u_p)$.\newline
Le sous-espace $H$ n'est pas forcément stable par $f$. On considère
\[
  g = p \circ f_H \in \mathcal{L}(H)
\]
où $p$ est la projection sur $H$ parallèlement à $\Vect(u_1)$. D'après $\mathcal{P}_{p-1}$ appliqué à $g$, il existe une base $(v_2, \cdots, v_p)$ de $H$ telle que $\MatB{V}{g}$ soit triangulaire supérieure. \`A cause de la projection, il existe des $\mu_i$ tels que $f(v_i) = g(v_i) + \mu_i u_1$. On en déduit 
\[
  \Mat_{(u_1,v_2,\cdots,v_p)}=
  \begin{pmatrix}
    \lambda                                        &   \begin{matrix} \mu_2 & \cdots & \mu_p \end{matrix}\\
    \begin{matrix} 0 \\ \vdots \\0 \end{matrix}    & \MatB{V}{g}
  \end{pmatrix}
\]
qui est triangulaire supérieure.\medskip

Méthode 2: hyperplan stable.\newline
Comme $\rg(f - \lambda \Id_E) < \dim(E)$, son image est incluse dans un hyperplan $H = \ker \varphi$. On en déduit
\[
  \varphi \circ (f - \lambda \Id_E) = 0_{\mathcal{L}(E)} \Rightarrow \varphi \circ f = \lambda \varphi.
\]
Ceci montre que $H = \ker \varphi$ est stable par $f$. En effet
\[
  x\in H \Rightarrow \varphi(f(x)) = \varphi \circ f(x) = \lambda \varphi(x) = O_E \Rightarrow f(x) \in H.
\]
La restriction $g$ de $f$ à $H$ est dans $\mathcal{L}(H)$ sans avoir besoin de projeter. D'après l'hypothèse de récurrence, il existe une base $(u_1, \cdots,u_{p-1})$ de $H$ dans laquelle la matrice de $g$ est triangulaire supérieure. Complétons en une base $\mathcal{U} = (u_1,\cdots, u_{p-1},u_p)$ de $E$. On se sait rien de $f(u_p)$, notons $\mu_1, \cdots, \mu_p$ ses coordonnées dans $\mathcal{U}$. Alors
\[
  \MatB{U}{f}=
  \begin{pmatrix}
    \Mat_{(u_1, \cdots, u_{p-1})}(g)          & \begin{matrix} \mu_1 \\ \vdots \\ \mu_{p-1}\end{matrix} \\
    \begin{matrix} 0 & \cdots & 0\end{matrix} & \mu_p
  \end{pmatrix}.
\]
qui est triangulaire supérieure.
\end{demo}
\begin{propn} \label{caracsupp}
  Soit $f\in \mathcal{L}(E)$, $M$ un polynôme minimal de $f$. On suppose $M$ scindé:
\[
  M = (X - \lambda_1)^{m_1} \cdots (X - \lambda_s)^{m_s} \text{ avec les $\lambda_i$ deux à deux distincts}
\]
On note $E(\lambda_i) = \ker((f - \lambda_i \Id_E)^{m_i})$ alors ils sont supplementaires
\[
  E(\lambda_1) \oplus \cdots \oplus E(\lambda_s) = E.
\]
\end{propn}
\begin{demo}
  Pour $i\in \llbracket 1, s \rrbracket$, on note $M_i = \prod_{\substack{ j \in \llbracket 1,s \rrbracket \\ j \neq i}} (X- \lambda_j)^{m_j}$. Les polynômes $(M_1, \cdots, M_s)$ sont premiers entre eux dans leur ensemble (mais pas deux à deux). D'après le théorème de Bezout, il existe des polynômes $U_1, \cdots , U_s$ tels que 
\[
  1 = U_1 M_1 + \cdots +U_s M_s.
\]
On note $p_i = U_i(f) \circ M_i(f)$. On montre que 
\begin{enumerate}
  \item $E(\lambda_i)$ est stable par $f$ et $\Im(p_i) \subset E(\lambda_i)$. 
  \item $ i\neq j \Rightarrow p_i \circ p_j = 0_{\mathcal{L}(E)}$.
  \item $p_1 + \cdots + p_s = \Id_E \Rightarrow E(\lambda_1)+ \cdots + E(\lambda_s) = E$ .
  \item Pour $i \in \llbracket 1,s\rrbracket$, $F_i = \sum_{\substack{j \in \llbracket 1,s\rrbracket \\ j \neq i}}E(\lambda_j)$ est stable par $f$, le polynôme $M_i$ est annulateur de $f_{F_i}$ et $(X- \lambda_i)^{m_i}$ est un polynôme annulateur de $f_{E(\lambda_i)}$ (restrictions).
  \item Que dire des polynômes annulateurs des restrictions de $f$ à $E(\lambda_i)\cap F_i$?
  \item Conclure.
\end{enumerate}
Preuve de 1.\newline
Le sous-espace $E(\lambda_i) = \ker(f- \lambda_i\Id_E)^{m_i}$ est stable par $f$ car $f$ commute avec $(f- \lambda_i\Id_E)^{m_i}$.\newline
Pour tout $i \in \llbracket 1,s \rrbracket$, $M = (X-\lambda)^{m_i}M_i$. Comme $M$ est annulateur: 
\[
  0_{\mathcal{L}(E)} = M(f) = (f-\lambda_i \Id_E)^{m_i}\circ M(i)(f) \Rightarrow \Im(M_i(f))\subset E(\lambda_i)
  \Rightarrow \Im(p_i) \subset U_i(f)(E(\lambda_i)) \subset E(\lambda_i)
\]
car $E(\lambda_i)$ est stable par $f$ donc par $U_i(f)$.\medskip\newline
Preuve de 2.\newline
Pour $i\neq j$, $M$ divise $M_i M_j$ donc dans $p_i\circ p_j$ figure un composé $M(f)$ qui est l'endomorphisme nul.\medskip\newline
Preuve de 3.\newline
En substituant $f$ à $X$:
\[
  1 = U_iM_i + \cdots + U_sM_s\Rightarrow \Id_E = p_1 + \cdots + p_s \Rightarrow E = \Im(p_1) + \cdots +\Im(p_s)
  = E(\lambda_1) + \cdots + E(\lambda_s).
\]
Preuve de 4.\newline
Comme les $E(\lambda_i)$ sont stables, leurs sommes le sont aussi en particulier les $F_i$.\newline
Par définition de $E(\lambda_i)$ comme noyau, $(X-\lambda_i)^{m_i}$ est annulateur de $f_{E(\lambda_i)}$. Pour tous les $j\neq i$, $(f-\lambda_i\Id_E)^{m_j}$ est un composée de $M_i(f)$ donc $M_i(f)$ est nul sur tous les sous-espaces qui composent $F_i$. On en déduit que $M_i$ est un polynome annulateur de $f_{F_1}$.\medskip\newline
Preuve de 5.\newline
Plaçons dans un cadre un peu plus général. Considérons des sous-espaces $A$ et $B$ stables par $f$ et notons $M_A$ (resp $M_B$) un polynôme minimal de $f_A$ (resp de $f_B$). Le sous-espace $C = A\cap B$ est encore stable par $f$, que peut-on dire du polynôme minimal $M_C$ de $f_C$?\newline
Il est clair que $M_A$ et $M_B$ sont annulateurs de $f_C$ donc $M_C$ divise les deux donc aussi leur pgcd. On en déduit que si $M_A$ et $M_B$ sont premiers entre eux, $1$ est anulateur ce qui signifie que $C$ est réduit au vecteur nul.\newline
Ceci s'applique ici avec $E(\lambda_i),(X-\lambda_i)^{m_i}$ et $F_i,M_i$. On en déduit $E(\lambda_i) \cap F_i = \left\lbrace 0_E \right\rbrace$ ce qui traduit que la somme est directe.\medskip\newline
6. conclusion\newline
On déduit de 3 et 5 que les sous-espaces caractéristiques sont supplémentaires.
\end{demo}

\begin{propn}
  Soit $f\in \mathcal{L}(E)$, $M$ un polynôme minimal de $f$. Montrer que 
\[
  f \text{ diagonalisable } \Leftrightarrow M \text{ scindé et toutes ses racines sont simples}.
\]
\end{propn}
\begin{demo}
D'après la proposition \ref{caracdiag}
\begin{multline*}
  f \text{ est diagonalisable } \Rightarrow \ker(f-\lambda_1 \Id_E) \oplus \cdots \oplus \ker(f-\lambda_s \Id_E) = E  \\
  \Rightarrow (f-\lambda_1 \Id_E) \circ \cdots \circ (f-\lambda_s \Id_E) = 0_{\mathcal{L}(E)}
  \Rightarrow (X-\lambda_1)\cdots (X-\lambda_s) \text{ annulateur}.
\end{multline*}
Réciproquement si le polynôme annulateur $M$ et scindé sans racine multiple, la proposition \ref{caracsupp} montre que les sous-espaces caractéristiques qui en l'occurrence sont aussi les sous-espaces propres sont supplémentaires donc $f$ est diagonalisable d'après l'autre implication de la proposition \ref{caracdiag}.
\end{demo}


\end{document}
