\input{courspdf.tex}
\hypersetup{pdftitle=2159}
\debutcours{Polynômes scindés}{obsolete}

\subsection{Définitions - Ensemble de racines}
\index{polynôme scindé}
\begin{defi}
 Un élément de $\K[X]$ est scindé dans $\K[X]$ lorsqu'il est produit de polynômes de degré 1 à coefficients dans $\K$.
\end{defi}
\begin{rems}
\begin{enumerate}
 \item Dans le cas où $\K$ est un sous-corps d'un autre corps $\K^\prime$. Il est possible qu'un polynôme $P$ dont les coefficients sont dans $\K$ soit scindé dans $\K^\prime$ mais pas dans $\K$. C'est le cas par exemple de $X^2+1$ qui est scindé dans $\C[X]$ mais pas dans $\R[X]$.
\item Tout polynôme de degré 1 est de la forme 
\begin{displaymath}
aX+b = a(X- \dfrac{b}{a}) \text{ avec } a\neq 0_K
\end{displaymath}
donc $-\frac{b}{a}$ est une racine de ce polynôme ainsi que de tout polynome dont il est un diviseur.
\item Lorsqu'un polynôme $P\in \K[X]$ de degré $p$ est scindé, il s'écrit sous la forme
\begin{equation}
 P = c(X-\alpha_1)\cdots (X-\alpha_p)
\end{equation}
où $\alpha_1, \cdots \alpha_p$ sont des éléments de $\K$ qui ne sont pas forcément deux à deux distincts et $c\neq0$ est le coefficient dominant de $P$.
\end{enumerate}
\end{rems}
 
On peut exploiter l'expression précédente de deux manières: soit en regroupant les racines soit en développant le produit.\newline
Le regroupement conduit à décomposition en facteurs irréductibles d'un polynôme scindé et à une caractérisation des polynômes scindés. Un polynôme est scindé si et seulement si la somme des multiplicitésde ses racines est égale à son degré.\newline
Le développement du produit conduit aux \emph{relations entre coefficients et racines} qui sont traitées dans la sous-section suivante après la définition des \emph{polynômes symétriques élémentaires}.
\index{polynôme irréductible}
\begin{defi}[polynôme irreductible]
 Un polynôme de degré $n$ est irréductible si et seulement si ses seuls diviseurs sont de degré $0$ ou $n$.
\end{defi}
Tout polynôme de degré $1$ est irréductible. Un polynôme de degré $2$ ou $3$ est irréductible si et seulement si il est sans racine.

\subsection{Relations entre coefficients et racines.}
\begin{defi}[polynômes symétriques élémentaires]\index{polynômes symétriques élémentaires}
 Soit $a_1,a_2,\cdots,a_p$ des éléments quelconques de $\K$. On définit $\sigma_1, \sigma_2, \cdots ,\sigma_k$ par les relations suivantes;
\begin{align*}
 \sigma_1 &= a_1+a_2+\cdots +a_p  = \sum_{i}a_i\\
\sigma_2 &= a_1a_2+a_2a_3+\cdots   = \sum_{i_1<i_2}a_{i_1}a_{i_2} \\
 &\vdots \\
\sigma_k &= a_1a_2\cdots a_k+a_2a_3\cdots a_k+\cdots   = \sum_{i_1<i_2<\cdots <i_k}a_{i_1}a_{i_2}\cdots a_{i_k} \\
 &\vdots \\
\sigma_p &= a_1a_2\cdots a_p
\end{align*}
\end{defi}
\begin{rem}
 On peut noter que la somme définissant $\sigma_k$ contient $\dbinom{p}{k}$ termes.
\end{rem}
\begin{prop}[relations entre coefficients et racines]\index{relations entre coefficients et racines}
 Lorsqu'un polynôme $P\in K[X]$ de degré $p$ est scindé, il s'écrit sous la forme
\begin{displaymath}
 P = a_0+a_1X +\cdots +a_pX^p = X-\alpha_1)\cdots (X-\alpha_p)
\end{displaymath}
 Alors, pour $i\in\{0,\cdots p-1\}$ :
\begin{displaymath}
 a_i = (-1)^{p-i}c \sigma_{n-i}
\end{displaymath}
\end{prop}


\end{document}
