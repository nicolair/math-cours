%<dscrpt>Fichier de déclarations Latex à inclure au début d'un élément de cours.</dscrpt>

\documentclass[a4paper]{article}
\usepackage[hmargin={1.8cm,1.8cm},vmargin={2.4cm,2.4cm},headheight=13.1pt]{geometry}

%includeheadfoot,scale=1.1,centering,hoffset=-0.5cm,
\usepackage[pdftex]{graphicx,color}
\usepackage[french]{babel}
%\selectlanguage{french}
\addto\captionsfrench{
  \def\contentsname{Plan}
}
\usepackage{fancyhdr}
\usepackage{floatflt}
\usepackage{amsmath}
\usepackage{amssymb}
\usepackage{amsthm}
\usepackage{stmaryrd}
%\usepackage{ucs}
\usepackage[utf8]{inputenc}
%\usepackage[latin1]{inputenc}
\usepackage[T1]{fontenc}


\usepackage{titletoc}
%\contentsmargin{2.55em}
\dottedcontents{section}[2.5em]{}{1.8em}{1pc}
\dottedcontents{subsection}[3.5em]{}{1.2em}{1pc}
\dottedcontents{subsubsection}[5em]{}{1em}{1pc}

\usepackage[pdftex,colorlinks={true},urlcolor={blue},pdfauthor={remy Nicolai},bookmarks={true}]{hyperref}
\usepackage{makeidx}

\usepackage{multicol}
\usepackage{multirow}
\usepackage{wrapfig}
\usepackage{array}
\usepackage{subfig}


%\usepackage{tikz}
%\usetikzlibrary{calc, shapes, backgrounds}
%pour la présentation du pseudo-code
% !!!!!!!!!!!!!!      le package n'est pas présent sur le serveur sous fedora 16 !!!!!!!!!!!!!!!!!!!!!!!!
%\usepackage[french,ruled,vlined]{algorithm2e}

%pr{\'e}sentation du compteur de niveau 2 dans les listes
\makeatletter
\renewcommand{\labelenumii}{\theenumii.}
\renewcommand{\thesection}{\Roman{section}.}
\renewcommand{\thesubsection}{\arabic{subsection}.}
\renewcommand{\thesubsubsection}{\arabic{subsubsection}.}
\makeatother


%dimension des pages, en-t{\^e}te et bas de page
%\pdfpagewidth=20cm
%\pdfpageheight=14cm
%   \setlength{\oddsidemargin}{-2cm}
%   \setlength{\voffset}{-1.5cm}
%   \setlength{\textheight}{12cm}
%   \setlength{\textwidth}{25.2cm}
   \columnsep=1cm
   \columnseprule=0.5pt

%En tete et pied de page
\pagestyle{fancy}
\lhead{MPSI-\'Eléments de cours}
\rhead{\today}
%\rhead{25/11/05}
\lfoot{\tiny{Cette création est mise à disposition selon le Contrat\\ Paternité-Pas d'utilisations commerciale-Partage des Conditions Initiales à l'Identique 2.0 France\\ disponible en ligne http://creativecommons.org/licenses/by-nc-sa/2.0/fr/
} }
\rfoot{\tiny{Rémy Nicolai \jobname}}


\newcommand{\baseurl}{http://back.maquisdoc.net/data/cours\_nicolair/}
\newcommand{\urlexo}{http://back.maquisdoc.net/data/exos_nicolair/}
\newcommand{\urlcours}{https://maquisdoc-math.fra1.digitaloceanspaces.com/}

\newcommand{\N}{\mathbb{N}}
\newcommand{\Z}{\mathbb{Z}}
\newcommand{\C}{\mathbb{C}}
\newcommand{\R}{\mathbb{R}}
\newcommand{\D}{\mathbb{D}}
\newcommand{\K}{\mathbf{K}}
\newcommand{\Q}{\mathbb{Q}}
\newcommand{\F}{\mathbf{F}}
\newcommand{\U}{\mathbb{U}}
\newcommand{\p}{\mathbb{P}}


\newcommand{\card}{\mathop{\mathrm{Card}}}
\newcommand{\Id}{\mathop{\mathrm{Id}}}
\newcommand{\Ker}{\mathop{\mathrm{Ker}}}
\newcommand{\Vect}{\mathop{\mathrm{Vect}}}
\newcommand{\cotg}{\mathop{\mathrm{cotan}}}
\newcommand{\sh}{\mathop{\mathrm{sh}}}
\newcommand{\ch}{\mathop{\mathrm{ch}}}
\newcommand{\argsh}{\mathop{\mathrm{argsh}}}
\newcommand{\argch}{\mathop{\mathrm{argch}}}
\newcommand{\tr}{\mathop{\mathrm{tr}}}
\newcommand{\rg}{\mathop{\mathrm{rg}}}
\newcommand{\rang}{\mathop{\mathrm{rg}}}
\newcommand{\Mat}{\mathop{\mathrm{Mat}}}
\newcommand{\MatB}[2]{\mathop{\mathrm{Mat}}_{\mathcal{#1}}\left( #2\right) }
\newcommand{\MatBB}[3]{\mathop{\mathrm{Mat}}_{\mathcal{#1} \mathcal{#2}}\left( #3\right) }
\renewcommand{\Re}{\mathop{\mathrm{Re}}}
\renewcommand{\Im}{\mathop{\mathrm{Im}}}
\renewcommand{\th}{\mathop{\mathrm{th}}}
\newcommand{\repere}{$(O,\overrightarrow{i},\overrightarrow{j},\overrightarrow{k})$}
\newcommand{\cov}{\mathop{\mathrm{Cov}}}

\newcommand{\absolue}[1]{\left| #1 \right|}
\newcommand{\fonc}[5]{#1 : \begin{cases}#2 \rightarrow #3 \\ #4 \mapsto #5 \end{cases}}
\newcommand{\depar}[2]{\dfrac{\partial #1}{\partial #2}}
\newcommand{\norme}[1]{\left\| #1 \right\|}
\newcommand{\se}{\geq}
\newcommand{\ie}{\leq}
\newcommand{\trans}{\mathstrut^t\!}
\newcommand{\val}{\mathop{\mathrm{val}}}
\newcommand{\grad}{\mathop{\overrightarrow{\mathrm{grad}}}}

\newtheorem*{thm}{Théorème}
\newtheorem{thmn}{Théorème}
\newtheorem*{prop}{Proposition}
\newtheorem{propn}{Proposition}
\newtheorem*{pa}{Présentation axiomatique}
\newtheorem*{propdef}{Proposition - Définition}
\newtheorem*{lem}{Lemme}
\newtheorem{lemn}{Lemme}

\theoremstyle{definition}
\newtheorem*{defi}{Définition}
\newtheorem*{nota}{Notation}
\newtheorem*{exple}{Exemple}
\newtheorem*{exples}{Exemples}


\newenvironment{demo}{\renewcommand{\proofname}{Preuve}\begin{proof}}{\end{proof}}
%\renewcommand{\proofname}{Preuve} doit etre après le begin{document} pour fonctionner

\theoremstyle{remark}
\newtheorem*{rem}{Remarque}
\newtheorem*{rems}{Remarques}

\renewcommand{\indexspace}{}
\renewenvironment{theindex}
  {\section*{Index} %\addcontentsline{toc}{section}{\protect\numberline{0.}{Index}}
   \begin{multicols}{2}
    \begin{itemize}}
  {\end{itemize} \end{multicols}}


%pour annuler les commandes beamer
\renewenvironment{frame}{}{}
\newcommand{\frametitle}[1]{}
\newcommand{\framesubtitle}[1]{}

\newcommand{\debutcours}[2]{
  \chead{#1}
  \begin{center}
     \begin{huge}\textbf{#1}\end{huge}
     \begin{Large}\begin{center}Rédaction incomplète. Version #2\end{center}\end{Large}
  \end{center}
  %\section*{Plan et Index}
  %\begin{frame}  commande beamer
  \tableofcontents
  %\end{frame}   commande beamer
  \printindex
}


\makeindex
\begin{document}
\noindent

\hypersetup{pdftitle=2159}
\debutcours{Polynômes scindés}{obsolete}

\subsection{Définitions - Ensemble de racines}
\index{polynôme scindé}
\begin{defi}
 Un élément de $\K[X]$ est scindé dans $\K[X]$ lorsqu'il est produit de polynômes de degré 1 à coefficients dans $\K$.
\end{defi}
\begin{rems}
\begin{enumerate}
 \item Dans le cas où $\K$ est un sous-corps d'un autre corps $\K^\prime$. Il est possible qu'un polynôme $P$ dont les coefficients sont dans $\K$ soit scindé dans $\K^\prime$ mais pas dans $\K$. C'est le cas par exemple de $X^2+1$ qui est scindé dans $\C[X]$ mais pas dans $\R[X]$.
\item Tout polynôme de degré 1 est de la forme 
\begin{displaymath}
aX+b = a(X- \dfrac{b}{a}) \text{ avec } a\neq 0_K
\end{displaymath}
donc $-\frac{b}{a}$ est une racine de ce polynôme ainsi que de tout polynome dont il est un diviseur.
\item Lorsqu'un polynôme $P\in \K[X]$ de degré $p$ est scindé, il s'écrit sous la forme
\begin{equation}
 P = c(X-\alpha_1)\cdots (X-\alpha_p)
\end{equation}
où $\alpha_1, \cdots \alpha_p$ sont des éléments de $\K$ qui ne sont pas forcément deux à deux distincts et $c\neq0$ est le coefficient dominant de $P$.
\end{enumerate}
\end{rems}
 
On peut exploiter l'expression précédente de deux manières: soit en regroupant les racines soit en développant le produit.\newline
Le regroupement conduit à décomposition en facteurs irréductibles d'un polynôme scindé et à une caractérisation des polynômes scindés. Un polynôme est scindé si et seulement si la somme des multiplicitésde ses racines est égale à son degré.\newline
Le développement du produit conduit aux \emph{relations entre coefficients et racines} qui sont traitées dans la sous-section suivante après la définition des \emph{polynômes symétriques élémentaires}.
\index{polynôme irréductible}
\begin{defi}[polynôme irreductible]
 Un polynôme de degré $n$ est irréductible si et seulement si ses seuls diviseurs sont de degré $0$ ou $n$.
\end{defi}
Tout polynôme de degré $1$ est irréductible. Un polynôme de degré $2$ ou $3$ est irréductible si et seulement si il est sans racine.

\subsection{Relations entre coefficients et racines.}
\begin{defi}[polynômes symétriques élémentaires]\index{polynômes symétriques élémentaires}
 Soit $a_1,a_2,\cdots,a_p$ des éléments quelconques de $\K$. On définit $\sigma_1, \sigma_2, \cdots ,\sigma_k$ par les relations suivantes;
\begin{align*}
 \sigma_1 &= a_1+a_2+\cdots +a_p  = \sum_{i}a_i\\
\sigma_2 &= a_1a_2+a_2a_3+\cdots   = \sum_{i_1<i_2}a_{i_1}a_{i_2} \\
 &\vdots \\
\sigma_k &= a_1a_2\cdots a_k+a_2a_3\cdots a_k+\cdots   = \sum_{i_1<i_2<\cdots <i_k}a_{i_1}a_{i_2}\cdots a_{i_k} \\
 &\vdots \\
\sigma_p &= a_1a_2\cdots a_p
\end{align*}
\end{defi}
\begin{rem}
 On peut noter que la somme définissant $\sigma_k$ contient $\dbinom{p}{k}$ termes.
\end{rem}
\begin{prop}[relations entre coefficients et racines]\index{relations entre coefficients et racines}
 Lorsqu'un polynôme $P\in K[X]$ de degré $p$ est scindé, il s'écrit sous la forme
\begin{displaymath}
 P = a_0+a_1X +\cdots +a_pX^p = X-\alpha_1)\cdots (X-\alpha_p)
\end{displaymath}
 Alors, pour $i\in\{0,\cdots p-1\}$ :
\begin{displaymath}
 a_i = (-1)^{p-i}c \sigma_{n-i}
\end{displaymath}
\end{prop}


\end{document}
