%!  pour pdfLatex
\documentclass[a4paper]{article}
\usepackage[hmargin={1.5cm,1.5cm},vmargin={2.4cm,2.4cm},headheight=13.1pt]{geometry}

\usepackage[pdftex]{graphicx,color}
%\usepackage{hyperref}

\usepackage[utf8]{inputenc}
\usepackage[T1]{fontenc}
\usepackage{lmodern}
%\usepackage[frenchb]{babel}
\usepackage[french]{babel}

\usepackage{fancyhdr}
\pagestyle{fancy}

%\usepackage{floatflt}

\usepackage{parcolumns}
\setlength{\parindent}{0pt}
\usepackage{xcolor}

%pr{\'e}sentation des compteurs de section, ...
\makeatletter
%\renewcommand{\labelenumii}{\theenumii.}
\renewcommand{\thepart}{}
\renewcommand{\thesection}{}
\renewcommand{\thesubsection}{}
\renewcommand{\thesubsubsection}{}
\makeatother

\newcommand{\subsubsubsection}[1]{\bigskip \rule[5pt]{\linewidth}{2pt} \textbf{ \color{red}{#1} } \newline \rule{\linewidth}{.1pt}}
\newlength{\parcoldist}
\setlength{\parcoldist}{1cm}

\usepackage{maths}
\newcommand{\dbf}{\leftrightarrows}
% remplace les commandes suivantes 
%\usepackage{amsmath}
%\usepackage{amssymb}
%\usepackage{amsthm}
%\usepackage{stmaryrd}

%\newcommand{\N}{\mathbb{N}}
%\newcommand{\Z}{\mathbb{Z}}
%\newcommand{\C}{\mathbb{C}}
%\newcommand{\R}{\mathbb{R}}
%\newcommand{\K}{\mathbf{K}}
%\newcommand{\Q}{\mathbb{Q}}
%\newcommand{\F}{\mathbf{F}}
%\newcommand{\U}{\mathbb{U}}

%\newcommand{\card}{\mathop{\mathrm{Card}}}
%\newcommand{\Id}{\mathop{\mathrm{Id}}}
%\newcommand{\Ker}{\mathop{\mathrm{Ker}}}
%\newcommand{\Vect}{\mathop{\mathrm{Vect}}}
%\newcommand{\cotg}{\mathop{\mathrm{cotan}}}
%\newcommand{\sh}{\mathop{\mathrm{sh}}}
%\newcommand{\ch}{\mathop{\mathrm{ch}}}
%\newcommand{\argsh}{\mathop{\mathrm{argsh}}}
%\newcommand{\argch}{\mathop{\mathrm{argch}}}
%\newcommand{\tr}{\mathop{\mathrm{tr}}}
%\newcommand{\rg}{\mathop{\mathrm{rg}}}
%\newcommand{\rang}{\mathop{\mathrm{rg}}}
%\newcommand{\Mat}{\mathop{\mathrm{Mat}}}
%\renewcommand{\Re}{\mathop{\mathrm{Re}}}
%\renewcommand{\Im}{\mathop{\mathrm{Im}}}
%\renewcommand{\th}{\mathop{\mathrm{th}}}


%En tete et pied de page
\lhead{Programme colle math}
\chead{Semaine 6 du 04/11/19 au 09/11/19}
\rhead{MPSI B Hoche}

\lfoot{\tiny{Cette création est mise à disposition selon le Contrat\\ Paternité-Partage des Conditions Initiales à l'Identique 2.0 France\\ disponible en ligne http://creativecommons.org/licenses/by-sa/2.0/fr/
} }
\rfoot{\tiny{Rémy Nicolai \jobname}}


\begin{document}
\subsection{Techniques fondamentales de calcul en analyse}
\subsubsection{C - Primitives et équations différentielles linéaires}

\subsubsubsection{c) \'Equations différentielles linéaires du second ordre à coefficients constants}
\begin{parcolumns}[rulebetween,distance=\parcoldist]{2}
  \colchunk{Notion d'équation différentielle linéaire du second ordre à coefficients constants:
  \begin{displaymath}
   y''+ay'+by = f(x)
  \end{displaymath}
où $a$ et $b$ sont des scalaires et $f$ est une application continue à valeurs dans $\R$ ou $\C$.}
  \colchunk{\'Equation homogène associée.}
  \colplacechunks

  \colchunk{Résolution de l'équation homogène.}
  \colchunk{Si $a$ et $b$ sont réels, description des solutions réelles.}
  \colplacechunks

  \colchunk{Forme des solutions: somme d'une solution particulière et de la solution générale de l'équation homogène.}
  \colchunk{Les étudiants doivent savoir déterminer une solution particulière dans le cas d'un second membre de la forme $x\mapsto Ae^{\lambda x}$ avec $(A,\lambda)\in \C^2$, $x\mapsto B\cos(\omega x)$ et $x\mapsto B\sin(\omega x)$ avec $(B,\omega)\in \R^2$.\newline 
  $\leftrightarrows$ PC: régime libre, régime forcé; régime transitoire, régime établi.}
  \colplacechunks

  \colchunk{Principe de superposition.}
  \colchunk{}
  \colplacechunks

  \colchunk{Existence et unicité de la solution d'un problème de Cauchy.}
  \colchunk{La démonstration de ce résultat est hors programme.\newline
  $\leftrightarrows$ PC et SI: modélisation des circuits électriques LC, RLC et de systèmes mécaniques linéaires.}
  \colplacechunks

  \colchunk{}
  \colchunk{}
  \colplacechunks
\end{parcolumns}



Révision des programmes précédents: techniques fondamentales de calcul algébrique, complexe, trigonométrique, en analyse.

\bigskip
\begin{center}
 \textbf{Questions de cours}
\end{center}
\textbf{méthodes}\newline
Savoir-faire: on constate que cela conduit au résultat sans justification théorique. 
\begin{itemize}
 \item Primitive de l'inverse d'un trinôme
 \item Primitive d'un polynôme-exponentiel (avec des coefficients indéterminés)
 \item Intégration par parties très simple
 \item Changement de variable très très simple
 \item Equ. diff. lin ordre 1 ou 2 à coeff. constant avec second membre polynôme-exponentiel
\end{itemize}

\textbf{démonstrations}
\begin{itemize}
 \item Ensemble des solutions d'une eq. dif. lin d'ordre 2 homogène (coeff constants).
 \item Existence d'une sol d'une eq. dif. lin d'ordre 2 (coeff constants).
 \item Existence et unicité pour les pbs de Cauchy 
\end{itemize}


\begin{center}
 \textbf{Prochain programme}
\end{center}
Nombres réels et suites numériques.

\end{document}
