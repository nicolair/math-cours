%!  pour pdfLatex
\documentclass[a4paper]{article}
\usepackage[hmargin={1.5cm,1.5cm},vmargin={2.4cm,2.4cm},headheight=13.1pt]{geometry}

\usepackage[pdftex]{graphicx,color}
%\usepackage{hyperref}

\usepackage[utf8]{inputenc}
\usepackage[T1]{fontenc}
\usepackage{lmodern}
%\usepackage[frenchb]{babel}
\usepackage[french]{babel}

\usepackage{fancyhdr}
\pagestyle{fancy}

%\usepackage{floatflt}

\usepackage{parcolumns}
\setlength{\parindent}{0pt}
\usepackage{xcolor}

%pr{\'e}sentation des compteurs de section, ...
\makeatletter
%\renewcommand{\labelenumii}{\theenumii.}
\renewcommand{\thepart}{}
\renewcommand{\thesection}{}
\renewcommand{\thesubsection}{}
\renewcommand{\thesubsubsection}{}
\makeatother

\newcommand{\subsubsubsection}[1]{\bigskip \rule[5pt]{\linewidth}{2pt} \textbf{ \color{red}{#1} } \newline \rule{\linewidth}{.1pt}}
\newlength{\parcoldist}
\setlength{\parcoldist}{1cm}

\usepackage{maths}
\newcommand{\dbf}{\leftrightarrows}
% remplace les commandes suivantes 
%\usepackage{amsmath}
%\usepackage{amssymb}
%\usepackage{amsthm}
%\usepackage{stmaryrd}

%\newcommand{\N}{\mathbb{N}}
%\newcommand{\Z}{\mathbb{Z}}
%\newcommand{\C}{\mathbb{C}}
%\newcommand{\R}{\mathbb{R}}
%\newcommand{\K}{\mathbf{K}}
%\newcommand{\Q}{\mathbb{Q}}
%\newcommand{\F}{\mathbf{F}}
%\newcommand{\U}{\mathbb{U}}

%\newcommand{\card}{\mathop{\mathrm{Card}}}
%\newcommand{\Id}{\mathop{\mathrm{Id}}}
%\newcommand{\Ker}{\mathop{\mathrm{Ker}}}
%\newcommand{\Vect}{\mathop{\mathrm{Vect}}}
%\newcommand{\cotg}{\mathop{\mathrm{cotan}}}
%\newcommand{\sh}{\mathop{\mathrm{sh}}}
%\newcommand{\ch}{\mathop{\mathrm{ch}}}
%\newcommand{\argsh}{\mathop{\mathrm{argsh}}}
%\newcommand{\argch}{\mathop{\mathrm{argch}}}
%\newcommand{\tr}{\mathop{\mathrm{tr}}}
%\newcommand{\rg}{\mathop{\mathrm{rg}}}
%\newcommand{\rang}{\mathop{\mathrm{rg}}}
%\newcommand{\Mat}{\mathop{\mathrm{Mat}}}
%\renewcommand{\Re}{\mathop{\mathrm{Re}}}
%\renewcommand{\Im}{\mathop{\mathrm{Im}}}
%\renewcommand{\th}{\mathop{\mathrm{th}}}


%En tete et pied de page
\lhead{Programme colle math}
\chead{Semaine 25 du 27/04/20 au 9/05/20}
\rhead{MPSI B Hoche}

\lfoot{\tiny{Cette création est mise à disposition selon le Contrat\\ Paternité-Partage des Conditions Initiales à l'Identique 2.0 France\\ disponible en ligne http://creativecommons.org/licenses/by-sa/2.0/fr/
} }
\rfoot{\tiny{Rémy Nicolai \jobname}}


\begin{document}
Attention aux vendredis 1 et 5 Mai fériés.


\subsection{Dénombrement}
\begin{itshape}
Ce chapitre est introduit essentiellement en vue de son utilisation en probabilités ; rattaché aux mathématiques discrètes, le dénombrement interagit également avec l'algèbre et l'informatique. Il permet de modéliser certaines situations combinatoires et offre un nouveau cadre à la représentation de certaines égalités.

Toute formalisation excessive est exclue. En particulier~:
\begin{itemize}
\item  parmi les propriétés du paragraphe a), les plus intuitives sont admises sans démonstration;
\item  l'utilisation systématique de bijections dans les problèmes de dénombrement n'est pas un attendu du programme.
\end{itemize}
\end{itshape}

\subsubsubsection{a) Cardinal d'un ensemble fini}
\begin{parcolumns}[rulebetween,distance=\parcoldist]{2}
  \colchunk{Cardinal d'un ensemble fini.}
  \colchunk{Notations $|A|$, $\card(A)$, $\# A$.}
  \colplacechunks

  \colchunk{Cardinal d'une partie d'un ensemble fini, cas d'égalité.}
  \colchunk{Tout fondement théorique des notions d'entier naturel et de cardinal est hors programme.}
  \colplacechunks

  \colchunk{Une application entre deux ensembles finis de même cardinal est bijective si et seulement si elle est injective, si et seulement si elle est surjective.}
  \colchunk{}
  \colplacechunks

  \colchunk{Cardinal d'un produit fini d'ensembles finis.}
  \colchunk{}
  \colplacechunks


  \colchunk{Cardinal de la réunion de deux ensembles finis.}
  \colchunk{La formule du crible est hors programme.}
  \colplacechunks

  \colchunk{Cardinal de l'ensemble des applications d'un ensemble fini dans un autre.}
  \colchunk{}
  \colplacechunks

  \colchunk{Cardinal de l'ensemble des parties d'un ensemble fini.}
  \colchunk{}
  \colplacechunks
\end{parcolumns}


\subsubsubsection{b) Listes et combinaisons}
\begin{parcolumns}[rulebetween,distance=\parcoldist]{2}
  \colchunk{Nombre de $p$-listes (ou $p$-uplets) d'éléments distincts d'un ensemble de cardinal $n$, nombre d'applications injectives d'un ensemble de cardinal $p$ dans un ensemble de cardinal $n$, nombre de permutations d'un ensemble de cardinal $n$.}
  \colchunk{}
  \colplacechunks

  \colchunk{Nombre de parties à $p$ éléments (ou $p$-combinaisons) d'un ensemble de cardinal $n$.}
  \colchunk{Démonstration combinatoire des formules de Pascal et du binôme.}
  \colplacechunks

\end{parcolumns}


\subsection{Probabilités}
\subsubsection{A - Probabilités sur un univers fini}
\begin{itshape}
 Les définitions sont motivées par la notion d'expérience aléatoire.
La modélisation de situations aléatoires simples fait partie des capacités attendues des étudiants.
\end{itshape}

\subsubsubsection{a) Expérience aléatoire et univers}
\begin{parcolumns}[rulebetween,distance=2.5cm]{2}
  \colchunk{L'ensemble des issues (ou résultats possibles ou réalisations) d'une expérience aléatoire est
appelé univers.}
  \colchunk{On se limite au cas où cet univers est fini.}
  \colplacechunks

  \colchunk{\'Evénement, événement élémentaire (singleton), événement contraire, événements \og $A$ et $B$ \fg, évènement \og$A$ ou $B$ \fg, événement impossible, événements incompatibles, système complet d'événements.}
  \colchunk{}
  \colplacechunks
\end{parcolumns}

\subsubsubsection{b) Espaces probabilisés finis}
\begin{parcolumns}[rulebetween,distance=2.5cm]{2}

  \colchunk{Une probabilité sur un univers fini $\Omega$ est une application $P$ de $\mathcal{P} (\Omega)$ dans $[ 0 , 1 ]$ telle que $P (\Omega) = 1$ et, pour toutes parties disjointes $A$ et $B$, $P (A \cup B) = P (A) + P (B).$}
  \colchunk{Un espace probabilisé fini est un couple $(\Omega , P)$ où $\Omega$ est un univers fini et $P$ une probabilité sur $\Omega$.}
  \colplacechunks

  \colchunk{Détermination d'une probabilité par les images des singletons.}
  \colchunk{}
  \colplacechunks
  
  \colchunk{Probabilité uniforme.}
  \colchunk{}
  \colplacechunks

  \colchunk{Propriétés : probabilité de la réunion de deux événements, de l'événement contraire, croissance.}
  \colchunk{}
  \colplacechunks
\end{parcolumns}

\subsubsubsection{c) Probabilités conditionnelles}
\begin{parcolumns}[rulebetween,distance=2.5cm]{2}
  \colchunk{Si $P (B) > 0$, la probabilité conditionnelle de $A$ sachant $B$ est définie par : $P (A|B) = P_B (A) = \dfrac{P (A \cap B)}{P (B)}$.}
  \colchunk{On justifiera cette définition par une approche heuristique fréquentiste.

L'application $P_B$ est une probabilité.}
  \colplacechunks

  \colchunk{Formule des probabilités composées.}
  \colchunk{}
  \colplacechunks

  \colchunk{Formule des probabilités totales.}
  \colchunk{}
  \colplacechunks

  \colchunk{Formules de Bayes :
\begin{enumerate}
\item
si $A$ et $B$ sont deux événements tels que $P(A)>0$ et $P(B)>0$, alors
$$P(A\,|\,B)=\frac{P(B\,|\,A)\,P(A)}{P(B)}$$
\item
si $(A_i)_{1\leqslant i\leqslant n}$ est un système complet d'événements de probabilités
non nulles et si $B$ est un événement de probabilité non nulle, alors
$$P(A_j\,|\,B)= \frac{P(B\,|\,A_j)\;P(A_j)}{\displaystyle{\sum_{i=1}^n P(B\,|\,A_i)\;P(A_i)}}$$
\end{enumerate}}
  \colchunk{On donnera plusieurs applications issues de la vie courante.}
  \colplacechunks
\end{parcolumns}

\subsubsubsection{d) \'Evénements indépendants}
\begin{parcolumns}[rulebetween,distance=2.5cm]{2}
  \colchunk{Couple d'événements indépendants.}
  \colchunk{Si $P (B) > 0$, l'indépendance de $A$ et $B$ s'écrit $P (A|B) = P (A)$.}
  \colplacechunks

  \colchunk{Famille finie d'événements mutuellement indépendants.}
  \colchunk{L'indépendance deux à deux des événements d'une famille $(A_1,\dots,A_n)$ n'implique pas l'indépendance mutuelle si $n \geq 3$.}
  \colplacechunks
\end{parcolumns}


\bigskip
\begin{center}
 \textbf{Questions de cours}
\end{center}
Preuves formalisées (classement et récurrence) des dénombrements usuels:\newline
 produit cartésien,  nombre de parties d'un ensemble, nombre de fonctions, nombre de fonctions injectives, nombre de parties à $p$ éléments: classement et triangle de Pascal. \newline
Formule des probabilités composées.\newline
Formule des probabilités totales. \newline
Formules de Bayes. 


\bigskip
\begin{center}
 \textbf{Prochain programme}
\end{center}
Variables aléatoires sur un espace probabilisé fini.
\end{document}
