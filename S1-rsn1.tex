\subsection{Nombres réels et suites numériques (début)}
L’objectif de ce chapitre est de fonder rigoureusement le cours d’analyse relatif aux propriétés des nombres réels. Il convient
d’insister sur l’aspect fondateur de la propriété de la borne supérieure.\newline
Dans l’étude des suites, on distingue les aspects qualitatifs (monotonie, convergence, divergence) des aspects quantitatifs
(majoration, encadrement, vitesse de convergence ou de divergence).\newline
Il convient de souligner l’intérêt des suites, tant du point de vue pratique (modélisation de phénomènes discrets) que
théorique (approximation de nombres réels).

\subsubsubsection{a) Ensembles de nombres usuels}
\begin{parcolumns}[rulebetween,distance=\parcoldist]{2}
  \colchunk{Entiers naturels relatifs, nombres décimaux, rationnels, réels, irrationnels.}
  \colchunk{La construction de $\R$ est hors programme.}  
  \colplacechunks
  
  \colchunk{Partie entière.}
  \colchunk{Notation $\lfloor x \rfloor$.}
  \colplacechunks
  
  \colchunk{Approximations décimales d'un réel.}
  \colchunk{Valeurs décimales approchées à la précision $10^{-n}$ par défaut et par excès. \newline
  $\leftrightarrows$ I: représentation des réels en machine.}
  \colplacechunks
  
  \colchunk{Tout intervalle ouvert non vide rencontre $\Q$ et $\R\setminus \Q$.\newline
  Droite achevée $\overline{\R}$.}
  \colchunk{}
  \colplacechunks

\end{parcolumns}


\subsubsubsection{b) Propriété de la borne supérieure}
\begin{parcolumns}[rulebetween,distance=\parcoldist]{2}
  \colchunk{Borne supérieure (resp. inférieure) d'une partie non vide majorée (resp. minorée) de $\R$.}
  \colchunk{}  
  \colplacechunks
  
  \colchunk{Une partie $X$ de $\R$ est un intervalle si et seulement si pour tous $a,b\in X$ tels que $a\leq b$, $[a,b]\subset X$.}
  \colchunk{Une partie convexe de $\R$ est un intervalle.}
  \colplacechunks
  
\end{parcolumns}

\subsubsubsection{c) Généralités sur les suites réelles}
\begin{parcolumns}[rulebetween,distance=\parcoldist]{2}
  
  \colchunk{Suite majorée, minorée, bornée. Suite stationnaire, monotone, strictement monotone.}
  \colchunk{Une suite $(u_n)_{n\in\N}$ est bornée si et seulement si $(|u_n|)_{n\in\N}$ est majorée.}
  \colplacechunks
\end{parcolumns}


\subsubsubsection{d) Limite d'une suite réelle}
\begin{parcolumns}[rulebetween,distance=\parcoldist]{2}
  
  \colchunk{Limite finie ou infinie d'une suite.}
  \colchunk{Pour $l\in\overline{\R}$, notation $u_n\rightarrow l$.\newline
  Les définitions sont énoncées avec des inégalités larges. Lien avec la définition vue en Terminale.}
  \colplacechunks
  
  \colchunk{Suite qui converge vers 0, suite convergente, divergente.\newline
  Toute suite convergente est bornée.}
  \colchunk{Une suite $u$ est convergente ssi il existe $l\in \R$ tel que $u-l$ converge vers 0.}
  \colplacechunks

  \colchunk{Stabilité des inégalités larges par passage à la limite.}
  \colchunk{}
  \colplacechunks

  \colchunk{Unicité de la limite.}
  \colchunk{Notation $\lim u_n$. (notation déconseillée pendant les premières semaines)}
  \colplacechunks
  
  \colchunk{Opérations sur les suites convergentes: combinaison linéaire, produit, quotient. Opérations avec des suites admettant une limite infinie.}
  \colchunk{Produit d'une suite bornée et d'une suite de limite nulle.}
  \colplacechunks
    
  \colchunk{Si $(u_n)_{n\in \N}$ converge vers $l>0$, alors $u_n>0$ à partir d'un certain rang.}
  \colchunk{}
  \colplacechunks
  
  \colchunk{Théorème de convergence par encadrement. Théorèmes de divergence par minoration ou majoration.}
  \colchunk{}
  \colplacechunks
  
\end{parcolumns}
