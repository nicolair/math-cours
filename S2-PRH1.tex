
\subsection{Espaces préhilbertiens réels}
\begin{itshape}
La notion de produit scalaire a été étudiée d'un point de vue élémentaire
dans l'enseignement secondaire. Les objectifs  de ce chapitre sont les suivants :
\begin{itemize}
\item
généraliser cette notion et  exploiter, principalement à travers l'étude des projections orthogonales, l'intuition acquise dans des situations géométriques en dimension $2$ ou $3$ pour traiter des problèmes posés dans un contexte plus abstrait ;
\item
approfondir l'étude de la géométrie euclidienne du plan, notamment à travers l'étude des isométries vectorielles.
\end{itemize}
Le cours doit être illustré par de nombreuses figures. Dans toute la suite, $E$ est un espace vectoriel réel.
\end{itshape}

\subsubsubsection{a) Produit scalaire}
\begin{parcolumns}[rulebetween,distance=2.5cm]{2}
  \colchunk{Produit scalaire.}
  \colchunk{Notations $\langle x , y \rangle$, $(x|y)$, $x \cdot y$.}
  \colplacechunks

  \colchunk{Produit scalaire canonique sur $\R^n$,\newline
produit scalaire $(f |g)=\int_a^b fg$ sur $\mathcal{C} \big( [ a , b ] , \R \big)$.}
  \colchunk{}
  \colplacechunks
\end{parcolumns}

\subsubsubsection{b) Norme associée à un produit scalaire}
\begin{parcolumns}[rulebetween,distance=2.5cm]{2}
  \colchunk{Norme associée à un produit scalaire, distance.}
  \colchunk{}
  \colplacechunks

  \colchunk{Inégalité de Cauchy-Schwarz, cas d'égalité.}
  \colchunk{Exemples : sommes finies, intégrales.}
  \colplacechunks
  
  \colchunk{Inégalité triangulaire, cas d'égalité.}
  \colchunk{}
  \colplacechunks

  \colchunk{Formule de polarisation :
\begin{displaymath}
 2 \big< x , y \big> = \| x+y \|^2 - \| x \|^2 - \| y \|^2 
\end{displaymath}}
  \colchunk{}
  \colplacechunks
  \end{parcolumns}

\subsubsubsection{c) Orthogonalité}
\begin{parcolumns}[rulebetween,distance=2.5cm]{2}
  \colchunk{Vecteurs orthogonaux, orthogonal d'une partie.}
  \colchunk{Notation $X^\perp$.

  L'orthogonal d'une partie est un sous-espace.}
  \colplacechunks

  \colchunk{Famille orthogonale, orthonormale (ou orthonormée).}
  \colchunk{}
  \colplacechunks
  
  \colchunk{Toute famille orthogonale de vecteurs non nuls est libre.}
  \colchunk{}
  \colplacechunks

  \colchunk{Théorème de Pythagore.}
  \colchunk{}
  \colplacechunks
  
  \colchunk{Algorithme d'orthonormalisation de Schmidt.}
  \colchunk{}
  \colplacechunks
\end{parcolumns}

\subsubsubsection{d) Bases orthonormales}
\begin{parcolumns}[rulebetween,distance=2.5cm]{2}
  \colchunk{Existence de bases orthonormales dans un espace euclidien. Théorème de la base orthonormale incomplète.}
  \colchunk{}
  \colplacechunks

  \colchunk{Coordonnées dans une base orthonormale, expressions du produit scalaire et de la norme.}
  \colchunk{$\dbf$ PC et SI : mécanique et électricité.}
  \colplacechunks

\end{parcolumns}

\subsubsubsection{e) Projection orthogonale sur un sous-espace de dimension finie}
\begin{parcolumns}[rulebetween,distance=2.5cm]{2}
  \colchunk{Supplémentaire orthogonal d'un sous-espace de dimension finie.}
  \colchunk{En dimension finie, dimension de l'orthogonal.}
  \colplacechunks

  \colchunk{Projection orthogonale. Expression du projeté orthogonal dans une base orthonormale.}
  \colchunk{}
  \colplacechunks
  
  \colchunk{Distance d'un vecteur à un sous-espace. Le projeté orthogonal de $x$ sur $V$ est l'unique élément de $V$ qui minimise la distance de $x$ à $V$.}
  \colchunk{Notation $d (x , V)$.}
  \colplacechunks
\end{parcolumns}
