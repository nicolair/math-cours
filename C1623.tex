\input{courspdf.tex}
\debutcours{Fractions rationnelles}{ 0.4 \tiny{ le \today}}


La partie du programme intitulée \og Polynômes et fractions rationnelles\fg~ est présentée dans trois documents distincts \href{\baseurl C1622.pdf}{Polynômes}, \href{\baseurl C2160.pdf}{Arithmétique polynomiale} et \href{\baseurl C1623.pdf}{Fractions rationnelles} (ce document) .

Sauf mention explicite, toutes les fractions rationnelles sont à coefficients complexes.

\section{Définitions}
\subsection{Définition axiomatique}
\index{corps des fractions d'un anneau intègre}
Le corps noté $\K$ est $\R$ ou $\C$. Il existe un ensemble noté $\K(X)$ appelé corps des fractions rationnelles à coefficients dans $K$ et vérifiant une présentation axiomatique.
\begin{pa}
 \item[c'est plus gros que ] : $\K[X]$ est un sous-anneau de $\K(X)$.
 \item[c'est bien ] : $\K(X)$ est un corps.
 \item[c'est pas trop gros] : pour tout $F\in K(X)$ il existe $A$ et $B$ dans $K[X]$, le polynôme $B$ étant non nul tels que
 \begin{displaymath}
   F = A B^{-1}\text{ on notera } F = \frac{A}{B}
 \end{displaymath}
\end{pa}
Comme d'habitude, on ne cherchera pas à construire $\K(X)$ c'est à dire à fabriquer un objet mathématique vérifiant ces propriétés.\newline
D'après le premier axiome, tout polynôme est une fraction rationnelle et tout polynôme non nul est inversible dans $\K(X)$. Le corps $\K(X)$ doit contenir toutes les fractions $\frac{A}{B}$. Le troisième axiome indique justement qu'il ne contient que celles là.\newline
Les règles de calcul usuelles (y compris les manipulations de fractions) sont valables dans un corps. On notera en particulier
\begin{displaymath}
  \frac{A_1}{B_1} = \frac{A_2}{B_2} \Leftrightarrow  \frac{A_1}{B_1} - \frac{A_2}{B_2} = 0 \Leftrightarrow
  \frac{A_1B_2 - B_1A_2}{B_1B_2} = 0 \Leftrightarrow A_1B_2 - B_1A_2 =0
\end{displaymath}
\index{réprésentants d'une fraction}Lorsque $F = \frac{A}{B}$ avec $A$ et $B$ deux polynômes ($B$ non nul), on dit que $(A,B)$ est un \emph{représentant de la fraction} 

\subsection{Propriétés}
\index{forme irréductible d'une fraction rationnelle}
\begin{prop}[Forme irréductible d'une fraction rationnelle.]
  Pour toute $F\in \K(X)$ non nulle, il existe des polynômes non nuls et premiers entre eux $A$, $B$ tels que $F = \frac{A}{B}$. On dit que  $(A,B)$ est un couple représentant irréductible de $F$. Tout autre couple représentant irréductible est de la forme $(\lambda A, \frac{1}{\lambda}B)$ avec $\lambda \in \K^*$.
\end{prop}
\begin{demo}
  Soit $(A_0,B_0)$ un couple représentant $F$ et $D= A_0 \wedge B_0$. Il existe alors $A$ et $B$ premiers entre eux tels que $A0=DA$, $B_0=DB$. Alors:
\begin{displaymath}
  F = \frac{A_0}{B_0}=\frac{DA}{DB}=\frac{A}{B}.
\end{displaymath}
Si $\frac{A}{B}$ et $\frac{A_1}{B_1}$ sont deux représentants irréductibles d'une même fraction, alors $AB_1=A_1 B$. Par le théorème de Gauss, $A\wedge B = 1$ entraine $A$ divise $A_1$ et $A_1 \wedge B_1=1$ entraine $A_1$ divise $A$. Les polynômes $A$ et $A_1$ se divisent mutuellement donc ils sont égaux à la multiplication près par un élément non nul du corps.
\end{demo}

\index{degré d'une fraction rationnelle} 
\begin{prop}[Définition du degré.]
  Soit $F \in \K(X)$. Pour tous les couples $(A,B)$ représentant la fraction $F$, la valeur de $\deg(A)-\deg(B)$ est la même. Cette valeur commune est appelée le \emph{degré} de la fraction $F$.
\end{prop}
\begin{demo}
  Soit $(A_1,B_1)$ et $(A_2,B_2)$ deux couples représentant la fraction. Alors
\begin{multline*}
  A_1B_2 - B_1A_2 = 0 \Rightarrow A_1B_2 = B_1A_2 \Rightarrow \deg(A_1)+\deg(B_2) = \deg(B_1)+\deg(A_2)\\
  \Rightarrow \deg(A_1)-\deg(B_1) = \deg(A_2) - \deg(B_2)
\end{multline*}
\end{demo}
\begin{rem}
  La fraction $\frac{X^4 + X +1}{X^4 + X^3 + 2}$ est de degré $0$. Il est inutile de prendre la forme irréductible pour évaluer le degré d'une fraction.
\end{rem}

\index{valuation d'une fraction rationnelle}
\begin{defi}[Définition de la valuation en $X-a$]
  Soit $F\in \K(X)$ non nulle et $a\in \K$. Il existe $m \in \Z$ et $A$, $B$ dans $\K[X]$ non nuls tels que $F = (X-a)^m \,\frac{A}{B}$ avec $\widetilde{A}(a) \neq 0$ et $\widetilde{B}(a) \neq 0$. Cet entier $m$ est appelé la \emph{valuation} de $F$ en $X-a$, il est noté $v_a(F)$.
\end{defi}

\begin{prop}
 Soit $F$ et $G$ dans $K(X)$ non nulles et $a \in \K$.
\begin{align*}
 &\deg(FG)=\deg(F)+\deg(G) & &\deg(F+G)\leq \max(\deg(F),\deg(G))  \hspace{0.5cm}\left( \text{égalité si } \deg(F) \neq \deg(G)\right). \\
 &v_a(FG)=v_a(F)+v_a(G) & &v_a(F+G)\geq \min(v_a(F),v_a(G)) \hspace{0.5cm}\left( \text{égalité si } v_a(F) \neq v_a(G))\right) .
\end{align*}
\end{prop}
\begin{demo}
 Introduisons $A_F$, $B_F$, $A_G$, $B_G$ dans $\C[X]$ avec $B_F, B_G \neq 0$ tels que $F = \frac{A_F}{B_F}$, $G = \frac{A_G}{B_G}$.
\begin{multline*}
F G = \frac{A_F A_G }{B_F B_G}
\Rightarrow \deg(FG) = \deg(A_F A_G )-\deg(B_F B_G) \\
= \deg(A_F)-\deg(B_F) + \deg(A_G )-\deg(B_G)
= \deg(F) + \deg(G).
\end{multline*}
\begin{multline*}
F + G = \frac{A_F B_G + B_F A_G}{B_F B_G}
\Rightarrow \deg(F + G) = \deg(A_F B_G + B_F A_G)-\deg(B_F B_G) \\
\leq \max\left( \deg(A_F B_G), \deg(B_F A_G)\right) - \deg(B_F B_G)\\
= \max\left( \deg(A_F B_G)  - \deg(B_F B_G), \deg(B_F A_G)  - \deg(B_F B_G)\right)\\
= \max\left( \deg(A_F)  - \deg(B_F), \deg(A_G)  - \deg(B_G)\right)
= \max\left( \deg(F), \deg(G)\right)
\end{multline*}
L'égalité avec le max des degrés se produit lorsque 
\[
 \deg(A_F B_G) = \deg(B_F A_G) \Leftrightarrow \deg(A_F) - \deg(B_F) = \deg(A_G) - \deg(B_G) \Leftrightarrow \deg(F) = \deg(G).
\]
Pour la valuation en $a$, introduisons $F_1$ et $G_1$ dans $\C(X)$ pour lesquelles $a$ n'est ni un pôle ni un zéro telles que 
\[
 F=(X-a)^{v_a(F)}F_1, \hspace{1cm} G=(X-a)^{v_a(G)}G_1.
\]
Alors $FG = (X-a)^{v_a(F)+v_a(G)}F_1G_1$ où $a$ n'est ni un pôle ni un zéro de$F_1G_1$ donc $v_a(FG) = v_a(F) + v_a(G)$.
Supposons $v_a(F) \leq v_a(G)$ ($F$ et $G$ jouent des rôles symétriques). 
\[
 F + G 
 = (X-a)^{v_a(F)}\left( \underset{ = H}{\underbrace{F_1 + (X-a)^{v_a(G) - v_a(F)}G_1}}\right) 
\]
Comme $a$ n'est un pôle ni de $F_1$ ni de $G_1$, il n'est pas non plus un pôle de $H$ donc 
\[
 v_a(F+G)\geq v_a(F) \geq \min(v_a(F),v_a(G)).
\]
Si $v_a(F) < v_a(G)$ alors $v_a(G) - v_a(F) >0$ donc $\widetilde{H}(a) = \widetilde{F_1}(a) \neq 0$ donc $v_a(F+G) = v_a(F) = \min(v_a(F),v_a(G))$.
\end{demo}

\index{pôle d'une fraction rationnelle} \index{zéro d'une fraction rationnelle}
\begin{defi}[Pôle, zéro, multiplicité]
Soit $F \in \K(X)$ non nulle et $a\in \K$.\newline
On dit que $a$ est un \emph{zéro} de $F$ si et seulement si $v_a(F) > 0$. La \emph{multiplicité} de $a$ comme zéro de $F$ est alors $v_a(F)$.\newline
On dit que $a$ est un \emph{pôle} de $F$ si et seulement si $v_a(F) < 0$. La \emph{multiplicité} de $a$ comme zéro de $F$ est alors $-v_a(F)$. 
\end{defi}

Si $F \in \C(X)$ et $a \in \C$ \emph{n'est pas un pôle} de $F$, on peut substituer $a$ à $X$ dans $F$.
\index{fonction rationnelle}
\begin{defi}[fonction rationnelle]
Soit $F = \frac{A}{B}\in \C(X)$ avec $A, B \in \C[X]$, $B\neq 0$ et $\mathcal{P}$ l'ensemble des pôles de $F$. La fonction rationnelle attachée à $F$ est la fonction de $\C \setminus \mathcal{P}$ dans $\C$ notée $\widetilde{F}$ définie par:
\[
 \forall a \in \C \setminus \mathcal{P}, \hspace{0.5cm} \widetilde{F}(a) = \frac{\widetilde{A}(a)}{\widetilde{B}(a)}.
\]
\end{defi}
\begin{rem}
  Dans n'importe quelle $F\in \K(X)$, on peut substituer à $X$ n'importe quelle $G\in K(X)$. La fraction obtenue est notée $\widehat{F}(G)$ ou $F \circ G$.
\[
  F = \frac{X + 1}{X^2 - 1},\; G = \frac{1}{X},\; \widehat{F}(\frac{1}{X}) = \frac{(1 + X)X}{1-X^2} .
\]
\end{rem}

\subsection{Exercice traité en classe}
\index{dérivée d'une fraction rationnelle}
On veut étendre l'opérateur de dérivation de $\K[X]$ à $K(X)$ par: 
\[
\forall (A,B)\in \K[X]^2,\; B\neq 0\;, \left( \frac{A}{B}\right)' = \frac{A' B - AB'}{B^2}. 
\]
\begin{enumerate}
 \item Montrer que $\frac{A_1}{B_1} = \frac{A_2}{B_2} \Rightarrow \left( \frac{A_1}{B_1}\right)'  = \left( \frac{A_2}{B_2}\right)'$.\newline Justifier que l'on a bien défini une fonction de dérivation sur l'ensemble des fractions rationnelles.
 \item Vérifier les formules suivantes
\[
 \forall (F,G) \in \K(X)^2, \; (F+G)' = F' + G', \; (FG)' = F'G + FG'.
\]
On vérifie aussi la formule de Leibniz.
 \item \'Etude du degré de la dérivée.
\begin{enumerate}
 \item Montrer que si $F$ est une fraction rationnelle de degré non nul alors $\deg(F')=\deg(F)-1$. 
 \item Montrer que si $F$ est une fraction rationnelle de degré $0$ qui n'est pas un complexe, alors $\deg(F')\leq2$.
 \item Montrer qu'une fraction de degré $-1$ n'est pas la dérivée d'une fraction rationnelle.
\end{enumerate}
\end{enumerate}
Détaillons seulement une preuve du premier point. Notons
\[
  T = A_1B_2 - B_1A_2, \; M =(A_1'B_1-A_1B_1')B_2^2 - (A_2'B_2-A_2B_2')B_1^2.
\]
Il s'agit de montrer que $T= 0 \Rightarrow M = 0$. Supposons $T=0$ et réarrangeons $M$.
\begin{multline*}
  M = (A_1'B_2)(B_1B_2) -\underset{=B_1A_2}{\underbrace{(A_1B_2)}}(B_1'B_2) - (B_1A_2')(B_1B_2) + \underset{=A_1B_2}{\underbrace{(B_1A_2)}}(B_1B_2')\\
  = (B_1B_2)\left(A_1'B_2 - A_2B_1' - B_1A_2' + A_1B_2' \right)
  = (B_1B_2)M' = 0.
\end{multline*}


\section{\'Etude locale - Décompositions}
Le corps de base est $\C$. Comme tout polynôme non constant admet au moins une racine, une fraction rationnelle est un polynôme si et seulement si elle n'a aucun pôle.

\subsection{Parties polaires}
\index{partie polaire d'une fraction rationnelle}
La partie polaire en $a$ d'une fraction rationnelle dont $a$ est un pôle est une fraction rationnelle qui concentre tout ce qui fait que $a$ est un pôle.
\begin{prop}[Existence et unicité des parties polaires]
 Soit $F \in \C(X), F \neq 0$ et $a$ un de ses pôles de multiplicité $m$. Il existe une unique fraction rationnelle notée $\Pi_a$ (appelée partie polaire en $a$) telle que :
\begin{displaymath}
 a \text{ est le seul pôle de } \Pi_a \text{ sa multiplicité est } m,\hspace{0,5cm}
 \deg(\Pi_a) <0 ,\hspace{0,5cm}
 a \text{ n'est pas un pôle de } F-\Pi_a 
\end{displaymath}
Il existe des nombres complexes $\lambda_1,\cdots\lambda_m$ tels que
\begin{displaymath}
 \Pi_a = 
\frac{\lambda_1}{(X-a)^m}+\frac{\lambda_2}{(X-a)^{m-1}}+\cdots +\frac{\lambda_i}{(X-a)^{m+1-i}}+\cdots+ \frac{\lambda_m}{(X-a)}.
\end{displaymath}
\end{prop}
\begin{demo}
 Cette proposition est démontrée en \ref{demo}\ref{ppol}.
\end{demo}
\begin{rems}
\begin{itemize}
 \item Si $a$ n'est pas un pôle de $F$, sa partie polaire est nulle.
 \item \index{coefficient facile} \index{résidu} J'appelle $\lambda_1$ le coefficient \og facile \fg~ (attention cette dénomination n'est pas utilisée en dehors de la classe).
 \item Le coefficient $\lambda_m$ est appelé \emph{résidu} (dénomination universelle). En général c'est le plus difficile à calculer.
\end{itemize}
\end{rems}

\index{éléments simples}
\begin{defi}
 Un éléments simple (de première espèce) est une fraction rationnelle de la forme
\begin{displaymath}
 \frac{\lambda}{(X-a)^m}\text{ où } \lambda\in \C,\; a\in\C,\; m\in \N^*.
\end{displaymath}
\end{defi}

\subsection{Partie entière}
\index{partie entière d'une fraction rationnelle}
\begin{prop}
 Soit $F \in \C(X), F \neq 0$. Il existe un unique polynôme noté $\Pi_\infty$ (appelée partie entière) tel que 
\begin{displaymath}
 \deg(F-\Pi_\infty) < 0
\end{displaymath}
De plus, $\Pi_\infty$ est le quotient de la division euclidienne du numérateur de $F$ par son dénominateur. Il est nul lorsque $\deg(F)<0$, sinon son degré est égal à celui de $F$.
\end{prop}
\begin{demo}
 Cette proposition est démontrée en \ref{demo}\ref{pent}.
\end{demo}
La partie entière (privée de son terme de degré $0$) peut être regardée comme une partie polaire en l'infini. En effet elle rassemble tout ce qui \og diverge\fg~ à l'infini comme la partie polaire en $a$ rassemble tout ce qui \og diverge\fg~ en $a$.

\subsection{Décomposition en éléments simples}
\begin{prop}
 Toute fraction rationnelle est la somme de sa partie entière et de ses parties polaires.
\end{prop}
\begin{demo}
 Notons $G$ la somme des parties polaires et de la partie entière de $F$ et $H=F-G$. Pour tout pôle $a$,
\begin{displaymath}
 H = (F-\Pi_a) + R_a
\end{displaymath}
où $R_a$ est la somme des parties polaires pour les pôles autres que $a$ (y compris $\infty$). D'après les propriétés des parties polaires, $a$ n'est un pôle ni de $F-\Pi_a$ ni de $R_a$. Comme ceci est valable pour tous les pôles de $F$ et que les autres nombres complexes ne sont évidemment pas non plus des pôles, on en déduit que $H$ \emph{n'admet aucun} pôle. Si $H$ est non nulle, elle ne peut être qu'un polynôme. Mais ceci aussi lui est refusé. en effet:
\begin{displaymath}
 H = (F-\Pi_\infty) + R_\infty
\end{displaymath}
D'après la définition de la partie entière et des parties polaires, $F-\Pi_\infty$ et $R_\infty$ sont des fractions rationnelles de degré strictement négatif. Leur somme ne peut être un polynôme que si celui ci est nul. 
\end{demo}

Toute fraction est la somme d'un polynôme et d'\emph{éléments simples}.
Soit $F\in \C(X), F\neq 0$. Ses pôles sont $a_1, \cdots, a_p$ avec les multiplicités $m_1,\cdots m_p$. Alors
\begin{multline*}
 F = \Pi_\infty 
 + \frac{\lambda_{1,1}}{(X-a_1)^{m_1}} + \frac{\lambda_{1,2}}{(X-a_1)^{m_1-1}} + \cdots + \frac{\lambda_{1,m_1}}{(X-a_1)} \\
 + \frac{\lambda_{2,1}}{(X-a_2)^{m_2}} + \frac{\lambda_{2,2}}{(X-a_2)^{m_2-1}} + \cdots + \frac{\lambda_{2,m_2}}{(X-a_2)} \\
 + \cdots \\
 + \frac{\lambda_{p,1}}{(X-a_p)^{m_p}} + \frac{\lambda_{p,2}}{(X-a_p)^{m_p-1}} + \cdots + \frac{\lambda_{p,m_p}}{(X-a_p)}.
\end{multline*}
où les $\lambda_{i,j}\in \C$.

\subsection{Décomposition en éléments simples réels}
\index{éléments simples réels (deuxième espèce)}
\begin{defi}
 Un éléments simple réel de deuxième espèce  est une fraction rationnelle de la forme
\begin{displaymath}
 \frac{u X + v}{A^m}\hspace{1cm} \text{ avec } u \in \R,\; v \in\R,\; A\in \R[X] \text{ de degré $2$ sans racine réelle}, \; m\in \N^*.
\end{displaymath}
\end{defi}
\begin{prop}
 Soit $F\in \R(X), F\neq 0$, soit $a_1,\cdots, a_p$ ses pôles réels de multiplicités $m_1,\cdots, m_p$, soit $z_1,\cdots, z_s$ ses pôles complexes non réels de multiplicités $n_1,\cdots, n_s$. Alors
\begin{multline*}
 F = \Pi_\infty 
 + \left( \frac{\lambda_{1,1}}{(X-a_1)^{m_1}}  + \cdots + \frac{\lambda_{1,m_1}}{(X-a_1)} \right) 
 + \cdots 
 + \left( \frac{\lambda_{p,1}}{(X-a_p)^{m_p}} + \cdots + \frac{\lambda_{p,m_p}}{(X-a_p)}\right)  \\
 + \left( \frac{u_{1,1} + v_{1,1}X}{(X^2-2\Re(z_1)X + |z_1|)^{n_1}} + \cdots + \frac{u_{1,n_1}+ v_{1,n_1}X}{(X^2-2\Re(z_1)X + |z_1|)}\right)  \\
 + \cdots 
 + \left( \frac{u_{s,1} + v_{s,1}X}{(X^2-2\Re(z_s)X + |z_s|)^{n_s}} + \cdots + \frac{u_{s,m_s}+ v_{s,m_s}X}{(X^2-2\Re(z_s)X + |z_s|)}\right) .
\end{multline*}
avec les $\lambda_{i,j}, u_{i,j}, v_{i,j}$ réels.
\end{prop}
\begin{demo}
 On peut démontrer ce résultat en commençant par regrouper les éléments simples complexes conjugués ou par une voie plus arithmétique. Cette démonstration n'est pas détaillée ici. Ce résultat est admis. 
\end{demo}


\section{Démonstrations} \label{demo}
\subsection{Supertildation}
\index{supertildation}
Soit $a$ un pôle de multiplicité $m$ d'une fraction rationnelle $F$. On ne peut pas prendre la valeur de $F$ en $a$ mais il est possible de définir une opération qui associe un complexe au triplet $(F,a,m)$. Il suffit de multiplier $F$ par $(X-a)^m$ \emph{avant} de substituer $a$ à $X$ car $a$ n'est plus alors un pôle de la fraction obtenue. 
J'appelle \emph{supertildation} cette opération\footnote{attention cette notation et ce vocabulaire sont spécifiques à ce cours.}
\begin{displaymath}
 \widetilde{\widetilde{F}}(a) = \widetilde{(X-a)^mF}(a)
\end{displaymath}
Attention, le $m$ est la multiplicité du pôle $a$ dans la fraction. Il dépend donc des deux à la fois. Cette supertildation est un outil fondamental aussi bien théorique que pratique de la décomposition des fractions de $\C(X)$.

\subsection{Parties polaires (algorithmique)} \label{ppol}
Soit $a\in \C$ un pôle de $F$ de multiplicité $m$. Démontrons l'existence, l'unicité et la forme de la partie polaire relative à $a$ par analyse-synthèse.

\textbf{Analyse.}\newline
Supposons qu'il existe une fraction $\Pi_a$ vérifiant les conditions. Alors $\deg(\Pi_a) <0$ et $a$ est son seul pôle. Il existe donc $\Lambda \in \C[X]$ tel que 
\[
 \Pi_a = \frac{\Lambda}{(X-a)^m} \; \text{ avec } \; \deg(\Lambda) \leq m -1 \; \text{ et } \; \widetilde{\Lambda}(a) \neq 0.
\]
En écrivant $\Lambda$ à l'aide de la formule de Taylor en $a$ puis en divisant par $(X-a)^m$, on obtient la formule annoncée:
\begin{multline*}
 \Lambda = \widetilde{\Lambda}(a) + \frac{\widetilde{\Lambda'}(a)}{1!}(X-a) + \cdots + \frac{\widetilde{\Lambda^{(m-1)}}(a)}{(m-1)!}(X-a)^{m-1} \\
\Rightarrow
\Pi_a =\frac{\widetilde{\Lambda}(a)}{(X-a)^m} + \frac{\widetilde{\Lambda'}(a)}{1!(X-a)^{m-1}} + \cdots + \frac{\widetilde{\Lambda^{(m-1)}}(a)}{(m-1)!(X-a)} \\
  = \frac{\lambda_1}{(X-a)^m}+\frac{\lambda_2}{(X-a)^{m-1}}+\cdots +\frac{\lambda_i}{(X-a)^{m+1-i}}+\cdots+ \frac{\lambda_m}{(X-a)}.
\end{multline*}
On peut remarquer que le coefficient facile est $\lambda_1 = \widetilde{\Lambda}(a) \neq 0$.\newline
En fait, comme $a$ n'est pas un pôle de $F-\Pi_a$, on a  $\lambda_1 = \widetilde{\widetilde{F}}(a)$. Posons alors
\[
 F_1 = F -\frac{\lambda_1}{(X-a)^m}
\]
et montrons que la multiplicité de $a$ comme pôle de $F_1$ est $\leq m-1$.\newline
Par définition de la multiplicité, il existe $A$ et $B$ dans $\C[X]$ tels que $F = \frac{A}{(X-a)^m\,B}$  avec $\widetilde{A}(a)\neq 0$, $\widetilde{B}(a)\neq 0$. Donc
\[
 \lambda_1 = \frac{\widetilde{A}(a)}{\widetilde{B}(a)}\\
 \Rightarrow F_1 = \frac{\widetilde{B}(a)A-\widetilde{A}(a)B}{\widetilde{B}(a)(X-a)^mB}=\frac{(X-a)A_1}{\widetilde{B}(a)(X-a)^mB}
=\frac{A_1}{\widetilde{B}(a)(X-a)^{m-1}B}
\]
avec $A_1\in \C[X]$ car $a$ est clairement racine du numérateur. On en déduit que la multiplicité de $a$ comme pôle de $F_1$ est inférieure ou égale à $m-1$. \newline
On peut recommencer la supertildation.
\begin{align*}
 &\lambda_2 = \widetilde{\widetilde{F_1}}(a),& &F_2 = F_1 -\frac{\lambda_2}{(X-a)^{m-1}},   &\text{ multiplicité de $a$ comme pôle de $F_2$ } \leq m-1\\
 &\lambda_3 = \widetilde{\widetilde{F_2}}(a),& &F_3 = F_2 -\frac{\lambda_3}{(X-a)^{m-2}}, &\text{ multiplicité de $a$ comme pôle de $F_3$ } \leq m-3\\
&  & \vdots & & \\
&\lambda_m = \widetilde{\widetilde{F_{m-1}}}(a),& &F_m = F_{m-1} -\frac{\lambda_m}{(X-a)}, &\text{ multiplicité de $a$ comme pôle de $F_m$} \leq 0
\end{align*}
Cette suite de relations est justifiée par le fait que la multiplicité de $a$ comme pôle de $F_i$ décroît à chaque étape exactement comme pour la première. 

\textbf{Synthèse.}\newline
Définissons des complexes $\lambda_i$ par les relations précédentes.
\begin{align*}
 &\lambda_1 = \widetilde{\widetilde{F}}(a)\neq 0,& &F_1 = F -\frac{\lambda_1}{(X-a)^m},   &\text{ multiplicité de $a$ comme pôle de $F_1$ } \leq m-1\\
 &\lambda_2 = \widetilde{\widetilde{F_1}}(a),& &F_2 = F_1 -\frac{\lambda_2}{(X-a)^{m-1}}, &\text{ multiplicité de $a$ comme pôle de $F_2$ } \leq m-2\\
&  & \vdots & & \\
&\lambda_m = \widetilde{\widetilde{F_{m-1}}}(a),& &F_m = F_{m-1} -\frac{\lambda_m}{(X-a)}, &\text{ multiplicité de $a$ comme pôle de $F_m$ } \leq 0
\end{align*}
et définissons $\Pi_a$ par
\[
 \Pi_a = \frac{\lambda_1}{(X-a)^m}+\frac{\lambda_2}{(X-a)^{m-1}}+\cdots +\frac{\lambda_i}{(X-a)^{m+1-i}}+\cdots+ \frac{\lambda_m}{(X-a)}.
\]
Alors $\deg(\Pi_a) < 0$ et $a$ est bien le seul pôle de $\Pi_a$. Sa multiplicité est $m$ et $a$ n'est pas un pôle de $F - \Pi_a = F_m$ puisque la multiplicité de $a$ comme pôle de $F_m$ est négative ou nulle.

\begin{rem}
 Cette méthode est un moyen pratique de calculer une décomposition en éléments simples.
\end{rem}

\subsection{Partie entière}
\label{pent}

\textbf{Analyse.}\newline
Soit $\Pi_\infty \in \C[X]$ tel que $\deg(F - \Pi_\infty) < 0$. Il existe alors $A$ et $B$ dans $\C[X]$, $B\neq 0$ tels que
\[
 F= \frac{A}{B} 
 \Rightarrow F - \Pi_\infty = \frac{A - B \Pi_\infty}{B}.\hspace{0.5cm}
 \deg(F - \Pi_\infty) < 0 \Rightarrow \deg(A - B \Pi_\infty) < \deg(B).
\]
Si $R = A - B \Pi_\infty$, alors $A = B \Pi_\infty + R$ est la division euclidienne de $A$ par $B$ car $\deg(R) < \deg(B)$. On en déduit que $\Pi_\infty$ est le quotient de la division euclidienne de $A$ par $B$ ce qui assure l'unicité de la partie entière. 

\textbf{Synthèse.}\newline
Soit $F=\frac{A}{B}$, $\Pi_\infty$ le quotient de la division euclidienne de $A$ par $B$ et $R$ son reste. Alors
\[
 F = \frac{B\Pi_\infty + R}{B} = \Pi_\infty + \frac{R}{B} \text{ avec }
 \deg(\frac{R}{B}) = \deg(R) - \deg(B) < 0.
\]
Ce qui assure que $\Pi_\infty$ est la partie entière.
\begin{rem}
 La partie entière d'une fraction rationnelle se calcule en général par cette méthode comme le quotient dans la division euclidienne du numérateur par le dénominateur.
\end{rem}

\subsection{Parties polaires (arithmétique)}
Démonstration de l'existence et de l'unicité de la partie polaire.
\begin{demo}
La fraction $F$ est de la forme $F=\frac{A}{(X-a)^mQ}$.
D'après les conditions imposées, $\Pi_a$ doit être de la forme $\frac{U}{(X-a)^m}$ avec $U$ un polynôme de degré strictement inférieur à $m$ tel que $\widetilde{U}(a)\neq0$. Alors, le reste $R=F-\Pi_a$ doit vérifier
\begin{displaymath}
 R= \frac{A}{(X-a)^mQ} - \frac{U}{(X-a)^m} = \frac{W}{(X-a)^mQ}
\end{displaymath}
et $(X-a)^m$ doit diviser $W$ car $a$ n'est pas un  pôle du reste. Il existe donc un polynôme $V$ tel que
\begin{displaymath}
 A = UQ + V(X-a)^m \text{ avec } \deg(U)< m
\end{displaymath}
On reconnait une \index{équation de Bezout} \href{\baseurl C5546.pdf}{équation de Bezout} aux inconnues $U$ et $V$ avec $Q$ et $(X-a)^m$ premiers entre eux car $\widetilde{Q}(a)\neq 0$. Elle admet un unique couple solution tel que $\deg(U)<m=\deg(X-a)^m$. Ceci démontre l'existence et l'unicité de la partie polaire.\\
\end{demo}

\section{Pratique - Compléments}

\subsection{Développements classiques}
Décomposition de $\frac{P'}{P}$.\newline
Si les racines de $P$ sont $z_1,\cdots, z_p$ avec les multiplicités $m_1,\cdots, m_p$,
\[
 \frac{P'}{P} = \frac{m_1}{X - z_1} + \cdots + \frac{m_p}{X - z_p}.
\]


\subsection{Méthode générale}
La fraction doit être donnée sous une forme factorisée. Tous les pôles et leurs multiplicités sont connus.
\begin{enumerate}
 \item \'Ecrire la décomposition avec des coefficients indéterminés.
 \item Exploiter l'unicité avec des symétries.
 \item Calculer les coefficients faciles
 \item S'il reste des coefficients à calculer, former de nouvelles relationset résoudre un système d'équations.
\end{enumerate}

\subsection{Exemples de symétries}

\subsection{Calculer les coefficients faciles}
Supertildation
\index{dérivée du dénominateur pour un pôle simple}
\index{astuce de la dérivée}
\begin{prop}[astuce de la dérivée]
Si $a$ est un pôle simple de $F=\frac{A}{B}$, le coefficient dans la partie polaire de $F$ en $a$ est $\frac{\widetilde{A}(a)}{\widetilde{B'}(a)}$. 
\end{prop}

\subsection{Former de nouvelles relations}
Multiplier par $X$ et \og tilder\fg~ à l'infini. Prendre une valeur qui n'est pas un pôle.

\subsection{Cas d'une multiplicité élevée}
Utilisation d'un développement limité.

\subsection{\'Eléments simples de deuxième espèce}
Supertider en un pôle complexe, les coefficients sont réels.

\end{document}
