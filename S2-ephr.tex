\subsection{Espaces préhilbertiens réels}
\begin{itshape}
La notion de produit scalaire a été étudiée d'un point de vue élémentaire
dans l'enseignement secondaire. Les objectifs  de ce chapitre sont les suivants :
\begin{itemize}
\item
généraliser cette notion et  exploiter, principalement à travers l'étude des projections orthogonales, l'intuition acquise dans des situations géométriques en dimension $2$ ou $3$ pour traiter des problèmes posés dans un contexte plus abstrait ;
\item
approfondir l'étude de la géométrie euclidienne du plan, notamment à travers l'étude des isométries vectorielles.
\end{itemize}
Le cours doit être illustré par de nombreuses figures. Dans toute la suite, $E$ est un espace vectoriel réel.
\end{itshape}

\subsubsubsection{a) Produit scalaire}
\begin{parcolumns}[rulebetween,distance=2.5cm]{2}
  \colchunk{Produit scalaire.}
  \colchunk{Notations $\langle x , y \rangle$, $(x|y)$, $x \cdot y$.}
  \colplacechunks

  \colchunk{Produit scalaire canonique sur $\R^n$,\newline
produit scalaire $(f |g)=\int_a^b fg$ sur $\mathcal{C} \big( [ a , b ] , \R \big)$.}
  \colchunk{}
  \colplacechunks
\end{parcolumns}

\subsubsubsection{b) Norme associée à un produit scalaire}
\begin{parcolumns}[rulebetween,distance=2.5cm]{2}
  \colchunk{Norme associée à un produit scalaire, distance.}
  \colchunk{}
  \colplacechunks

  \colchunk{Inégalité de Cauchy-Schwarz, cas d'égalité.}
  \colchunk{Exemples : sommes finies, intégrales.}
  \colplacechunks
  
  \colchunk{Inégalité triangulaire, cas d'égalité.}
  \colchunk{}
  \colplacechunks

  \colchunk{Formule de polarisation :
\begin{displaymath}
 2 \big< x , y \big> = \| x+y \|^2 - \| x \|^2 - \| y \|^2 
\end{displaymath}}
  \colchunk{}
  \colplacechunks
  \end{parcolumns}

\subsubsubsection{c) Orthogonalité}
\begin{parcolumns}[rulebetween,distance=2.5cm]{2}
  \colchunk{Vecteurs orthogonaux, orthogonal d'une partie.}
  \colchunk{Notation $X^\perp$.

  L'orthogonal d'une partie est un sous-espace.}
  \colplacechunks

  \colchunk{Famille orthogonale, orthonormale (ou orthonormée).}
  \colchunk{}
  \colplacechunks
  
  \colchunk{Toute famille orthogonale de vecteurs non nuls est libre.}
  \colchunk{}
  \colplacechunks

  \colchunk{Théorème de Pythagore.}
  \colchunk{}
  \colplacechunks
  
  \colchunk{Algorithme d'orthonormalisation de Schmidt.}
  \colchunk{}
  \colplacechunks
\end{parcolumns}

\subsubsubsection{d) Bases orthonormales}
\begin{parcolumns}[rulebetween,distance=2.5cm]{2}
  \colchunk{Existence de bases orthonormales dans un espace euclidien. Théorème de la base orthonormale incomplète.}
  \colchunk{}
  \colplacechunks

  \colchunk{Coordonnées dans une base orthonormale, expressions du produit scalaire et de la norme.}
  \colchunk{$\dbf$ PC et SI : mécanique et électricité.}
  \colplacechunks
  
  \colchunk{Produit mixte dans un espace euclidien orienté.}
  \colchunk{Notation $\big[ x_1 , \ldots , x_n \big]$.

  Interprétation géométrique en termes de volume orienté, effet d'une application linéaire.}
  \colplacechunks
\end{parcolumns}

\subsubsubsection{e) Projection orthogonale sur un sous-espace de dimension finie}
\begin{parcolumns}[rulebetween,distance=2.5cm]{2}
  \colchunk{Supplémentaire orthogonal d'un sous-espace de dimension finie.}
  \colchunk{En dimension finie, dimension de l'orthogonal.}
  \colplacechunks

  \colchunk{Projection orthogonale. Expression du projeté orthogonal dans une base orthonormale.}
  \colchunk{}
  \colplacechunks
  
  \colchunk{Distance d'un vecteur à un sous-espace. Le projeté orthogonal de $x$ sur $V$ est l'unique élément de $V$ qui minimise la distance de $x$ à $V$.}
  \colchunk{Notation $d (x , V)$.}
  \colplacechunks
\end{parcolumns}

\subsubsubsection{f) Hyperplans affines d'un espace euclidien}
\begin{parcolumns}[rulebetween,distance=2.5cm]{2}
  \colchunk{Vecteur normal à un hyperplan affine d'un espace euclidien. Si l'espace est orienté, orientation d'un hyperplan par un vecteur normal.}
  \colchunk{Lignes de niveau de $M \mapsto \overrightarrow{AM} \cdot \vec{n}$.}
  \colplacechunks

  \colchunk{\'Equations d'un hyperplan affine dans un repère orthonormal.}
  \colchunk{Cas particuliers de $\R^2$ et $\R^3$.}
  \colplacechunks
  
  \colchunk{Distance à un hyperplan affine défini par un point $A$ et un vecteur normal unitaire $\vec{n}$ : $\quad \big| \overrightarrow{AM} \cdot \vec{n} \big|$.}
  \colchunk{Cas particuliers de $\R^2$ et $\R^3$.}
  \colplacechunks
\end{parcolumns}

\subsubsubsection{g) Isométries vectorielles d'un espace euclidien}
\begin{parcolumns}[rulebetween,distance=2.5cm]{2}
  \colchunk{Isométrie vectorielle (ou automorphisme orthogonal) : définition par la linéarité et la conservation des normes, caractérisation par la conservation du produit scalaire, caractérisation par l'image d'une base orthonormale.}
  \colchunk{}
  \colplacechunks
  
  \colchunk{Symétrie orthogonale, réflexion.}
  \colchunk{}
  \colplacechunks


  \colchunk{Groupe orthogonal.}
  \colchunk{Notation $\mbox{O} (E)$.}
  \colplacechunks
\end{parcolumns}

\subsubsubsection{h) Matrices orthogonales}
\begin{parcolumns}[rulebetween,distance=2.5cm]{2}
  \colchunk{Matrice orthogonale : définition ${}^t\!\! A A = I_n$, caractérisation par le caractère orthonormal de la famille des colonnes, des lignes.}
  \colchunk{}
  \colplacechunks

  \colchunk{Groupe orthogonal.}
  \colchunk{Notations $\mbox{O}_n (\R)$, $\mbox{O} (n)$.}
  \colplacechunks
  
  \colchunk{Lien entre les notions de base orthonormale, isométrie et matrice orthogonale.}
  \colchunk{}
  \colplacechunks

  \colchunk{Déterminant d'une matrice orthogonale, d'une isométrie. Matrice orthogonale positive, négative ; isométrie positive, négative.}
  \colchunk{}
  \colplacechunks
  
  \colchunk{Groupe spécial orthogonal.}
  \colchunk{Notations $\mbox{SO} (E)$, $\mbox{SO}_n (\R)$, $\mbox{SO} (n)$.}
  \colplacechunks
\end{parcolumns}

\subsubsubsection{i) Isométries vectorielles en dimension 2}
\begin{parcolumns}[rulebetween,distance=2.5cm]{2}
  \colchunk{Description des matrices orthogonales et orthogonales positives de taille 2.}
  \colchunk{Lien entre les éléments de $\mbox{SO}_2 (\R)$ et les nombres complexes de module 1.}
  \colplacechunks

  \colchunk{Rotation vectorielle d'un plan euclidien orienté.}
  \colchunk{On introduira à cette occasion, sans soulever de difficulté sur la notion d'angle, la notion de mesure d'un angle orienté de vecteurs.

$\dbf$ SI : mécanique.}
  \colplacechunks
  
  \colchunk{Classification des isométries d'un plan euclidien orienté.}
  \colchunk{}
  \colplacechunks
\end{parcolumns}
