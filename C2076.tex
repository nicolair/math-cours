%<dscrpt>Fichier de déclarations Latex à inclure au début d'un élément de cours.</dscrpt>

\documentclass[a4paper]{article}
\usepackage[hmargin={1.8cm,1.8cm},vmargin={2.4cm,2.4cm},headheight=13.1pt]{geometry}

%includeheadfoot,scale=1.1,centering,hoffset=-0.5cm,
\usepackage[pdftex]{graphicx,color}
\usepackage[french]{babel}
%\selectlanguage{french}
\addto\captionsfrench{
  \def\contentsname{Plan}
}
\usepackage{fancyhdr}
\usepackage{floatflt}
\usepackage{amsmath}
\usepackage{amssymb}
\usepackage{amsthm}
\usepackage{stmaryrd}
%\usepackage{ucs}
\usepackage[utf8]{inputenc}
%\usepackage[latin1]{inputenc}
\usepackage[T1]{fontenc}


\usepackage{titletoc}
%\contentsmargin{2.55em}
\dottedcontents{section}[2.5em]{}{1.8em}{1pc}
\dottedcontents{subsection}[3.5em]{}{1.2em}{1pc}
\dottedcontents{subsubsection}[5em]{}{1em}{1pc}

\usepackage[pdftex,colorlinks={true},urlcolor={blue},pdfauthor={remy Nicolai},bookmarks={true}]{hyperref}
\usepackage{makeidx}

\usepackage{multicol}
\usepackage{multirow}
\usepackage{wrapfig}
\usepackage{array}
\usepackage{subfig}


%\usepackage{tikz}
%\usetikzlibrary{calc, shapes, backgrounds}
%pour la présentation du pseudo-code
% !!!!!!!!!!!!!!      le package n'est pas présent sur le serveur sous fedora 16 !!!!!!!!!!!!!!!!!!!!!!!!
%\usepackage[french,ruled,vlined]{algorithm2e}

%pr{\'e}sentation du compteur de niveau 2 dans les listes
\makeatletter
\renewcommand{\labelenumii}{\theenumii.}
\renewcommand{\thesection}{\Roman{section}.}
\renewcommand{\thesubsection}{\arabic{subsection}.}
\renewcommand{\thesubsubsection}{\arabic{subsubsection}.}
\makeatother


%dimension des pages, en-t{\^e}te et bas de page
%\pdfpagewidth=20cm
%\pdfpageheight=14cm
%   \setlength{\oddsidemargin}{-2cm}
%   \setlength{\voffset}{-1.5cm}
%   \setlength{\textheight}{12cm}
%   \setlength{\textwidth}{25.2cm}
   \columnsep=1cm
   \columnseprule=0.5pt

%En tete et pied de page
\pagestyle{fancy}
\lhead{MPSI-\'Eléments de cours}
\rhead{\today}
%\rhead{25/11/05}
\lfoot{\tiny{Cette création est mise à disposition selon le Contrat\\ Paternité-Pas d'utilisations commerciale-Partage des Conditions Initiales à l'Identique 2.0 France\\ disponible en ligne http://creativecommons.org/licenses/by-nc-sa/2.0/fr/
} }
\rfoot{\tiny{Rémy Nicolai \jobname}}


\newcommand{\baseurl}{http://back.maquisdoc.net/data/cours\_nicolair/}
\newcommand{\urlexo}{http://back.maquisdoc.net/data/exos_nicolair/}
\newcommand{\urlcours}{https://maquisdoc-math.fra1.digitaloceanspaces.com/}

\newcommand{\N}{\mathbb{N}}
\newcommand{\Z}{\mathbb{Z}}
\newcommand{\C}{\mathbb{C}}
\newcommand{\R}{\mathbb{R}}
\newcommand{\D}{\mathbb{D}}
\newcommand{\K}{\mathbf{K}}
\newcommand{\Q}{\mathbb{Q}}
\newcommand{\F}{\mathbf{F}}
\newcommand{\U}{\mathbb{U}}
\newcommand{\p}{\mathbb{P}}


\newcommand{\card}{\mathop{\mathrm{Card}}}
\newcommand{\Id}{\mathop{\mathrm{Id}}}
\newcommand{\Ker}{\mathop{\mathrm{Ker}}}
\newcommand{\Vect}{\mathop{\mathrm{Vect}}}
\newcommand{\cotg}{\mathop{\mathrm{cotan}}}
\newcommand{\sh}{\mathop{\mathrm{sh}}}
\newcommand{\ch}{\mathop{\mathrm{ch}}}
\newcommand{\argsh}{\mathop{\mathrm{argsh}}}
\newcommand{\argch}{\mathop{\mathrm{argch}}}
\newcommand{\tr}{\mathop{\mathrm{tr}}}
\newcommand{\rg}{\mathop{\mathrm{rg}}}
\newcommand{\rang}{\mathop{\mathrm{rg}}}
\newcommand{\Mat}{\mathop{\mathrm{Mat}}}
\newcommand{\MatB}[2]{\mathop{\mathrm{Mat}}_{\mathcal{#1}}\left( #2\right) }
\newcommand{\MatBB}[3]{\mathop{\mathrm{Mat}}_{\mathcal{#1} \mathcal{#2}}\left( #3\right) }
\renewcommand{\Re}{\mathop{\mathrm{Re}}}
\renewcommand{\Im}{\mathop{\mathrm{Im}}}
\renewcommand{\th}{\mathop{\mathrm{th}}}
\newcommand{\repere}{$(O,\overrightarrow{i},\overrightarrow{j},\overrightarrow{k})$}
\newcommand{\cov}{\mathop{\mathrm{Cov}}}

\newcommand{\absolue}[1]{\left| #1 \right|}
\newcommand{\fonc}[5]{#1 : \begin{cases}#2 \rightarrow #3 \\ #4 \mapsto #5 \end{cases}}
\newcommand{\depar}[2]{\dfrac{\partial #1}{\partial #2}}
\newcommand{\norme}[1]{\left\| #1 \right\|}
\newcommand{\se}{\geq}
\newcommand{\ie}{\leq}
\newcommand{\trans}{\mathstrut^t\!}
\newcommand{\val}{\mathop{\mathrm{val}}}
\newcommand{\grad}{\mathop{\overrightarrow{\mathrm{grad}}}}

\newtheorem*{thm}{Théorème}
\newtheorem{thmn}{Théorème}
\newtheorem*{prop}{Proposition}
\newtheorem{propn}{Proposition}
\newtheorem*{pa}{Présentation axiomatique}
\newtheorem*{propdef}{Proposition - Définition}
\newtheorem*{lem}{Lemme}
\newtheorem{lemn}{Lemme}

\theoremstyle{definition}
\newtheorem*{defi}{Définition}
\newtheorem*{nota}{Notation}
\newtheorem*{exple}{Exemple}
\newtheorem*{exples}{Exemples}


\newenvironment{demo}{\renewcommand{\proofname}{Preuve}\begin{proof}}{\end{proof}}
%\renewcommand{\proofname}{Preuve} doit etre après le begin{document} pour fonctionner

\theoremstyle{remark}
\newtheorem*{rem}{Remarque}
\newtheorem*{rems}{Remarques}

\renewcommand{\indexspace}{}
\renewenvironment{theindex}
  {\section*{Index} %\addcontentsline{toc}{section}{\protect\numberline{0.}{Index}}
   \begin{multicols}{2}
    \begin{itemize}}
  {\end{itemize} \end{multicols}}


%pour annuler les commandes beamer
\renewenvironment{frame}{}{}
\newcommand{\frametitle}[1]{}
\newcommand{\framesubtitle}[1]{}

\newcommand{\debutcours}[2]{
  \chead{#1}
  \begin{center}
     \begin{huge}\textbf{#1}\end{huge}
     \begin{Large}\begin{center}Rédaction incomplète. Version #2\end{center}\end{Large}
  \end{center}
  %\section*{Plan et Index}
  %\begin{frame}  commande beamer
  \tableofcontents
  %\end{frame}   commande beamer
  \printindex
}


\makeindex
\begin{document}
\noindent

\debutcours{Espaces vectoriels (sans dimension)}{0.2 \tiny{le \today}}
Le cours d'algèbre linéaire (hors calcul matriciel) est réparti sur trois documents:  \href{\baseurl C2076.pdf}{Espaces vectoriels (sans dimension)} (ce texte),  \href{\baseurl C2112.pdf}{Dimensions des espaces vectoriels} et \href{\baseurl C9587.pdf}{Applications linéaires}.
\section{Définitions}
\subsection{Espace vectoriel}
\index{espace vectoriel}
\begin{defi}[espace vectoriel]
 Soit $\K$ un corps et $E$ un ensemble muni d'une opération interne notée $+$ et d'une multiplication externe
\begin{displaymath}
 \begin{aligned}
  &\K\times E &\rightarrow E\\
 &(\lambda,x) &\rightarrow \lambda x
 \end{aligned}
\end{displaymath}
  On dira que $E$ est un $\K$-espace vectoriel lorsque les opérations vérifient les propriétés suivantes
\begin{itemize}
 \item $(E,+)$ est un groupe commutatif.
 \item Pour tout $x\in E$, $1_\K\,x = x$.
 \item Pour tous $\lambda$ et $\mu$ dans $\K$ et tout $x$ dans $E$, $(\lambda\,\mu)x = \lambda (\mu x)$ et $(\lambda + \mu)x = \lambda x + \mu x$.
 \item (distributivité) Pour tous $\lambda$ dans $\K$ et $x$, $y$ dans $E$, $\lambda(x+y)=\lambda x + \lambda y$.
\end{itemize}
\end{defi}
\begin{rem} Il faut bien noter que pour $\lambda \in \K$ (on dit $\lambda$ \emph{scalaire}) et tout vecteur $x\in E$, la notation $x\lambda$ avec $\lambda$ à gauche n'a aucun sens. Vous ne devriez jamais avoir à écrire une telle expression. Elle provient toujours d'une erreur. 
\end{rem}

\begin{prop}[règles de calcul]
Soit $\K$ un corps et $E$ un $\K$-espace vectoriel. 
\begin{align*}
  &\forall \lambda \in \K :  \lambda 0_E = 0_E\\
  &\forall x \in E : 0_\K x = 0_E\\
  &\forall \lambda\in \K, \forall x\in E : -(\lambda x) = (-\lambda)x = \lambda (-x).
\end{align*}
\end{prop}
\begin{demo}
 \begin{displaymath}
\lambda 0_E + \lambda 0_E =\lambda(0_E+0_E)=\lambda 0_E \Rightarrow \lambda 0_E = 0_E  
 \end{displaymath}
en ajoutant $-(\lambda 0_E)$. De même :
\begin{displaymath}
 0_\K x + 0_\K x = (0_\K + 0_\K)x = 0_\K x \Rightarrow  0_\K x = 0_E
\end{displaymath}
 en ajoutant $-( 0_\K x) $. La troisième relation se déduit facilement des précédentes.
\end{demo}

\begin{rem}
 En particulier : $0_\K 0_E =0_E\text{ et } -x=(-1_\K)x$ .
\end{rem}

\subsection{Exemples - Constructions}
\subsubsection{Sous-corps}
En considérant la multiplication interne sur un corps $\K$ comme une multiplication externe (!!), on remarque que tout corps $\K$ est naturellement un $\K$-espace vectoriel. Plus généralement, on peut considérer les sous-corps.
\begin{prop}
 Soit $\K$ un corps et $k$ un sous-corps de $\K$. L'addition et la multiplication de $\K$ définissent sur $\K$ une structure de $k$-espace vectoriel.
\end{prop}
\begin{exple}
 Ainsi, $\C$ est un $\R$-espace vectoriel, $\R$ est un $\Q$-espace vectoriel.
\end{exple}

\subsubsection{Espaces produits}
\begin{prop}
 Soit $E_1, E_2, \cdots, E_p$ des espaces vectoriels sur un même corps $\K$. Les opérations obtenues par produit sur $E_1\times E_2 \times \cdots \times E_p$ des additions et multiplications externes définissent une structure de $\K$-espace vectoriel.
\end{prop}
\begin{exple}
 On définit ainsi une structure de $\K$-espace vectoriel sur $\K^n$ pour tout entier naturel non nul $n$.\newline
En particulier, notons \index{base canonique de $\K^n$}
\begin{align*}
 \varepsilon_1 &= (1_\K,0_\K,\cdots,0_\K)\\
 \varepsilon_2 &= (0_\K,1_\K,\cdots,0_\K)\\
\vdots& \\
\varepsilon_i &= (0_\K,\cdots,0_\K,1_\K,0_\K,\cdots,0_\K)\hspace{1cm}\text{ (le $1$ en position $i$)}\\
\vdots& \\
\varepsilon_n &= (0_\K,\cdots,0_\K,1_\K)\\
\end{align*}
\label{base_canonique}
Alors:
\begin{displaymath}
 \forall (x_1,x_2,\cdots,x_n)\in \K^n,\hspace{0.5cm}
(x_1,x_2,\cdots,x_n) = x_1\varepsilon_1 + x_2\varepsilon_2 + \cdots + x_n\varepsilon_n
\end{displaymath}
Ainsi, tout vecteur de $\K^n$ est une combinaison linéaire des vecteurs de la famille $(\varepsilon_1,\varepsilon_2,\cdots,\varepsilon_n)$. Cette famille est appelée la \emph{base canonique} de $\K^n$. Le terme de base sera justifié plus loin.
\end{exple}

\subsubsection{Espaces fonctionnels}
\begin{prop}
  Soit $\Omega$ un ensemble quelconque et $E$ un $\K$-espace vectoriel. Sur l'ensemble $\mathcal{F}(\Omega,E)$ des fonctions de $\Omega$ dans $E$, sont définies une addition fonctionnelle et une multiplication externe fonctionnelle:
\begin{displaymath}
  \forall(f,g)\in \mathcal{F}(\Omega,E)^2, \forall \lambda \in \K,\\
\left\lbrace \begin{aligned}
&f+g \text{ est défini par : }\forall \omega \in \Omega:\: (f+g)(\omega) = f(\omega) + g(\omega)\\
&\lambda f \text{ est défini par : }\forall \omega \in \Omega:\: (\lambda f)(\omega) = \lambda f(\omega)  
\end{aligned} \right. 
\end{displaymath}
Ces opérations définissent une structure de $\K$-espace vectoriel sur $\mathcal{F}(\Omega,E)$.
\end{prop}
Les suites et les fonctions à valeurs réelles héritent ainsi de structures de $\R$-espace vectoriel.

\subsubsection{Espaces de polynômes}
Pour tout corps $\K$, l'ensemble $\K[X]$ est un $\K$-espace vectoriel. Pour un entier $n$ fixé, l'ensemble $\K_n[X]$ des polynômes de degré inférieur ou égal à $n$ est un $\K$-espace vectoriel.

\subsection{Familles de vecteurs}
La proposition suivante joue un rôle très important dans la théorie de la dimension.
\begin{prop}
 Soit $\K$ un corps et $E$ un $\K$-espace vectoriel. Pour tout $\lambda \in \K$ et $x\in E$ :
\begin{displaymath}
  \lambda x = 0_E \Rightarrow \lambda = 0_\K \text{ ou } x=0_E 
\end{displaymath}
\end{prop}
\begin{demo}
 Si $\lambda\neq 0_K$ alors il est inversible car $\K$ est un corps. Il existe donc $\mu \in \K$ tel que $\mu \lambda =1$. En multipliant scalairement par $\mu$, on obtient
\begin{displaymath}
 \lambda x = 0_E \Rightarrow \mu(\lambda x )= \mu 0_E \Rightarrow (\mu \lambda)x = 0_E
\Rightarrow 1_\K x =0_E \Rightarrow x= 0_E.
\end{displaymath}
\end{demo}
\index{combinaison linéaire}
\begin{defi}[Combinaison linéaire]
 Une \emph{combinaison linéaire} d'une famille finie de vecteurs $(x_1,x_2,\cdots,x_p)$ est un vecteur de la forme
\begin{displaymath}
 \lambda_1x_1 + \lambda_2x_2 + \cdots + \lambda_p x_p 
\end{displaymath}
avec $\lambda_1,\cdots,\lambda_p$ dans $\K$. On dit que les $\lambda_i$ sont les coefficients de la combinaison linéaire.
\end{defi}

\index{relation linéaire}
\begin{defi}[Relation linéaire]
 Une \emph{relation linéaire} entre les vecteurs d'une famille finie $(x_1,x_2,\cdots,x_p)$ est une combinaison linéaire nulle dont un au moins des coefficients est non nul.
\end{defi}

\begin{exple} Dans le $\R$-espace vectoriel $\C$, il existe une relation linéaire entre les vecteurs de la famille $(1,j,j^2)$. (la somme est nulle, les trois coefficients étant égaux à $1$)
\end{exple}

\index{famille liée}
\begin{defi}[Famille liée]
  Une famille est \emph{liée} si et seulement si il existe une relation linéaire entre ses vecteurs. 
\end{defi}

\index{famille libre}
\begin{defi}[Famille libre]
  Une famille est \emph{libre} si et seulement si il n'existe pas de relation linéaire entre ses vecteurs. 
\end{defi}

\begin{prop}[caractérisations du caractère libre ou lié]
Soit $E$ un $\K$-espace vectoriel et $(a_1, \cdots,a_p)\in E^p$.
\[
  (a_1, \cdots,a_p) \text{ liée } \Leftrightarrow \exists (\lambda_1, \cdots, \lambda_p)\in \K^p \text{ tq }
\left\lbrace
\begin{aligned}
  &(\lambda_1, \cdots, \lambda_p) \neq (0_K, \cdots, 0_k)\\
  &\lambda_1 a_1 + \cdots + \lambda_p a_p = 0_E
\end{aligned}
\right.
\]
\[
  (a_1, \cdots,a_p) \text{ libre } \Leftrightarrow \forall (\lambda_1, \cdots, \lambda_p)\in \K^p, \;
\left( \lambda_1 a_1 + \cdots + \lambda_p a_p = 0_E \Rightarrow (\lambda_1, \cdots, \lambda_p) = (0_K, \cdots, 0_k)\right).
\]
\end{prop}
\begin{demo}
  Il s'agit d'une simple reformulation des définitions.
\end{demo}

\begin{rems}
\begin{enumerate}
  \item Une famille est liée si et seulement si elle n'est pas libre.
  \item Une famille $(x)$ formée d'un seul vecteur est libre si et seulement si $x\neq 0_E$.
  \item Pour une famille de scalaires ou de vecteurs, bien faire la différence entre les expression \og non tous nuls\fg~ et \og tous non nuls\fg.
\end{enumerate}
\end{rems}
\index{famille génératrice}
\begin{defi}[Famille génératrice]
  Une famille d'un $\K$-espace vectoriel $E$ est \emph{génératrice} si et seulement si tout vecteur de $E$ est une combinaison linéaire de la famille.
\end{defi}
\index{base}
\begin{defi}[Base]
Une \emph{base} est une famille libre et génératrice.  
\end{defi}
\begin{prop}[Coordonnées dans une base]
  Si $(a_1,\cdots,a_p)$ est une base de $E$, pour tout vecteur $x$, il existe un unique $p$-uplet $(\lambda_1,\cdots,\lambda_p)\in \K^p$ tel que
\begin{displaymath}
  x = \lambda_1 a_1 + \cdots + \lambda_p a_p
\end{displaymath}
Ces $\lambda_i$ sont appelés les \emph{coordonnées} de $x$ dans la base.
\end{prop}
\begin{demo}
L'existence vient du caractère générateur de la base. L'unicité vient du caractère libre. S'il existait deux $p$-uplets distincts, en soustrayant on obtiendrait une relation entre les vecteurs de la base.  
\end{demo}
Exemples de bases: base canonique  de la partie \ref{base_canonique} Espaces produits, polynômes de degrés échelonnés.

Les questions liées à l'existence de bases, aux coordonnées et aux propriétés des espaces vectoriels qui admettent des bases seront traitées dans la section \href{\baseurl C2112.pdf}{Dimension des espaces vectoriels}.


\section{Sous espaces vectoriels}
\subsection{Définition}
\begin{defi}
  Une partie $A$ d'un $\K$-espace vectoriel $E$ est un sous-espace vectoriel si et seulement si elle est non vide et stable pour la multiplication externe et l'addition.
\begin{displaymath}
  \forall (a,a')\in A^2,\; \forall \lambda \in \K,\; a+a'\in A\text{ et } \lambda a \in A
\end{displaymath}
\end{defi}
\begin{rems}
\begin{itemize}
  \item Un sous-espace vectoriel contient toujours le vecteur nul.
  \item La restriction des opérations à un sous-espace définit une structure d'espace vectoriel sur celui ci.
  \item On trouve parfois des caractérisations mélant les deux opérations ( par exemple $a+\lambda a' \in A$) mais cela ne sert pas à grand-chose.
\end{itemize}
\end{rems}

\begin{prop}
  Une partie non vide $A$ d'un $\K$-espace vectoriel $E$ est un sous-espace vectoriel si et seulement si $A$ est stable pour les combinaisons linéaires des familles de vecteurs de $A$.
\end{prop}
\begin{demo}
  Immédiat à partir de la définition.
\end{demo}


\subsection{Intersection}
\begin{prop}
  L'intersection d'une famille de sous-espaces vectoriels est encore un sous-espace vectoriel.
\end{prop}
\begin{demo}
  On vérifie les stabilités par le jeu du quantificateur $\forall$.
\end{demo}
\begin{rem}
  En revanche, si $A$ et $B$ sont eux sous-espaces, $A\cup B$ n'est un sous-espace que dans le cas trivial où l'un est inclus dans l'autre. Par exemple, si $A\not \subset B$, il existe $a\in A$ tel que $a\notin B$. Alors, pour tout $b\in B$, notons 
\begin{displaymath}
  s = \underset{\in A\cup B}{\underbrace{a}} + \underset{\in A\cup B}{\underbrace{b}} \in A\cup B \text{ stabilité du sev}
\end{displaymath}
Si $s$ appartenait à $B$, on aurait $a =s - b\in B$ en contradiction avec la définition de $a$. Donc 
\begin{displaymath}
  s\in A \Rightarrow b = s - a \in A
\end{displaymath}
\end{rem}


\subsection{Sous espace engendré par une partie}
\begin{defi}
 Soit $E$ un $\K$-espace vectoriel et $A$ une partie de $E$. L'intersection de tous les sous-espaces vectoriels contenant $A$ est appelé le sous-espace vectoriel \emph{engendré} par $A$. Il est noté $\Vect(A)$.
\end{defi}
\index{sous-espace engendré}
\begin{rem}
  Si $V$ est un sous-espace vectoriel contenant $A$ alors $\Vect(A)\subset V$.
\end{rem}

\begin{prop}
 Si $A=\left\lbrace a_1,\cdots,a_p\right\rbrace$, le sous-espace engendré par $A$ est l'ensemble des combinaisons linéaires.
\begin{displaymath}
 \Vect(a_1,\cdots,a_p)=
\left\lbrace
\lambda_1 a_1 + \cdots + \lambda_p a_p
 \right\rbrace 
\end{displaymath}
\end{prop}
\begin{demo}
 On a vu qu'une combinaison linéaire de vecteurs qui sont des combinaisons linéaires de $a_1,\cdots,a_p$ est encore une combinaison linéaire de $a_1,\cdots,a_p$. On en déduit que l'ensemble des combinaisons linéaires est un sous-espace. Il est contenu dans n'importe quel sous-espace $U$ contenant les $a_i$ par stabilité de $U$.
\end{demo}

\begin{rem}
 En particulier dans le cas d'un singleton, $\Vect(a)=\{\lambda a, \lambda\in \K\}$ est aussi noté $\K a$. Une \emph{droite vectorielle} est un sous-espace de la forme $\K a$ pour un vecteur $a$ non nul\index{droite vectorielle}. \index{plan vectoriel}Un \emph{plan vectoriel} est un espace engendré par une famille \emph{libre} de deux vecteurs.
\end{rem}
Exercice traité en classe\newline
Dans $E=\R^3$, on note $x=(2,3,-1)$, $y=(1,-1,-2)$  et $u=(3,7,0)$, $v=(5,0,-7)$.\newline Montrer que $\Vect(x,y)=\Vect(u,v)$.\newline
Si on exprime $x$ et $y$ comme des combinaisons linéaires de $u$ et $v$, on prouve que toute combinaison linéaire de $x$ et $y$ est une combinaison linéaire de $u$ et $v$ d'où $\Vect(x,y)\subset \Vect(u,v)$. On obtient l'autre inclusion en exprimant $u$ et $v$ en fonction de $x$ et $y$.\newline
On cherche à prouver l'existence de $\lambda$ et $\mu$ tels que $x=\lambda u + \mu v$. Cela se traduit par le système suivant aux inconnues $\lambda$ et $\mu$:
\[
  \begin{aligned}
    u &= (3,7,0)  &\times \lambda \\
    v &= (5,0,-7) &\times \mu\\ \hline
    x &= (2,3,-1)&
  \end{aligned}
\hspace{0.5cm}
\left\lbrace
\begin{aligned}
  3\lambda + 5 \mu &= 2\\
  7\lambda &= 3\\
  -7\mu &= -1
\end{aligned}
\right. \Rightarrow x = \frac{3}{7}u + \frac{1}{7}v.
\]
De même pour $y = \lambda u + \mu v$.
\[
  \begin{aligned}
    u &= (3,7,0)  &\times \lambda \\
    v &= (5,0,-7) &\times \mu\\ \hline
    y &= (1,-1,-2)&
  \end{aligned}
\hspace{0.5cm}
\left\lbrace
\begin{aligned}
  3\lambda + 5 \mu &= -1\\
  7\lambda &= -1\\
  -7\mu &= -2
\end{aligned}
\right. \Rightarrow y = -\frac{1}{7}u + \frac{2}{7}v.
\]
Les relations obtenues montrent que $\Vect(x,y) \subset \Vect(u,v)$. Toujours avec des techniques de systèmes linéaires, onn peut inverser les relations:
\begin{displaymath}
 \left\lbrace 
\begin{aligned}
 x&= \frac{3}{7}u + \frac{1}{7}v \\ y &= -\frac{1}{7}u + \frac{2}{7}v
\end{aligned}
\right. 
\Rightarrow
 \left\lbrace 
\begin{aligned}
 u&= 2x -y \\ v &= x+3y
\end{aligned}
\right. 
\Rightarrow \Vect(u,v) \subset \Vect(x,y).
\end{displaymath}

\subsection{Somme de sous-espaces}
\index{somme de deux sous-espaces}
\begin{defi}
 Soit $E$ un $\K$-espace vectoriel et $A$, $B$ deux sous-espaces de $E$. La somme de $A$ et $B$ est l'ensemble noté $A+B$ des sommes d'un vecteur quelconque de $A$ et d'un vecteur quelconque de $B$.
\begin{displaymath}
 A+B = \left\lbrace a+b, (a+b\in A\times B\right\rbrace 
\end{displaymath}
\end{defi}
\begin{prop}
 La somme de deux sous-espaces vectoriels est un sous-espace vectoriel.
\end{prop}
\begin{demo}
 On vérifie immédiatement les stabilités requises.
\end{demo}
\index{espaces supplémentaires}
\begin{defi}[sous-espaces supplémentaires]
 Soit $E$ un $\K$-espace vectoriel. Deux sous-espaces $A$, $B$ de $E$ sont dits \emph{supplémentaires} si et seulement si tout vecteur de $E$ se décompose de manière unique en la somme d'un vecteur de $A$ et d'un vecteur de $B$.
\end{defi}
\begin{prop}
 Soit $E$ un $\K$-espace vectoriel et $A$, $B$ deux sous-espaces de $E$. Ils sont supplémentaires si et seulement si $A\cap B=\{0_E\}$ et $A+B=E$.
\end{prop}
\begin{demo}
Supposons $A$ et $B$ supplémentaires. Alors $A+B=E$ car tout vecteur se décompose en la somme d'un vecteur de $A$ et d'un vecteur de $B$. Montrons que $A\cap B = \left\lbrace 0_E \right\rbrace$.\newline
Soit $x\in A\cap B$:
\begin{displaymath}
  0_E = \underset{\in A}{\underbrace{0_E}} + \underset{\in B}{\underbrace{0_E}} =
  \underset{\in A\cap B \subset A}{\underbrace{x}} + \underset{\in A\cap B \subset B}{\underbrace{(-x)}}
\Rightarrow x = 0_E\hspace{1cm} \text{(unicité de la décomposition de $0_E$)} 
\end{displaymath}

Supposons que $A\cap B=\{0_E\}$ et $A+B=E$. Il s'agit seulement de prouver l'unicité de la décomposition d'un vecteur quelconque de $E$. Si un $x\in E$ se décompose de deux manières
\begin{multline*}
\exists (a,b)\in A\times B \text{ et } (a',b')\in A\times B \text{ tq }
\left\lbrace 
\begin{aligned}
  x &= a+ b \\ x &= a' + b'
\end{aligned}
\right. \Rightarrow 0_E = (a-a') + (b-b') \\
\Rightarrow \underset{\in A}{\underbrace{a-a'}} = \underset{\in B}{\underbrace{b'-b}} \in A\cap B = \left\lbrace 0_E\right\rbrace 
\Rightarrow
\left\lbrace 
\begin{aligned}
  a &=a' \\ b &= b'
\end{aligned}
\right. 
\end{multline*}
\end{demo}
\begin{exples}
Pour les exemples listés au dessous $A$ et $B$ sont des sous-espaces supplémentaires de $E$.
  \begin{enumerate}
    \item Soit $E$ le $\R$-espace vectoriel des fonctions de $\R$ dans $\R$, $A$ est l'ensemble des fonctions paires de $\R$ dans $\R$ et $B$ l'ensemble des fonctions impaires. 
    \item Soit $E$ le $\R$-espace vectoriel des suites convergentes de réels, $A$ l'ensemble des suites constantes et $B$ l'ensemble des suites qui convergent vers $0$.
  \end{enumerate}
\end{exples}
\begin{rems}
  Soit $A, B, C$ des sous-espaces d'un $\K$-espace vectoriel $E$.
\[
  A+B = \Vect(A\cup B), \hspace{0.5cm} \left(A\subset C \text{ et } B \subset C\right) \Rightarrow A+B \subset C.
\]
\end{rems}


On peut étendre la notion de somme à plus de deux sous-espaces.
\index{somme de sous-espaces}
\begin{defi}[Somme de sous-espaces]
Soit $A_1,\cdots,A_p$ des sous-espaces d'un $\K$-espace vectoriel $E$. On appelle \emph{somme} des $A_i$ la partie de $E$
\begin{displaymath}
  A_1 + \cdots + A_p = \left\lbrace a_1+\cdots+a_p,\; (a_1,\cdots,a_p)\in A_1\times\cdots\times A_p \right\rbrace 
\end{displaymath}
\end{defi}
\begin{rem}
  Soit $x\in E$,
\begin{displaymath}
  x\in A_1+\cdots +A_p \Leftrightarrow \exists (a_1,\cdots,a_p)\in A_1\times\cdots\times A_p \text{ tq } x = a_1+\cdots + a_p
\end{displaymath}
Autrement dit un vecteur est dans la somme si et seulement il se décompose en une somme de vecteurs de chacun des sous-espaces.
\end{rem}

\begin{prop}
  Une somme de sous-espaces est un sous-espace.
\end{prop}
\begin{demo}
  Vérification immédiate
\end{demo}
On vérifie facilement que l'opération \og somme de sous-espaces\fg~ est associative.
\index{somme directe}
\begin{defi}
  Soit $A_1,\cdots,A_p$ des sous-espaces d'un $\K$-espace vectoriel $E$. On dira que la somme $A_1+\cdots + A_p$ est \emph{directe} si et seulement si
les vecteurs de la somme se décomposent \emph{de manière unique} en une somme de vecteurs des sous-espaces. 
\end{defi}
\begin{nota}
 Lorsque les sous-espaces $A_1,\cdots,A_p$ sont en somme directe, on note
\[
 A_1 \oplus \cdots \oplus A_p = A_1 + \cdots + A_p.
\]
En particulier, $E = A \oplus B$ si $A$ et $B$ sont supplémentaires dans $E$.
\end{nota}


\end{document}
