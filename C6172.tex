\input{courspdf.tex}
\debutcours{Espace euclidien orienté en dimension 2 ou 3}{alpha}

On a vu que dans tout espace euclidien orienté, le déterminant d'une famille de vecteurs est indépendant de la base \emph{orthonormée directe} utilisée pour le calculer. On peut donc parler du déterminant d'une famille sans préciser de base étant entendu que l'on peut utiliser n'importe quelle base orthonormée directe.\newline

Ce déterminant est lié à l'aire et au sinus

En dimension 3 le déterminant est lié au volume on le note parfois $vol$. Il permet de définir le produit vectoriel de deux vecteurs. Un autre point important est que l'espace des matrices antisymétriques est de dimension $3$ comme l'espace lui même. Toute matrice antisymétrique est la matrice dans une base orthonormée directe d'un fonction obtenue par un produit vectoriel avec un vecteur unique et fixé.
\end{document}
