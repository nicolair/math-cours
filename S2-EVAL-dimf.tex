\subsubsection{B - Espaces de dimension finie}
\subsubsubsection{Existence de bases}
\begin{parcolumns}[rulebetween,distance=\parcoldist]{2}

 \colchunk{Un espace vectoriel est dit de dimension finie s'il possède une famille génératrice finie.}
 \colchunk{}
 \colplacechunks

 \colchunk{Si ${(x_i)}_{1\le i\le n}$ engendre $E$ et si ${(x_i)}_{i\in I}$ est libre pour une certaine partie $I$ de $\{1,\dots,n\}$, alors il existe une partie $J$ de $\{1,\dots,n\}$ contenant $I$ pour laquelle ${(x_j)}_{j\in J}$ est une base de $E$.}
  \colchunk{Existence de bases en dimension finie. \\ Théorème de la base extraite~: de toute famille génératrice on peut extraire une base.\\Théorème de la base incomplète~: toute famille libre peut être complétée en une base.}
  \colplacechunks
\end{parcolumns}

\subsubsubsection{Dimension d'un espace de dimension finie}
\begin{parcolumns}[rulebetween,distance=\parcoldist]{2}

 \colchunk{Dans un espace engendré par $n$ vecteurs, toute famille de $n+1$ vecteurs est liée.}
  \colchunk{}
  \colplacechunks
 \colchunk{Dimension d'un espace de dimension finie. }
  \colchunk{Dimensions de $\K^n$, de $\K_n[X]$, de l'espace des solutions d'une équation différentielle linéaire homogène d'ordre $1$, de l'espace des solutions d'une équation différentielle linéaire homogène d'ordre $2$ à coefficients constants, de l'espace des suites vérifiant une relation de récurrence linéaire homogène d'ordre $2$ à coefficients constants.}
  \colplacechunks
 \colchunk{En dimension $n$, une famille de $n$ vecteurs est une base si et seulement si elle est libre, si et seulement si elle est génératrice.}
  \colchunk{}
  \colplacechunks
 \colchunk{Dimension d'un produit fini d'espaces vectoriels de dimension finie.}
  \colchunk{}
  \colplacechunks
 \colchunk{Rang d'une famille finie de vecteurs.}
  \colchunk{Notation $\rg(x_1,\dots,x_n)$.}
  \colplacechunks

\end{parcolumns}

\subsubsubsection{Sous-espaces et dimension}
\begin{parcolumns}[rulebetween,distance=\parcoldist]{2}


 \colchunk{Dimension d'un sous-espace d'un espace de dimension finie, cas d'égalité.}
  \colchunk{Sous-espace de $\R^2$ et $\R^3$.}
  \colplacechunks
 \colchunk{Tout sous-espace d'un espace de dimension finie possède un supplémentaire.}
  \colchunk{Dimension commune des supplémentaires.}
  \colplacechunks
 \colchunk{Base adaptée à un sous-espace, à une décomposition en somme directe d'un nombre fini de sous-espaces.}
  \colchunk{}
  \colplacechunks
 \colchunk{Dimension d'une somme de deux sous-espaces~; formule de Grassmann. Caractérisation des couples de sous-espaces supplémentaires.}
  \colchunk{}
  \colplacechunks
 \colchunk{Si $F_1,\dots,F_p$ sont des sous-espaces de dimension finie, alors~: $\dim\displaystyle\sum_{i=1}^pF_i\le \displaystyle \sum_{i=1}^p\dim F_i$, avec égalité si et seulement si la somme est directe.}
  \colchunk{}
  \colplacechunks
\end{parcolumns}

