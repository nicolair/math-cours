\subsubsection{A - Espaces vectoriels}

\subsubsubsection{Espaces vectoriels}
\begin{parcolumns}[rulebetween,distance=\parcoldist]{2}
  \colchunk{Structure de $\K$ espace vectoriel}
  \colchunk{Espaces $\K^n$, $\K[X]$.}
  \colplacechunks
  
  \colchunk{Produit d'un nombre fini d'espaces vectoriels.}
  \colchunk{}
  \colplacechunks

 \colchunk{Espace vectoriel des  fonctions d'un ensemble dans un espace vectoriel.}
  \colchunk{Espace $\K^\N$ des suites d'éléments de $\K$.}
  \colplacechunks

 \colchunk{Famille presque nulle (ou à support fini) de scalaires, combinaison linéaire  d'une famille de vecteurs.}
  \colchunk{On commence par la notion de combinaison linéaire d'une famille finie de vecteurs.}
  \colplacechunks
  
\end{parcolumns}

\subsubsubsection{Sous-espaces vectoriels}
\begin{parcolumns}[rulebetween,distance=\parcoldist]{2}

 \colchunk{Sous-espace vectoriel~: définition, caractérisation.}
  \colchunk{Sous-espace nul. Droites vectorielles de $\R^2$, droites et plans vectoriels de $\R^3$. Sous-espaces $K_n[X]$ de $\K[X]$.}
  \colplacechunks
 \colchunk{Instersection d'une famille de sous-espaces vectoriels. }
  \colchunk{}
  \colplacechunks
 \colchunk{Sous-espace vectoriel engendré par une partie $X$.}
  \colchunk{Notations $\Vect(X)$, $\Vect{(x_i)}_{i\in I}$.\\Tout sous-espace contenant $X$ contient $\Vect(X)$.}
  \colplacechunks
\end{parcolumns}

\subsubsubsection{Familles de vecteurs}
\begin{parcolumns}[rulebetween,distance=\parcoldist]{2}
 \colchunk{Familles et parties génératrices.}
  \colchunk{}
  \colplacechunks
 \colchunk{Familles et parties libres.}
  \colchunk{}
  \colplacechunks
 \colchunk{Base, coordonnées.}
  \colchunk{Bases canoniques de $K^n$, $K_n[X]$, $\K[X]$.}
  \colplacechunks

\end{parcolumns}
\subsubsubsection{Somme d'un nombre fini de sous-espaces}
\begin{parcolumns}[rulebetween,distance=\parcoldist]{2}

 \colchunk{Somme de deux sous-espaces.}
  \colchunk{}
  \colplacechunks
 \colchunk{Somme directe de deux sous-espaces. Caractérisation par l'intersection.}
  \colchunk{La somme $F+G$ est directe si la décomposition de tout vecteur de $F+G$ comme somme d'un élément de $F$ et d'un élément de $G$ est unique.}

  \colplacechunks
 \colchunk{Sous-espaces supplémentaires.}
  \colchunk{}
  \colplacechunks
 \colchunk{Somme d'un nombre fini de sous-espaces.}
  \colchunk{}
  \colplacechunks
 \colchunk{Somme directe d'un nombre fini de sous-espaces. Caractérisation par l'unicité de la décomposition du vecteur nul.}
  \colchunk{La somme $F_1+\cdots+F_p$ est directe si la décomposition de tout vecteur de $F_1+\cdots+F_p$ sous la forme $x_1+\cdots+x_p$ avec $x_i\in F_i $ est unique.}
  \colplacechunks

\end{parcolumns}
