%! iTeXMac(project): pdftex

\input fr

\def\date{}

%\input macros


\catcode`\@=11

\font\goth=eufm10
\font\ineg=msam8
\font\star=msam10
\font\vid=msbm10
\font\bsl=cmbxsl10 at 10pt % gras-pench{\'e}
\font\large=cmr10 at 12pt
\font\Large=cmr10 at 14pt
\font\largeb=cmbx10 at 12pt
\font\Largeb=cmbx10 at 14pt
\font\pcar=cmr8 at 8pt % pour {\'e}crire les si\`ecles
\font\tenbb=cmssbx10 at 10pt % police provisoire pour R,N,Q,Z
\font\sevenbb=cmbx10 at 7pt
\font\fivebb=cmbx10 at 5pt

\everymath{\displaystyle}
\newfam\bbfam
\textfont\bbfam=\tenbb
\scriptfont\bbfam=\sevenbb
\scriptscriptfont\bbfam=\fivebb
\def\bb{\fam\bbfam\tenbb}

\catcode`\;=\active
\def;{\relax\ifhmode\ifdim\lastskip>\z@
\unskip\fi\kern.2em\fi\string;}

\catcode`\:=\active
\def:{\relax\ifhmode\ifdim\lastskip>\z@\unskip\fi
\penalty\@M\ \fi\string:}

\catcode`\!=\active
\def!{\relax\ifhmode\ifdim\lastskip>\z@
\unskip\fi\kern.2em\fi\string!}

\catcode`\?=\active
\def?{\relax\ifhmode\ifdim\lastskip>\z@
\unskip\fi\kern.2em\fi\string?}

%\def\^#1{\if#1i{\accent"5E\i}\else{\accent"5E #1}\fi}
%\def\"#1{\if#1i{\accent"7F\i}\else{\accent"7F #1}\fi}

\newif\ifpagetitre \pagetitretrue
\newtoks\hautpagetitre
\hautpagetitre={\tenrm\hfil\the\premiertitre\hfil}
\newtoks\baspagetitre \baspagetitre={\hfil}

\newtoks\partiecourante \partiecourante={\hfil}
\newtoks\titrecourant \titrecourant={\hfil}
\newtoks\premiertitre \premiertitre={\hfil}

\newtoks\hautpagegauche \newtoks\hautpagedroite
\hautpagegauche={\tenrm\folio\hfill{\the\partiecourante}}
\hautpagedroite={\tenrm{\the\titrecourant}\hfill\folio}

\newtoks\baspagegauche \baspagegauche={\hfil}
\newtoks\baspagedroite \baspagedroite={\hfil}

\headline={\ifnum\pageno=1\the\hautpagetitre\else\the\hautpagedroite \fi}

\footline={\hfil}

\def\nopagenumbers{\def\folio{\hfil}}

\catcode`\@=12

\let\optionkeymacros\null
\let\dis=\displaystyle
\let\scr=\scriptstyle
\let\so=\medskip
\let\eps=\varepsilon % Le "bon" epsilon
\let\vphi=\varphi % Le phi usuel
\let\tend=\rightarrow
\let\Tend=\longrightarrow
\let\ssi=\Longleftrightarrow

\def\op{{\star F}}
\def\frac#1#2{{#1\over#2}}
\def\text#1{\hbox{\rm #1}}
\def\d{\,\hbox{\rm d}\,}
\def\trait{\par\centerline{\hbox{\vrule height .4pt depth 0pt width 12cm}}}
\def\ie{\mathrel{\hbox{\ineg 6}}} % <= fran{\c c}ais
\def\le{\mathrel{\hbox{\ineg 6}}} % <= fran{\c c}ais
\def\leq{\mathrel{\hbox{\ineg 6}}} % <= fran{\c c}ais
\def\se{\mathrel{\hbox{\ineg >}}} % >= fran{\c c}ais
\def\ge{\mathrel{\hbox{\ineg >}}} % >= fran{\c c}ais
\def\geq{\mathrel{\hbox{\ineg >}}} % >= fran{\c c}ais
\def\vide{\hbox{\vid~?}}
\def\Z{{\bb Z}}
\def\R{{\bb R}}
\def\C{{\bb C}}
\def\N{{\bb N}}
\def\Q{{\bb Q}}
\def\K{{\bb K}}
\def\U{{\bb U}}
\def\dim{{\rm dim}\,}
\def\sev{{\rm sous-espace vectoriel}}
\def\ker{{\rm Ker}\,}
\def\Ker{{\rm Ker}\,}
\def\re{{\rm Re}\,}
\def\im{{\rm Im}\,}
\def\gav{{\rm GA}(E)}
\def\gle{{\rm GL}(E)}
\def\mnpk{{\cal M}_{n,p}(\K)}
\def\mnk{{\cal M}_n(\K)}
\def\glnk{{\rm GL}_n(\K)}
\def\det{{\rm Det}\,}
\def\card{{\rm Card}}
\def\tr{{\rm Tr}\,}
\def\e{{\rm e}}
\def\ch{\mathop{\rm ch}\nolimits}
\def\sh{\mathop{\rm sh}\nolimits}
\def\th{\mathop{\rm th}\nolimits}
\def\argch{\mathop{\rm Arg\,ch}\nolimits}
\def\argsh{\mathop{\rm Arg\,sh}\nolimits}
\def\argth{\mathop{\rm Arg\,th}\nolimits}
\def\arccos{\mathop{\rm Arc\,cos}\nolimits}
\def\arcsin{\mathop{\rm Arc\,sin}\nolimits}
\def\arctan{\mathop{\rm Arc\,tan}\nolimits}
\def\adh#1{\overline{\!#1}}
\def\rond#1{\buildrel\;\circ\over #1}
\def\cnp#1#2{{\displaystyle\Big({{\textstyle #1}\atop%
{\textstyle #2}}\Big)}}
\def\hfl#1#2{\smash{\mathop{\hbox to 4mm{\rightarrowfill}}
\limits^{#1}_{#2}}}
\def\vect#1{\overrightarrow{#1}} % vecteur
\def\Frac#1#2{{\displaystyle#1\over\displaystyle#2}}
\def\frac#1#2{{\scriptstyle#1\over\scriptstyle#2}}
\def\Der#1#2{\Frac{\hbox{d}#1}{\hbox{d}#2}} % Ex:\Der{y}{x}
\def\Derr#1#2{\Frac{\hbox{d}^2#1}{\hbox{d}#2^2}}
\def\Dron#1#2{\Frac{\partial#1}{\partial#2}}
\def\dron#1#2{\frac{\partial#1}{\partial#2}}
\def\<<{\leavevmode\raise.3ex\hbox{$\scriptscriptstyle\langle\!\langle$}}
\def\>>{\leavevmode\raise.3ex\hbox{$\scriptscriptstyle\rangle\!\rangle$}}
\def\implique{\ \Longrightarrow\ }
\def\vabs#1{\vert#1\vert}
\def\norme#1{\Vert#1\Vert}
\def\Norme#1{\vert\vert\vert#1\vert\vert\vert}
\def\[{[\![}
\def\]{]\!]}

% SUPERPOSITION
\def\up#1{\raise 1ex\hbox{\septtm#1}}
% op{\'e}rateur avec dessous: build {op{\'e}rateur} {dessous}
\def\build#1#2{\mathrel{\mathop{\kern 0pt#1}\limits_{#2}}}
% op{\'e}rateur avec deux dessous: Build {op{\'e}rateur} {dessous1}{dessous2}
\def\Build#1#2#3{\build{{#1}}{\scriptstyle{#2}\atop\scriptstyle{#3}}}
% fl\`eche double : fleche {variable} {valeur}
\def\fleche#1#2{\build{\hbox to 9mm{\rightarrowfill}}{{#1}\rightarrow{#2}}}
% fl\`eche triple : Fleche {variable} {valeur} {3i\`eme ligne}
\def\Fleche#1#2#3{\build{\hbox to 9mm{\rightarrowfill}}
{\scriptstyle{#1}\rightarrow{#2}\atop\scriptstyle{#3}}}
% encadrement d'une bo{\^i}te: cadre {largeur blanc} {bo{\^i}te}
\long\def\cadre#1#2{\vbox{\hrule\hbox{\vrule%
\vbox spread#1{\vfil\hbox spread#1{\hfil#2\hfil}\vfil}\vrule}\hrule}\par}

\def\tp{\centerline{\bsl Travaux pratiques}\so}
\def\tit#1{{\parindent=-2mm{\bf#1}\smallskip}}
\long\def\TITRE#1{\bigskip\bigskip

\centerline{\Large#1}

\bigskip}
\long\def\TIT#1{\bigskip\centerline{\largeb#1}\bigskip}
\long\def\Titre#1{\bigskip{\large#1}\bigskip}
\long\def\titre#1{\bigskip\centerline{\hfill{\bf #1}\hfill}
\medskip}
\long\def\tx#1#2{\hbox{\hbox to 94mm{\vtop{\hsize=90mm#1\vfill}\hfill}
\hfill\hbox to 74mm{\vtop{\hsize=70mm#2\vfill}\hfill}}}%\filbreak}


\hsize=170mm \vsize=250mm
\hoffset=-4mm \voffset=-1mm
\pretolerance=500 \tolerance=1000 \brokenpenalty=5000

%\fhyph
\frenchspacing
\overfullrule=0cm %\emergencystretch=10pt

\null
\vskip 0.5cm

\parindent=0mm
\abovedisplayskip=6pt plus 2pt minus 4pt
\abovedisplayshortskip=0pt plus 2pt
\belowdisplayskip=6pt plus 2pt minus 4pt
\belowdisplayshortskip=0pt plus 2pt

\def\bg{\bigskip}
\def\md{\medskip}
\def\cl{\centerline}
\def\info{informatique}

\long\def\tx#1#2{\hbox{\hbox to 94mm{\vtop{\hsize=90mm#1\vfill}\hfill}
\hfill\hbox to 74mm{\vtop{\hsize=70mm#2\vfill}\hfill}}\filbreak}

\long\def\txv#1#2{\hbox{\vrule\hskip2mm\hbox to
94mm{\vtop{\hsize=90mm#1\vfill}\hfill} \hfill\hbox to
74mm{\vtop{\hsize=70mm#2\vfill}\hfill}}\filbreak}

\long\def\txz#1#2{\hbox{\hbox to
94mm{\vtop{\hsize=90mm#1\vfill}\hfill} \hfill\hbox to
74mm{\vtop{\hsize=70mm#2\vfill}\hfill}\hskip2mm\vrule width 1mm}\filbreak}

\def\vrg{\raise.3mm\hbox{$\hskip.2mm,\ $}}





\titrecourant={\bf MPSI\ }
\partiecourante={\bf MPSI\ }
\premiertitre={\bf CLASSE DE PREMI\`ERE ANN\'EE MPSI} \bg {\sl Le 
programme de premi\`ere ann\'ee MPSI est organis\'e en trois parties. Dans
une premi\`ere partie figurent les notions et les objets qui doivent
\^etre \'etudi\'es d\`es le d\'ebut de l'ann\'ee scolaire. Il s'agit
essentiellement, en partant du programme de la classe de Terminale S et en
s'appuyant sur les connaissances pr\'ealables des \'etudiants, d'introduire des notions de base
n{\'e}cessaires tant en math{\'e}matiques que dans les autres disciplines
scientifiques (physique, chimie, sciences industrielles\dots). Certains de ces objets seront
consid\'er\'es comme d\'efinitivement acquis (nombres complexes, coniques,
\dots) et il n'y aura pas lieu de reprendre ensuite leur \'etude dans le
cours de math\'ematiques; d'autres, au contraire, seront revus plus tard
dans un cadre plus g\'en\'eral ou dans une pr\'esentation plus th\'eorique
(groupes, produit scalaire, \'equations diff\'erentielles, \dots).

Les deuxi\`eme et troisi\`eme parties correspondent \`a un d\'ecoupage classique
entre l'analyse et ses applications g\'eom\'etriques d'une part, l'alg\`ebre et
la g\'eom\'etrie euclidienne d'autre part. }


\TITRE{PROGRAMME DE D\'EBUT D'ANN\'EE}

\Titre{I. NOMBRES COMPLEXES ET G\'EOM\'ETRIE \'EL\'EMENTAIRE}


\titre{1- Nombres complexes}

{\sl L'objectif est de consolider et d'approfondir les notions sur les
nombres complexes d\'ej\`a abord\'ees en classe de Terminale. Le programme
combine l'\'etude du corps des nombres complexes et de l'exponentielle
complexe avec les applications des nombres complexes aux \'equations
alg\'ebriques, \`a la trigonom\'etrie et \`a la g\'eom\'etrie.
\so

Il est souvent commode d'identifier \C\ au plan euclidien
notamment pour les probl\`emes d'origine g\'eom\'etrique, ce qui permet
d'exploiter le langage de la g\'eom\'etrie pour l'\'etude des nombres
complexes et, inversement, d'utiliser les nombres complexes pour
traiter certaines questions de g\'eom\'etrie plane.
En particulier,
les \'etudiants doivent savoir interpr\'eter \`a l'aide des nombres complexes
les notions suivantes de la g\'eom\'etrie euclidienne plane~:
calcul vectoriel, barycentre, alignement, orthogonalit\'e, distance, mesure
d'angle.}



\so

\tit{a) Corps \C\ des nombres complexes}

\tx{Corps \C\ des nombres complexes. Parties
r\'eelle et imaginaire d'un nombre complexe, conjugaison dans \C.
}
{La construction du corps \C\ n'est pas exigible des
\'etudiants.

Notations $\re z$, $\im z$, $\bar z$.}
\so

\tx{Le plan \'etant muni d'un rep\`ere orthonormal,
affixe d'un point, d'un vecteur; image d'un nombre complexe.
}
{Interpr\'etation g\'eom\'etrique des transformations $z\mapsto\bar z$, $z\mapsto
z+b$.}
\so

\tx{Module d'un nombre complexe, module d'un produit, d'un quotient.
In\'egalit\'e triangulaire; interpr\'etation en termes de distances.
}
{Notation $|z|$; relation $|z|^2=\bar zz$.

Interpr\'etation g\'eom\'etrique de $|z|$, de $|z-a|$; disque ouvert (ferm\'e) de
centre $a$.}

\so


\tit{b) Groupe \U\ des nombres complexes de module 1}


\tx{ D\'efinition du groupe \U\ des nombres complexes de module 1. Cercle trigonom{\'e}trique.

D\'efinition de ${\rm e}^{{\rm i}\theta}$, relations d'Euler.

Morphisme $\theta\mapsto{\rm e}^{{\rm i}\theta}$ de \R\ dans \U. Formule de Moivre.
}
{On se contentera d'une br{\`e}ve pr{\'e}sentation de la structure de groupe.

Par d\'efinition, ${\rm e}^{{\rm i}\theta}=\cos\theta+{\rm i}\,\sin\theta$
o\`u $\theta\in\R$. La continuit\'e, la d\'erivabilit\'e et les variations des
fonctions cosinus, sinus et tangente sont suppos\'ees connues, ainsi que
leurs formules d'addition.}

\so
\tx{\S\ Lin\'earisation et factorisation d'expressions trigonom\'e\-triques.}{Les \'etudiants doivent conna\^{\i}tre les
formules donnant $\cos(a+b)$,  $\sin(a+b)$,
$\tan(a+b)$, $\cos 2x$, $\sin 2x$, $\tan 2x$. Ils doivent savoir exprimer
$\sin\theta$, $\cos\theta$, $\tan\theta$ et ${\rm e}^{{\rm i}\theta}$ \`a
l'aide de $\dis\tan{\theta\over2}$ et relier ces formules \`a la
repr\'esentation param\'etrique rationnelle du cercle trigonom\'etrique priv\'e de
$-1$.}
\so

\tx{Arguments d'un nombre complexe non nul. \'Ecriture d'un nombre complexe
$z\not=0$ sous la forme $\rho\,{\rm e}^{{\rm i}\theta}$ o\`u $\rho>0$ et
$\theta\in\R$ (forme trigonom\'etrique).{}

\so
Racines $n$-i{\`e}mes de l'unit{\'e}. R{\'e}solution de l'{\'e}quation $z^n=a$.}

\so

\tit{c) \'Equations du second degr\'e}

\tx{R\'esolution des \'equations du second degr\'e \`a coefficients
complexes; discriminant. Relations entre coefficients et racines.}{}

\so

\tit{d) Exponentielle complexe}

\tx{D\'efinition de l'exponentielle d'un nombre complexe:
$${\rm e}^z={\rm e}^x\,{\rm e}^{{\rm i}y}\quad\hbox{\rm o\`u}\quad z=x+{\rm
i}y.$$
Propri\'et\'es.}
{La continuit\'e, la d\'erivabilit\'e et les variations de la fonction
exponentielle r\'eelle sont suppos\'ees connues, ainsi que son \'equation fonctionnelle.}
\so

\tit{e) Nombres complexes et g\'eom\'etrie plane}

\tx{Interpr\'etation g{\'e}om{\'e}trique des transformations: $$z\mapsto az,\ z\mapsto az+b, z\mapsto\overline z.$$

Interpr\'etation du module et de l'argument de $z\mapsto{1\over z}\vrg\dis{z-a\over z-b}\cdotp$ }
{Les \'etudiants doivent savoir interpr\'eter \`a l'aide des nombres
complexes les notions suivantes de la g\'eom\'etrie euclidienne plane:
distance, mesure d'angle, barycentre, alignement, orthogonalit\'e.} \so




\titre{2- G\'eom\'etrie \'el\'ementaire du plan}

{\sl \`A l'issue de la Terminale, les \'etudiants connaissent le plan
g\'eom\'etrique euclidien et l'espace
g\'eom\'etrique euclidien de dimension 3 
en tant qu'ensemble de points. Ils connaissent en particulier la
fa\c{c}on d'associer \`a deux points $A$ et $B$ le vecteur $\vect{AB}$,
ainsi que les propri\'et\'es op\'eratoires usuelles. Il convient de faire constater
que l'ensemble des vecteurs du plan (respectivement de l'espace) est muni d'une structure
de plan vectoriel r\'eel (respectivement d'espace vectoriel r\'eel de dimension 3), d\'efini comme espace vectoriel sur
$\R$ dont tout vecteur s'exprime comme combinaison lin\'eaire de
deux vecteurs ind\'ependants, c'est-\`a-dire non colin\'eaires (respectivement trois vecteurs ind\'ependants, c'est-\`a-dire non coplanaires). Toute
th\'eorie g\'en\'erale des espaces vectoriels est exclue \`a ce
stade.

Les notions suivantes sont suppos\'ees connues :
calcul vectoriel et barycentrique, distance euclidienne, orthogonalit\'e,
rep\`ere orthonormal, angles, angles orient{\'e}s dans le plan euclidien.

La donn\'ee d'un rep\`ere orthonormal identifie le plan \`a $\R^2$ ou \`a $\C$  (respectivement l'espace \`a $\R^3$).}
\so
\tit{a) Modes de rep\'erage dans le plan}
\tx{Rep\`ere cart\'esien du plan, coordonn\'ees cart\'esiennes. Rep{\`e}re orthonormal direct, changement de rep{\`e}re.}{Les formules de changement de rep{\`e}re sont {\`a} conna{\^i}tre uniquement dans le cas o{\`u} les deux rep{\`e}res sont orthonormaux directs.}
\so
\tx{Coordonn\'ees polaires d'un point du plan suppos\'e muni d'un rep\`ere orthonormal.}
{Le rep\`ere orthonormal
identifie le plan \`a $\C$.}

\tx{\'Equation
polaire d'une droite, d'un cercle passant par $O$.}{}
\so
\tx{
Rep\`ere polaire $(\vec{u},\vec{v})$ du plan euclidien $\R^2$ d\'efini,
pour tout nombre r\'eel $\theta$, par 
$$\vec{u}\,(\theta)=\,\,\cos\theta\,\,\vec{e_1}+\sin\theta\,\,\vec{e_2},$$
$$\vec{v}\,(\theta)=-\sin\theta\,\,\vec{e_1}+\cos\theta\,\,\vec{e_2}$$ o\`u
$(\vec{e_1},\vec{e_2})$ est la base canonique de $\R^2$.}
{Identification $\vec{u}=\e^{{\rm i}\theta}$, $\vec{v}={\rm
i}\e^{{\rm i}\theta}$.}
\so



\so
\tit{b) Produit scalaire}
\tx{D\'efinition g\'eom\'etrique du produit scalaire. Si $\vec{u}$ et
$\vec{v}$ sont non nuls
$$\vec{u}\cdot\vec{v}
=\|\vec{u}\|\,\|\vec{v}\|\cos(\vec{u},\vec{v}),$$
et $\vec{u}\cdot\vec{v}=0$ sinon.}
{Interpr\'etation en terme de projection.}
\so
\tx{Bilin\'earit\'e, sym\'etrie, expression en base orthonormale.}
{Dans $\C$, interpr\'etation g\'eom\'etrique de $\re(\bar{a}b)$.}

\so
\tit{c) D\'eterminant}
\tx{D\'efinition g\'eom\'etrique du d\'eterminant.
Si $\vec{u}$ et

$\vec{v}$ sont non nuls
$$\det(\vec{u},\vec{v})
=\|\vec{u}\|\,\|\vec{v}\|\sin(\vec{u},\vec{v}),$$
et $\det(\vec{u},\vec{v})
=0$ sinon.}
{La notion d'orientation du plan est admise, ainsi que celle de
base orthonormale directe.}
\so
\tx{Bilin\'earit\'e, antisym\'etrie, expression en base orthonormale
directe.}
{Dans $\C$, interpr\'etation de $\im(\bar{a}b)$ comme d\'eter\-mi\-nant
des vecteurs associ\'es \`a $a$ et $b$.
Interpr\'etation g\'eom\'etrique de $|\det(\vec{u},\vec{v})|$ comme
aire du parall\'elogramme construit sur $\vec{u}$ et $\vec{v}$.
}

\so

\tit{d) Droites}

\tx{Applications du d\'eterminant \`a la
colin\'earit\'e de deux vecteurs, l'alignement de trois points.}

{Lignes de niveau de
$M\mapsto \vec{u}\cdot \vect{AM}$ et de
$M\mapsto\det(\vec{u},\vect{AM})$.}

\tx{
Param\'etrage et \'equation cart\'esienne d'une droite d\'efinie
par un point et un vecteur directeur, par deux points distincts,
par un point et un vecteur normal.

Distance \`a une droite, \'equation normale d'une
droite.}{}

\so

\tit{e) Cercles}

\tx{\'Equation cart\'esienne d'un cercle.
Intersection d'un cercle et d'une droite. Intersection de deux cercles.
}
{Caract\'erisation d'un cercle de diam{\`e}tre $[AB]$ par l'\'equation
$\vect{MA}\cdot\vect{MB}=0$.

Angles de droites. \'Etant donn{\'e}s deux points distincts $A$ et $B$, ensemble des points $M$ tels que $(MA,MB)=\alpha $, ensemble des points $M$ tels que $MB=kMA$.}

\so

\titre{3- G\'eom\'etrie \'el\'ementaire de l'espace}
\tit{a) Modes de rep\'erage dans l'espace}
\tx{Coordonn\'ees cart\'esiennes, cylindriques, sph\'eriques.

Changements de rep{\`e}re.}
{Pour les coordonn\'ees sph\'eriques, on convient de noter $\theta$
la colatitude, mesure dans $[0,\pi]$ de l'angle entre $Oz$ et $OM$.}


\tit{b) Produit scalaire}
\tx{D\'efinition g\'eom\'etrique du produit scalaire.
Bilin\'earit\'e, sym\'etrie, expression en base orthonormale.}
{Expression de la distance de deux points dans un rep\`ere orthonormal.}

\so
\tit{c) Produit vectoriel}

\tx{D\'efinition g\'eom\'etrique du produit vectoriel de deux vecteurs.
Si $\vec{u}$ et
$\vec{v}$ sont non nuls, le produit vectoriel de $\vec{u}$ et
$\vec{v}$ est le vecteur de norme
$\|\vec{u}\|\,\|\vec{v}\|\sin(\vec{u},\vec{v})$ directement
orthogonal \`a $(\vec{u},\vec{v})$; sinon le produit vectoriel
est le vecteur nul.

Notation $\vec{u}\wedge\vec{v}$.
}
{La notion d'orientation de l'espace est admise, ainsi que celle de
base orthonormale directe. Il convient de donner les conventions
physiques usuelles.

Interpr\'etation de $\|\vec{u}\wedge\vec{v}\|$ comme aire du
parall\'elogramme construit sur $\vec{u}$ et $\vec{v}$.}

\so

\tx{ Bilin\'earit\'e, antisym\'etrie. Expression dans un rep\`ere
ortho\-normal direct. Condition de colin\'earit\'e de deux
vecteurs.}{} \so 

\tit{d) D\'eterminant ou
produit mixte}

\tx{D\'efinition du produit mixte (ou d\'eterminant) de trois vecteurs~:
$$\det(\vec{u},\vec{v},\vec{w})=(\vec{u}\wedge \vec{v})\cdot \vec{w}.$$
\so
Trilin\'earit\'e, antisym\'etrie. Expression en rep\`ere orthonormal
direct. Condition pour que trois vecteurs soient coplanaires.}
{Interpr\'etation de $|\det(\vec{u},\vec{v},\vec{w})|$
comme volume du
parall\'el\'epip\`ede construit sur $\vec{u}$, $\vec{v}$ et $\vec{w}$.}


\so
\tit{e) Droites et plans}

\tx{Param\'etrage d'une droite d\'efinie par un point et un vecteur
directeur, deux points distincts, deux plans s\'ecants.


\'Equation d'un plan d\'efini par un point et deux vecteurs
ind\'ependants, un point et un vecteur normal, trois points non
align\'es. \'Equation normale d'un plan; distance \`a un plan.

Perpendiculaire commune.

Distance {\`a} une droite.}{}

\so
\tit{f) Sph\`eres}

\tx{\'Equation cart\'esienne d'une sph\`ere en rep\`ere orthonormal.
Intersection d'une sph\`ere et d'une droite, d'une
sph\`ere et d'un plan, de deux sph{\`e}res.}
\so
\so

\Titre{II. FONCTIONS USUELLES ET \'EQUATIONS DIFF\'ERENTIELLES LIN\'EAIRES}

{\sl
Les propri\'et\'es \'el\'ementaires li\'ees \`a la continuit\'e
et \`a la d\'erivabilit\'e des fonctions r\'eelles d'une variable r\'eelle sont
suppos\'ees connues. Les d\'eriv\'ees des fonctions circulaires
r\'eciproques seront d\'etermin\'ees en admettant le th\'eor\`eme sur
la d\'erivabilit\'e d'une fonction r\'eciproque.}
\so
\titre{1- Fonctions usuelles}

{\sl Les propri\'et\'es {\sl des fonctions polynomiales
et rationnelles} et des fonctions $\exp$, (sur $\R$), $\ln$, $\cos$, $\sin$
sont rappel\'ees sans d\'emonstration.}
\so
\tit{a) Fonctions exponentielles, logarithmes, puissances}

\tx{Fonctions exponentielles r\'eelles, fonctions logarithmes. Fonc\-tions
puissances. Croissances compar{\'e}es de ces fonctions.}
{Les \'etudiants doivent savoir
d\'eriver une fonction de la forme $x\mapsto u(x)^{v(x)}$.}
\so


\tx{Fonctions hyperboliques ch, sh et th. Fonctions hyperboliques
r\'eciproques $\argch$, $\argsh$ et $\argth$.
}{Les {\'e}tudiants doivent conna{\^i}tre les d{\'e}riv{\'e}es, les variations et les repr{\'e}sentations graphiques
des fonctions hyperboliques directes et r{\'e}ci\-pro\-ques.

En ce qui concerne la trigonom\'etrie hyperbolique, la seule formule
exigible des \'etudiants est la relation $\hbox{\rm ch}^2t-\hbox{\rm
sh}^2t=1$ et son interpr\'etation g\'eom\'etrique.}

\so



\tit{b) Fonctions circulaires}

\tx{Fonctions circulaires cos, sin et tan.

\so
Fonctions circulaires r\'eciproques $\arcsin$, $\arccos$, $\arctan$.}
{Les \'etudiants doivent conna\^{\i}tre les d\'eriv\'ees, les variations et les
repr\'esentations graphiques des fonctions circulaires directes et r{\'e}ciproques.}
\so

\tit{c) Fonction exponentielle complexe}

\tx{D\'erivation de $t\mapsto {\rm e}^{at}$ o\`u $a\in\C$; d\'erivation de
$t\mapsto {\rm e}^{\varphi(t)}$, o\`u $\varphi$ est \`a valeurs complexes.
}
{La d\'eriv\'ee d'une fonction \`a valeurs complexes est d\'efinie par
d\'erivation des parties r\'eelle et imaginaire.}
\so


\titre{2- \'Equations diff\'erentielles lin\'eaires}
{\sl Il convient ici de rappeler la notion de primitive et d'admettre
le th\'eor\`eme fondamental la reliant \`a la notion d'int\'egrale. Toute
th\'eorie g\'en\'erale de l'int\'egration est exclue \`a ce stade.

\so
L'objectif, tr\`es modeste, est d'\'etudier les \'equations diff\'erentielles
lin\'eaires du premier ordre et les \'equations lin\'eaires du second ordre \`a
coefficients constants.

Il convient de relier cette \'etude \`a l'enseignement des autres disciplines
scientifiques (syst\`emes m\'ecaniques ou \'electriques gouvern\'es par une loi
d'\'evolution et une condition initiale, traitement du signal) en d\'egageant la signification de certains param\`etres ou
comportements: stabilit\'e, r\'egime permanent, oscillation, amortissement,
fr\'equences propres, r\'esonance.}
\so

\tit{a) \'Equations lin\'eaires du premier ordre}

\tx{Caract\'erisation de la fonction $t\mapsto {\rm e}^{at}$ ($a\in\C$) par
l'\'equation diff\'erentielle $y'=a\,y$ et la condition initiale $y(0)=1$.
}
{\'Equation fonctionnelle $f(t+u)=f(t)f(u)$ o\`u $f$ est une
fonction d\'erivable de $\R$ dans $\C$.}
\so

\tx{\'Equation $y'+a(t)y=b(t)$, o\`u $a,\,b,\,c$ sont des fonctions
continues \`a valeurs r\'eelles ou complexes. \'Equation sans second membre
associ\'ee.
}
{Cons\'equences de la lin\'earit\'e de l'\'equation~: structure de
l'ensemble des solutions; la solution g\'en\'erale de l'\'equation avec
second membre est somme d'une solution particuli\`ere et de la
solution g\'en\'erale de l'\'equation sans second membre; principe de
superposition lorsque $b=b_1+b_2$.}

\tx
{Existence et unicit\'e de la solution satisfaisant \`a une condition initiale
donn\'ee. Droite
vectorielle des solutions de l'\'equation sans second membre associ\'ee.
Expression des solutions sous forme int\'egrale.}{}
\so
\tit{b) M\'ethode d'Euler}

\tx{\S\ M\'ethode d'Euler de r\'esolution approch\'ee dans le cas d'une
\'equation diff\'erentielle lin\'eaire du premier
ordre.}{Interpr\'etation graphique.}
\so
\tit{c) \'Equations lin\'eaires du second ordre \`a coefficients constants}

\tx{\'Equation $ay''+by'+cy=f(t)$, o\`u $a,\,b,\,c$ sont des nombres
complexes, $a\not=0$, et $f$ une somme de fonctions de type $t\mapsto
{\rm e}^{\alpha t} P(t)$, o\`u $\alpha\in\C$ et $P\in\C[X]$.

\'Equation sans second membre associ\'ee.}
{Cons\'equences de la lin\'earit\'e de l'\'equation~: structure de
l'ensemble des solutions; la solution g\'en\'erale de l'\'equation avec
second membre est somme d'une solution particuli\`ere et de la
solution g\'en\'erale de l'\'equation sans second membre; principe de
superposition lorsque $f=f_1+f_2$.}

\tx{Existence et unicit\'e de la solution satisfaisant \`a une condition initiale
donn\'ee. Plan vectoriel des solutions de l'\'equation sans
second membre associ\'ee.}{}
\so

\titre{3- Courbes param\'etr\'ees. Coniques}
{\sl On adopte ici le point de vue suivant. Par d\'efinition, la fonction
vectorielle $f$ tend vers le vecteur $l$ si $\|f-l\|$ tend vers
z\'ero; cela \'equivaut au fait que les fonctions coordonn\'ees
de $f$ tendent vers les coordonn\'ees de $l$.}

\so
\tit{a) Courbes planes param\'etr\'ees}
\tx{D\'erivation de $(f|g)$, $\|f\|$, $\det(f,g)$ lorsque $f$ et $g$
sont deux fonctions ${\cal C}^1$ \`a valeurs dans $\R^2$.}{}
\so


\tx{\S\ Courbe d\'efinie par une repr\'esentation param\'etrique de classe
${\cal C}^k$ $$t\mapsto\vect{OM}(t)=f(t).$$ Point r\'egulier, tangente en
un point r\'egulier.} {} \so

\tx{Interpr\'etation cin\'ematique~: mouvement d'un point
mobile, trajectoire, vitesse, acc\'el\'eration.}{}

\so \tx{Branches infinies~: directions asymptotiques, asymptotes.}{}

\tx{\S\ Courbe d\'efinie par une repr\'esentation polaire
$$f(t)=\rho(t)\,\vec{u}\big(\theta(t)\big),$$
o\`u $\rho$ et
$\theta$ sont deux fonctions r\'eelles de classe ${\cal C}^k$ sur
un intervalle $I$ et $(\vec{u},\vec{v})$ d\'esigne
le rep\`ere polaire.

Calcul des coordonn\'ees de la vitesse et de l'acc\'el\'eration dans le rep\`ere
polaire.
}

\so
\tx{\S\ Courbe d\'efinie par une \'equation polaire $\theta\mapsto\rho(\theta)$ o\`u
$\rho$ est de classe ${\cal C}^k$ et \`a valeurs r\'eelles. Expression dans le
rep\`ere polaire de vecteurs directeurs de la tangente et de la normale.}
{Les seules connaissances sp\'ecifiques exigibles des \'etudiants concernant
l'\'etude de courbes d\'efinies par une \'equation polaire sont celles indiqu\'ees
ci-contre.}
\so
\tit{b) Coniques}

\tx{Dans le plan, lignes de niveau de $\dis{MF\over MH};$ d\'efinition
par excentricit\'e, foyer et directrice d'une parabole, d'une ellipse, d'une
hyperbole.  \'Equations r\'eduites, centres, sommets, foyers. Asymptotes d'une
hyperbole.
}
{Caract\'erisation des ellipses et des hyperboles \`a l'aide des lignes de
niveau de $MF+MF'$ et de $|MF-MF'|$ (d\'efinition bifocale).}
\so
\tx{\'Equation polaire d'une conique de foyer $O$.}{}
\so
\tx{D{\'e}termination en coordonn\'ees
cart\'esiennes ou en coordonn\'ees polaires des tangentes \`a une conique.}

\so \tx{ Image d'un cercle par une affinit\'e orthogonale. } { Projection
orthogonale d'un cercle de l'espace sur un plan.}{}

 \so

\so
\tx{\'Etude des ensembles d\'efinis par une \'equation cart\'esienne (dans un rep\`ere
orthonormal) de la forme $P(x,y)=0$,
o\`u $P$ est un polyn\^ome du second degr\'e \`a deux variables. \'Equation
r\'eduite.}
{Les \'etudiants doivent savoir distinguer la nature
de la conique \`a l'aide du discriminant.}

\vfill\eject
\TITRE{ANALYSE ET G\'EOM\'ETRIE DIFF\'ERENTIELLE}

{\sl Le programme d'analyse est organis\'e autour des concepts
fondamentaux de suite et de fonction. La ma\^{\i}trise du calcul
diff\'erentiel et int\'egral \`a une variable et de ses interventions en
g\'eom\'etrie diff\'erentielle plane constitue un objectif essentiel. \so

Le cadre d'\'etude est bien d\'elimit\'e: suites de nombres r\'eels et de
nombres complexes, fonctions d\'efinies sur un intervalle de \R\ \`a
valeurs r\'eelles ou complexes, courbes planes, notions \'el\'ementaires
sur les fonctions de deux variables r\'eelles.\so

Le programme combine l'\'etude globale des suites et des fonctions
(op\'erations, majorations, caract\`ere lipschitzien, monotonie,
convexit\'e, existence d'extremums$\ldots$) et l'\'etude de leur
comportement local ou asymptotique. En particulier, il convient de mettre
en valeur le caract\`ere local des notions de limite, de continuit\'e, de
d\'erivabilit\'e et de tangente.\so

Il combine aussi l'\'etude de probl\`emes qualitatifs (monotonie d'une
suite ou d'une fonction, existence de limites, continuit\'e, existence de
z\'eros et d'extremums de fonctions, existence de tangentes$\ldots$) avec
celle des probl\`emes quantitatifs (majorations, \'evaluations
asymptotiques de suites et de fonctions, approximations de z\'eros et
d'extremums de fonctions, propri\'et\'es m\'etriques des courbes
planes$\ldots$).\so

En analyse, les majorations et les encadrements jouent un r\^ole
essentiel. Tout au long de l'ann\'ee, il convient donc de d\'egager les
m\'ethodes usuelles d'obtention de majorations et de minorations:
op\'erations sur les in\'egalit\'es, emploi de la valeur absolue ou du
module, emploi du calcul diff\'erentiel et int\'egral (recherche
d'extremums, in\'egalit\'es des accroissements finis et de la moyenne,
majorations tayloriennes$\ldots$). Pour comparer des nombres, des suites
ou des fonctions, on utilise syst\'ematiquement des in\'egalit\'es larges
(qui sont compatibles avec le passage \`a la limite), en r\'eservant les
in\'egalit\'es strictes aux cas o\`u elles sont indispensables. \so

En ce qui concerne l'usage des quantificateurs, il convient
d'entra\^{\i}ner les \'etudiants \`a savoir les employer pour formuler de
fa\c{c}on pr\'ecise certains \'enonc\'es et leurs n\'egations (caract\`ere
born\'e, caract\`ere croissant, existence d'une limite, continuit\'e en un
point, continuit\'e sur un intervalle, d\'erivabilit\'e en un
point$\ldots$). En revanche, il convient d'\'eviter tout recours
syst\'ematique aux quantificateurs. A fortiori, leur emploi abusif
(notamment sous forme d'abr\'eviations) est exclu. \so

Le programme d'analyse et g\'eom\'etrie diff\'erentielle comporte la
construction, l'analyse et l'emploi d'algorithmes num\'eriques
(approximations de solutions d'\'equations num\'eriques, approximations
d'une int\'egrale$\ldots$) et d'algorithmes de calcul formel
(d\'erivation, primitivation$\ldots$); plus largement, le point de vue
algorithmique est \`a prendre en compte pour l'ensemble de ce programme,
notamment pour le trac\'e de courbes.} \so

\Titre{I. NOMBRES R\'EELS, %ET COMPLEXES
SUITES ET FONCTIONS}



\titre{1- Suites de nombres r\'eels}

{\sl Pour la notion de limite d'une suite $(u_n)$ de nombres r\'eels, on adopte
les d\'efinitions suivantes :

$-$~\'etant donn\'e un nombre r\'eel $a$, on dit que $(u_n)$ admet $a$ pour
limite si, pour tout nombre r\'eel $\varepsilon>0$, il existe un entier
$N$ tel que, pour tout entier $n$, la relation $n\se N$ implique la
relation $|u_n-a|\ie\varepsilon$. Lorsqu'un tel nombre $a$
existe, on dit que la suite $(u_n)$ est convergente, ou qu'elle admet une
limite finie; le nombre $a$ est alors unique, et on le
note $\dis\lim_{n\rightarrow {\infty}}u_n$. Dans le cas contraire, on dit que $(u_n)$ est divergente.

$-$~on d\'efinit de mani\`ere analogue la notion de limite lorsque
$a=+\infty$ ou $a=-\infty$; on dit alors que la suite $(u_n)$ tend vers
$+\infty$ ou vers $-\infty$.
\so

En ce qui concerne le comportement global et asymptotique d'une suite, il
convient de combiner l'\'etude de probl\`emes qualitatifs (monotonie,
convergence, divergence$\ldots$) avec celle de probl\`emes quantitatifs
(majorations, encadrements, vitesse de convergence ou de divergence par
comparaison aux suites de r\'ef\'erence usuelles\dots).} 
\so
\tit{a) Corps \R\ des nombres r\'eels}

\tx{Corps \R\ des nombres r\'eels; relation d'ordre, compatibilit\'e avec
l'addition, la multiplication. } {La construction du corps des nombres
r\'eels et la notion de corps totalement ordonn\'e sont hors programme.}
 \so
%%%%
\tx{Valeur absolue d'un nombre r\'eel, distance de deux points.

In\'egalit\'es triangulaires $$||x|-|y||\ie|x+y|\ie|x|+|y|.$$ }{} \so

\tx{D\'efinition d'une borne sup\'erieure, d'une borne inf\'erieure.
Toute partie major\'ee non vide admet une borne sup\'erieure. D\'efinition
de la droite r\'eelle achev\'ee $\overline\R$. }{\null

Propri{\'e}t{\'e} admise.} \so

\tx{D\'efinition des intervalles de \R. Tout intervalle $]a,b[$ non vide
rencontre \Q\ et son compl\'ementaire. } {Toute partie convexe de \R\ est
un intervalle.} \so


\tx{Partie enti\`ere d'un nombre r\'eel. Valeurs d\'ecimales ap\-pro\-ch\'ees
\`a la pr\'ecision $10^{-n}$; approximation par d\'efaut, par exc\`es. }
{La notion de d\'eveloppement d\'ecimal illimit\'e est hors programme.}
\so


\tit{b) Suites de nombres r\'eels}

\tx{Espace vectoriel des suites de nombres r\'eels, relation d'ordre.

Suites major\'ees, minor\'ees. Suites born\'ees.

Suites monotones, strictement monotones. } {Pour la pr\'esentation du
cours, le programme se place dans le cadre des suites index\'ees par \N.
On effectue ensuite une br\`eve extension aux autres cas usuels.} \so

\tit{c) Limite d'une suite}

\tx{Limite d'une suite, convergence et divergence.

Lorsque $a\in\R$, la relation $u_n\rightarrow a$ \'equivaut \`a
$u_n-a\rightarrow0$. } {Tout nombre r{\'e}el est limite d'une suite de nombres rationnels.} \so

\tx{Toute suite convergente est born\'ee. } {Toute suite de nombres
r\'eels convergeant vers un nombre r\'eel strictement positif est
minor\'ee, \`a partir d'un certain rang, par un nombre r\'eel strictement
positif.} \so

\tx{Espace vectoriel des suites convergeant vers 0; produit d'une suite
born\'ee et d'une suite convergeant vers 0. } {} \so

\tx{Op\'erations alg\'ebriques sur les limites; compatibilit\'e du passage
\`a la limite avec la relation d'ordre. } {Si $|u_n|\leq\alpha_n\ {\rm
et}\ \alpha_n\tend 0$, alors $u_n\tend 0$. \so

Si $v_n\leq u_n\leq w_n$, et si $v_n\tend a$ et $w_n\tend a$, alors
$u_n\tend a$.

Si $v_n\ie u_n$ et si $v_n\rightarrow+\infty$, alors
$u_n\rightarrow+\infty$.} \so

\tx{Suites extraites d'une suite. Toute suite extraite d'une suite
convergeant vers $a$ converge vers $a$. } {Application \`a la divergence
d'une suite born\'ee: il suffit d'exhiber deux suites extraites
convergeant vers des limites diff\'erentes.

La notion de valeur d'adh\'erence d'une suite est hors programme.} \so

\tit{d) Relations de comparaison}

\tx{\'Etant donn\'ee une suite $(\alpha_n)$ de nombres r\'eels non nuls,
d\'efinition d'une suite $(u_n)$ de nombres r\'eels domin\'ee par
$(\alpha_n)$, n\'egligeable devant $(\alpha_n)$. } {Notations $u_n={\rm
O}(\alpha_n),\ u_n={\rm o}(\alpha_n)$. \so

Caract\'erisations \`a l'aide du quotient
$\displaystyle{u_n\over\alpha_n}\cdot$} \so

\tx{D\'efinition de l'\'equivalence de deux suites $(u_n)$ et $(v_n)$ de
nombres r\'eels non nuls. \'Equivalent d'un produit, d'un quotient. \so

Si $u_n=\alpha_n+w_n$, o\`u $w_n$ est n\'egligeable devant $\alpha_n$,
alors $u_n\sim\alpha_n$. } {Notation $u_n\sim v_n$. \so

Caract\'erisation \`a l'aide du quotient $\displaystyle{u_n\over
v_n}\cdot$

Si $u_n\sim v_n$, alors, \`a partir d'un certain rang, le signe de $u_n$
est \'egal \`a celui de $v_n$.} \so

\tx{Comparaison des suites de r\'ef\'erence: $$n\mapsto a^n,\ n\mapsto
n^{\alpha}, \ n\mapsto (\ln n)^{\beta},\ n\mapsto n!$$ o\`u $a>0$,
$\alpha\in\R$, $\beta\in\R$. } {} \so
\tx{
Exemples simples de d\'eveloppements asymptotiques.}{Toute \'etude syst\'ematique
est exclue; en particulier, la notion g\'en\'erale d'\'echelle de
comparaison est hors programme.} \so


\tit{e) Th\'eor\`emes d'existence de limites}

\tx{Toute suite croissante major\'ee $(u_n)$ converge, et
$$\lim_nu_n=\sup_nu_n.$$ } {Extension au cas d'une suite croissante non
major\'ee.} \so

\tx{Suites adjacentes. Th\'eor\`eme des segments embo\^{\i}t\'es. } {Les \'etudiants doivent
conna\^{\i}tre et savoir exploiter la notion de suite dichotomique
d'intervalles.} \so

\tx{Th\'eor\`eme de Bolzano-Weierstrass: de toute suite born\'ee de
nombres r\'eels, on peut extraire une suite convergente. } {La d{\'e}monstration de ce th{\'e}or{\`e}me n'est pas exigible des {\'e}tudiants}


\so
\tit{f) Br{\`e}ve extension aux suites complexes}

\tx{Suites \`a valeurs complexes; parties r\'eelle et
imaginaire d'une suite; conjugaison.
\so
Suites born\'ees. } {Notations $\re u_n$, $\im u_n$, $\bar u_n$, $|u_n|$.} \so

\tx{
Limite d'une suite \`a valeurs complexes; caract\'erisation \`a l'aide des parties
r\'eelle et imaginaire.}
{Toute suite convergente est born\'ee.}

\so
\tx{Op\'erations alg\'ebriques sur les limites. \so
}{} \so


\titre{2- Fonctions d'une variable r\'eelle \`a valeurs r\'eelles}

{\sl Pour la notion de limite d'une fonction $f$ en un point $a$ (appartenant
\`a $I$ ou extr\'emit\'e de $I$), on adopte les d\'efinitions suivantes:

-~\'Etant donn\'es des nombres r\'eels $a$ et $b$, on dit que $f$ admet
$b$ pour limite au point $a$ si, pour tout nombre r\'eel $\varepsilon>0$,
il existe un nombre r\'eel $\delta>0$ tel que, pour tout \'el\'ement $x$
de $I$, la relation $|x-a|\ie\delta$ implique la relation
$|f(x)-b|\ie\varepsilon$; le nombre $b$ est alors unique, et on le note
$\lim_{x\rightarrow a}f$. Lorsqu'un tel nombre $b$ existe, on dit que $f$
admet une limite finie au point $a$.

-~On d\'efinit de mani\`ere analogue la notion de limite lorsque
$a=+\infty$ ou $a=-\infty$, ou lorsque $b=+\infty$ ou $b=-\infty$. \so

Dans un souci d'unification, on dit qu'une propri\'et\'e portant sur une
fonction d\'efinie sur $I$ est vraie au voisinage de $a$ si elle est vraie
sur l'intersection de $I$ avec un intervalle ouvert de centre $a$ lorsque
$a\in\R$, avec un intervalle $]c,+\infty[$ lorsque $a=+\infty$ et avec un
intervalle $]-\infty,c\,[$ lorsque $a=-\infty$.\so

En ce qui concerne le comportement global et local (ou asymptotique) d'une
fonction, il convient de combiner l'\'etude de probl\`emes qualitatifs
(monotonie, existence de z\'eros, existence d'extremums, existence de
limites, continuit\'e, d\'erivabilit\'e$\ldots$) avec celle de probl\`emes
quantitatifs (majorations, encadrements, caract\`ere lipschitzien,
comparaison aux fonctions de r\'ef\'erence au voisinage d'un
point\dots).} \so

\tit{a) Fonctions d'une variable r\'eelle \`a valeurs r\'eelles}

\tx{Espace vectoriel des fonctions \`a valeurs r\'eelles, relation d'ordre.

Fonctions major\'ees, minor\'ees. Fonctions born\'ees. }
{D\'efinition de $|f|$, $\sup(f,g)$, $\inf(f,g)$.} \so

\tx{D\'efinition d'un extremum, d'un extremum local. } {Notations
$\dis\max_{x\in I}f(x)$ et $\dis\max_If$.} \so

\tx{D\'efinition de la borne sup\'erieure (inf\'erieure) d'une fonction. }
{Notations $\dis\sup_{x\in I}f(x)$ et $\dis\sup_If$.} \so

\tx{Fonctions monotones, strictement monotones; composition. } {} \so

\tx{Sous-espace vectoriel des fonctions paires, des fonctions impaires. }{}
 \so

\tx{Fonctions $T$-p\'eriodiques, op\'erations. }{} \so

\tx{D\'efinition des fonctions lipschitziennes. }{} \so

\tit{b) \'Etude locale d'une fonction}

\tx{Limite d'une fonction $f$ en un point $a$, continuit\'e en un point.

Lorsque $b\in\R$, la relation $f(x)\rightarrow b$ \'equivaut \`a la
relation $f(x)-b\rightarrow 0$.

Lorsque $a\in\R$, la relation $f(x)\rightarrow b$ lorsque $x\rightarrow a$
\'equivaut \`a la relation $f(a+h)\rightarrow b$ lorsque $h\rightarrow0$.
} {Lorsque $a\in I$, dire que $f$ a une limite finie en $a$ \'equivaut \`a
la continuit\'e de $f$ en ce point. Lorsque $a\not\in I$, $f$ a une limite
finie en $a$ si et seulement si $f$ se prolonge par continuit\'e en ce
point.} \so

\tx{Limite \`a gauche, limite \`a droite.

Continuit\'e \`a gauche, continuit\'e \`a droite. } {Les limites \`a
gauche (ou \`a droite) en $a$ sont d\'efinies par restriction de $f$ \`a
$I\,\cap\,]-\infty,a\,[$ (\`a $I\,\cap\,]a,+\infty[$).} \so

\tx{Toute fonction admettant une limite finie en un point est born\'ee au
voisinage de ce point. } {Toute fonction admettant une limite strictement
positive en un point est minor\'ee, au voisinage de ce point, par un
nombre r\'eel strictement positif.} \so

\tx{Espace vectoriel des fonctions tendant vers 0 en un point $a$; produit
d'une fonction d'une fonction born\'ee au voisinage de $a$ par une
fonction tendant vers 0 en $a$. }{} \so

\tx{Op\'erations alg\'ebriques sur les limites; compatibilit\'e du passage
\`a la limite avec la relation d'ordre. } {Si $|f(x)|\leq g(x)\ {\rm et}\
g(x)\tend 0$, alors $f(x)\tend 0$. \so

Si $g(x)\leq f(x)\leq h(x)$, et si $g(x)\tend b$ et $h(x)\tend b$, alors
$f(x)\tend b$.} \so

\tx{Limite d'une fonction compos\'ee. Image d'une suite convergente. }{}

\so

\tx{Existence d'une limite d'une fonction monotone. } {Comparaison des
bornes (sup\'erieure ou inf\'erieure) et des limites (\`a gauche ou \`a
droite).} \so

\tit{c) Relations de comparaison}

\tx{\'Etant donn\'es un point $a$ (appartenant \`a $I$ ou extr\'emit\'e de
$I$) et une fonction $\varphi$ \`a valeurs r\'eelles et ne s'annulant pas
sur $I$ priv\'e de $a$, d\'efinition d'une fonction $f$ \`a valeurs
r\'eelles, domin\'ee par $\varphi$ (n\'egligeable devant $\varphi$) au
voisinage de $a$. } {Notations $f={\rm O}(\varphi), \ f={\rm o}(\varphi)$.
\so

Caract\'erisations \`a l'aide du quotient $\dis{f\over\varphi}\cdot$} \so

\tx{D\'efinition de l'\'equivalence au voisinage de $a$ de deux fonctions
$f$ et $g$ \`a valeurs r\'eelles ne s'annulant pas sur $I$ priv\'e de $a$.
\'Equivalent d'un produit, d'un quotient. \so

Si $f=\varphi+h$, o\`u $h$ est n\'egligeable devant $\varphi$, alors
$f\sim\varphi$. } {Notation $f\sim g$. \so

Caract\'erisation \`a l'aide du quotient $\dis{f\over g}\cdot$ \so

Si $f\sim g$ alors, au voisinage de $a$, le signe de $f(x)$ est \'egal \`a
celui de $g(x)$.} \so

\tx{Application {\`a} la comparaison des fonctions usuelles.}{{}} \so

\tit{d) Fonctions continues sur un intervalle}

\tx{Espace vectoriel ${\cal C}(I)$ des fonctions continues sur $I$ et \`a valeurs
r\'eelles.

Compos\'ee de deux fonctions continues. } {Si $f$ et $g$ sont continues, $|f|$, $\sup(f,g)$, $\inf(f,g)$ le sont.} \so

\tx{Restriction d'une fonction continue \`a un intervalle $J$ contenu
dans $I$.

Prolongement par continuit\'e en une extr\'emit\'e de $I$. }{} \so

\tx{Image d'un intervalle par une fonction continue. Th{\'e}or{\`e}me des valeurs interm{\'e}diaires.
\so
Image d'un segment
par une fonction continue. } {La d\'emonstration de ces  r\'esultats
n'est pas exigible.

Les {\'e}tudiants doivent savoir utiliser les m{\'e}thodes dichotomiques pour la recherche des z{\'e}ros d'une fonction continue.
} \so

\tx{Continuit\'e de la fonction r\'eciproque d'une fonction continue
strictement monotone. } {Comparaison des repr\'esentations graphiques
d'une bijection et de la bijection r\'eciproque.} \so

\tx{D\'efinition de la continuit\'e uniforme. Continuit\'e uniforme d'une
fonction continue sur un segment. } {La d\'emonstration de ce r\'esultat
n'est pas exigible des \'etudiants.

Toute \'etude syst\'ematique des fonctions uniform\'ement continues est exclue.}
\so


\tit{e) Br{\`e}ve extension aux fonctions \`a valeurs complexes}

\tx{Fonctions \`a valeurs complexes; parties r\'eelle et
imaginaire d'une fonction; conjugaison.
\so
Fonctions born\'ees. } {Notations $\re f$, $\im f$, $\bar f$, $|f|$.} \so

\tx{
Limite d'une fonction \`a valeurs complexes en un point $a$,
continuit\'e en $a$;
caract\'erisation \`a l'aide des parties
r\'eelle et imaginaire.}
{Toute fonction admettant une limite en un point est born\'ee au voisinage
de ce point.}

\so
\tx{Op\'erations alg\'ebriques sur les limites. \so

Ensemble ${\cal C}(I)$ des fonctions continues sur $I$ \`a valeurs
complexes. }{} \so


%%%%%%%%

\Titre{II. CALCUL DIFF\'ERENTIEL ET INT\'EGRAL}

{\sl
Le programme est organis\'e autour de trois axes:

-~D\'erivation en un point et sur un intervalle; notions sur la convexit\'e.

-~Int\'egration sur un segment des fonctions continues par morceaux, \`a partir
de l'int\'egration des fonctions en escalier.

-~Th\'eor\`eme fondamental reliant l'int\'egration et la d\'erivation; exploitation
de ce th\'eor\`eme pour le calcul diff\'erentiel et int\'egral, et notamment pour
les formules de Taylor.
\so

L'\'etude g\'en\'erale de la d\'erivation et de l'int\'egration doit \^etre illustr\'ee
par de nombreux exemples portant sur les fonctions usuelles (vues en d\'ebut d'ann\'ee) et celles qui s'en
d\'eduisent.
}
\so

\titre{1- D\'erivation des fonctions \`a valeurs r\'eelles}

\tit{a) D\'eriv\'ee en un point, fonction d\'eriv\'ee}

\tx{D\'erivabilit\'e en un point: d\'eriv\'ee, d\'eriv\'ee \`a gauche, \`a droite.
\so

Extremums locaux des fonctions d\'erivables.
}
{Les \'etudiants doivent conna\^{\i}tre et savoir exploiter l'interpr\'etation
graphique et l'inter\-pr\'e\-ta\-tion cin\'ematique de la notion de d\'eriv\'ee en un
point.}
\so

\tx{D\'erivabilit\'e sur un intervalle, fonction d\'eriv\'ee. Op\'erations sur les
d\'eriv\'ees: lin\'earit\'e, produit, quotient, fonctions compos\'ees, fonctions
r\'eciproques.
}
{Notations $f'$, D$f$, $\dis{\hbox{\rm d}f\over \hbox{\rm d}x}\cdot$}
\so

\tx{Pour $0\ie k\ie+\infty$, ensemble ${\cal C}^k(I)$ des fonctions de classe ${\cal C}^k$ ; op\'erations. D\'eriv\'ee $n$-i\`eme d'un produit (formule de Leibniz).
}
{Notations $f^{(k)}$, $D^{\,k}f$, $\dis{\hbox{\rm d}^kf\over\hbox{\rm
d}x^k}\cdot$}
\so

\tx{Br{\`e}ve extension aux fonctions {\`a} valeurs complexes}{}

\so

\tit{b) \'Etude globale des fonctions d\'erivables}

\tx{Th\'eor\`eme de Rolle, \'egalit\'e des accroissements finis.
\so

In\'egalit\'e des accroissements finis:

-~si $m\ie f'\ie M$, alors $m(b-a)\ie f(b)-f(a)\ie M(b-a);$

-~si $|f'|\ie k$, alors $f$ est $k$-lipschitzienne.
\so

Caract\'erisation des fonctions constantes, monotones et strictement
monotones parmi les fonctions d\'erivables.
}
{Pour le th\'eor\`eme de Rolle, l'\'egalit\'e et l'in\'egalit\'e des accroissements
finis, ainsi que pour la caract\'erisation des fonctions monotones, on
suppose $f$ continue sur $[a,b]$ et d\'erivable sur $]a,b[$.
\so

Les \'etudiants doivent conna\^{\i}tre l'interpr\'etation graphique et cin\'ematique
de ces r\'esultats. 

\S\ Ils doivent savoir {\'e}tudier des suites de nombres r\'eels d\'efinies par une
relation de r\'ecurrence $u_{n+1}=f(u_n)$ et utiliser une telle suite pour
l'approximation d'un point fixe $a$ de $f$.}
\so
\tx{\S\ Application de l'in{\'e}galit{\'e} des accroissements finis {\`a} l'{\'e}tude des suites d{\'e}finies par une relation de r{\'e}currence$$u_{n+1}=f(u_{n})$$}{Voir le chapitre  {\bf 4- Approximation.}}\so

\tx{Si $f$ est continue sur $[a,b]$, de classe ${\cal C}^1$ sur $]a,b]$ et
si $f'$ a une limite finie en $a$, alors $f$ est de classe ${\cal C}^1$ sur
$[a,b]$.
}
{Br\`eve extension au cas d'une limite infinie.}
\so

\tit{c) Fonctions convexes}

\tx{D\'efinition, interpr\'etation graphique (tout sous-arc est sous sa corde).

Croissance des pentes des s\'ecantes dont on fixe une extr\'emit\'e.
}
{In\'egalit\'e de convexit\'e: si $\lambda_j\geq0$ et
$\dis\sum_{j=1}^n\lambda_j=1$, alors
$$\dis{f\big(\sum_{j=1}^n\lambda_j a_j\big)\leq
\sum_{j=1}^n\lambda_jf(a_j)}.$$}
\so

\tx{Si $f$ est de classe ${\cal C}^1$, $f$ est convexe si et seulement si
$f'$ est croissante. La courbe est alors situ\'ee au dessus de chacune de ses
tangentes.
}
{L'\'etude de la continuit\'e et de la d\'erivabilit\'e des fonctions convexes est
hors programme.}
\so

\tit{d) Br\`eve extension aux fonctions \`a valeurs complexes}

\tx{D\'erivabilit\'e en un point, caract\'erisation \`a l'aide des parties r\'eelle
et imaginaire; op\'erations sur les fonctions d\'erivables. Espace vectoriel ${\cal
C}^k(I)$ des fonctions de classe ${\cal C}^k$ \`a valeurs complexes, o\`u $0\ie
k\ie+\infty$; d\'eriv\'ee $n$-i\`eme d'un produit.
}
{In{\'e}galit{\'e} des accroissements finis. Caract\'e\-ri\-sa\-tion des fonctions constantes.

Il convient de montrer, \`a l'aide d'un contre-exemple, que le th\'eor\`eme de
Rolle ne s'\'etend pas.}
\so
\titre{2- Int\'egration sur un segment des fonctions \`a valeurs r\'eelles}

{\sl Le programme se limite \`a l'int\'egration des fonctions continues par
morceaux sur un segment. Les notions de fonction r\'egl\'ee et de fonction
int\'egrable au sens de Riemann sont hors programme.

%En vue de l'enseignement des autres disciplines scientifiques, il convient
%de d\'efinir la convergence absolue de l'int\'egrale d'une fonction continue
%sur un intervalle quelconque, mais, en math\'ematiques, aucune connaissance
%sp\'ecifique sur ce point n'est exigible des \'etudiants.
}
\so

\tit{a) Fonctions continues par morceaux}

\tx{D\'efinition d'une fonction $\varphi$ en escalier sur $[a,b]$, d'une
subdivision de $[a,b]$ subordonn\'ee \`a $\varphi$. Ensemble des fonctions en
escalier sur un segment.
}
{}
\so

\tx{Ensemble des fonctions continues par morceaux sur un segment; op\'erations.
}{}
\so

\tx{Approximation des fonctions continues par morceaux sur un segment par
des fonctions en escalier: \'etant donn\'ee une fonction $f$ continue par
morceaux sur $[a,b]$, pour tout r\'eel $\eps>0$, il existe des fonctions
$\varphi$ et $\psi$ en escalier sur $[a,b]$ telles que: $$\varphi\leq
f\leq\psi\quad{\rm et}\quad\psi-\varphi\leq\eps.$$ } {} \so

\tit{b) Int\'egrale d'une fonction continue par morceaux}

\tx{Int\'egrale d'une fonction en escalier sur un segment. Lin\'earit\'e.
Croissance.

Int\'egrale d'une fonction continue par morceaux sur un segment.

\so
Notations $\int_{I}f$, $\int_{[a,b]}f$.
\so

D{\'e}finition de $\int_a^bf(t)dt$, o{\`u} $a$ et $b$ appartiennent {\`a} $I$.

\so  Lin\'earit\'e.
Croissance; in\'egalit\'e $\left |\int_If\right |\leq\int_I|f|$.

Additivit\'e par rapport \`a l'intervalle d'int\'egration, relation de Chasles.

\so
Invariance de l'int\'egrale par translation.
}
{Il convient d'interpr\'eter graphiquement l'int\'e\-grale d'une fonction \`a
valeurs positives en termes d'aire. Aucune difficult\'e th\'eorique ne doit
\^etre soulev\'ee sur la notion d'aire.}
\so

\tx{Valeur moyenne d'une fonction.
}
{}
\so

\tx{In\'egalit\'e de la moyenne $$\left|\int_{[a,b]}fg\right|\leq
\sup\limits_{[a,b]}|f|\int_{[a,b]}|g|.$$
}
{En particulier $$\left|\int_{[a,b]}f\right|\leq(b-a)\,
\sup\limits_{[a,b]}|f|.$$

Toute autre formule ou \'egalit\'e dite de la moyenne est hors programme.}
\so

\tx{Une fonction $f$ continue et \`a valeurs positives sur un segment est
nulle si et seulement si son int\'egrale est nulle.

Produit scalaire $(f,g)\mapsto\int_Ifg$ sur l'espace vectoriel ${\cal
C}(I)$; in\'egalit\'e de Cauchy-Schwarz.
}
{}
\so

\tx{Approximation de l'int\'egrale d'une fonction $f$ continue sur $[a,b]$
par les sommes de Riemann $$R_n(f)={b-a\over n}\,\sum_{j=0}
^{n-1}f(a_j)$$ o\`u $(a_0,\ldots,a_n)$ est une subdivision \`a pas constant.
}
{Cas o\`u $f$ est $k$-lipschitzienne sur $[a,b]$.}
\so



\tx{\S\ Approximation d'une int\'egrale par la m\'ethode des trap\`ezes.
}
{}
\so

\tit{c) Br{\`e}ve extension aux fonctions {\`a} valeurs complexes}

\tx{Par d\'efinition, $$\int_If=\int_I\re f+{\rm i}\int_I\im f.$$
\so

Lin{\'e}arit{\'e}, relation de Chasles et in{\'e}galit{\'e} de la moyenne.}{}



\titre{3- Int\'egration et d\'erivation}

{\sl Dans cette partie, les fonctions consid{\'e}r{\'e}es sont {\`a} valeurs r{\'e}elles ou complexes.}\so

\tit{a) Primitives et int\'egrale d'une fonction continue}

\tx{D\'efinition d'une primitive d'une fonction continue.

Deux primitives d'une m\^eme fonction diff\`erent d'une cons\-tante.}
{Il convient de montrer sur des exemples que cette d\'efinition ne peut \^etre
\'etendue sans changement au cas des fonctions continues par morceaux.}
\so

\tx{Th\'eor\`eme fondamental: \'etant donn\'es une fonction $f$ continue sur un
intervalle $I$ et un point $a\in I$,
\so

-~la fonction $x\mapsto\int_a^x f(t)\,\hbox{\rm d}t$ est
l'unique primitive de $f$ qui s'annule en $a$;
}{}
\so

\tx{-~pour toute primitive $h$ de $f$ sur $I$, $$\int_a^x f(t)\,\hbox{\rm
d}t=h(x)-h(a).$$
}
{Pour toute fonction $f$ de classe ${\cal C}^1$ sur $I$,
$$f(x)-f(a)=\int_a^x f'(t)\,\hbox{\rm d}t.$$}
\so

\tit{b) \S\ Calcul des primitives}

\tx{Int\'egration par parties pour des fonctions de classe ${\cal C}^1$.
}
{}
\so

\tx{Changement de variable: \'etant donn\'ees une fonction $f$ continue sur $I$
et une fonction $\varphi$ \`a valeurs dans $I$ et de classe ${\cal C}^1$ sur
$[\alpha,\beta]$,
$$\int_{\varphi(\alpha)}^{\varphi(\beta)}f(t)\,\hbox{\rm d}t
=\int_\alpha^\beta f\big(\varphi(u)\big)\,\varphi'(u)\,\hbox{\rm d}u.$$
}
{Il convient de mettre en valeur l'int\'er\^et de changements de variable
affines, notamment pour exploiter la p\'eriodicit\'e et les sym\'etries, ou pour
se ramener, par param\'etrage du segment $[a,b]$, au cas o\`u l'intervalle
d'int\'egration est $[0,1]$ ou $[-1,1]$.}
\so

\tx{Primitives des fonctions usuelles.}{}\so






\tit{c) Formules de Taylor}

\tx{Pour une fonction de classe ${\cal C}^{p+1}$ sur $I$, formule de Taylor
avec reste int{\'e}gral \`a l'ordre $p$ en un point $a$ de $I$.


Majoration du reste: in\'egalit\'e de Taylor-Lagrange.
}{Relation $f(x)=T_p(x)+R_p(x)$, o\`u $$T_p(x)=\dis{\sum_{n=0}^p{(x-a)^n\over
n!}D^{\,n}f(a)}.$$}

\so
\tit{d) D{\'e}veloppement limit{\'e}s}

\tx{\S\ D\'eveloppement limit\'e \`a l'ordre $n$ d'une fonction au voisinage
d'un point; op\'erations alg\'ebriques sur les {d\'e}\-velop\-pe\-ments limit\'es:
somme, produit; d\'eveloppement limit\'e de $\dis u\mapsto{1\over 1-u}$,
application au quotient. } {Les \'etudiants doivent savoir d\'eterminer
sur des exemples simples le d\'eveloppement limit\'e d'une fonction
compos\'ee. Aucun r\'esultat g\'en\'eral sur ce point n'est exigible.}

\so\tx{D{\'e}veloppement limit{\'e} des fonctions usuelles exp, ln, sin,cos, $x\mapsto (1+x)^{\alpha }$}{}

\so
\tx{Application {\`a} l'{\'e}tude des points singuliers (ou stationnaires) des courbes
param{\'e}tr{\'e}es planes.}{}\so

\tx{Existence d'un d\'eveloppement limit\'e \`a l'ordre $p$ pour une fonction de
classe ${\cal C}^p$: formule de Taylor-Young.

\so
D\'eveloppement limit\'e d'une primitive, d'une d\'eriv\'ee.
}{}

\so

\tx{\S\ 
Exemples simples de d\'eveloppements asymptotiques.}{Toute \'etude syst\'ematique
est exclue; en particulier, la notion g\'en\'erale d'\'echelle de
comparaison est hors programme.} \so


\so

\titre{4- Approximation}

{\sl Dans cette partie sont regroup\'ees un certain nombre de m\'ethodes d\'ebouchant sur des calculs approch\'es par la mise en place d'algorithmes. Aucune connaissance n'est exigible concernant les erreurs; seul leurs ordres de grandeur doivent \^etre connus des \'etudiants. \`A cette occasion, on pourra introduire la notion de rapidit\'e de convergence d'une suite mais aucune connaissance n'est exigible sur ce point.

\so
Les algorithmes pr\'esent\'es ne sont regroup\'es que pour la commodit\'e de la pr\'esentation; leur \'etude doit intervenir au fur et \`a mesure de l'avancement du programme.}

\so
\tit{a) Calcul approch\'e des z\'eros d'une fonction}

{\sl On consid\`ere ici des fonctions $f:I\tend \R$, o\`u $I$ est un intervalle de $\R$.}

\so
\tx{M\'ethode de dichotomie.}{Pratique d'un test d'arr\^et.}

\so

\tx{Utilisation de suites r\'ecurrentes (m\'ethode d'approximations successives).}
{Le th\'eor\`eme du point fixe de Cauchy est hors programme sous forme g\'en\'erale mais les \'etudiants doivent savoir utiliser l'in\'egalit\'e des accroissements finis pour justifier une convergence.}

\so
\tx{M\'ethode de Newton et algorithme de Newton-Raphson.
}
{On d\'egagera, sur des exemples, le caract\`ere quadratique de la convergence.

Convergence dans le cas d'une fonction $f$ de classe ${\cal C}^2$ telle que $f''$ ne s'annule pas, si la valeur initiale $x_0$ est telle que $f(x_0)f''(x_0)\geq0$.}

\so
\tit{b) Calcul approch\'e d'une int\'egrale}

\tx{Pr\'esentation d'un algorithme associ\'e {\`a} la m\'ethode des trap\`ezes. Il convient de souligner l'int{\'e}r{\^e}t des subdivisions dichotomiques.}{On admettra que pour une fonction de classe ${\cal C}^1$, l'erreur est un $O\Bigl({1\over n^2}\Bigr)$, o{\`u} $n$ est le nombre de points de la subdivision.}

\so
\tit{c) Valeur approch\'ee de r\'eels}

\tx{On pr\'esentera, sur des exemples, quelques algorithmes de calcul de nombres r\'eels remarquables ($\pi,\ \e, \sqrt2$, etc.).}{On pourra utiliser les algorithmes vus {pr\'ec\'e}\-dem\-ment.}





\so

\Titre{III. NOTIONS SUR LES FONCTIONS DE DEUX VARIABLES R\'EELLES}

{\sl Cette partie constitue une premi\`ere prise de contact avec les fonctions de
plusieurs variables; toute technicit\'e est \`a \'eviter aussi bien pour la
pr\'esentation du cours qu'au niveau des exercices et probl\`emes.
\so

L'objectif, tr\`es modeste, est triple:

$-$~\'etudier quelques notions de base sur les fonctions de deux variables
r\'eelles (continuit\'e et d\'erivation) ;


$-$~introduire la notion d'int{\'e}grale double ;


$-$~exploiter les r\'esultats obtenus pour l'\'etude de probl\`emes, issus
notamment des autres disciplines scientifiques.
\so

En vue de l'enseignement de ces disciplines, il convient d'\'etendre
bri\`evement ces notions aux fonctions de trois variables r\'eelles. Mais, en
math\'ematiques, les seules connaissances exigibles des \'etudiants ne portent
que sur les fonctions de deux variables.}
\so

\titre{1- Espace $\R^2$, fonctions continues}

{\sl Les fonctions consid\'er\'ees dans ce chapitre sont d\'efinies sur une
partie $A$ de $\R^2$ qui est muni de la norme euclidienne usuelle ; l'\'etude g\'en\'erale
des normes sur $\R^2$ est hors programme. Pour la pratique,
on se limite aux cas o\`u $A$ est
d\'efinie par des conditions simples.
\so

Pour d\'efinir la notion de limite, on proc\`ede comme pour les fonctions d'une
variable r\'eelle.}
\so



\tx{Espace vectoriel des fonctions d\'efinies sur $A$ et \`a valeurs r\'eelles.
Applications partielles associ\'ees \`a une telle fonction.
}
{}
\so

\tx{Limite et continuit\'e en un point $a$ d'une fonction d\'efinie sur une
partie $A$ et \`a valeurs r\'eelles.

\so
Espace vectoriel des fonctions continues sur $A$ et \`a valeurs r\'eelles;
op\'erations.}
{Il convient de montrer, sur un exemple simple, que la continuit\'e
des applications partielles n'implique pas la continuit\'e, mais l'\'etude de
la continuit\'e partielle est hors programme.}

\so

\tx{Extension des notions de limite et de continuit\'e \`a une application de $A$
dans $\R^2$; caract\'erisation \`a l'aide des coordonn\'ees.

Continuit\'e d'une application compos\'ee.
}{}\so

\tx{D\'efinition des parties ouvertes de $\R^2$. } {Les op\'erations sur
les ouverts, ainsi que les notions de partie ferm\'ee, de voisinage,
d'int\'erieur et d'adh\'erence d'une partie sont hors programme.} \so

\titre{2- 
Calcul diff\'erentiel}

{\sl Les fonctions \'etudi\'ees dans ce chapitre sont d\'efinies sur un ouvert
$U$ de $\R^2$ et \`a valeurs r\'eelles.
\so

L'objectif essentiel est d'introduire quelques notions de base: d\'eriv\'ee
selon un vecteur, d\'eriv\'ees partielles, d\'eveloppement limit\'e \`a l'ordre 1,
gradient et de les appliquer aux extremums locaux et aux coordonn\'ees
polaires; en revanche, les notions de fonction diff\'erentiable et de
diff\'erentielle en un point sont hors programme.

En vue de l'enseignement des autres disciplines scientifiques, il convient
d'\'etendre bri\`evement ces notions au cas o\`u $f$ est d\'efinie sur un ouvert de
$\R^3$. \S\ Il convient {\'e}galement de donner quelques
notions sur les courbes d\'efinies par une \'equation implicite
$F(x,y)=\lambda$ (tangente et normale en un point r\'egulier). En
math\'ematiques, aucune connaissance sur ce point n'est exigible des
\'etudiants.}
\so

\tit{a) D\'eriv\'ees partielles premi\`eres}

\tx{D\'efinition de la d\'eriv\'ee de $f$ en un point $a$ de $U$ selon un vecteur
$h$, not\'ee D${}_hf(a)$. D\'efinition des d\'eriv\'ees partielles, not\'ees
D${}_jf(a)$ ou $\dis{\partial f\over\partial x_j}(a)$.
}
{Il existe un nombre r\'eel $\delta>0$ tel que, pour tout \'el\'ement
$t\in[-\delta,\delta]$, $a+th$ appartienne \`a $U$; on pose alors
$\varphi_h(t)=f(a+th)$. Si $\varphi_h$ est d\'erivable \`a l'origine, on
dit que $f$ admet une d\'eriv\'ee au point $a$ de $U$ selon le vecteur $h$, et
l'on pose D${}_hf(a)=\varphi_h'(0)$.}
\so

\tx{D\'efinition des fonctions de classe ${\cal C}^1$ sur $U$ (les d{\'e}riv{\'e}es
partielles sont continues).
}
{}
\so

\tx{Th\'eor\`eme fondamental: si les d\'eriv\'ees partielles sont continues sur
$U$, alors $f$ admet, en tout point $a$ de $U$, un d\'eveloppement limit\'e \`a
l'ordre un, ainsi qu'une d\'eriv\'ee selon tout vecteur $h$, et $$\hbox{\rm
D}{}_hf(a)=h_1\hbox{\rm D}{}_1f(a)+h_2\hbox{\rm D}{}_2f(a).$$ En
particulier, $f$ est de classe ${\cal C}^1$ sur $U$ et l'application
$h\mapsto$D${}_hf(a)$ est une forme lin\'eaire. Le gradient de $f$ est
d\'efini, dans le plan euclidien $\R^2$, par la relation}
{La d\'emonstration de ce r\'esultat est hors programme.
\so

En vue de l'enseignement des autres disciplines scientifiques, il convient
de donner la notation diff\'erentielle d$f$, mais aucune connaissance sur ce
point n'est exigible en math\'ema\-ti\-ques.}

\tx{$$\hbox{\rm
D}{}_hf(a)=(\hbox{\rm grad}f(a)|h).$$}{
\vglue2mm
Interpr{\'e}tation g{\'e}om{\'e}trique du gradient.}
\so

\tx{Espace vectoriel ${\cal C}^1(U)$ des fonctions de classe ${\cal C}^1$ sur $U$.
}
{}
\so

\tx{D\'eriv\'ee d'une fonction compos\'ee de la forme $f\circ\varphi$, o\`u
$\varphi$ est de classe ${\cal C}^1$ sur un intervalle $I$ et \`a valeurs
dans $U$.

\so Application au calcul des d\'eriv\'ees partielles d'une fonction compos\'ee de
la forme $f\circ\varphi$, o\`u $\varphi$ est une application de classe ${\cal
C}^1$ sur un ouvert $V$ de $\R^2$ et \`a valeurs dans $U$.}{}
\so

\tx{En un point de $U$ o\`u une fonction $f$ de classe ${\cal C}^1$ sur $U$
pr\'esente un extremum local, ses d\'eriv\'ees partielles sont nulles.
}
{}
\so

\tit{b) D\'eriv\'ees partielles d'ordre $2$}

\tx{Th\'eor\`eme de Schwarz pour une fonction de classe ${\cal C}^2$ sur $U$.

\so

Espace vectoriel ${\cal C}^2(U)$ des fonctions de classe ${\cal C}^2$ sur $U$.
}
{La d\'emonstration est hors programme.}

\so \tx{\S\ Exemples simples d'{\'e}quations aux d{\'e}riv{\'e}es partielles,
{\'e}quation des cordes vibrantes.}{} \so

 \titre{3- Calcul int\'egral}

\tx{D\'efinition de l'int\'egrale double d'une fonction $f$ continue sur
un rectangle $R=[a,b]\times[c,d]$ et \`a valeurs r\'eelles.

Notation $\int\!\!\int_Rf=\int\!\!\int_Rf(x,y)\,\hbox{\rm d}\,x\,\hbox{\rm
d}\,y.$

Lin\'earit\'e, croissance, invariance par translation. Additivit\'e par
rapport au domaine d'int\'egration.
 

Th\'eor\`eme de Fubini: expression de l'int\'egrale double \`a l'aide de
deux int\'egrations successives. } {Br\`eve extension au cas d'une
fonction continue sur une partie $A$ born\'ee de $\R^2$ d\'efinie par
conditions simples ; extension du th\'eor\`eme de Fubini lorsque $A$ est
constitu\'ee des points $(x,y)\in\R^2$ tels que $a\ie x\ie b$ et
$\varphi(x)\ie y\ie\psi(x)$ o\`u $\varphi$ et $\psi$ sont des fonctions
continues sur $[a,b]$. Aucune d\'emonstration sur le th{\'e}or{\`e}me de Fubini n'est exigible des {\'e}tudiants.} \so

\tx{Changement de variables affine, int\'egration sur un
parall\'elogramme. Passage en coordonn\'ees polaires. Int\'egration sur un
disque, une couronne ou un secteur angulaire. } {La d\'emonstration de ces
r\'esultats, ainsi que tout \'enonc\'e g\'en\'eral concernant les
changements de variables, sont hors programme.} \so


\Titre{IV. G\'EOM\'ETRIE DIFF\'ERENTIELLE} \so {\sl Les fonctions
consid\'er\'ees dans ce chapitre sont de classe ${\cal C}^k$ sur un
intervalle $I$ de \R\ (o\`u $1\ie k\ie+\infty$) et sont \`a valeurs dans
le plan euclidien $\R^2$. En outre, pour la pr\'esentation des notions du
cours, on suppose que les arcs param\'etr\'es $\Gamma$ ainsi d\'efinis
sont r\'eguliers \`a l'ordre 1, c'est-\`a-dire que tous leurs points sont
r\'eguliers.} \so



\titre{1- \'Etude m{\'e}trique des courbes planes}

{\sl L'objectif est d'{\'e}tudier quelques propri\'et\'es m\'etriques
fondamentales des courbes planes
(abscisse curviligne, rep\`ere de Frenet, courbure).
\so
}
\so

\tx{Pour un arc orient\'e $\Gamma$ r\'egulier \`a l'ordre 1, rep\`ere de Frenet
$(\vect{T},\vect{N})$, abscisse curviligne. L'abscisse curviligne est un
param\'etrage admissible (la notion de param{\'e}trage admissible sera introduite {\`a} cette occasion); repr\'esentation normale d'un arc. Longueur d'un arc.
}
{Par d\'efinition, une abscisse curviligne est une fonction $s$ de classe
${\cal C}^1$ sur $I$ telle que $$s'(t)=\|f'(t)\|.$$ La longueur d'un
arc est d\'efinie \`a l'aide de l'abscisse curviligne; toute d\'efinition
g\'eom\'etrique d'une telle longueur est hors programme.}
\so

\tx{Si $f$ est de classe ${\cal C}^k$ sur $I$, o\`u $2\ie k<+\infty$,
existence d'une fonction $\alpha$ de classe ${\cal C}^{k-1}$ sur $I$ telle
que, pour tout $t\in I$,
$\vect{T}(t)=\cos\alpha(t)\,\vect{e_1}+\sin\alpha(t)\,\vect{e_2}$.
}
{La d\'emonstration de ce r\'esultat est hors programme.

Relations $$\dis{{\hbox{\rm d}f\over\hbox{\rm d}s}=\vect{T},\quad
{\hbox{\rm d}x\over\hbox{\rm d}s}=\cos\alpha,\quad{\hbox{\rm
d}y\over\hbox{\rm d}s}=\sin\alpha.}$$}
\so

\tx{D\'efinition de la courbure $\dis{\gamma={\hbox{\rm
d}\alpha\over\hbox{\rm d}s}}$; caract\'erisation des points bir\'eguliers.
\so

Relations $$\dis{{\hbox{\rm d}\vect{T}\over\hbox{\rm
d}s}=\gamma\vect{N},\quad{\hbox{\rm d}\vect{N}\over\hbox{\rm
d}s}=-\gamma\vect{T}.}$$
}
{Aucune connaissance sp\'ecifique sur le centre de courbure, le cercle
osculateur, les d\'evelopp\'ees et les d\'eveloppantes n'est exigible des
\'etudiants.}
\so

\tx{Dans le cas d'un arc $\Gamma$ bir\'egulier, $\alpha$ est un param\'etrage
admissible de l'arc de classe ${\cal C}^{k-1}$ sous-jacent. Rayon de courbure.
}
{Relations $\dis{\quad{\hbox{\rm d}\vect{T}\over\hbox{\rm
d}\alpha}=\vect{N},\quad{\hbox{\rm d}\vect{N}\over\hbox{\rm
d}\alpha}=-\vect{T}.}$}
\so

\tx{Calcul des coordonn\'ees de la vitesse et de l'acc\'el\'eration dans le
rep\`ere de Frenet.
}
{}
\so

\titre{2- Champs de vecteurs du plan et de l'espace}

 {\sl En vue de l'enseignement des autres disciplines scientifiques, il
convient de donner quelques notions sur les champs de vecteurs du plan et
de l'espace.}\so


\tx{Potentiel scalaire, caract\'erisation des champs admettant un
potentiel scalaire.\so

%-~D\'eriv\'ees partielles, divergence, rotationnel.

Circulation, int\'egrale curviligne. Formule de Green-Riemann dans le
plan.} {\vskip 8mm Aucune d\'emonstration  n'est exigible des \'etudiants
sur ces diff\'erents points.} \so

\vfill\eject

\TITRE{ALG\`EBRE ET G\'EOM\'ETRIE}

{\sl Le programme d'alg\`ebre et g\'eom\'etrie est organis\'e autour des
concepts fondamentaux d'espace vectoriel et d'application lin\'eaire, et
de leurs interventions en alg\`ebre, en analyse et en g\'eom\'etrie. La
ma\^{\i}trise de l'alg\`ebre lin\'eaire \'el\'ementaire en dimension finie
constitue un objectif essentiel. \so

Le cadre d'\'etude est bien d\'elimit\'e: br\`eve mise en place des
concepts d'espace vectoriel, d'application lin\'eaire, de sous-espaces
vectoriels suppl\'ementaires, de produit scalaire, sous
leur forme g\'en\'erale, en vue notamment des interventions en analyse; en
dimension finie, \'etude des concepts de base, de dimension et de rang,
mise en place du calcul matriciel, \'etude des espaces vectoriels
euclidiens; interventions de l'alg\`ebre lin\'eaire en g\'eom\'etrie
affine et en g\'eom\'etrie euclidienne. La ma\^{\i}trise de l'articulation entre le point de vue g\'eom\'etrique
(vecteurs et points) et le point de vue matriciel constitue un objectif
majeur.  \so

Pour les groupes, les anneaux et les corps, le programme se limite \`a
quelques d\'efinitions de base et aux exemples usuels; toute \'etude
g\'en\'erale de ces structures est hors programme. \so

Le point de vue algorithmique est
\`a prendre en compte pour l'ensemble de ce programme.} \so

\Titre{I. NOMBRES ET STRUCTURES ALG\'EBRIQUES USUELLES}

\titre{1- Vocabulaire relatif aux ensembles et aux applications}

\so

{\sl Le programme se limite strictement aux notions de base figurant
ci-dessous. Ces notions doivent \^etre acquises progressivement par les
\'etudiants au cours de l'ann\'ee, au fur et \`a mesure des exemples
rencontr\'es dans les diff\'erents chapitres d'alg\`ebre, d'analyse et de
g\'eom\'etrie. Elles ne doivent en aucun cas faire l'objet d'une \'etude
exhaustive bloqu\'ee en d\'ebut d'ann\'ee.} \so



Ensembles, appartenance, inclusion. Ensemble ${\cal P}(E)$ des parties de
$E$. Op\'erations sur les parties: intersection, r\'eunion,
compl\'ementaire. Produit de deux ensembles.\so


Application de $E$ dans (vers) $F$ ; graphe d'une application.\so

Ensemble ${\cal F}(E,F)$ des applications de $E$ dans $F$. Ensemble $E^I$
des familles $(x_i)_{i\in I}$ d'\'el\'ements d'un ensemble $E$ index\'ees
par un ensemble $I$.\so

Compos\'ee de deux applications, application identique. Restriction et
prolongements d'une application.\so

\'Equations, applications injectives, surjectives, bijectives. Application
r\'eciproque d'une bijection. Compos\'ee de deux injections, de deux
surjections, de deux bijections.\so



Images directe et r\'eciproque d'une partie.\so

D\'efinition d'une loi de composition interne. Associativit\'e,
commutativit\'e, \'el\'ement neutre. D\'efinition des \'el\'ements
inversibles pour une loi associative admettant un \'el\'ement neutre.\so


Relation d'ordre, ordre total, ordre partiel. Majorants, minorants, plus
grand et plus petit \'el\'ement.\so


\titre{2- Nombres entiers naturels, ensembles finis, d\'enombrements}


{\sl En ce qui concerne les nombres entiers naturels et les ensembles finis,
l'objectif principal est d'acqu\'erir la ma\^{\i}trise du raisonnement par
r\'ecurrence. Les propri\'et\'es de l'addition, de la multiplication et de
la relation d'ordre dans \N\ sont suppos\'ees connues; toute construction
et toute axiomatique de \N\ sont hors programme. \so

L'\'equipotence des ensembles infinis et la notion d'ensemble
d\'enombrable sont hors programme. \so

En ce qui concerne la combinatoire, l'objectif est de consolider les
acquis de la classe de Terminale S; le programme se limite strictement aux
exemples fondamentaux indiqu\'es ci-dessous. \so

La d\'emonstration des r\'esultats de ce chapitre n'est pas exigible des
\'etudiants. }\so

\tit{a) Nombres entiers naturels}

\tx{Propri\'et\'es fondamentales de l'ensemble \N\ des nombres entiers
naturels. Toute partie non vide a un plus petit \'el\'ement; principe de
r\'ecurrence. Toute partie major\'ee non vide a un plus grand \'el\'ement.
} {Les \'etudiants doivent ma\^{\i}triser le raisonnement par r\'ecurrence
simple ou avec pr\'ed\'ecesseurs.} \so

\tx{Suites d'\'el\'ements d'un ensemble $E$ (index\'ees par une partie de
\N). Suite d\'efinie par une relation de r\'ecurrence et une condition
initiale. } {} \so

\tx{Exemples d'utilisation des notations $a_1+a_2+\ldots+a_p+\ldots+a_n$,
$a_1a_2\ldots a_p\ldots a_n$, $\dis\sum_{1\leq p\leq n}a_p$,
$\dis\prod_{1\leq p\leq n}a_p$.

Suites arithm\'etiques, suites g\'eom\'etriques. Notations $na$ et $a^n$.
} {Symbole $n!$ (on convient que $0!=1$).} \so

\tit{b) Ensembles finis}

\tx{D\'efinition: il existe $n$ et une bijection de $\[1,n\]$ sur $E$; cardinal
(ou nombres d'\'el\'ements) d'un ensemble fini, notation $\card\,E$. On
convient que l'ensemble vide est fini et que $\card\vide=0$. } {On admet
que s'il existe une bijection de $\[1,p\]$ sur $\[1,n\]$, alors $p=n$.

\'Etant donn\'es deux ensembles finis $E$ et $F$ de m\^eme cardinal, et
une application $f$ de $E$ dans $F$, $f$ est bijective si et seulement si
$f$ est surjective ou injective.} \so

\tx{Toute partie $E'$ d'un ensemble fini $E$ est finie et
$$\card\,E'\leq\card\,E,$$ avec \'egalit\'e si et seulement si $E'=E$.}
{Une partie non vide $P$ de \N\ est finie si et seulement si elle est
major\'ee. Si $P$ est finie non vide, il existe une bijection strictement


croissante et une seule de l'intervalle $\[1,n\]$ sur $P$, o\`u
$n=\card\,P$.} \so

\tit{c) \S\ Op\'erations sur les ensembles finis, d\'enombrements}

\tx{Si $E$ et $F$ sont des ensembles finis, $E\cup F$ l'est aussi;
cardinal d'une r\'eunion finie de parties finies disjointes.

Si $E$ et $F$ sont des ensembles finis, $E\times F$ l'est aussi et
$$\card\,(E\times F)=\card\,E\cdot\card\,F.$$ } {Les \'etudiants doivent
conna\^{\i}tre la relation $\card(A\cup B)=\card\,A+\card\,B-\card(A\cap
B).$} \so

\tx{Cardinal de l'ensemble ${\cal F}(E,F)$ des applications de $E$ dans
$F$; cardinal de l'ensemble ${\cal P}(E)$ des parties de $E$. Cardinal de
l'ensemble des bijections (permutations) de $E$. } {} \so

\tx{Cardinal $\cnp{n}{p}$ de l'ensemble des parties ayant $p$ \'el\'ements
d'un ensemble $E$ \`a $n$ \'el\'ements. Combinaisons.



Relations
$$\dis\cnp{n}{p}=\cnp{n}{n-p},\qquad\sum_{p=0}^n\cnp{n}{p}=2^n,$$
$$\cnp{n}{p}=\cnp{n-1}{p}+\cnp{n-1}{p-1}\,\hbox{\rm (triangle de
Pascal).}$$\so Interpr\'etation ensembliste de ces relations} {} \so

\tx{Ensemble \Z\ des nombres entiers, ensemble \Q\ des nombres
rationnels. Relation d'ordre, valeur absolue. } {La construction de \Z\ et
de \Q\ est hors programme.} \so


\titre{3- Structures alg\'ebriques usuelles}

\so

{\sl Le programme se limite strictement aux notions de base indiqu\'ees
ci-dessous. Ces notions doivent \^etre acquises progressivement par les
\'etudiants au cours de l'ann\'ee, au fur et \`a mesure des exemples
rencontr\'es dans les diff\'erents chapitres d'alg\`ebre, d'analyse et de
g\'eom\'etrie. Elles ne doivent pas faire l'objet d'une \'etude exhaustive
bloqu\'ee en d\'ebut d'ann\'ee.
} \so

\tit{a) Vocabulaire relatif aux groupes et aux anneaux}

\tx{D\'efinition d'un groupe, d'un sous-groupe, d'un morphisme de groupes,
d'un isomorphisme. Noyau et image d'un morphisme de groupes.

Groupe additif \Z\ des nombres entiers. } {Ces notions doivent \^etre
illustr\'ees par des exemples issus :

des ensembles de nombres, notamment \Z, \R\ et \C;

des applications exponentielle et logarithme ;

de l'alg\`ebre lin\'eaire et de la g\'eom\'etrie.}
\so



\tx{D\'efinition d'un anneau (ayant un \'el\'ement unit\'e), d'un
sous-anneau. Distributivit\'e du produit par rapport au symbole sommatoire
$\sum$.

D\'efinition d'un corps (commutatif et non r\'eduit \`a $\{0\}$), d'un
sous-corps. } {Ces notions doivent \^etre illustr\'ees par des exemples
issus:

-~des ensembles de nombres \Z, \Q, \R, \C;

-~des polyn\^omes et fractions rationnelles.} \so

\tx{Anneau \Z\ des nombres entiers, corps \Q\ des nombres rationnels. } {}
\so

\tit{b) \S\ Arithm\'etique dans \Z. Calculs dans \R\ ou \C.}

 \tx{Multiples et
diviseurs d'un entier. Division euclidienne dans \Z, algorithme de la
division euclidienne.\so

Diviseurs communs \`a deux nombres entiers; nombres premiers entre eux.
PGCD de deux entiers; algorithme d'Euclide. PPCM de deux entiers; forme
irr\'eductible d'un nombre rationnel.\so

Th\'eor\`eme de B\'ezout. Th\'eor\`eme de Gauss.\so
}
{La d\'efinition des id\'eaux de \Z\ est hors programme.

Les \'etudiants doivent conna\^{\i}tre l'algorithme donnant les coefficients de
l'\'egalit\'e de B\'ezout, ainsi que l'algorithme d'exponentiation rapide.}
\so



\tx{D\'efinition des nombres premiers. Existence et unicit\'e de la
d\'ecomposition d'un entier strictement positif en produit de facteurs
premiers. } {La d\'emonstration de l'existence et de l'unicit\'e de la
d\'ecomposition en facteurs premiers n'est pas exigible des {\'e}tudiants.} \so

\tx{ Formule du bin\^ome. Relation
$$x^n-y^n=(x-y)\sum_{k=0}^{n-1}x^{n-k-1}y^k.$$

Somme des $n$ premiers termes d'une suite g\'eom\'etrique. } {On
g\'en\'eralisera en cours d'ann\'ee au cas d'\'el\'e\-ments qui commutent dans les
anneaux de matrices ou d'endomorphismes.} \so



\Titre{II. ALG\`EBRE LIN\'EAIRE ET POLYN\^OMES}


{\sl L'objectif est double:

-~Acqu\'erir les notions de base sur les espaces vectoriels de dimension
finie (ind\'ependance lin\'eaire, bases, dimension, sous-espaces
vectoriels suppl\'ementaires et projecteurs, rang) et le calcul matriciel.

-~Ma\^{\i}triser les relations entre le point de vue g\'eom\'etrique
(vecteurs et applications lin\'eaires) et
le point de vue matriciel.

Il convient d'\'etudier conjointement l'alg\`ebre lin\'eaire et la
g\'eom\'etrie affine du plan et de l'espace et, dans les deux cas,
d'illustrer les notions et les r\'esultats par de nombreuses figures. \so

En alg\`ebre lin\'eaire, le programme se limite au cas o\`u le corps de
base est \K, o\`u \K\ d\'esigne \R\ ou \C.}
\so

\titre{1- Espaces vectoriels}

\tit{a) Espaces vectoriels}

\tx{D\'efinition d'un espace vectoriel sur \K, d'un
sous-espace vectoriel.}{Exemples~: espace $\K^n$, espaces
vectoriels de
suites ou de fonctions.}
\so
\tx{Intersection de sous-espaces
vectoriels. Sous-espace engendr\'e par une partie.}{}
\so
\tx{Somme de deux sous-espaces vectoriels.
Sous-espaces {suppl\'e}\-men\-taires.}
{La notion g\'en\'erale de somme directe est hors programme.}
\so
\tx{Espace vectoriel produit $E\times F$.

\so
Espace vectoriel ${\cal F}(X,F)$ des
applications d'un ensemble $X$ dans un espace vectoriel $F$.
}{}
\so

\tit{b) Translations, sous-espaces affines}

\tx{Translations d'un espace vectoriel $E$.

D\'efinition d'un sous-espace affine~: partie de $E$
de la forme $a+F$, o\`u $F$ est un sous-espace vectoriel de $E$. Direction
d'un sous-espace affine.

Sous-espaces affines parall\`eles~: $W$ est parall\`ele \`a
$W'$ si la direction de $W$ est incluse dans celle de $W'$.

Intersection de deux sous-espaces affines, direction de
cette intersection lorsqu'elle n'est pas vide.

Barycentres. Parties convexes (lorsque \K=\R). } {Il convient d'illustrer
ces notions par la g\'eom\'etrie du plan et de l'espace, d\'ej\`a
abord\'ee dans les classes ant\'erieures et en d\'ebut d'ann\'ee. Il
convient de souligner que le choix d'une origine du plan ou de l'espace
permet d'identifier points et vecteurs. On \'evitera cependant de faire
syst\'ematiquement cette identification.}

\so

\tit{c) Applications lin\'eaires}
\tx{D\'efinition d'une application lin\'eaire, d'une forme lin\'eaire,
d'un endomorphisme.

Espace
vectoriel ${\cal L}(E,F)$ des applications lin\'eaires de $E$ dans $F$.}
{Homoth\'eties.
Projecteurs associ\'es \`a deux sous-espaces suppl\'ementaires.
Sym\'etries, affinit\'es.
Exemples d'applications lin\'eaires en analyse, en g\'eom\'etrie.}

\so \tx{Compos\'ee de deux applications lin\'eaires, r\'eciproque d'une
application lin\'eaire bijective. D\'efinition d'un isomorphisme, d'un
automorphisme. Lin\'earit\'e des applications $v\mapsto v\circ u$ et
$u\mapsto v\circ u$. D\'efinition du groupe lin\'eaire $\gle$} {L'\'etude
g\'en\'erale du groupe lin\'eaire est hors programme.}


\so
\tx{\'Equations lin\'eaires; noyau et image d'une application lin\'eaire.
Description de l'ensemble des solutions de \hfill

$u(x)=b$.
}
{Structure de l'ensemble des solutions d'une \'equation diff\'erentielle lin\'eaire. Structure de l'ensemble des suites $(u_{n})$ d{\'e}finies par une relation de r{\'e}currence de la forme $u_{n+2}=au_{n+1}+bu_{n}$.}
\so
\tx{Caract\'erisation des projecteurs par la relation $p^2=p$.
 Caract\'erisation des sym\'etries par la relation
$s^2=I_E$.}{}

\so



\titre{2- Dimension des espaces vectoriels}

\tit{a) Familles de vecteurs}
\tx{D\'efinition des combinaisons lin\'eaires de $p$ vecteurs \break
$x_1,x_2,
\ldots,x_p$ d'un espace vectoriel; image par une ap\-pli\-ca\-tion lin\'eaire d'une
combinaison lin\'eaire.
}{Le cas
des familles index\'ees par un ensemble infini est hors programme.}

\so
\tx{Sous-espace engendr\'e par une famille finie de vecteurs.
D\'efinition d'une famille g\'en\'eratrice.

Ind\'ependance lin\'eaire~: d\'efinition d'une famille libre,
li\'ee.

D\'efinition d'une base; coordonn\'ees (ou composantes) d'un vecteur dans une
base. Base canonique de $\K^n$.

\'Etant donn\'es un espace vectoriel $E$ muni d'une base $(e_1,\ldots,e_p)$
et une famille $(f_1,\ldots,f_p)$ de vecteurs d'un espace vectoriel $F$, il
existe une application lin\'eaire $u$ et une seule de $E$ dans $F$ telle que
$u(e_j)=f_j$.
}
{La donn\'ee d'une famille de $p$ vecteurs \break
$(x_1,x_2, \ldots,x_p)$ d'un
\K-espace vectoriel $E$ {d\'e}\-termine une application lin\'eaire de $\K^p$ dans
$E$; noyau et image de cette application; caract\'erisation des bases de $E$,
des familles g\'en\'eratrices, des familles libres.}
\so

\tit{b) Dimension d'un espace vectoriel}

\tx{D\'efinition d'un espace vectoriel de dimension finie (espace vectoriel
admettant une famille g\'en\'eratrice finie). Th\'eor\`eme de la base incompl\`ete,
existence de bases.
}
{}
\so

\tx{Toutes les bases d'un espace vectoriel $E$ de dimension finie sont
finies et ont le m\^eme nombre d'\'el\'ements, appel\'e dimension de $E$. On
convient que l'espace vectoriel r\'eduit \`a $\{0\}$ est de dimension nulle.

Tout espace vectoriel de dimension $n$ est isomorphe \`a $\K^n$; deux espaces
vectoriels de dimension finie $E$ et $F$ sont isomorphes si et seulement si
$\dim E=\dim F$.
}
{\'Etant donn\'ee une famille $S$ de vecteurs d'un espace vectoriel de
dimension $n$:

-~si $S$ est libre, alors $p\ie n$, avec \'egalit\'e si et seulement si $S$ est
une base;

-~si $S$ est g\'en\'eratrice, alors $p\se n$, avec \'egalit\'e si et seulement si
$S$ est une base.}
\so

\tx{Base de $E{\times}F$ associ\'ee \`a des bases de $E$ et de $F$; dimension
de $E\times F$.
}
{}
\so

\tx{\'Etant donn\'es un espace vectoriel $E$ muni d'une base \break
$B=(e_j)$ et un
espace vectoriel $F$ muni d'une base $C=(f_i)$, une application lin\'eaire
$u$ de $E$ dans $F$ et un vecteur $x$ de $E$, expression des coordonn\'ees de
$y=u(x)$ dans $C$ en fonction des coordonn\'ees de $x$ dans $B$.
}
{\'Etant donn\'ee une forme lin\'eaire $\varphi$ sur $E$, expression de
$\varphi(x)$ en fonction des coordonn\'ees de $x$ dans $B$.}
\so

\tit{c) Dimension d'un sous-espace vectoriel}

\tx{Tout \sev\ $E'$ d'un espace vectoriel de dimension finie $E$ est de
dimension finie et $\dim E'\leq \dim E$, avec \'egalit\'e si et seulement si
$E'=E$. Rang d'une famille de vecteurs.
}
{}
\so

\tx{Existence de sous-espaces vectoriels suppl\'ementaires d'un sous-espace
vectoriel donn\'e; dimension d'un suppl\'ementaire.
}
{Les \'etudiants doivent conna\^{\i}tre la relation $\dim(E+F)=\dim E+\dim F
-\dim(E\cap F)$.}
\so

\tit{d) Rang d'une application lin\'eaire}

\tx{\'Etant donn\'ee une application lin\'eaire $u$ de $E$ dans $F$, $u$
d\'efinit un isomorphisme de tout suppl\'ementaire de $\Ker u$ sur $\im u$;
en particulier, $$\dim E=\dim\Ker u+\dim\im u.$$
}
{Cas d'une forme lin\'eaire: caract\'erisation et \'equations d'un hyperplan.}
\so

\tx{Rang d'une application lin\'eaire, caract\'erisation des
iso\-mor\-phis\-mes. } {Invariance du rang par composition avec un
isomorphisme.} \tx{Caract\'erisation des \'el\'ements inversibles de
${\cal L}(E)$.}{} \so 


\titre{3- Polyn\^omes}



{\sl L'objectif est d'\'etudier, par des m\'ethodes \'el\'ementaires, les
propri\'et\'es de base des polyn\^omes et des fractions rationnelles, et
d'exploiter ces objets formels pour la r\'esolution de probl\`emes portant
sur les \'equations alg\'ebriques et les fonctions num\'eriques. \so

Le programme se limite au cas o\`u le corps de base est \K, o\`u \K\
d\'esigne \R\ ou \C.} \so

\tit{a) Polyn\^omes \`a une ind\'etermin\'ee
% et corps $\K(X)$
}

\tx{Espace vectoriel $\K[X]$ des polyn\^omes \`a une ind\'etermin\'ee \`a
coef\-fi\-cients dans \K; op\'erations.}
{Aucune connaissance sur la construction de $\K[X]$
n'est exigible des \'etudiants.}

\so \tx{ Degr\'e d'un polyn\^ome (on convient que le degr\'e de 0 est
$-\infty$), coefficient dominant, polyn\^ome unitaire (ou normalis\'e).
Degr\'e d'un produit, d'une somme; les polyn\^omes de degr\'e inf\'erieur
ou \'egal \`a $p$ constituent un sous-espace vectoriel de $\K[X]$.

L'anneau $\K[X]$ est int\`egre. Corps $\K(X)$ des fractions rationnelles, degr\'e
d'une fraction rationnelle.
} {Notation $a_0+a_1\,X+\cdots+a_p\,X^p$ ou, le cas \'ech\'eant,
$\dis\sum_{n=0}^{+\infty}a_n X^n$.} \so

 \so

\tx{Multiples et diviseurs d'un polyn\^ome, polyn\^omes associ\'es.
Division euclidienne dans $\K[X]$, algorithme de la division euclidienne.
} {} \so

\tit{b) Fonctions polynomiales
et rationnelles
}

\tx{Fonction polynomiale associ\'ee \`a un polyn\^ome. \'Equations
alg\'ebriques. Z\'eros (ou racines) d'un polyn\^ome; ordre de
multiplicit\'e. Isomorphisme entre polyn\^omes et fonctions polynomiales.
} {Reste de la division euclidienne d'un polyn\^ome $P$ par $X-a$;
caract\'erisation des z\'eros de $P$. \so

\S\ Algorithme de Horner pour le calcul des valeurs d'une fonction
polynomiale.
} \so

\tx{Fonction rationnelle associ\'ee \`a une fraction rationnelle. Z\'eros
et p\^oles d'une fraction rationnelle; ordre de multiplicit\'e. } {} \so

\so

\tx{D\'efinition du polyn\^ome d\'eriv\'e. Lin\'earit\'e de la
d\'erivation, d\'eriv\'ee d'un produit. D\'eriv\'ees successives,
d\'eriv\'ee $n$-i\`eme d'un produit (formule de Leibniz). Formule de
Taylor, application \`a la recherche de l'ordre de multiplicit\'e d'un
z\'ero. } {Les \'etudiants doivent conna\^{\i}tre les relations
$$P(X)=\sum_{n=0}^{+\infty}\,{(X-a)^n\over n!}\,{P^{(n)}}(a),$$
$$P(a+X)=\sum_{n=0}^{+\infty}\,{X^n\over n!}\,{P^{(n)}}(a).$$} \so

\tit{c) Polyn\^omes scind\'es}

\tx{D\'efinition d'un polyn\^ome scind\'e sur \K; relations entre les
coefficients et les racines d'un polyn\^ome scind\'e. } {Aucune
connaissance sp\'ecifique sur le calcul des fonctions sym\'etriques des
racines d'un polyn\^ome n'est exigible des \'etudiants.} \so

\tx{Th\'eor\`eme de d'Alembert-Gauss. Description des polyn\^omes
irr\'eductibles de $\C[X]$ et de $\R[X]$. } {La d\'emonstration du
th\'eor\`eme de d'Alembert-Gauss est hors programme.} \so

\tx{D\'ecomposition d'un polyn\^ome en produit de facteurs {irr\'e}\-duc\-tibles
sur \C\ et sur \R. } {D\'ecomposition dans $\C[X]$ de $X^n-1$.} \so

\tit{d) Divisibilit\'e dans l'anneau $\K[X]$}

\tx{\S\ Diviseurs communs \`a deux polyn\^omes, polyn\^omes premiers entre eux. PGCD
de deux polyn\^omes; algorithme d'Euclide. PPCM de deux polyn\^omes; forme
irr\'eductible d'une fraction rationnelle.
}
{La d\'efinition des id\'eaux de $\K[X]$ est hors programme.}
\so

\tx{Th\'eor\`eme de B\'ezout. Th\'eor\`eme de Gauss.
}
{Les \'etudiants doivent conna\^{\i}tre l'algorithme donnant les coefficients de
l'\'egalit\'e de B\'ezout.}
\so

\tx{Polyn\^omes irr\'eductibles. Existence et unicit\'e de la
d\'ecompo\-sition d'un polyn\^ome en produit de facteurs irr\'eductibles.
} {Pour la pratique de la d\'ecomposition en produit de facteurs
irr\'eductibles, le programme se limite au cas o\`u $\K=\R$ ou \C. Aucune
connaissance sp\'ecifique sur l'irr\'eductibilit\'e sur \Q\ n'est exigible
des \'etudiants.} \so

\tit{e) \'Etude locale d'une fraction rationnelle}

\tx{\S\ Existence et unicit\'e de la partie enti\`ere d'une fraction
rationnelle $R$; existence et unicit\'e de la partie polaire de $R$
relative \`a un p\^ole $a$. Lorsque $a$ est un p\^ole simple de $R$,
expressions de la partie polaire relative \`a ce p\^ole. } {Les
\'etudiants doivent savoir calculer la partie polaire en un p\^ole double.
En revanche, des indications sur la m\'ethode \`a suivre doivent \^etre
fournies pour des p\^oles d'ordre sup\'erieur ou \'egal \`a 3. La division
des polyn\^omes suivant les puissances croissantes est hors programme.}
\so

\tx{Lorsque $\K=\C$, toute fraction rationnelle $R$ est \'egale \`a la
somme de sa partie enti\`ere et de ses parties polaires. Existence et
unicit\'e de la d\'ecomposition de $R$ en \'el\'ements simples.
D\'ecomposition en \'el\'ements simples de $\displaystyle{P'\over P}\cdot$
} {Aucune connaissance sp\'ecifique sur la d\'ecom\-po\-si\-tion en
\'el\'ements simples sur un corps autre que \C\ n'est exigible des
\'etudiants.} \so


\titre{4- \S\ Calcul matriciel}


\tit{a) Op\'erations sur les matrices}

\tx{Espace vectoriel $\mnpk$ des matrices \`a $n$ lignes et $p$ colonnes sur
\K. Base canonique $(E_{i,j})$ de $\mnpk$; dimension de $\mnpk$.
Isomorphisme canonique de ${\cal L}(\K^p,\K^n)$ sur $\mnpk$. D\'efinition du
produit matriciel, bilin\'earit\'e.
}
{Identification des matrices colonnes et des vecteurs de $\K^n$, des
matrices lignes et des formes lin\'eaires sur $\K^p$.

\'Ecriture matricielle $Y=M\,X$ de l'effet d'une application lin\'eaire sur
un vecteur.}
\so

\tx{Anneau $\mnk$ des matrices carr\'ees \`a $n$ lignes. Isomorphisme
canonique de l'anneau ${\cal L}(\K^n)$ sur l'anneau $\mnk$. Matrices
carr\'ees inversibles; d\'efinition du groupe lin\'eaire $\glnk$.
}
{Sous-anneau des matrices diagonales, des matrices triangulaires
sup\'erieures (ou inf\'erieures).}
\so

\tx{Transpos\'ee d'une matrice. Compatibilit\'e avec les op\'erations alg\'ebriques
sur les matrices.
}
{}
\so

\tx{Matrices carr\'ees sym\'etriques, antisym\'etriques.
}
{}
\so

\tit{b) Matrices et applications lin\'eaires}

\tx{Matrice $M_{B,C}(u)$ associ\'ee \`a une application lin\'eaire $u$ d'un
espace vectoriel $E$ muni d'une base $B$ dans un espace vectoriel $F$ muni
d'une base $C$. L'application $u\mapsto M_{B,C}(u)$ est un isomorphisme de
${\cal L}(E,F)$ sur $\mnpk$; dimension de ${\cal L}(E,F)$.

Matrice $M_B(u)$ associ\'ee \`a un endomorphisme $u$ d'un espace vectoriel
$E$ muni d'une base $B$. L'application $u\mapsto M_B(u)$ est un
isomorphisme (lin{\'e}aire) d'anneaux.

Matrice dans une base d'une famille finie de vecteurs, d'une famille finie
de formes lin\'eaires.
}
{La $j$-i\`eme colonne de $M_{B,C}(u)$ est constitu\'ee des coordonn\'ees dans la
base $C$ de l'image par $u$ du $j$-i\`eme vecteur de la base $B$.}
\so

\tx{Matrice de passage d'une base $B$ \`a une base $B'$ d'un espace vectoriel
$E$; effet d'un changement de base(s) sur les coordonn\'ees d'un vecteur, sur
l'expression d'une forme lin\'eaire, sur la matrice d'une application
lin\'eaire, sur la matrice d'un endomorphisme.
}
{La matrice de passage de la base $B$ \`a la base $B'$ est, par d\'efinition,
la matrice de la famille $B'$ dans la base $B$: sa $j$-i\`eme colonne est
constitu\'ee des coordonn\'ees dans la base $B$ du $j$-i\`eme vecteur de la base
$B'$. Cette matrice est aussi $M_{B',B}(I_E)$.}
\so

\tit{c) Op\'erations \'el\'ementaires sur les matrices}

\tx{
\vglue6pt
Op\'erations (ou manipulations) \'el\'ementaires sur les lignes (ou les
colonnes) d'une matrice.

\so
Interpr\'etation des op\'erations \'el\'ementaires en
termes de produits matriciels.
}
{Les op\'erations \'el\'ementaires sur les lignes sont les suivantes:

-~addition d'un multiple d'une ligne \`a une autre (codage: $L_i\leftarrow
L_i+\alpha L_j$);

-~multiplication d'une ligne par un scalaire non nul (codage:
$L_i\leftarrow \alpha L_i$);

-~\'echange de deux lignes (codage: $L_i\leftrightarrow L_j$).}
\so

\tx{Application \`a l'inversion d'une matrice carr\'ee par l'algorithme du pivot
de Gauss.
}
{Cet algorithme permet en outre d'\'etudier l'inversibilit\'e de la matrice.}
\so

\tit{d) Rang d'une matrice}

\tx{D\'efinition du rang d'une matrice (rang de l'application lin\'eaire
canoniquement associ\'ee, ou encore rang des vecteurs colonnes).
}
{Pour toute application lin\'eaire $u$ de $E$ dans $F$, le rang de $u$ est
\'egal au rang de $M_{B,C}(u)$, o\`u $B$ est une base de $E$ et $C$ une base de
$F$.
}
\so

\tx{Une matrice de $\mnpk$ est de rang $r$ si et seulement si elle est de
la forme $UJ_rV$ o\`u $U$ et $V$ sont des matrices carr\'ees inversibles.
Invariance du rang par transposition.

\so
Emploi des op\'erations \'el\'ementaires pour le calcul du rang d'une matrice.
}
{\vglue6pt
La matrice $J_r$ est l'\'el\'ement $(\alpha_{i,j})$ de $\mnpk$ d\'efini par les
relations: $$\alpha_{i,j}=\cases{1 & si $i=j\leq r$\cr 0 & dans tous les
autres cas.\cr}$$}
\so

\tit{e) Syst\`emes d'\'equations lin\'eaires}

\tx{
D\'efinition, syst\`eme homog\`ene associ\'e; interpr\'etations. Des\-crip\-tion de
l'ensemble des solutions.

\so
Rang d'un syst\`eme lin\'eaire. Dimension de l'espace vectoriel des solutions
d'un syst\`eme lin\'eaire homog\`ene.
}
{Les \'etudiants doivent conna\^{\i}tre l'interpr\'etation d'un syst\`eme de $n$
\'equations lin\'eaires \`a $p$ inconnues, \`a l'aide des vecteurs de $\K^n$, des
formes lin\'eaires sur $\K^p$, et d'une application lin\'eaire de $\K^p$ dans
$\K^n$ (ainsi que la traduction matricielle correspondante).}
\so

\tx{Existence et unicit\'e de la solution lorsque $r=n=p$ (syst\`emes de
Cramer). R\'esolution des syst\`emes de Cramer triangulaires. Algorithme
du pivot de Gauss pour la r\'esolution des syst\`emes de Cramer.
}
{Le th\'eor\`eme de Rouch\'e-Fonten\'e et les matrices bordantes sont hors
programme.}
\so



\titre{5- D\'eterminants}
\tit{a) Groupe sym\'etrique} 

\tx{D\'efinition du groupe {\goth S}${}_n$ des permutations de $\[1,n\]$;
cycles, transpositions. D\'ecomposition d'une permutation en produit de
transpositions. Signature $\varepsilon(\sigma)$ d'une permutation $\sigma$,
signature d'une transposition.
}
{}
\so

\tx{L'application $\sigma\mapsto\varepsilon(\sigma)$ est un morphisme de
{\goth S}${}_n$ dans le groupe multiplicatif $\{-1,1\}$; d\'efinition du
sous-groupe altern\'e {\goth A}${}_n$.
}
{La d\'emonstration de ce r\'esultat n'est pas exigible des \'etudiants.}
\so

\so
\tit{b) Applications multilin\'eaires}

\tx{D\'efinition d'une application $n$-lin\'eaire, applications
$n$-lin\'e\-ai\-res sym\'etriques, antisym\'etriques, altern\'ees. } {Il
convient de donner de nombreux exemples d'applications bilin\'eaires issus
de l'alg\`ebre, de l'analyse et de la g\'eom\'etrie. En revanche,
l'\'etude g\'en\'erale des applications bilin\'eaires et multilin\'eaires
est hors programme.} \so

\tit{c) D\'eterminant de $n$ vecteurs}

\tx{Formes $n$-lin\'eaires altern\'ees sur un espace vectoriel de
dimension $n$. D\'eterminant de $n$ vecteurs dans une base d'un espace
vectoriel de dimension $n$. Caract\'erisation des bases.

Application \`a l'expression de la solution d'un syst\`eme de Cramer.
}
{La d\'emonstration de l'existence et de l'unicit\'e du d\'eterminant n'est pas
exigible des \'etudiants.}

\so

\tit{d) D\'eterminant d'un endomorphisme}

\tx{D\'eterminant d'un endomorphisme, du compos\'e de deux endomorphismes;
caract\'erisation des automorphismes. } {Application \`a l'orientation
d'un espace vectoriel r\'eel de dimension finie; la donn\'ee d'une base
d\'etermine une orientation. Bases directes d'un espace vectoriel
orient\'e.} \so

\tit{e) D\'eterminant d'une matrice carr\'ee}

\tx{D\'eterminant d'une matrice carr\'ee. D\'eterminant du produit de deux
matrices, de la transpos\'ee d'une matrice. D\'eveloppement par rapport
\`a une ligne ou une colonne; cofacteurs. } {Relation $$M.{}^t\hbox{\rm
Com}\,M={}^t\hbox{\rm Com}\,M.M=(\det M)\,I_n,$$ o\`u $\hbox{\rm Com}\,M$
d\'esigne la matrice des cofacteurs de $M$.

Expression de l'inverse d'une matrice carr\'ee.}


\so

\Titre{III. ESPACES VECTORIELS EUCLIDIENS ET G\'EOM\'ETRIE EUCLIDIENNE}


{\sl L'objectif est double:

-~Refonder la th\'eorie des espaces vectoriels euclidiens de dimension 2
ou 3 (bases orthonormales, suppl\'ementaires orthogonaux) et  la
g\'eom\'etrie euclidienne du plan et de l'espace (distances, angles).

-~D\'evelopper les notions de base sur les automorphismes orthogonaux, les
isom\'etries et les similitudes.


Dans toute cette partie, le corps de base est \R.} \so

%\titre{1- Produit scalaire, espaces vectoriels euclidiens}

\tit{a) Produit scalaire}

\tx{Produit scalaire $(x,y)\mapsto (x|y)$  sur un \R-espace vectoriel.
In\'egalit\'e de Cauchy-Schwarz; norme euclidienne, distance associ\'ee,
in\'egalit\'e triangulaire.

Vecteurs unitaires. Vecteurs orthogonaux, sous-espaces vectoriels
orthogonaux, orthogonal d'un \sev. Familles orthogonales, familles
orthonormales; relation de Pythagore pour une famille orthogonale finie. }{}
 \so

\tx{Relations entre produit scalaire et norme. Identit\'e du
parall\`elogramme. Identit\'es de polarisation. } {Les \'etudiants doivent
conna\^{\i}tre l'interpr\'etation g\'eom\'etrique de ces relations.} \so

\tx{D\'efinition d'un espace vectoriel euclidien. } {Un espace vectoriel
euclidien est un espace vectoriel de dimension finie muni d'un produit
scalaire.} \so

\tit{b) Orthogonalit\'e}

\tx{\S\ Existence de bases orthonormales, compl\'etion d'une famille
orthonormale en une base orthonormale.}{}\so


\tx{L'orthogonal d'un sous-espace vectoriel $F$ est un
suppl\'e\-men\-taire de ce sous-espace vectoriel, appel\'e
suppl\'e\-men\-taire orthogonal de $F$, et not\'e $F^\perp$ ou $F^\circ$.
} {Distance {\`a} un sous-espace vectoriel. Proc\'ed\'e d'orthonormalisation de Gram-Schmidt.\so Toute forme
lin\'eaire $f$ s'\'ecrit de mani\`ere unique sous la forme $f(x)=(a|x)$,
o\`u $a$ est un vecteur.} \so

\tx{Projecteurs orthogonaux, sym\'etries orthogonales, r\'eflexions. }
{Expression de la projection orthogonale d'un vecteur sur un sous-espace
muni d'une base orthonormale.} \so

%\tx{Dans un plan (resp. un espace) euclidien orient\'e, la donn\'ee d'une
%orientation d'une droite $D$ induit une orientation de la droite (resp. du
%plan) $D^\perp$. } {} \so
%
%\tx{Matrices carr\'ees sym\'etriques, antisym\'etriques. } {Les matrices
%sym\'etriques et les matrices anti\-sym\'etri\-ques constituent des
%sous-espaces suppl\'ementaires.}{} \so

\tit{c) Isom\'etries affines du plan et de l'espace}

\tx{D\'efinition d'une isom\'etrie du plan (resp. de l'espace).
%: transformation
%affine conservant les distances. D\'efinition d'un d\'eplacement.
Applications affines. Exemples : translations, r\'eflexions.
%, rotations. Les isom\'etries, les d\'eplacements, constituent des
%sous-groupes du groupe affine du plan (resp. de l'espace).
} {Image d'un barycentre par une application affine. } \so

\tx{\'Etant donn\'es deux points distincts $A$ et $B$ du plan ou de
l'espace, il existe une r\'eflexion affine et une seule \'echangeant $A$
et $B$. }{}

 \so

\tx{Expression d'une isom\'etrie comme compos\'ee de r\'eflexions.
D\'eplacements. } {Toute isom\'etrie est affine.} \so



\tit{d) Automorphismes orthogonaux du plan vectoriel euclidien}

\tx{D\'efinition d'un automorphisme orthogonal d'un plan eu\-cli\-dien $E$
(c'est-\`a-dire un automorphisme de $E$ conservant le produit scalaire).
Caract\'erisation \`a l'aide de la conservation de la norme. }
{Caract\'erisation d'un automorphisme orthogonal par l'image d'une (de
toute) base orthonormale.} \so

\tx{D\'efinition du groupe orthogonal $O(E)$; sym\'etries ortho\-go\-nales,
r\'eflexions. } {L'\'etude g\'en\'erale du groupe orthogonal est hors
programme.} \so

\tx{Matrice d'un automorphisme orthogonal.D\'efinition des matrices orthogonales et du groupe $O(2)$.
Caract\'erisation des matrices orthogonales par leurs vecteurs colonnes.
\so
D\'efinition du groupe sp\'ecial orthogonal $SO(E)$ (rotations), du groupe
$SO(2)$. } {Caract\'erisation d'une rotation par l'image d'une (de toute)
base orthonormale directe.} \so

\tx{D{\'e}composition d'un automorphisme orthogonal en produit de  r\'eflexions. } {} \so


\tx{Matrice dans une base orthonormale directe d'une rotation, mesure de
l'angle d'une rotation; matrice de rotation $R(\theta)$ associ\'ee \`a un
nombre r\'eel $\theta$; morphisme $\theta\mapsto R(\theta)$ de \R\ sur
$SO(2)$. } {Si $u$ est la rotation d'angle de mesure $\theta$, alors pour
tout vecteur unitaire $a$,
$$\cos\theta=\bigl(a\bigm|u(a)\bigr),\quad\sin\theta=\det\big(a,u(a)\big).$$} \so

\tit{e) Automorphismes orthogonaux de l'espace}

\tx{Automorphismes orthogonaux d'un espace vectoriel euclidien de dimension 3,
groupe orthogonal $O(E)$, sym\'etries orthogonales, r\'eflexions. Matrices
orthogonales, groupe $O(3)$. D\'efinition du groupe des rotations $SO(E)$, du groupe $SO(3)$. } {Il
s'agit d'une br\`eve extension \`a l'espace des notions d\'ej\`a
\'etudi\'ees dans le cas du plan.

\'Etude de la d\'ecomposition d'une rotation en produit de deux
r\'eflexions. } \so



\tx{Axe et mesure de l'angle d'une rotation d'un espace euclidien
orient\'e de dimension 3. \'Etant donn\'ee une rotation $u$ d'axe dirig\'e
par un vecteur unitaire $a$ et d'angle de mesure $\theta$ (modulo $2\pi$),
l'image d'un vecteur $x$ orthogonal \`a l'axe est donn\'ee par
$$u(x)=(\cos\theta)\,x+(\sin\theta)\,a\wedge x.$$ } {Les \'etudiants
doivent savoir d\'eterminer l'axe et la mesure de l'angle d'une rotation,
ainsi que l'image d'un vecteur quelconque et la matrice associ\'ee \`a
cette rotation.

En revanche, l'\'etude g\'en\'erale de la r\'eduction des automorphismes
orthogonaux de l'espace est hors programme.} \so

\tit{f) D{\'e}placements}

\tx{D{\'e}finition d'un d{\'e}placement.

Tout d\'eplacement du plan est soit une translation, soit une
rotation.

Tout d\'eplacement de l'espace est soit une translation, soit une
rotation, soit un vissage. } {} \so

\tit{g) Similitudes directes du plan}

\tx{D\'efinition d'une similitude (transformation affine multipliant les
distances dans un rapport donn\'e); rapport de similitude. D\'efinition
d'une similitude directe. Homoth\'eties de rapport non nul, translations,
rotations.
\'Ecriture complexe d'une similitude directe. Centre et mesure de l'angle
d'une similitude directe distincte d'une translation. } {Les \'etudiants doivent conna\^{\i}tre l'effet d'une similitude directe
sur les angles orient\'es et les aires.} \so

\tx{\'Etant donn\'es deux segments $[AB]$ et $[A'B']$ de longueur non nulle,
il existe une similitude directe et une seule transformant $A$ en $A'$ et
$B$ en $B'$. } {Les \'etudiants doivent savoir d\'eterminer le rapport, la
mesure de l'angle et le centre de cette similitude directe lorsqu'elle
n'est pas une translation.} \so



\bye


