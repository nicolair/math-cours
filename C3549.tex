\input{courspdf.tex}
\debutcours{Applications - Familles}{alpha}


\section{Définitions - Vocabulaire}

\subsection{Graphe fonctionnel}
ensemble fonctionnel $\mathcal F(E,F)$.
\newline
Notation fonctionnelle usuelle.

\subsection{Restrictions et prolongement}
\index{restriction d'une application} Restriction

\index{prolongements d'une fonction}prolongements d'une fonction

\index{corestriction d'une fonction}corestriction d'une fonction

\subsection{Exemples}
\index{fonction caractéristique}fonction caractéristique

fonctions identités

\subsection{Composition}

\section{Images directes et réciproques}
\subsection{Définitions}
notations provisoires $\Phi$ et $\varphi$.
propriétés de $\varphi$ union, intersection, complémentaire.
inclusions pour les composées

Exercice. Cas d'égalité.

Notation définitive

\section{Injectivité, surjectivité, bijectivité}
Expression avec les images directes et réciproques
\begin{prop}
 \begin{align*}
  f \text{ injective et } g \text{ injective } &\Rightarrow g\circ f \text{ injective} \\
  f \text{ surjective et } g \text{ surjective } &\Rightarrow g\circ f \text{ surjective} \\
  f \text{ bijective et } g \text{ bijective } &\Rightarrow g\circ f \text{ bijective} \\
  g\circ f \text{ injective } &\Rightarrow f \text{ injective} \\
  g\circ f \text{ surjective } &\Rightarrow g \text{ surjective} 
 \end{align*}
\end{prop}
\begin{demo}
 
\end{demo}

\begin{prop}
 $f$ est bijective si et seulement si il existe une fonction $g$ telle que
\begin{align*}
 f\circ g =\Id _F & & g\circ f = \Id _E
\end{align*}
Lorsque la fonction est bijective cette fonction $g$ est unique, elle est notée $f^{-1}$ et appelée la bijection réciproque de $f$.
\end{prop}
\begin{demo}
 
\end{demo}
\begin{rem}
 Attention, $g\circ f$ bijective n'entraine pas que $f$ et $g$ soient bijective. Exemple du \og pousser \fg : $P$ et du \og tirer \fg :$T_0$.
\end{rem}

\begin{prop}
 Si $E$ et $F$ sont deux ensembles finis avec le même nombre d'éléments, l'injectivité est équivalente à la surjectivité est équivalente à la bijectivité.
\end{prop}

\section{Familles}

\end{document}
