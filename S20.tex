%!  pour pdfLatex
\documentclass[a4paper]{article}
\usepackage[hmargin={1.5cm,1.5cm},vmargin={2.4cm,2.4cm},headheight=13.1pt]{geometry}

\usepackage[pdftex]{graphicx,color}
%\usepackage{hyperref}

\usepackage[utf8]{inputenc}
\usepackage[T1]{fontenc}
\usepackage{lmodern}
%\usepackage[frenchb]{babel}
\usepackage[french]{babel}

\usepackage{fancyhdr}
\pagestyle{fancy}

%\usepackage{floatflt}

\usepackage{parcolumns}
\setlength{\parindent}{0pt}
\usepackage{xcolor}

%pr{\'e}sentation des compteurs de section, ...
\makeatletter
%\renewcommand{\labelenumii}{\theenumii.}
\renewcommand{\thepart}{}
\renewcommand{\thesection}{}
\renewcommand{\thesubsection}{}
\renewcommand{\thesubsubsection}{}
\makeatother

\newcommand{\subsubsubsection}[1]{\bigskip \rule[5pt]{\linewidth}{2pt} \textbf{ \color{red}{#1} } \newline \rule{\linewidth}{.1pt}}
\newlength{\parcoldist}
\setlength{\parcoldist}{1cm}

\usepackage{maths}
\newcommand{\dbf}{\leftrightarrows}
% remplace les commandes suivantes 
%\usepackage{amsmath}
%\usepackage{amssymb}
%\usepackage{amsthm}
%\usepackage{stmaryrd}

%\newcommand{\N}{\mathbb{N}}
%\newcommand{\Z}{\mathbb{Z}}
%\newcommand{\C}{\mathbb{C}}
%\newcommand{\R}{\mathbb{R}}
%\newcommand{\K}{\mathbf{K}}
%\newcommand{\Q}{\mathbb{Q}}
%\newcommand{\F}{\mathbf{F}}
%\newcommand{\U}{\mathbb{U}}

%\newcommand{\card}{\mathop{\mathrm{Card}}}
%\newcommand{\Id}{\mathop{\mathrm{Id}}}
%\newcommand{\Ker}{\mathop{\mathrm{Ker}}}
%\newcommand{\Vect}{\mathop{\mathrm{Vect}}}
%\newcommand{\cotg}{\mathop{\mathrm{cotan}}}
%\newcommand{\sh}{\mathop{\mathrm{sh}}}
%\newcommand{\ch}{\mathop{\mathrm{ch}}}
%\newcommand{\argsh}{\mathop{\mathrm{argsh}}}
%\newcommand{\argch}{\mathop{\mathrm{argch}}}
%\newcommand{\tr}{\mathop{\mathrm{tr}}}
%\newcommand{\rg}{\mathop{\mathrm{rg}}}
%\newcommand{\rang}{\mathop{\mathrm{rg}}}
%\newcommand{\Mat}{\mathop{\mathrm{Mat}}}
%\renewcommand{\Re}{\mathop{\mathrm{Re}}}
%\renewcommand{\Im}{\mathop{\mathrm{Im}}}
%\renewcommand{\th}{\mathop{\mathrm{th}}}


%En tete et pied de page
\lhead{Programme colle math}
\chead{Semaine 20 du 09/03/20 au 14/03/20}
\rhead{MPSI B Hoche}

\lfoot{\tiny{Cette création est mise à disposition selon le Contrat\\ Paternité-Partage des Conditions Initiales à l'Identique 2.0 France\\ disponible en ligne http://creativecommons.org/licenses/by-sa/2.0/fr/
} }
\rfoot{\tiny{Rémy Nicolai \jobname}}


\begin{document}
\subsection{Intégration}
\begin{itshape}
L'objectif majeur de ce chapitre est de définir l'intégrale d'une fonction continue par morceaux sur un segment à valeurs réelles ou complexes et d'en établir les propriétés élémentaires, notamment le lien entre intégration et primitivation. On achève ainsi la justification des propriétés présentées dans le chapitre \og Techniques fondamentales de calcul en analyse\fg.

Ce chapitre permet de consolider la pratique des techniques usuelles de calcul intégral. Il peut également offrir l'occasion de revenir sur l'étude des équations différentielles rencontrées au premier semestre.



La notion de continuité uniforme est introduite uniquement en vue de la construction de l'intégrale. L'étude systématique des fonctions uniformément continues est exclue.

Dans tout le chapitre, \,$\K$ désigne \, $\R$ ou \,$\C$.
\end{itshape}

\subsubsubsection{a) Continuité uniforme}
\begin{parcolumns}[rulebetween,distance=\parcoldist]{2}
  \colchunk{Continuité uniforme.}
  \colchunk{}
  \colplacechunks

  \colchunk{Théorème de Heine.}
  \colchunk{La démonstration n'est pas exigible.}
  \colplacechunks
\end{parcolumns}


\subsubsubsection{b) Fonctions continues par morceaux}
\begin{parcolumns}[rulebetween,distance=\parcoldist]{2}
  \colchunk{Subdivision d'un segment, pas d'une subdivision.}
  \colchunk{}
  \colplacechunks

  \colchunk{Fonction en escalier.}
  \colchunk{}
  \colplacechunks

  \colchunk{Fonction continue par morceaux sur un segment, sur un intervalle.}
  \colchunk{Une fonction est continue par morceaux sur un intervalle $I$ si sa restriction à tout segment inclus dans $I$ est continue par morceaux.}
  \colplacechunks
\end{parcolumns}

\subsubsubsection{c) Intégrale d'une fonction continue par morceaux sur un segment}
\begin{parcolumns}[rulebetween,distance=\parcoldist]{2}
  \colchunk{Intégrale d'une fonction continue par morceaux sur un segment.}
  \colchunk{Le programme n'impose pas de construction particulière.

Interprétation géométrique.

$\dbf$ PC et SI : valeur moyenne.

Aucune difficulté théorique relative à la notion d'aire ne doit être soulevée.

Notations $\displaystyle \int_{[a,b]}f$, $\displaystyle \int_{a}^bf$, $\displaystyle\int_a^bf(t)\,dt$.}
  \colplacechunks

  \colchunk{Linéarité, positivité et croissance de l'intégrale.}
  \colchunk{Les étudiants doivent savoir majorer et minorer des intégrales.}
  \colplacechunks

  \colchunk{Inégalité: $\displaystyle \left\lvert \int_{[a,b]}f \right\rvert \leqslant \int_{[a,b]} \lvert f \rvert$.}
  \colchunk{}
  \colplacechunks

  \colchunk{Relation de Chasles.}
  \colchunk{Extension de la notation $\displaystyle\int_a^bf(t)\,dt$ au cas où $b\leqslant a$. Propriétés correspondantes.}
  \colplacechunks

  \colchunk{L'intégrale sur un segment d'une fonction continue de signe constant est nulle si et seulement si la fonction est nulle.}
  \colchunk{}
  \colplacechunks
\end{parcolumns}

\subsubsubsection{d) Sommes de Riemann}
\begin{parcolumns}[rulebetween,distance=\parcoldist]{2}
  \colchunk{Si $f$ est une fonction continue par morceaux sur le segment $[a,b]$ à valeurs dans $\R$, alors
\[
\frac{b-a}{n}\sum_{k=0}^{n-1}f\left(a+k\frac{b-a}{n}\right) \xrightarrow[n \to +\infty]{} \int_a^bf(t)\,dt.
\]
}
  \colchunk{Interprétation géométrique.

Démonstration dans le cas où $f$ est de classe $\mathcal{C}^1$.

$\dbf$ I : méthodes des rectangles, des trapèzes.}
  \colplacechunks
\end{parcolumns}

\subsubsubsection{e) Intégrale fonction de sa borne supérieure}
\begin{parcolumns}[rulebetween,distance=\parcoldist]{2}
  \colchunk{Dérivation de $\displaystyle x \mapsto \int_a^x f(t)\, dt$ pour $f$ continue. Calcul d'une intégrale au moyen d'une primitive. Toute fonction continue sur un intervalle possède des primitives.}
  \colchunk{Intégration par parties, changement de variable.}
  \colplacechunks
\end{parcolumns}


\bigskip
\begin{center}
\textbf{Questions de cours}
\end{center}
Théorème de Heine avec démonstration.\newline
Preuve pour la relation de Chasles.\newline
Preuve pour la linéarité. \newline
Approximation par une suite de sommes de Riemann dans le cas $\mathcal{C}^1$.\newline
Dérivation de l'intégrale fonction de sa borne supérieure.\newline

\begin{center}
\textbf{Prochain programme}
\end{center}
Intégration: calculs de primitives, formule de Taylor avec reste intégral.\newline
Matrices. Calcul matriciel. (les matrices pour elles mêmes).
\end{document}
