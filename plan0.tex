%! iTeXMac(project): pdftex

\input fr

\def\date{}

%\input macros


\catcode`\@=11

\font\goth=eufm10
\font\ineg=msam8
\font\star=msam10
\font\vid=msbm10
\font\bsl=cmbxsl10 at 10pt % gras-pench{\'e}
\font\large=cmr10 at 12pt
\font\Large=cmr10 at 14pt
\font\largeb=cmbx10 at 12pt
\font\Largeb=cmbx10 at 14pt
\font\pcar=cmr8 at 8pt % pour {\'e}crire les si\`ecles
\font\tenbb=cmssbx10 at 10pt % police provisoire pour R,N,Q,Z
\font\sevenbb=cmbx10 at 7pt
\font\fivebb=cmbx10 at 5pt

\everymath{\displaystyle}
\newfam\bbfam
\textfont\bbfam=\tenbb
\scriptfont\bbfam=\sevenbb
\scriptscriptfont\bbfam=\fivebb
\def\bb{\fam\bbfam\tenbb}

\catcode`\;=\active
\def;{\relax\ifhmode\ifdim\lastskip>\z@
\unskip\fi\kern.2em\fi\string;}

\catcode`\:=\active
\def:{\relax\ifhmode\ifdim\lastskip>\z@\unskip\fi
\penalty\@M\ \fi\string:}

\catcode`\!=\active
\def!{\relax\ifhmode\ifdim\lastskip>\z@
\unskip\fi\kern.2em\fi\string!}

\catcode`\?=\active
\def?{\relax\ifhmode\ifdim\lastskip>\z@
\unskip\fi\kern.2em\fi\string?}

%\def\^#1{\if#1i{\accent"5E\i}\else{\accent"5E #1}\fi}
%\def\"#1{\if#1i{\accent"7F\i}\else{\accent"7F #1}\fi}

\newif\ifpagetitre \pagetitretrue
\newtoks\hautpagetitre
\hautpagetitre={\tenrm\hfil\the\premiertitre\hfil}
\newtoks\baspagetitre \baspagetitre={\hfil}

\newtoks\partiecourante \partiecourante={\hfil}
\newtoks\titrecourant \titrecourant={\hfil}
\newtoks\premiertitre \premiertitre={\hfil}

\newtoks\hautpagegauche \newtoks\hautpagedroite
\hautpagegauche={\tenrm\folio\hfill{\the\partiecourante}}
\hautpagedroite={\tenrm{\the\titrecourant}\hfill\folio}

\newtoks\baspagegauche \baspagegauche={\hfil}
\newtoks\baspagedroite \baspagedroite={\hfil}

\headline={\ifnum\pageno=1\the\hautpagetitre\else\the\hautpagedroite \fi}

\footline={\hfil}

\def\nopagenumbers{\def\folio{\hfil}}

\catcode`\@=12

\let\optionkeymacros\null
\let\dis=\displaystyle
\let\scr=\scriptstyle
\let\so=\medskip
\let\eps=\varepsilon % Le "bon" epsilon
\let\vphi=\varphi % Le phi usuel
\let\tend=\rightarrow
\let\Tend=\longrightarrow
\let\ssi=\Longleftrightarrow

\def\op{{\star F}}
\def\frac#1#2{{#1\over#2}}
\def\text#1{\hbox{\rm #1}}
\def\d{\,\hbox{\rm d}\,}
\def\trait{\par\centerline{\hbox{\vrule height .4pt depth 0pt width 12cm}}}
\def\ie{\mathrel{\hbox{\ineg 6}}} % <= fran{\c c}ais
\def\le{\mathrel{\hbox{\ineg 6}}} % <= fran{\c c}ais
\def\leq{\mathrel{\hbox{\ineg 6}}} % <= fran{\c c}ais
\def\se{\mathrel{\hbox{\ineg >}}} % >= fran{\c c}ais
\def\ge{\mathrel{\hbox{\ineg >}}} % >= fran{\c c}ais
\def\geq{\mathrel{\hbox{\ineg >}}} % >= fran{\c c}ais
\def\vide{\hbox{\vid~?}}
\def\Z{{\bb Z}}
\def\R{{\bb R}}
\def\C{{\bb C}}
\def\N{{\bb N}}
\def\Q{{\bb Q}}
\def\K{{\bb K}}
\def\U{{\bb U}}
\def\dim{{\rm dim}\,}
\def\sev{{\rm sous-espace vectoriel}}
\def\ker{{\rm Ker}\,}
\def\Ker{{\rm Ker}\,}
\def\re{{\rm Re}\,}
\def\im{{\rm Im}\,}
\def\gav{{\rm GA}(E)}
\def\gle{{\rm GL}(E)}
\def\mnpk{{\cal M}_{n,p}(\K)}
\def\mnk{{\cal M}_n(\K)}
\def\glnk{{\rm GL}_n(\K)}
\def\det{{\rm Det}\,}
\def\card{{\rm Card}}
\def\tr{{\rm Tr}\,}
\def\e{{\rm e}}
\def\ch{\mathop{\rm ch}\nolimits}
\def\sh{\mathop{\rm sh}\nolimits}
\def\th{\mathop{\rm th}\nolimits}
\def\argch{\mathop{\rm Arg\,ch}\nolimits}
\def\argsh{\mathop{\rm Arg\,sh}\nolimits}
\def\argth{\mathop{\rm Arg\,th}\nolimits}
\def\arccos{\mathop{\rm Arc\,cos}\nolimits}
\def\arcsin{\mathop{\rm Arc\,sin}\nolimits}
\def\arctan{\mathop{\rm Arc\,tan}\nolimits}
\def\adh#1{\overline{\!#1}}
\def\rond#1{\buildrel\;\circ\over #1}
\def\cnp#1#2{{\displaystyle\Big({{\textstyle #1}\atop%
{\textstyle #2}}\Big)}}
\def\hfl#1#2{\smash{\mathop{\hbox to 4mm{\rightarrowfill}}
\limits^{#1}_{#2}}}
\def\vect#1{\overrightarrow{#1}} % vecteur
\def\Frac#1#2{{\displaystyle#1\over\displaystyle#2}}
\def\frac#1#2{{\scriptstyle#1\over\scriptstyle#2}}
\def\Der#1#2{\Frac{\hbox{d}#1}{\hbox{d}#2}} % Ex:\Der{y}{x}
\def\Derr#1#2{\Frac{\hbox{d}^2#1}{\hbox{d}#2^2}}
\def\Dron#1#2{\Frac{\partial#1}{\partial#2}}
\def\dron#1#2{\frac{\partial#1}{\partial#2}}
\def\<<{\leavevmode\raise.3ex\hbox{$\scriptscriptstyle\langle\!\langle$}}
\def\>>{\leavevmode\raise.3ex\hbox{$\scriptscriptstyle\rangle\!\rangle$}}
\def\implique{\ \Longrightarrow\ }
\def\vabs#1{\vert#1\vert}
\def\norme#1{\Vert#1\Vert}
\def\Norme#1{\vert\vert\vert#1\vert\vert\vert}
\def\[{[\![}
\def\]{]\!]}

% SUPERPOSITION
\def\up#1{\raise 1ex\hbox{\septtm#1}}
% op{\'e}rateur avec dessous: build {op{\'e}rateur} {dessous}
\def\build#1#2{\mathrel{\mathop{\kern 0pt#1}\limits_{#2}}}
% op{\'e}rateur avec deux dessous: Build {op{\'e}rateur} {dessous1}{dessous2}
\def\Build#1#2#3{\build{{#1}}{\scriptstyle{#2}\atop\scriptstyle{#3}}}
% fl\`eche double : fleche {variable} {valeur}
\def\fleche#1#2{\build{\hbox to 9mm{\rightarrowfill}}{{#1}\rightarrow{#2}}}
% fl\`eche triple : Fleche {variable} {valeur} {3i\`eme ligne}
\def\Fleche#1#2#3{\build{\hbox to 9mm{\rightarrowfill}}
{\scriptstyle{#1}\rightarrow{#2}\atop\scriptstyle{#3}}}
% encadrement d'une bo{\^i}te: cadre {largeur blanc} {bo{\^i}te}
\long\def\cadre#1#2{\vbox{\hrule\hbox{\vrule%
\vbox spread#1{\vfil\hbox spread#1{\hfil#2\hfil}\vfil}\vrule}\hrule}\par}

\def\tp{\centerline{\bsl Travaux pratiques}\so}
\def\tit#1{{\parindent=-2mm{\bf#1}\smallskip}}
\long\def\TITRE#1{\bigskip\bigskip

\centerline{\Large#1}

\bigskip}
\long\def\TIT#1{\bigskip\centerline{\largeb#1}\bigskip}
\long\def\Titre#1{\bigskip{\large#1}\bigskip}
\long\def\titre#1{\bigskip\centerline{\hfill{\bf #1}\hfill}
\medskip}
\long\def\tx#1#2{\hbox{\hbox to 94mm{\vtop{\hsize=90mm#1\vfill}\hfill}
\hfill\hbox to 74mm{\vtop{\hsize=70mm#2\vfill}\hfill}}}%\filbreak}


\hsize=170mm \vsize=250mm
\hoffset=-4mm \voffset=-1mm
\pretolerance=500 \tolerance=1000 \brokenpenalty=5000

%\fhyph
\frenchspacing
\overfullrule=0cm %\emergencystretch=10pt

\null
\vskip 0.5cm

\parindent=0mm
\abovedisplayskip=6pt plus 2pt minus 4pt
\abovedisplayshortskip=0pt plus 2pt
\belowdisplayskip=6pt plus 2pt minus 4pt
\belowdisplayshortskip=0pt plus 2pt

\def\bg{\bigskip}
\def\md{\medskip}
\def\cl{\centerline}
\def\info{informatique}

\long\def\tx#1#2{\hbox{\hbox to 94mm{\vtop{\hsize=90mm#1\vfill}\hfill}
\hfill\hbox to 74mm{\vtop{\hsize=70mm#2\vfill}\hfill}}\filbreak}

\long\def\txv#1#2{\hbox{\vrule\hskip2mm\hbox to
94mm{\vtop{\hsize=90mm#1\vfill}\hfill} \hfill\hbox to
74mm{\vtop{\hsize=70mm#2\vfill}\hfill}}\filbreak}

\long\def\txz#1#2{\hbox{\hbox to
94mm{\vtop{\hsize=90mm#1\vfill}\hfill} \hfill\hbox to
74mm{\vtop{\hsize=70mm#2\vfill}\hfill}\hskip2mm\vrule width 1mm}\filbreak}

\def\vrg{\raise.3mm\hbox{$\hskip.2mm,\ $}}





\titrecourant={\bf MPSI\ }
\partiecourante={\bf MPSI\ }
\premiertitre={\bf CLASSE DE PREMI\`ERE ANN\'EE MPSI} \bg {\sl Le
programme de premi\`ere ann\'ee MPSI est organis\'e en trois parties. Dans
une premi\`ere partie figurent les notions et les objets qui doivent
\^etre \'etudi\'es d\`es le d\'ebut de l'ann\'ee scolaire. Il s'agit
essentiellement, en partant du programme de la classe de Terminale S et en
s'appuyant sur les connaissances pr\'ealables des \'etudiants, d'introduire des notions de base
n{\'e}cessaires tant en math{\'e}matiques que dans les autres disciplines
scientifiques (physique, chimie, sciences industrielles\dots). Certains de ces objets seront
consid\'er\'es comme d\'efinitivement acquis (nombres complexes, coniques,
\dots) et il n'y aura pas lieu de reprendre ensuite leur \'etude dans le
cours de math\'ematiques; d'autres, au contraire, seront revus plus tard
dans un cadre plus g\'en\'eral ou dans une pr\'esentation plus th\'eorique
(groupes, produit scalaire, \'equations diff\'erentielles, \dots).

Les deuxi\`eme et troisi\`eme parties correspondent \`a un
d\'ecoupage classique entre l'analyse et ses applications
g\'eom\'etriques d'une part, l'alg\`ebre et la g\'eom\'etrie
euclidienne d'autre part. } \vfill\eject

\TITRE{PROGRAMME DE D\'EBUT D'ANN\'EE}

\Titre{I. NOMBRES COMPLEXES ET G\'EOM\'ETRIE \'EL\'EMENTAIRE}


\titre{1- Nombres complexes} \so

\tit{a) Corps \C\ des nombres complexes} \so

\tit{b) Groupe \U\ des nombres complexes de module 1} \so

\tit{c) \'Equations du second degr\'e} \so

\tit{d) Exponentielle complexe} \so

\tit{e) Nombres complexes et g\'eom\'etrie plane} \so

\titre{2- G\'eom\'etrie \'el\'ementaire du plan} \so

\tit{a) Modes de rep\'erage dans le plan} \so

\tit{b) Produit scalaire} \so

\tit{c) D\'eterminant} \so

\tit{d) Droites} \so

\tit{e) Cercles} \so

\titre{3- G\'eom\'etrie \'el\'ementaire de l'espace}

\tit{a) Modes de rep\'erage dans l'espace} \so \tit{c) Produit
vectoriel} \so

\tit{e) Droites et plans} \so

\tit{f) Sph\`eres} \so

\Titre{II. FONCTIONS USUELLES ET \'EQUATIONS DIFF\'ERENTIELLES
LIN\'EAIRES} \so

 \titre{1- Fonctions usuelles} \so

 \tit{a) Fonctions exponentielles, logarithmes, puissances} \so

\tit{b) Fonctions circulaires} \so

\tit{c) Fonction exponentielle complexe} \so

\titre{2- \'Equations diff\'erentielles lin\'eaires} \so

\tit{a) \'Equations lin\'eaires du premier ordre} \so

\tit{b)M\'ethode d'Euler} \so

\tit{c) \'Equations lin\'eaires du second ordre \`a coefficients
constants} \so

\titre{3- Courbes param\'etr\'ees. Coniques} \so \tit{a) Courbes
planes param\'etr\'ees} \so

\tit{b) Coniques} \vfill\eject

\TITRE{ANALYSE ET G\'EOM\'ETRIE DIFF\'ERENTIELLE} \so

\Titre{I. NOMBRES R\'EELS, ET COMPLEXES SUITES ET FONCTIONS} \so

\titre{1- Suites de nombres r\'eels} \so

\tit{a) Corps \R\ des nombres r\'eels} \so

\tit{b) Suites de nombres r\'eels} \so

\tit{c) Limite d'une suite} \so

\tit{d) Relations de comparaison} \so

\tit{e) Th\'eor\`emes d'existence de limites} \so

\tit{f) Br{\`e}ve extension aux suites complexes} \so

\titre{2- Fonctions d'une variable r\'eelle \`a valeurs r\'eelles}
\so

\tit{a) Fonctions d'une variable r\'eelle \`a valeurs r\'eelles}
\so

\tit{b) \'Etude locale d'une fonction} \so

\tit{c) Relations de comparaison} \so

\tit{d) Fonctions continues sur un intervalle} \so

\tit{e) Br{\`e}ve extension aux fonctions \`a valeurs complexes}
\so

\Titre{II. CALCUL DIFF\'ERENTIEL ET INT\'EGRAL} \so

\titre{1- D\'erivation des fonctions \`a valeurs r\'eelles} \so

\tit{a) D\'eriv\'ee en un point, fonction d\'eriv\'ee} \so

\tit{b) \'Etude globale des fonctions d\'erivables}\so

\tit{c) Fonctions convexes} \so

\tit{d) Br\`eve extension aux fonctions \`a valeurs complexes}\so

\titre{2- Int\'egration sur un segment des fonctions \`a valeurs
r\'eelles} \so

\tit{a) Fonctions continues par morceaux} \so

\tit{b) Int\'egrale d'une fonction continue par morceaux} \so

\tit{c) Br{\`e}ve extension aux fonctions {\`a} valeurs complexes}
\so

\titre{3- Int\'egration et d\'erivation} \so

\tit{a) Primitives et int\'egrale d'une fonction continue} \so

\tit{b) \S\ Calcul des primitives} \so

\tit{c) Formules de Taylor}\so

\tit{d) D{\'e}veloppement limit{\'e}s}\so

\titre{4- Approximation}\so

\tit{b) Calcul approch\'e d'une int\'egrale}\so

\tit{c) Valeurapproch\'ee de r\'eels}\so

\Titre{III. NOTIONS SUR LES FONCTIONS DE DEUX VARIABLES
R\'EELLES}\so

\titre{1- Espace $\R^2$, fonctions continues} \so

\titre{2- Calcul diff\'erentiel} \so

\tit{a) D\'eriv\'ees partielles premi\`eres} \so

\tit{b) D\'eriv\'ees partielles d'ordre $2$} \so

 \titre{3- Calcul int\'egral} \so


\Titre{IV. G\'EOM\'ETRIE DIFF\'ERENTIELLE} \so

\titre{1- \'Etude m{\'e}trique des courbes planes} \so

\titre{2- Champs de vecteurs du plan et de l'espace} \so

\vfill\eject

\TITRE{ALG\`EBRE ET G\'EOM\'ETRIE} \so

\Titre{I. NOMBRES ET STRUCTURES ALG\'EBRIQUES USUELLES} \so

\titre{1- Vocabulaire relatif aux ensembles et aux applications}
\so

\titre{2- Nombres entiers naturels, ensembles finis,
d\'enombrements} \so

\tit{a) Nombres entiers naturels} \so

\tit{b) Ensembles finis} \so

\tit{c) \S\ Op\'erations sur les ensembles finis, d\'enombrements}
\so

\titre{3- Structures alg\'ebriques usuelles} \so

\tit{a) Vocabulaire relatif aux groupes et aux anneaux} \so

\tit{b) \S\ Arithm\'etique dans \Z. Calculs dans \R\ ou \C.} \so

\Titre{II. ALG\`EBRE LIN\'EAIRE ET POLYN\^OMES} \so

\titre{1- Espaces vectoriels} \so

\tit{a) Espaces vectoriels} \so

\tit{b) Translations, sous-espaces affines} \so

\tit{c) Applications lin\'eaires} \so

\titre{2- Dimension des espaces vectoriels} \so

\tit{a) Familles de vecteurs} \so

\tit{b) Dimension d'un espace vectoriel} \so

\tit{c) Dimension d'un sous-espace vectoriel} \so

\tit{d) Rang d'une application lin\'eaire} \so

\titre{3- Polyn\^omes} \so

\tit{a) Polyn\^omes \`a une ind\'etermin\'ee et corps $\K(X)$} \so

\tit{b) Fonctions polynomiales et rationnelles} \so

\tit{c) Polyn\^omes scind\'es} \so

\tit{d) Divisibilit\'e dans l'anneau $\K[X]$} \so

\tit{e) \'Etude locale d'une fraction rationnelle} \so

\titre{4- \S\ Calcul matriciel} \so

\tit{a) Op\'erations sur les matrices} \so

\tit{b) Matrices et applications lin\'eaires} \so

\tit{c) Op\'erations \'el\'ementaires sur les matrices} \so

\tit{d) Rang d'une matrice} \so

\tit{e) Syst\`emes d'\'equations lin\'eaires} \so

\titre{5- D\'eterminants} \so

\tit{a) Groupe sym\'etrique} \so

\tit{b) Applications multilin\'eaires} \so

\tit{c) D\'eterminant de $n$ vecteurs} \so

\tit{d) D\'eterminant d'un endomorphisme} \so

\tit{e) D\'eterminant d'une matrice carr\'ee} \so

\Titre{III. ESPACES VECTORIELS EUCLIDIENS ET G\'EOM\'ETRIE
EUCLIDIENNE} \so

\titre{1- Produit scalaire, espaces vectoriels euclidiens} \so

\tit{a) Produit scalaire}\so

\tit{b) Orthogonalit\'e} \so

\tit{c) Isom\'etries affines du plan et de l'espace} \so

\tit{d) Automorphismes orthogonaux du plan vectoriel euclidien}
\so

\tit{e) Automorphismes orthogonaux de l'espace} \so

\tit{f) D{\'e}placements} \so

\tit{g) Similitudes directes du plan} \so



\bye
