\subsection{Raisonnement et vocabulaire ensembliste}
 Ce chapitre regroupe les différents points de vocabulaire, notations et raisonnement nécessaires aux étudiants pour la conception et la rédaction efficace d’une démonstration mathématique. Ces notions doivent être introduites de manière progressive en vue d’être acquises en fin de premier semestre.
Le programme se limite strictement aux notions de base figurant ci-dessous. Toute étude systématique de la logique ou de la théorie des ensembles est hors programme.

\subsubsubsection{a) Rudiments de logique}
\begin{parcolumns}[rulebetween,distance=\parcoldist]{2}
  \colchunk{Quantificateurs}
  \colchunk{L'emploi de quantificateurs en guise d'abréviation est exclue.}
  \colplacechunks
  
  \colchunk{Implication, contraposition, équivalence}
  \colchunk{Les étudiants doivent savoir formuler la négation d’une proposition.}
  \colplacechunks
  \colchunk{Modes de raisonnement : par récurrence (faible et forte), par contraposition, par l’absurde, par analyse-synthèse.}
  \colchunk{On pourra relier le raisonnement par récurrence au fait que toute partie non vide de $\N$ possède un plus petit élément. Toute construction et toute axiomatique de N sont hors programme.\newline
  Le raisonnement par analyse-synthèse est l’occasion de préciser les notions de condition nécessaire et condition suffisante.}
  \colplacechunks

\end{parcolumns}

\subsubsubsection{b) Ensembles}
\begin{parcolumns}[rulebetween,distance=\parcoldist]{2}
  \colchunk{Ensemble, appartenance, inclusion. Sous-ensemble (ou partie).}
  \colchunk{Ensemble vide.}
  \colplacechunks
  
  \colchunk{Opérations sur les parties d’un ensemble: réunion, intersection, différence, passage au complémentaire. Produit cartésien d’un nombre fini d’ensembles.}
  \colchunk{Notation $A \setminus B$ pour la différence et $E \setminus A$, $\overline{A}$ et  $C^E_A$ pour le complémentaire.}
  \colplacechunks

  \colchunk{Ensemble des parties d’un ensemble.}
  \colchunk{Notation $\mathcal{P}(E)$.}
  \colplacechunks

\end{parcolumns}

\subsubsubsection{c) Applications et relations}
\begin{parcolumns}[rulebetween,distance=\parcoldist]{2}
  \colchunk{Application d’un ensemble dans un ensemble. Graphe d’une application.}
  \colchunk{Le point de vue est intuitif : une application de $E$ dans $F$ associe à tout élément de $E$ un unique élément de $F$. Le programme ne distingue pas les notions de fonction et d’application. Notations $\mathcal{F}(E , F )$ et $\mathcal{F}( E)$.}
  \colplacechunks
  
  \colchunk{Famille d’éléments d’un ensemble.}
  \colchunk{}
  \colplacechunks

  \colchunk{Fonction indicatrice d’une partie d’un ensemble.}
  \colchunk{Notation $1_A$.}
  \colplacechunks

  \colchunk{Restriction et prolongement.}
  \colchunk{Notation $f_{|A}$.}
  \colplacechunks

  \colchunk{Image directe.}
  \colchunk{Notation $f(A)$.}
  \colplacechunks

  \colchunk{Image réciproque.}
  \colchunk{Notation $f^{-1}(B)$. Cette notation pouvant prêter à confusion, on peut provisoirement en utiliser une autre. 
  \newline
  Pour une fonction $f$ de $E$ dans $F$, j'utilise $\Phi(A)$ au lieu de $f(A)$ pour une partie $A$ de $E$ et $\varphi(X)$ au lieu de $f^{-1}(X)$ pour une partie $X$ de $F$.}
  \colplacechunks

  \colchunk{Composition.}
  \colchunk{}
  \colplacechunks

  \colchunk{Injection, surjection. Composée de deux injections, de deux surjections.}
  \colchunk{$f\circ g$ injective implique $g$ injective. $f\circ g$ surjective implique $g$ surjective. }
  \colplacechunks
  
  \colchunk{Bijection, réciproque. Composée de deux bijections, réciproque de la composée.}
  \colchunk{Compatibilité de la notation $f^{-1}$ avec la notation d'une image réciproque.}
  \colplacechunks
  
  \colchunk{Relation binaire sur un ensemble.}
  \colchunk{}
  \colplacechunks
  
  \colchunk{Relation d’équivalence, classes d’équivalence.}
  \colchunk{La notion d’ensemble quotient est hors programme.}
  \colplacechunks
  
  \colchunk{Relations de congruence modulo un réel sur $\R$, modulo un entier sur $\Z$.}
  \colchunk{}
  \colplacechunks
  
  \colchunk{Relation d’ordre. Ordre partiel, total.}
  \colchunk{}
  \colplacechunks
  
\end{parcolumns}
