\subsubsection{A - Probabilités sur un univers fini}
\begin{itshape}
 Les définitions sont motivées par la notion d'expérience aléatoire.
La modélisation de situations aléatoires simples fait partie des capacités attendues des étudiants.
\end{itshape}

\subsubsubsection{a) Expérience aléatoire et univers}
\begin{parcolumns}[rulebetween,distance=2.5cm]{2}
  \colchunk{L'ensemble des issues (ou résultats possibles ou réalisations) d'une expérience aléatoire est
appelé univers.}
  \colchunk{On se limite au cas où cet univers est fini.}
  \colplacechunks

  \colchunk{\'Evénement, événement élémentaire (singleton), événement contraire, événements \og $A$ et $B$ \fg, évènement \og$A$ ou $B$ \fg, événement impossible, événements incompatibles, système complet d'événements.}
  \colchunk{}
  \colplacechunks
\end{parcolumns}

\subsubsubsection{b) Espaces probabilisés finis}
\begin{parcolumns}[rulebetween,distance=2.5cm]{2}

  \colchunk{Une probabilité sur un univers fini $\Omega$ est une application $P$ de $\mathcal{P} (\Omega)$ dans $[ 0 , 1 ]$ telle que $P (\Omega) = 1$ et, pour toutes parties disjointes $A$ et $B$, $P (A \cup B) = P (A) + P (B).$}
  \colchunk{Un espace probabilisé fini est un couple $(\Omega , P)$ où $\Omega$ est un univers fini et $P$ une probabilité sur $\Omega$.}
  \colplacechunks

  \colchunk{Détermination d'une probabilité par les images des singletons.}
  \colchunk{}
  \colplacechunks
  
  \colchunk{Probabilité uniforme.}
  \colchunk{}
  \colplacechunks

  \colchunk{Propriétés : probabilité de la réunion de deux événements, de l'événement contraire, croissance.}
  \colchunk{}
  \colplacechunks
\end{parcolumns}

\subsubsubsection{c) Probabilités conditionnelles}
\begin{parcolumns}[rulebetween,distance=2.5cm]{2}
  \colchunk{Si $P (B) > 0$, la probabilité conditionnelle de $A$ sachant $B$ est définie par : $P (A|B) = P_B (A) = \dfrac{P (A \cap B)}{P (B)}$.}
  \colchunk{On justifiera cette définition par une approche heuristique fréquentiste.

L'application $P_B$ est une probabilité.}
  \colplacechunks

  \colchunk{Formule des probabilités composées.}
  \colchunk{}
  \colplacechunks

  \colchunk{Formule des probabilités totales.}
  \colchunk{}
  \colplacechunks

  \colchunk{Formules de Bayes :
\begin{enumerate}
\item
si $A$ et $B$ sont deux événements tels que $P(A)>0$ et $P(B)>0$, alors
$$P(A\,|\,B)=\frac{P(B\,|\,A)\,P(A)}{P(B)}$$
\item
si $(A_i)_{1\leqslant i\leqslant n}$ est un système complet d'événements de probabilités
non nulles et si $B$ est un événement de probabilité non nulle, alors
$$P(A_j\,|\,B)= \frac{P(B\,|\,A_j)\;P(A_j)}{\displaystyle{\sum_{i=1}^n P(B\,|\,A_i)\;P(A_i)}}$$
\end{enumerate}}
  \colchunk{On donnera plusieurs applications issues de la vie courante.}
  \colplacechunks
\end{parcolumns}

\subsubsubsection{d) \'Evénements indépendants}
\begin{parcolumns}[rulebetween,distance=2.5cm]{2}
  \colchunk{Couple d'événements indépendants.}
  \colchunk{Si $P (B) > 0$, l'indépendance de $A$ et $B$ s'écrit $P (A|B) = P (A)$.}
  \colplacechunks

  \colchunk{Famille finie d'événements mutuellement indépendants.}
  \colchunk{L'indépendance deux à deux des événements d'une famille $(A_1,\dots,A_n)$ n'implique pas l'indépendance mutuelle si $n \geq 3$.}
  \colplacechunks
\end{parcolumns}
