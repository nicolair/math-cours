\subsection{Analyse asymptotique}
\begin{itshape}
 L’objectif de ce chapitre est de familiariser les étudiants avec les techniques asymptotiques de base, dans les cadres discret
et continu. Les suites et les fonctions y sont à valeurs réelles ou complexes, le cas réel jouant un rôle prépondérant.\newline
On donne la priorité à la pratique d’exercices plutôt qu’à la vérification de propriétés élémentaires relatives aux relations
de comparaison.\newline
Les étudiants doivent connaître les développements limités usuels et savoir rapidement mener à bien des calculs asymptotiques simples. En revanche, les situations dont la gestion manuelle ne relèverait que de la technicité seront traitées à
l’aide d’outils logiciels.
\end{itshape}

\subsubsubsection{a) Relations de comparaison: cas des suites}
\begin{parcolumns}[rulebetween,distance=\parcoldist]{2}
  \colchunk{Relations de domination, de négligeabilité, d'équivalence.}
  \colchunk{Notations $u_n=O(v_n)$, $u_n=o(v_n)$, $u_n\sim v_n$.\newline
  On définit ces relations à partir du quotient $\frac{u_n}{v_n}$ sous l'hypothèse que la suite $(v_n)_{n\in\N}$ ne s'annule pas à partir d'un certain rang.\newline
  Traduction à l'aide du symbole $o$ des croissances comparées des suites de termes généraux $\ln^\beta(n)$, $n^\alpha$, $e^{\gamma n}$.}
  \colplacechunks
  
  \colchunk{Liens entre les relations de comparaison.}
  \colchunk{\'Equivalence des relations $u_n\sim v_n$ et $u_n-v_n = o(v_n)$.}
  \colplacechunks

  \colchunk{Opérations sur les équivalents: produit, quotient, puissances.}
  \colchunk{}
  \colplacechunks

  \colchunk{Propriétés conservées par équivalence: signe, limite.}
  \colchunk{}
  \colplacechunks
\end{parcolumns}


\subsubsubsection{b) Relations de comparaison: cas des fonctions}
\begin{parcolumns}[rulebetween,distance=\parcoldist]{2}
  \colchunk{Adaptation aux fonctions des définitions et résultats précédents.}
  \colchunk{}
  \colplacechunks
\end{parcolumns}


\subsubsubsection{c) Développements limités}
\begin{parcolumns}[rulebetween,distance=\parcoldist]{2}
  \colchunk{Développement limité, unicité des coefficients, troncature.}
  \colchunk{Développement limité en $0$ d'une fonction paire, impaire.}
  \colplacechunks
  
  \colchunk{Forme normalisée d'un développement limité:
  \begin{displaymath}
   f(a+h) \underset{h \rightarrow a}{=} h^p\left(a_0+a_1h+\cdots+a_nh^n+o(h^n) \right)
  \end{displaymath}
  avec $a_0\neq 0$.}
  \colchunk{\'Equivalence $f(a+h)\underset{h\rightarrow 0}{\sim} a_0h^p$; signe de $f$ au voisinage de $a$.}
  \colplacechunks

  \colchunk{Opérations sur les développements limités: combinaison linéaire, produit, quotient.}
  \colchunk{Utilisation de la forme normalisée pour prévoir l'ordre d'un développement.\newline
  Les étudiants doivent savoir déterminer sur des exemples simples le développement limité d'une composée, mais aucun résultat général n'est exigible.\newline
  La division suivant les puissances croissantes est hors programme.}
  \colplacechunks

  \colchunk{Primitivation d'un développement limité.}
  \colchunk{}
  \colplacechunks

  \colchunk{Formule de Taylor-Young: développement limité à l'ordre $n$ en un point d'une fonction de classe $\mathcal{C}^n$.}
  \colchunk{La formule de Taylor-Young peut être admise à ce stade et justifiée dans le chapitre \og Intégration\fg.}
  \colplacechunks

  \colchunk{Développement limité à tout ordre en $0$ de $\exp$, $\sin$, $\cos$, $\sh$, $\ch$, $x\mapsto \ln(1+x)$, $x\mapsto (1+x)^\alpha$, $\arctan$ et de $\tan$ à l'ordre $3$.}
  \colchunk{}
  \colplacechunks

  \colchunk{Utilisation des développements limités pour préciser l'allure d'une courbe au voisinage d'un point.}
  \colchunk{Condition nécessaire, condition suffisante à l'ordre $2$ pour un extrémum local.}
  \colplacechunks
\end{parcolumns}


\subsubsubsection{d) Exemples de développements asymptotiques}
\begin{parcolumns}[rulebetween,distance=\parcoldist]{2}
  \colchunk{}
  \colchunk{La notion de développement asymptotique est présentée sur des exemples simples. \newline
  La notion d'échelle de comparaison est hors programme.}
  \colplacechunks
  
  \colchunk{Formule de Stirling.}
  \colchunk{La démonstration n'est pas exigible.}
  \colplacechunks
\end{parcolumns}
