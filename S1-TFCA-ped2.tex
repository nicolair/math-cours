\subsubsection{C - Primitives et équations différentielles linéaires}

\subsubsubsection{c) \'Equations différentielles linéaires du second ordre à coefficients constants}
\begin{parcolumns}[rulebetween,distance=\parcoldist]{2}
  \colchunk{Notion d'équation différentielle linéaire du second ordre à coefficients constants:
  \begin{displaymath}
   y''+ay'+by = f(x)
  \end{displaymath}
où $a$ et $b$ sont des scalaires et $f$ est une application continue à valeurs dans $\R$ ou $\C$.}
  \colchunk{\'Equation homogène associée.}
  \colplacechunks

  \colchunk{Résolution de l'équation homogène.}
  \colchunk{Si $a$ et $b$ sont réels, description des solutions réelles.}
  \colplacechunks

  \colchunk{Forme des solutions: somme d'une solution particulière et de la solution générale de l'équation homogène.}
  \colchunk{Les étudiants doivent savoir déterminer une solution particulière dans le cas d'un second membre de la forme $x\mapsto Ae^{\lambda x}$ avec $(A,\lambda)\in \C^2$, $x\mapsto B\cos(\omega x)$ et $x\mapsto B\sin(\omega x)$ avec $(B,\omega)\in \R^2$.\newline 
  $\leftrightarrows$ PC: régime libre, régime forcé; régime transitoire, régime établi.}
  \colplacechunks

  \colchunk{Principe de superposition.}
  \colchunk{}
  \colplacechunks

  \colchunk{Existence et unicité de la solution d'un problème de Cauchy.}
  \colchunk{La démonstration de ce résultat est hors programme.\newline
  $\leftrightarrows$ PC et SI: modélisation des circuits électriques LC, RLC et de systèmes mécaniques linéaires.}
  \colplacechunks

  \colchunk{}
  \colchunk{}
  \colplacechunks
\end{parcolumns}
