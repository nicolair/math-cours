\begin{itshape}
 Le paragraphe A - a) consiste largement en des adaptations au cas continu de notions déjà abordées pour les suites. Afin d'éviter les répétitions, le professeur a la liberté d'admettre certains résultats.\newline
 Pour la pratique du calcul des limites, on se borne à ce stade a des calculs très simples, en attendant de pouvoir disposer d'outils efficaces (développement limités).\end{itshape}

\subsubsection{A - Limites et continuité (études locales)}

\subsubsubsection{a) Limite d'une fonction en un point}
\begin{parcolumns}[rulebetween,distance=\parcoldist]{2}
  \colchunk{\'Etant donné un point $a \in\overline{\R}$ appartenant à $I$ ou extrémité de $I$, limite, finie ou infinie d'une fonction en $a$.}
  \colchunk{Notation $f(x) \underset{x\rightarrow a}{\longrightarrow}l$. \newline  Les propriétés sont énoncées avec des inégalités larges.}
  \colplacechunks
    
  \colchunk{Stabilité des inégalités larges par passage à la limite. Unicité de la limite.}
  \colchunk{Notations $\underset{x\rightarrow a}{\lim}f(x)$, $\underset{a}{\lim}f$.\newline }
  \colplacechunks
    
  \colchunk{Si $f$ est définie en $a$ et possède une limite en $a$, alors $\underset{x\rightarrow a}{\lim}f(x)=f(a)$.}
  \colchunk{}
  \colplacechunks
    
  \colchunk{Si $f$ possède une limite finie en $a$, $f$ est bornée au voisinage de $a$.}
  \colchunk{}
  \colplacechunks
    
  \colchunk{Limite à droite, limite à gauche.}
  \colchunk{Notations $\underset{\underset{x>a}{x\rightarrow a}}{\lim}f(x)=f(a)$ ou $\underset{x\rightarrow a^+}{\lim}f(x)$.}
  \colplacechunks
    
  \colchunk{Extension de la notion de limite en $a$ lorsque $f$ est définie sur $I\setminus\{a\}$.\newline
  Caractérisation séquentielle de la limite (finie ou infinie).\newline
  Opérations sur les limites: combinaison linéaire, produit, quotient, composition.\newline
  Théorèmes d'encadrement (limite finie), de minoration (limite $+\infty$), de majoration (limite $-\infty$).\newline
  Théorème de la limite monotone. Si $f$ est monotone sur $I$ et si $f$ n'est pas continue en $a$ alors $f(I)$ n'est pas un intervalle.
  }
  \colchunk{}
  \colplacechunks
\end{parcolumns}

\subsubsubsection{b) Continuité}
\begin{parcolumns}[rulebetween,distance=\parcoldist]{2}
  \colchunk{Continuité, prolongement par continuité en un point.\newline
  Continuité à gauche, à droite.\newline
  Caractérisation séquentielle de la continuité en un point.\newline
  Opérations sur les fonctions continues en un point: combinaison linéaire, produit, quotient, composition.\newline
  Continuité sur un intervalle.}
  \colchunk{}
  \colplacechunks    
\end{parcolumns}

\subsubsubsection{f) Fonctions complexes}
\begin{parcolumns}[rulebetween,distance=\parcoldist]{2}
  \colchunk{Brève extension des définitions et résultats précédents.}
  \colchunk{Caractérisation de la limite et de la continuité à l'aide des parties réelle et imaginaire.}
  \colplacechunks
\end{parcolumns}
