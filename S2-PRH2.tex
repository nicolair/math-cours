\subsection{Espaces préhilbertiens réels (fin)}

\subsubsubsection{f) Hyperplans affines d'un espace euclidien}
\begin{parcolumns}[rulebetween,distance=2.5cm]{2}
  \colchunk{Vecteur normal à un hyperplan affine d'un espace euclidien. Si l'espace est orienté, orientation d'un hyperplan par un vecteur normal.}
  \colchunk{Lignes de niveau de $M \mapsto \overrightarrow{AM} \cdot \vec{n}$.}
  \colplacechunks

  \colchunk{\'Equations d'un hyperplan affine dans un repère orthonormal.}
  \colchunk{Cas particuliers de $\R^2$ et $\R^3$.}
  \colplacechunks
  
  \colchunk{Distance à un hyperplan affine défini par un point $A$ et un vecteur normal unitaire $\vec{n}$ : $\quad \big| \overrightarrow{AM} \cdot \vec{n} \big|$.}
  \colchunk{Cas particuliers de $\R^2$ et $\R^3$.}
  \colplacechunks
\end{parcolumns}

\subsubsubsection{g) Isométries vectorielles d'un espace euclidien}
\begin{parcolumns}[rulebetween,distance=2.5cm]{2}
  \colchunk{Isométrie vectorielle (ou automorphisme orthogonal) : définition par la linéarité et la conservation des normes, caractérisation par la conservation du produit scalaire, caractérisation par l'image d'une base orthonormale.}
  \colchunk{}
  \colplacechunks
  
  \colchunk{Symétrie orthogonale, réflexion.}
  \colchunk{}
  \colplacechunks


  \colchunk{Groupe orthogonal.}
  \colchunk{Notation $\mbox{O} (E)$.}
  \colplacechunks
\end{parcolumns}

\subsubsubsection{h) Matrices orthogonales}
\begin{parcolumns}[rulebetween,distance=2.5cm]{2}
  \colchunk{Matrice orthogonale : définition ${}^t\!\! A A = I_n$, caractérisation par le caractère orthonormal de la famille des colonnes, des lignes.}
  \colchunk{}
  \colplacechunks

  \colchunk{Groupe orthogonal.}
  \colchunk{Notations $\mbox{O}_n (\R)$, $\mbox{O} (n)$.}
  \colplacechunks
  
  \colchunk{Lien entre les notions de base orthonormale, isométrie et matrice orthogonale.}
  \colchunk{}
  \colplacechunks

  \colchunk{Déterminant d'une matrice orthogonale, d'une isométrie. Matrice orthogonale positive, négative ; isométrie positive, négative.}
  \colchunk{}
  \colplacechunks
  
  \colchunk{Produit mixte dans un espace euclidien orienté.}
  \colchunk{Notation $\big[ x_1 , \ldots , x_n \big]$.\newline
  Interprétation géométrique en termes de volume orienté, effet d'une application linéaire.}
  \colplacechunks  
  \colchunk{Groupe spécial orthogonal.}
  \colchunk{Notations $\mbox{SO} (E)$, $\mbox{SO}_n (\R)$, $\mbox{SO} (n)$.}
  \colplacechunks
\end{parcolumns}

\subsubsubsection{i) Isométries vectorielles en dimension 2}
\begin{parcolumns}[rulebetween,distance=2.5cm]{2}
  \colchunk{Description des matrices orthogonales et orthogonales positives de taille 2.}
  \colchunk{Lien entre les éléments de $\mbox{SO}_2 (\R)$ et les nombres complexes de module 1.}
  \colplacechunks

  \colchunk{Rotation vectorielle d'un plan euclidien orienté.}
  \colchunk{On introduira à cette occasion, sans soulever de difficulté sur la notion d'angle, la notion de mesure d'un angle orienté de vecteurs.

$\dbf$ SI : mécanique.}
  \colplacechunks
  
  \colchunk{Classification des isométries d'un plan euclidien orienté.}
  \colchunk{}
  \colplacechunks
\end{parcolumns}
