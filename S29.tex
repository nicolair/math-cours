%!  pour pdfLatex
\documentclass[a4paper]{article}
\usepackage[hmargin={1.5cm,1.5cm},vmargin={2.4cm,2.4cm},headheight=13.1pt]{geometry}

\usepackage[pdftex]{graphicx,color}
%\usepackage{hyperref}

\usepackage[utf8]{inputenc}
\usepackage[T1]{fontenc}
\usepackage{lmodern}
%\usepackage[frenchb]{babel}
\usepackage[french]{babel}

\usepackage{fancyhdr}
\pagestyle{fancy}

%\usepackage{floatflt}

\usepackage{parcolumns}
\setlength{\parindent}{0pt}
\usepackage{xcolor}

%pr{\'e}sentation des compteurs de section, ...
\makeatletter
%\renewcommand{\labelenumii}{\theenumii.}
\renewcommand{\thepart}{}
\renewcommand{\thesection}{}
\renewcommand{\thesubsection}{}
\renewcommand{\thesubsubsection}{}
\makeatother

\newcommand{\subsubsubsection}[1]{\bigskip \rule[5pt]{\linewidth}{2pt} \textbf{ \color{red}{#1} } \newline \rule{\linewidth}{.1pt}}
\newlength{\parcoldist}
\setlength{\parcoldist}{1cm}

\usepackage{maths}
\newcommand{\dbf}{\leftrightarrows}
% remplace les commandes suivantes 
%\usepackage{amsmath}
%\usepackage{amssymb}
%\usepackage{amsthm}
%\usepackage{stmaryrd}

%\newcommand{\N}{\mathbb{N}}
%\newcommand{\Z}{\mathbb{Z}}
%\newcommand{\C}{\mathbb{C}}
%\newcommand{\R}{\mathbb{R}}
%\newcommand{\K}{\mathbf{K}}
%\newcommand{\Q}{\mathbb{Q}}
%\newcommand{\F}{\mathbf{F}}
%\newcommand{\U}{\mathbb{U}}

%\newcommand{\card}{\mathop{\mathrm{Card}}}
%\newcommand{\Id}{\mathop{\mathrm{Id}}}
%\newcommand{\Ker}{\mathop{\mathrm{Ker}}}
%\newcommand{\Vect}{\mathop{\mathrm{Vect}}}
%\newcommand{\cotg}{\mathop{\mathrm{cotan}}}
%\newcommand{\sh}{\mathop{\mathrm{sh}}}
%\newcommand{\ch}{\mathop{\mathrm{ch}}}
%\newcommand{\argsh}{\mathop{\mathrm{argsh}}}
%\newcommand{\argch}{\mathop{\mathrm{argch}}}
%\newcommand{\tr}{\mathop{\mathrm{tr}}}
%\newcommand{\rg}{\mathop{\mathrm{rg}}}
%\newcommand{\rang}{\mathop{\mathrm{rg}}}
%\newcommand{\Mat}{\mathop{\mathrm{Mat}}}
%\renewcommand{\Re}{\mathop{\mathrm{Re}}}
%\renewcommand{\Im}{\mathop{\mathrm{Im}}}
%\renewcommand{\th}{\mathop{\mathrm{th}}}


%En tete et pied de page
\lhead{Programme colle math}
\chead{Semaine 29 du 02/06/20 au 06/06/20}
\rhead{MPSI B Hoche}

\lfoot{\tiny{Cette création est mise à disposition selon le Contrat\\ Paternité-Partage des Conditions Initiales à l'Identique 2.0 France\\ disponible en ligne http://creativecommons.org/licenses/by-sa/2.0/fr/
} }
\rfoot{\tiny{Rémy Nicolai \jobname}}


\begin{document}
Attention au lundi de Pentecôte.


\subsection{Espaces préhilbertiens réels (fin)}

\subsubsubsection{f) Hyperplans affines d'un espace euclidien}
\begin{parcolumns}[rulebetween,distance=2.5cm]{2}
  \colchunk{Vecteur normal à un hyperplan affine d'un espace euclidien. Si l'espace est orienté, orientation d'un hyperplan par un vecteur normal.}
  \colchunk{Lignes de niveau de $M \mapsto \overrightarrow{AM} \cdot \vec{n}$.}
  \colplacechunks

  \colchunk{\'Equations d'un hyperplan affine dans un repère orthonormal.}
  \colchunk{Cas particuliers de $\R^2$ et $\R^3$.}
  \colplacechunks
  
  \colchunk{Distance à un hyperplan affine défini par un point $A$ et un vecteur normal unitaire $\vec{n}$ : $\quad \big| \overrightarrow{AM} \cdot \vec{n} \big|$.}
  \colchunk{Cas particuliers de $\R^2$ et $\R^3$.}
  \colplacechunks
\end{parcolumns}

\subsubsubsection{g) Isométries vectorielles d'un espace euclidien}
\begin{parcolumns}[rulebetween,distance=2.5cm]{2}
  \colchunk{Isométrie vectorielle (ou automorphisme orthogonal) : définition par la linéarité et la conservation des normes, caractérisation par la conservation du produit scalaire, caractérisation par l'image d'une base orthonormale.}
  \colchunk{}
  \colplacechunks
  
  \colchunk{Symétrie orthogonale, réflexion.}
  \colchunk{}
  \colplacechunks


  \colchunk{Groupe orthogonal.}
  \colchunk{Notation $\mbox{O} (E)$.}
  \colplacechunks
\end{parcolumns}

\subsubsubsection{h) Matrices orthogonales}
\begin{parcolumns}[rulebetween,distance=2.5cm]{2}
  \colchunk{Matrice orthogonale : définition ${}^t\!\! A A = I_n$, caractérisation par le caractère orthonormal de la famille des colonnes, des lignes.}
  \colchunk{}
  \colplacechunks

  \colchunk{Groupe orthogonal.}
  \colchunk{Notations $\mbox{O}_n (\R)$, $\mbox{O} (n)$.}
  \colplacechunks
  
  \colchunk{Lien entre les notions de base orthonormale, isométrie et matrice orthogonale.}
  \colchunk{}
  \colplacechunks

  \colchunk{Déterminant d'une matrice orthogonale, d'une isométrie. Matrice orthogonale positive, négative ; isométrie positive, négative.}
  \colchunk{}
  \colplacechunks
  
  \colchunk{Produit mixte dans un espace euclidien orienté.}
  \colchunk{Notation $\big[ x_1 , \ldots , x_n \big]$.\newline
  Interprétation géométrique en termes de volume orienté, effet d'une application linéaire.}
  \colplacechunks  
  \colchunk{Groupe spécial orthogonal.}
  \colchunk{Notations $\mbox{SO} (E)$, $\mbox{SO}_n (\R)$, $\mbox{SO} (n)$.}
  \colplacechunks
\end{parcolumns}

\subsubsubsection{i) Isométries vectorielles en dimension 2}
\begin{parcolumns}[rulebetween,distance=2.5cm]{2}
  \colchunk{Description des matrices orthogonales et orthogonales positives de taille 2.}
  \colchunk{Lien entre les éléments de $\mbox{SO}_2 (\R)$ et les nombres complexes de module 1.}
  \colplacechunks

  \colchunk{Rotation vectorielle d'un plan euclidien orienté.}
  \colchunk{On introduira à cette occasion, sans soulever de difficulté sur la notion d'angle, la notion de mesure d'un angle orienté de vecteurs.

$\dbf$ SI : mécanique.}
  \colplacechunks
  
  \colchunk{Classification des isométries d'un plan euclidien orienté.}
  \colchunk{}
  \colplacechunks
\end{parcolumns}


\bigskip
\begin{center}
 \textbf{Questions de cours}
\end{center}
\'Etude des matrices orthogonales 2x2. Angle de la composée de deux réflexions en fonction de l'angle orienté entre les deux droites. Pas vraiment de résultats substantiels dans le programme de cette semaine pour être posé en question de cours. Je suis sorti du programme en définissant le produit vectoriel et en prouvant ses principales propriétés. J'ai aussi classifié les isométries en dimension 3 et donné une méthode pratique pour déterminer les éléments géométriques d'une matrice 3x3 orthogonale.
\begin{center}
 \textbf{Prochain programme}
\end{center}
Séries numériques.
Révision sur toute l'année.
\end{document}
