%!  pour pdfLatex
\documentclass[a4paper]{article}
\usepackage[hmargin={1.5cm,1.5cm},vmargin={2.4cm,2.4cm},headheight=13.1pt]{geometry}

\usepackage[pdftex]{graphicx,color}
%\usepackage{hyperref}

\usepackage[utf8]{inputenc}
\usepackage[T1]{fontenc}
\usepackage{lmodern}
%\usepackage[frenchb]{babel}
\usepackage[french]{babel}

\usepackage{fancyhdr}
\pagestyle{fancy}

%\usepackage{floatflt}

\usepackage{parcolumns}
\setlength{\parindent}{0pt}
\usepackage{xcolor}

%pr{\'e}sentation des compteurs de section, ...
\makeatletter
%\renewcommand{\labelenumii}{\theenumii.}
\renewcommand{\thepart}{}
\renewcommand{\thesection}{}
\renewcommand{\thesubsection}{}
\renewcommand{\thesubsubsection}{}
\makeatother

\newcommand{\subsubsubsection}[1]{\bigskip \rule[5pt]{\linewidth}{2pt} \textbf{ \color{red}{#1} } \newline \rule{\linewidth}{.1pt}}
\newlength{\parcoldist}
\setlength{\parcoldist}{1cm}

\usepackage{maths}
\newcommand{\dbf}{\leftrightarrows}
% remplace les commandes suivantes 
%\usepackage{amsmath}
%\usepackage{amssymb}
%\usepackage{amsthm}
%\usepackage{stmaryrd}

%\newcommand{\N}{\mathbb{N}}
%\newcommand{\Z}{\mathbb{Z}}
%\newcommand{\C}{\mathbb{C}}
%\newcommand{\R}{\mathbb{R}}
%\newcommand{\K}{\mathbf{K}}
%\newcommand{\Q}{\mathbb{Q}}
%\newcommand{\F}{\mathbf{F}}
%\newcommand{\U}{\mathbb{U}}

%\newcommand{\card}{\mathop{\mathrm{Card}}}
%\newcommand{\Id}{\mathop{\mathrm{Id}}}
%\newcommand{\Ker}{\mathop{\mathrm{Ker}}}
%\newcommand{\Vect}{\mathop{\mathrm{Vect}}}
%\newcommand{\cotg}{\mathop{\mathrm{cotan}}}
%\newcommand{\sh}{\mathop{\mathrm{sh}}}
%\newcommand{\ch}{\mathop{\mathrm{ch}}}
%\newcommand{\argsh}{\mathop{\mathrm{argsh}}}
%\newcommand{\argch}{\mathop{\mathrm{argch}}}
%\newcommand{\tr}{\mathop{\mathrm{tr}}}
%\newcommand{\rg}{\mathop{\mathrm{rg}}}
%\newcommand{\rang}{\mathop{\mathrm{rg}}}
%\newcommand{\Mat}{\mathop{\mathrm{Mat}}}
%\renewcommand{\Re}{\mathop{\mathrm{Re}}}
%\renewcommand{\Im}{\mathop{\mathrm{Im}}}
%\renewcommand{\th}{\mathop{\mathrm{th}}}


%En tete et pied de page
\lhead{Programme colle math}
\chead{Semaine 1 du 16/09/19 au 21/09/19}
\rhead{MPSI B Hoche}

\lfoot{\tiny{Cette création est mise à disposition selon le Contrat\\ Paternité-Partage des Conditions Initiales à l'Identique 2.0 France\\ disponible en ligne http://creativecommons.org/licenses/by-sa/2.0/fr/
} }
\rfoot{\tiny{Rémy Nicolai \jobname}}


\begin{document}
\subsection{Nombres complexes et trigonométrie (début)}
L’objectif de ce chapitre est de consolider et d’approfondir les notions sur les nombres complexes acquises en classe de 
Terminale. Le programme combine les aspects suivants :
\begin{itemize}
 \item l’étude algébrique du corps $\C$, équations algébriques (équations du second degré, racines $n$-ièmes d'un nombre complexe) ;  
 \item l’interprétation géométrique des nombres complexes et l’utilisation des nombres complexes en géométrie plane;
 \item l’exponentielle complexe et ses applications à la trigonométrie.
\end{itemize}
Il est recommandé d’illustrer le cours par de nombreuses figures.

\subsubsubsection{a) Nombres complexes}
\begin{parcolumns}[rulebetween,distance=2.5cm]{2}
  \colchunk{Parties réelle et imaginaire.}
  \colchunk{La construction de $\C$ n'est pas exigible.}
  \colplacechunks

  \colchunk{Opérations sur les nombres complexes.}
  \colchunk{}
  \colplacechunks

  \colchunk{Conjugaison, compatibilité avec les opérations.}
  \colchunk{}
  \colplacechunks

  \colchunk{Point du plan associé à un nombre complexe, affixe d'un point, affixe d'un vecteur.}
  \colchunk{On identifie $\C$ au plan usuel muni d'un repère orthonormé direct.}
  \colplacechunks
\end{parcolumns}

\subsubsubsection{b) Module}
\begin{parcolumns}[rulebetween,distance=2.5cm]{2}
  \colchunk{Module}
  \colchunk{Interprétation géométrique de $|z-z'|$, cercles et disques.}
  \colplacechunks

  \colchunk{Relation $|z|^2=z\bar{z}$, module d'un produit, d'un quotient.}
  \colchunk{}
  \colplacechunks

  \colchunk{Inégalité triangulaire, cas d'égalité.}
  \colchunk{}
  \colplacechunks
\end{parcolumns}


\subsubsubsection{e) \'Equations du second degré}
\begin{parcolumns}[rulebetween,distance=2.5cm]{2}
  \colchunk{Résolution des équations du second degré dans $\C$.}
  \colchunk{Calcul des racines carrées d'un nombre donné sous forme algébrique.}
  \colplacechunks

  \colchunk{Somme et produit des racines.}
  \colchunk{}
  \colplacechunks
\end{parcolumns}

\subsection{Calculs algébriques (début)}
Ce chapitre a pour but de présenter quelques notations et techniques fondamentales de calcul algébrique.

\subsubsubsection{a) Sommes et produits}
\begin{parcolumns}[rulebetween,distance=\parcoldist]{2}
  \colchunk{Somme et produit d'une famille finie de nombres complexes.}
  \colchunk{Notations $\sum_{i\in I}a_i$, $\sum_{i=1}^na_i$, $\prod_{i\in I}a_i$, $\prod_{i=1}^na_i$.\newline
  Sommes et produits télescopiques, exemples de changements d'indice et de regroupements de termes.}  
  \colplacechunks
  
  \colchunk{Expressions simplifiées de $\sum_{k=1}^nk$, $\sum_{k=1}^nk^2$, $\sum_{k=1}^nx^k$.}
  \colchunk{}
  \colplacechunks
  
  \colchunk{Factorisation de $a^n-b^n$ pour $n\in \N^*$.}
  \colchunk{}
  \colplacechunks
  
  \colchunk{Sommes doubles. Produit de deux sommes finies, sommes triangulaires.}
  \colchunk{}
  \colplacechunks

\end{parcolumns}


\bigskip
\begin{center}
 \textbf{Questions de cours}
\end{center}
Racines carrées d'un complexe non nul: il en existe exactement deux, calcul pratique. Factorisation canonique et équation du second degré. Inégalité triangulaire avec cas d'égalité. Calculs de sommes de produits d'entiers consécutifs.


\begin{center}
 \textbf{Prochain programme}
\end{center}
Calculs algébriques et trigonométriques
\end{document}
