\subsubsection{C - Primitives et équations différentielles linéaires}
\subsubsubsection{a) Calcul de primitives}
\begin{parcolumns}[rulebetween,distance=\parcoldist]{2}
  \colchunk{Primitives d'une fonction définie sur un intervalle à valeurs complexes}
  \colchunk{Description de l'ensemble des primitives d'une fonction sur un intervalle connaissant l'une d'entre elles.\newline
  Les étudiants doivent savoir utiliser les primitives de $x\mapsto e^{\lambda x}$ pour calculer celles de $x\mapsto e^{a x}\cos(bx)$ et de $x\mapsto e^{a x}\sin(bx)$.\newline
  $\leftrightarrows$ PC et SI: cinématique.}
  \colplacechunks

  \colchunk{Primitives des fonctions puissances, trigonométriques et hyperboliques, exponentielle, logarithme,
  \begin{displaymath}
   x\mapsto \frac{1}{1+x^2}, x\mapsto \frac{1}{\sqrt{1-x^2}}
  \end{displaymath}}
  \colchunk{Les étudiants doivent savoir calculer les primitives des fonctions du type
  \begin{displaymath}
   x\mapsto \frac{1}{ax^2+bx+c}
  \end{displaymath}
et reconnaître les dérivées de fonctions composées.}
  \colplacechunks

  \colchunk{Dérivée de $x\mapsto \int_{x_0}^xf(t)dt$ où $f$ est continue.}
  \colchunk{Résultat admis à ce stade.}
  \colplacechunks

  \colchunk{Toute fonction continue admet des primitives.}
  \colchunk{}
  \colplacechunks

  \colchunk{Calcul d'une intégrale au moyen d'une primitive.}
  \colchunk{}
  \colplacechunks

  \colchunk{Intégration par parties pour des fonctions de classe $\mathcal{C}^1$. Changement de variable: si $\varphi$ est de classe $\mathcal{C}^1$ sur $I$ et si $f$ est continue sur $\varphi(I)$, alors, pour tous $a$ et $b$ dans $I$
  \begin{displaymath}
   \int_{\varphi(a)}^{\varphi(b)}f(x)dx = \int_a^bf(\varphi(t))\varphi'(t)dt
  \end{displaymath}  }
  \colchunk{On définit à cette occasion la classe $\mathcal{C}^1$. Application au calcul de primitives.}
  \colplacechunks
  \end{parcolumns}

\subsubsubsection{b) \'Equations différentielles linéaires du premier ordre}
\begin{parcolumns}[rulebetween,distance=\parcoldist]{2}
  \colchunk{Notion d'équation différentielle linéaire du premier ordre:
  \begin{displaymath}
   y'+a(x)y = b(x)
  \end{displaymath}
où $a$ et $b$ sont des fonctions continues définies sur un intervalle $I$ de $\R$ à valeurs réelles ou complexes.}
  \colchunk{\'Equation homogène associée. \newline
  Cas particulier où la fonction $a$ est constante.}
  \colplacechunks

  \colchunk{Résolution d'une équation homogène.}
  \colchunk{}
  \colplacechunks

  \colchunk{Forme des solutions: somme d'une solution particulière et de la solution générale de l'équation homogène.}
  \colchunk{$\leftrightarrows$ PC: régime libre, régime forcé; régime transitoire, régime établi.}
  \colplacechunks

  \colchunk{Principe de superposition.}
  \colchunk{}
  \colplacechunks

  \colchunk{Méthode de la variation de la constante.}
  \colchunk{}
  \colplacechunks

  \colchunk{Existence et unicité de la solution d'un problème de Cauchy.}
  \colchunk{$\leftrightarrows$ PC et SI: modélisation de circuits électriques RC, RL ou de systèmes mécaniques linéaires.}
  \colplacechunks
\end{parcolumns}
