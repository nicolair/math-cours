%!  pour pdfLatex
\documentclass[a4paper]{article}
\usepackage[hmargin={1.5cm,1.5cm},vmargin={2.4cm,2.4cm},headheight=13.1pt]{geometry}

\usepackage[pdftex]{graphicx,color}
%\usepackage{hyperref}

\usepackage[utf8]{inputenc}
\usepackage[T1]{fontenc}
\usepackage{lmodern}
%\usepackage[frenchb]{babel}
\usepackage[french]{babel}

\usepackage{fancyhdr}
\pagestyle{fancy}

%\usepackage{floatflt}

\usepackage{parcolumns}
\setlength{\parindent}{0pt}
\usepackage{xcolor}

%pr{\'e}sentation des compteurs de section, ...
\makeatletter
%\renewcommand{\labelenumii}{\theenumii.}
\renewcommand{\thepart}{}
\renewcommand{\thesection}{}
\renewcommand{\thesubsection}{}
\renewcommand{\thesubsubsection}{}
\makeatother

\newcommand{\subsubsubsection}[1]{\bigskip \rule[5pt]{\linewidth}{2pt} \textbf{ \color{red}{#1} } \newline \rule{\linewidth}{.1pt}}
\newlength{\parcoldist}
\setlength{\parcoldist}{1cm}

\usepackage{maths}
\newcommand{\dbf}{\leftrightarrows}
% remplace les commandes suivantes 
%\usepackage{amsmath}
%\usepackage{amssymb}
%\usepackage{amsthm}
%\usepackage{stmaryrd}

%\newcommand{\N}{\mathbb{N}}
%\newcommand{\Z}{\mathbb{Z}}
%\newcommand{\C}{\mathbb{C}}
%\newcommand{\R}{\mathbb{R}}
%\newcommand{\K}{\mathbf{K}}
%\newcommand{\Q}{\mathbb{Q}}
%\newcommand{\F}{\mathbf{F}}
%\newcommand{\U}{\mathbb{U}}

%\newcommand{\card}{\mathop{\mathrm{Card}}}
%\newcommand{\Id}{\mathop{\mathrm{Id}}}
%\newcommand{\Ker}{\mathop{\mathrm{Ker}}}
%\newcommand{\Vect}{\mathop{\mathrm{Vect}}}
%\newcommand{\cotg}{\mathop{\mathrm{cotan}}}
%\newcommand{\sh}{\mathop{\mathrm{sh}}}
%\newcommand{\ch}{\mathop{\mathrm{ch}}}
%\newcommand{\argsh}{\mathop{\mathrm{argsh}}}
%\newcommand{\argch}{\mathop{\mathrm{argch}}}
%\newcommand{\tr}{\mathop{\mathrm{tr}}}
%\newcommand{\rg}{\mathop{\mathrm{rg}}}
%\newcommand{\rang}{\mathop{\mathrm{rg}}}
%\newcommand{\Mat}{\mathop{\mathrm{Mat}}}
%\renewcommand{\Re}{\mathop{\mathrm{Re}}}
%\renewcommand{\Im}{\mathop{\mathrm{Im}}}
%\renewcommand{\th}{\mathop{\mathrm{th}}}


%En tete et pied de page
\lhead{Programme colle math}
\chead{Semaine 4 du 07/10/19 au 12/10/19}
\rhead{MPSI B Hoche}

\lfoot{\tiny{Cette création est mise à disposition selon le Contrat\\ Paternité-Partage des Conditions Initiales à l'Identique 2.0 France\\ disponible en ligne http://creativecommons.org/licenses/by-sa/2.0/fr/
} }
\rfoot{\tiny{Rémy Nicolai \jobname}}


\begin{document}
\subsection{Techniques fondamentales de calcul en analyse}
\subsubsection{A - Inégalités dans $\R$}
\begin{parcolumns}[rulebetween,distance=\parcoldist]{2}
  \colchunk{Relation d'ordre sur $\R$. Compatibilité avec les opérations}
  \colchunk{Exemple de majoration et de minoration de sommes, de produit et de quotient.}
  \colplacechunks
  
  \colchunk{Parties positive et négative d'un réel. Valeur absolue. Inégalité triangulaire.}
  \colchunk{Notations $x^+$, $x^-$.}
  \colplacechunks
  
  \colchunk{Intervalles de $\R$.}
  \colchunk{Interprétation sur la droite réelle d'inégalités du type $|x-a|\leq b$.}
  \colplacechunks

  \colchunk{Parties majorées, minorées, bornées.\newline Majorant, minorant, maximum (ou plus grand élément), minimum (ou plus petit élémént).}
  \colchunk{Le \og plus simple des encadrements\fg (terminologie locale) :
\begin{displaymath}
n \min(x_1, \cdots, x_n) \leq \sum_{i=1}^n x_i \leq n \max(x_1, \cdots, x_n)
\end{displaymath}
Les notions de borne supérieure ou inférieure ne seront introduites que lors du cours de présentation axiomatique de $\R$.}

  \colplacechunks
  
  \colchunk{Exemples.}
  \colchunk{Inégalité de Cauchy-Schwarz. Preuve de la divergence de la série harmonique avec des puissances de $2$ et le plus simple des encadrements.}
    
\end{parcolumns}


\subsubsection{B - Fonction de la variable réelle à valeurs réelles ou complexes}
\subsubsubsection{a) Généralités sur les fonctions}
\begin{parcolumns}[rulebetween,distance=\parcoldist]{2}
  \colchunk{Ensemble de définition}
  \colchunk{}
  \colplacechunks
    
  \colchunk{Représentation graphique d'une fonction $f$ à valeurs réelles.}
  \colchunk{Graphes des fonctions $x\mapsto f(x)+a$, $x\mapsto f(x+a)$, $x\mapsto f(a-x)$, $x\mapsto f(ax)$, $x\mapsto af(x)$.\newline
  Résolution graphique d'équations et d'inéquations du type $f(x)=\lambda$ et $f(x)\geq \lambda$.}
  \colplacechunks
    
  \colchunk{Parité, imparité, périodicité.}
  \colchunk{Interprétation géométrique de ces propriétés.}
  \colplacechunks
    
  \colchunk{Somme, produit, composée.}
  \colchunk{}
  \colplacechunks
    
  \colchunk{Monotonie (large et stricte).}
  \colchunk{}
  \colplacechunks
    
  \colchunk{Fonctions majorées, minorées, bornées.}
  \colchunk{Traduction géométrique de ces propriétés. \newline
  Une fonction est bornée si et seulement si $|f|$ est majorée.}
  \colplacechunks    
\end{parcolumns}

\subsubsubsection{b) Dérivation}
\begin{parcolumns}[rulebetween,distance=\parcoldist]{2}
  \colchunk{\'Equation de la tangente en un point.}
  \colchunk{}
  \colplacechunks

  \colchunk{Dérivée d'une combinaison linéaire, d'un produit, d'un quotient, d'une composée.}
  \colchunk{Ces résultats sont admis à ce stade.\newline
  $\rightleftarrows$ SI: étude cinématique.\newline
  $\rightleftarrows$ PC: exemples de calculs de dérivées partielles.\newline
  \`A ce stade, toute théorie sur les fonctions de plusieurs variables est hors programme.}
  \colplacechunks

  \colchunk{Caractérisation des fonctions dérivables constantes, monotones, strictement monotones sur un intervalle.}
  \colchunk{Résultat admis à ce stade. Les étudiants doivent savoir introduire des fonctions pour établir des inégalités.}
  \colplacechunks

  \colchunk{Tableau de variation.}
  \colchunk{}
  \colplacechunks

  \colchunk{Graphe d'une réciproque.}
  \colchunk{}
  \colplacechunks

  \colchunk{Dérivée d'une réciproque.}
  \colchunk{Interprétation géométrique de la dérivabilité et du calcul de la dérivée d'une bijection réciproque.}
  \colplacechunks

  \colchunk{Dérivées d'ordre supérieur.}
  \colchunk{}
  \colplacechunks
  \end{parcolumns}
  
\subsubsubsection{c) \'Etude d'une fonction}
\begin{parcolumns}[rulebetween,distance=\parcoldist]{2}
  \colchunk{Détermination des symétries et des périodicités afin de réduire le domaine d'étude, tableau de variations, asymptotes verticales et horizontales, tracé du graphe.}
  \colchunk{Application à la recherche d'extrémums et à l'obtention d'inégalités.}
  \colplacechunks
  \end{parcolumns}
  
\subsubsubsection{d) Fonctions usuelles}
\begin{parcolumns}[rulebetween,distance=\parcoldist]{2}
  \colchunk{Fonctions exponentielle, logarithme népérien, puissances.}
  \colchunk{Dérivée, variation et graphe.\newline
  Les fonctions puissances sont définies sur $\R_+^*$ et prolongées en $0$ le cas échéant. Seules les fonctions puissances entières sont en outre défines sur $\R_-^*$.\newline
  $\leftrightarrows$ SI: logarithme décimal pour la représentation des diagrammes de Bode.}
  \colplacechunks

  \colchunk{Relations $(xy)^\alpha=x^\alpha y^\alpha$, $x^{\alpha+\beta}=x^\alpha x^\beta$, $(x^\alpha)^\beta=x^{\alpha \beta}$.\newline
  Croissances comparées des fonctions logarithme, puissances et exponentielle.}
  \colchunk{}
  \colplacechunks

  \colchunk{Fonction sinus, cosinus, tangente.}
  \colchunk{$\leftrightarrows$ PC et SI}
  \colplacechunks

  \colchunk{Fonctions circulaires réciproques.}
  \colchunk{Notation $\arcsin$, $\arccos$, $\arctan$.}
  \colplacechunks

  \colchunk{Fonctions hyperboliques.}
  \colchunk{Notations $\sh$, $\ch$, $\th$. \newline
  Seule relation de trigonométrie hyperbolique exigible: $\ch^2x - \sh^2 x =1$.\newline
  Les fonctions hyperboliques réciproques sont hors programmes.}
  \colplacechunks
\end{parcolumns}



\bigskip
\begin{center}
 \textbf{Questions de cours}
\end{center}
Inégalité de Cauchy-Schwarz. Preuve de la divergence de la série harmonique avec des puissances de $2$ et le plus simple des encadrements. \'Enoncé du théorème de dérivabilité d'une bijection réciproque. Fonctions réciproques en trigonométrie circulaire: définition, dérivées. Graphe de $\arcsin \circ \sin$. 

\begin{center}
 \textbf{Prochain programme}
\end{center}

Primitives et équations différentielles linéaires.
\end{document}
