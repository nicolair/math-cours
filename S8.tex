%!  pour pdfLatex
\documentclass[a4paper]{article}
\usepackage[hmargin={1.5cm,1.5cm},vmargin={2.4cm,2.4cm},headheight=13.1pt]{geometry}

\usepackage[pdftex]{graphicx,color}
%\usepackage{hyperref}

\usepackage[utf8]{inputenc}
\usepackage[T1]{fontenc}
\usepackage{lmodern}
%\usepackage[frenchb]{babel}
\usepackage[french]{babel}

\usepackage{fancyhdr}
\pagestyle{fancy}

%\usepackage{floatflt}

\usepackage{parcolumns}
\setlength{\parindent}{0pt}
\usepackage{xcolor}

%pr{\'e}sentation des compteurs de section, ...
\makeatletter
%\renewcommand{\labelenumii}{\theenumii.}
\renewcommand{\thepart}{}
\renewcommand{\thesection}{}
\renewcommand{\thesubsection}{}
\renewcommand{\thesubsubsection}{}
\makeatother

\newcommand{\subsubsubsection}[1]{\bigskip \rule[5pt]{\linewidth}{2pt} \textbf{ \color{red}{#1} } \newline \rule{\linewidth}{.1pt}}
\newlength{\parcoldist}
\setlength{\parcoldist}{1cm}

\usepackage{maths}
\newcommand{\dbf}{\leftrightarrows}
% remplace les commandes suivantes 
%\usepackage{amsmath}
%\usepackage{amssymb}
%\usepackage{amsthm}
%\usepackage{stmaryrd}

%\newcommand{\N}{\mathbb{N}}
%\newcommand{\Z}{\mathbb{Z}}
%\newcommand{\C}{\mathbb{C}}
%\newcommand{\R}{\mathbb{R}}
%\newcommand{\K}{\mathbf{K}}
%\newcommand{\Q}{\mathbb{Q}}
%\newcommand{\F}{\mathbf{F}}
%\newcommand{\U}{\mathbb{U}}

%\newcommand{\card}{\mathop{\mathrm{Card}}}
%\newcommand{\Id}{\mathop{\mathrm{Id}}}
%\newcommand{\Ker}{\mathop{\mathrm{Ker}}}
%\newcommand{\Vect}{\mathop{\mathrm{Vect}}}
%\newcommand{\cotg}{\mathop{\mathrm{cotan}}}
%\newcommand{\sh}{\mathop{\mathrm{sh}}}
%\newcommand{\ch}{\mathop{\mathrm{ch}}}
%\newcommand{\argsh}{\mathop{\mathrm{argsh}}}
%\newcommand{\argch}{\mathop{\mathrm{argch}}}
%\newcommand{\tr}{\mathop{\mathrm{tr}}}
%\newcommand{\rg}{\mathop{\mathrm{rg}}}
%\newcommand{\rang}{\mathop{\mathrm{rg}}}
%\newcommand{\Mat}{\mathop{\mathrm{Mat}}}
%\renewcommand{\Re}{\mathop{\mathrm{Re}}}
%\renewcommand{\Im}{\mathop{\mathrm{Im}}}
%\renewcommand{\th}{\mathop{\mathrm{th}}}


%En tete et pied de page
\lhead{Programme colle math}
\chead{Semaine 8 du 18/11/19 au 23/11/19}
\rhead{MPSI B Hoche}

\lfoot{\tiny{Cette création est mise à disposition selon le Contrat\\ Paternité-Partage des Conditions Initiales à l'Identique 2.0 France\\ disponible en ligne http://creativecommons.org/licenses/by-sa/2.0/fr/
} }
\rfoot{\tiny{Rémy Nicolai \jobname}}


\begin{document}

\subsection{Nombres réels et suites numériques (fin)}

\subsubsubsection{e) Suites monotones}
\begin{parcolumns}[rulebetween,distance=\parcoldist]{2}
  \colchunk{Théorème de la limite monotone: toute suite monotone possède une limite.}
  \colchunk{Toute suite croissante majorée converge, toute suite croissante non majorée tend vers $+\infty$.}
  \colplacechunks

  \colchunk{Théorème des suites adjacentes.}
  \colchunk{}
  \colplacechunks
\end{parcolumns}

\subsubsubsection{f) Suites extraites}
\begin{parcolumns}[rulebetween,distance=\parcoldist]{2}
  
  \colchunk{Suite extraite}
  \colchunk{}
  \colplacechunks

  \colchunk{Si une suite possède une limite, toutes ses suites extraites possèdent la même limite.}
  \colchunk{Utiisation pour montrer la divergence d'une suite. Si $(u_{2n})$ et $(u_{2n+1})$ tendent vers $l$ alors $(u_n)$ tend vers $l$.}
  \colplacechunks
  
  \colchunk{Théorème de Bolzano-Weierstrass.}
  \colchunk{Les étudiants doivent connaître le principe de démonstration par dichotomie, mais la formalisation précise n'est pas exigible.\newline
  La notion de valeur d'adhérence est hors programme.}
  \colplacechunks
\end{parcolumns}

\subsubsubsection{g) Traduction séquentielle de certaines propriétés}
\begin{parcolumns}[rulebetween,distance=\parcoldist]{2}
  
  \colchunk{Partie dense de $\R$.}
  \colchunk{Une partie est dense dans $\R$ si elle rencontre tout intervalle ouvert non vide.\newline
  Densité de l'ensemble des décimaux, des rationnels, des irrationnels.}
  \colplacechunks

  \colchunk{Caractérisation séquentielle de la densité.}
  \colchunk{}
  \colplacechunks

  \colchunk{Si $X$ est une partie non vide majorée (resp. non majorée) de $\R$, il existe une suite d'éléments de $X$ dont la limite est $\sup X$ (resp. $+\infty$).}
  \colchunk{}
  \colplacechunks

\end{parcolumns}

\subsubsubsection{h) Suites complexes}
\begin{parcolumns}[rulebetween,distance=\parcoldist]{2}
  
  \colchunk{Brève extension des définitions et résultats précédents. Par définition, une suite complexe $\left( z_n \right)_{n \in \N}$ converge vers un nombre complexe $z$ si et seulement si la suite réelle $\left( |z_n - z| \right)_{n \in \N}$ converge vers $0$. Opérations sur les suites convergentes.}
  \colchunk{Caractérisation de la limite en termes de parties réelle et imaginaire.}
  \colplacechunks

  \colchunk{Théorème de Bolzano-Weierstrass.}
  \colchunk{La démonstration n'est pas exigible par le programme de mpsi mais figure comme question de cours pour la classe.}
  \colplacechunks

\end{parcolumns}

\subsubsubsection{i) Suites particulières}
\begin{parcolumns}[rulebetween,distance=\parcoldist]{2}
  
  \colchunk{Suite arithmétique, géométrique.\newline Suite arithmético-géométrique. Suite récurrente linéaire homogène d'ordre 2 à coefficients constants.}
  \colchunk{Les étudiants doivent savoir déterminer une expression du terme général de ces suites.}
  \colplacechunks

  \colchunk{Exemples de suites définies par une relation de récurrence $u_{n+1}=f(u_n)$.}
  \colchunk{Seul résultat exigible: si $(u_n)_{n\in\N}$ converge vers $l$ et si $f$ est continue en $l$, alors $f(l)=l$.}
  \colplacechunks

\end{parcolumns}


\bigskip
\begin{center}
 \textbf{Questions de cours}
\end{center}
Démonstration du théorème sur la convergence des suites monotones.

Principe de démonstration du théorème de Bolzano-Weierstrass pour les suites réelles.

Démonstration de la propriété : \og Si $X$ est une partie non vide majorée (resp. non majorée) de $\R$, il existe une suite d'éléments de $X$ dont la limite est $\sup X$ (resp. $+\infty$).\fg

Preuve de la caractérisation de la convergence d'une suite complexe à l'aide des parties réelles et imaginaires. Démonstration du théorème de Bolzano-Weierstrass pour les suites complexes.

\textbf{méthodes}\newline
Savoir déterminer une expression du terme général des suites vérifiant une relation de récurrence des types particuliers figurant dans le cours.
\begin{center}
 \textbf{Prochain programme}
\end{center}
Limites d'une fonction. Continuité sur un intervalle.
\end{document}
