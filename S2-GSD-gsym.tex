\subsubsection{A - Groupe symétrique}
\begin{itshape}
Le groupe symétrique est introduit exclusivement en vue de l'étude des déterminants.
\end{itshape}

\subsubsubsection{a) Généralités}
\begin{parcolumns}[rulebetween,distance=2.5cm]{2}
  \colchunk{Groupe des permutations de l'ensemble $\big\{ 1 , \ldots , n \big\}$.}
  \colchunk{Notation $S_n$.}
  \colplacechunks

  \colchunk{Cycle, transposition.}
  \colchunk{Notation $(a_1 \ a_2 \ \ldots \ a_p)$.}
  \colplacechunks

  \colchunk{Décomposition d'une permutation en produit de cycles à supports disjoints : existence et unicité.}
  \colchunk{La démonstration n'est pas exigible, mais les étudiants doivent savoir décomposer une permutation.

  Commutativité de la décomposition.}
  \colplacechunks
\end{parcolumns}


\subsubsubsection{b) Signature d'une permutation}
\begin{parcolumns}[rulebetween,distance=2.5cm]{2}
  \colchunk{Tout élément de $S_n$ est un produit de transpositions.}
  \colchunk{}
  \colplacechunks

  \colchunk{Signature : il existe une et une seule application $\varepsilon$ de $S_n$ dans $\{ -1 , 1 \}$ telle que $\varepsilon (\tau) = -1$ pour toute transposition $\tau$ et $\varepsilon (\sigma \sigma') = \varepsilon (\sigma) \varepsilon (\sigma')$ pour toutes permutations $\sigma$ et $\sigma'$.}
  \colchunk{La démonstration n'est pas exigible.}
  \colplacechunks

\end{parcolumns}
