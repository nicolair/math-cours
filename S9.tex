%!  pour pdfLatex
\documentclass[a4paper]{article}
\usepackage[hmargin={1.5cm,1.5cm},vmargin={2.4cm,2.4cm},headheight=13.1pt]{geometry}

\usepackage[pdftex]{graphicx,color}
%\usepackage{hyperref}

\usepackage[utf8]{inputenc}
\usepackage[T1]{fontenc}
\usepackage{lmodern}
%\usepackage[frenchb]{babel}
\usepackage[french]{babel}

\usepackage{fancyhdr}
\pagestyle{fancy}

%\usepackage{floatflt}

\usepackage{parcolumns}
\setlength{\parindent}{0pt}
\usepackage{xcolor}

%pr{\'e}sentation des compteurs de section, ...
\makeatletter
%\renewcommand{\labelenumii}{\theenumii.}
\renewcommand{\thepart}{}
\renewcommand{\thesection}{}
\renewcommand{\thesubsection}{}
\renewcommand{\thesubsubsection}{}
\makeatother

\newcommand{\subsubsubsection}[1]{\bigskip \rule[5pt]{\linewidth}{2pt} \textbf{ \color{red}{#1} } \newline \rule{\linewidth}{.1pt}}
\newlength{\parcoldist}
\setlength{\parcoldist}{1cm}

\usepackage{maths}
\newcommand{\dbf}{\leftrightarrows}
% remplace les commandes suivantes 
%\usepackage{amsmath}
%\usepackage{amssymb}
%\usepackage{amsthm}
%\usepackage{stmaryrd}

%\newcommand{\N}{\mathbb{N}}
%\newcommand{\Z}{\mathbb{Z}}
%\newcommand{\C}{\mathbb{C}}
%\newcommand{\R}{\mathbb{R}}
%\newcommand{\K}{\mathbf{K}}
%\newcommand{\Q}{\mathbb{Q}}
%\newcommand{\F}{\mathbf{F}}
%\newcommand{\U}{\mathbb{U}}

%\newcommand{\card}{\mathop{\mathrm{Card}}}
%\newcommand{\Id}{\mathop{\mathrm{Id}}}
%\newcommand{\Ker}{\mathop{\mathrm{Ker}}}
%\newcommand{\Vect}{\mathop{\mathrm{Vect}}}
%\newcommand{\cotg}{\mathop{\mathrm{cotan}}}
%\newcommand{\sh}{\mathop{\mathrm{sh}}}
%\newcommand{\ch}{\mathop{\mathrm{ch}}}
%\newcommand{\argsh}{\mathop{\mathrm{argsh}}}
%\newcommand{\argch}{\mathop{\mathrm{argch}}}
%\newcommand{\tr}{\mathop{\mathrm{tr}}}
%\newcommand{\rg}{\mathop{\mathrm{rg}}}
%\newcommand{\rang}{\mathop{\mathrm{rg}}}
%\newcommand{\Mat}{\mathop{\mathrm{Mat}}}
%\renewcommand{\Re}{\mathop{\mathrm{Re}}}
%\renewcommand{\Im}{\mathop{\mathrm{Im}}}
%\renewcommand{\th}{\mathop{\mathrm{th}}}


%En tete et pied de page
\lhead{Programme colle math}
\chead{Semaine 9 du 25/11/19 au 30/11/19}
\rhead{MPSI B Hoche}

\lfoot{\tiny{Cette création est mise à disposition selon le Contrat\\ Paternité-Partage des Conditions Initiales à l'Identique 2.0 France\\ disponible en ligne http://creativecommons.org/licenses/by-sa/2.0/fr/
} }
\rfoot{\tiny{Rémy Nicolai \jobname}}


\begin{document}


\subsection{Limites, continuité, dérivabilité}

\begin{itshape}
 Le paragraphe A - a) consiste largement en des adaptations au cas continu de notions déjà abordées pour les suites. Afin d'éviter les répétitions, le professeur a la liberté d'admettre certains résultats.\newline
 Pour la pratique du calcul des limites, on se borne à ce stade a des calculs très simples, en attendant de pouvoir disposer d'outils efficaces (développement limités).\end{itshape}

\subsubsection{A - Limites et continuité (études locales)}

\subsubsubsection{a) Limite d'une fonction en un point}
\begin{parcolumns}[rulebetween,distance=\parcoldist]{2}
  \colchunk{\'Etant donné un point $a \in\overline{\R}$ appartenant à $I$ ou extrémité de $I$, limite, finie ou infinie d'une fonction en $a$.}
  \colchunk{Notation $f(x) \underset{x\rightarrow a}{\longrightarrow}l$. \newline  Les propriétés sont énoncées avec des inégalités larges.}
  \colplacechunks
    
  \colchunk{Stabilité des inégalités larges par passage à la limite. Unicité de la limite.}
  \colchunk{Notations $\underset{x\rightarrow a}{\lim}f(x)$, $\underset{a}{\lim}f$.\newline }
  \colplacechunks
    
  \colchunk{Si $f$ est définie en $a$ et possède une limite en $a$, alors $\underset{x\rightarrow a}{\lim}f(x)=f(a)$.}
  \colchunk{}
  \colplacechunks
    
  \colchunk{Si $f$ possède une limite finie en $a$, $f$ est bornée au voisinage de $a$.}
  \colchunk{}
  \colplacechunks
    
  \colchunk{Limite à droite, limite à gauche.}
  \colchunk{Notations $\underset{\underset{x>a}{x\rightarrow a}}{\lim}f(x)=f(a)$ ou $\underset{x\rightarrow a^+}{\lim}f(x)$.}
  \colplacechunks
    
  \colchunk{Extension de la notion de limite en $a$ lorsque $f$ est définie sur $I\setminus\{a\}$.\newline
  Caractérisation séquentielle de la limite (finie ou infinie).\newline
  Opérations sur les limites: combinaison linéaire, produit, quotient, composition.\newline
  Théorèmes d'encadrement (limite finie), de minoration (limite $+\infty$), de majoration (limite $-\infty$).\newline
  Théorème de la limite monotone. Si $f$ est monotone sur $I$ et si $f$ n'est pas continue en $a$ alors $f(I)$ n'est pas un intervalle.
  }
  \colchunk{}
  \colplacechunks
\end{parcolumns}

\subsubsubsection{b) Continuité}
\begin{parcolumns}[rulebetween,distance=\parcoldist]{2}
  \colchunk{Continuité, prolongement par continuité en un point.\newline
  Continuité à gauche, à droite.\newline
  Caractérisation séquentielle de la continuité en un point.\newline
  Opérations sur les fonctions continues en un point: combinaison linéaire, produit, quotient, composition.\newline
  Continuité sur un intervalle.}
  \colchunk{}
  \colplacechunks    
\end{parcolumns}

\subsubsubsection{f) Fonctions complexes}
\begin{parcolumns}[rulebetween,distance=\parcoldist]{2}
  \colchunk{Brève extension des définitions et résultats précédents.}
  \colchunk{Caractérisation de la limite et de la continuité à l'aide des parties réelle et imaginaire.}
  \colplacechunks
\end{parcolumns}


\subsubsection{B - Continuité sur un intervalle}

\subsubsubsection{c) Image d'un intervalle par une fonction continue}
\begin{parcolumns}[rulebetween,distance=\parcoldist]{2}
  \colchunk{Théorème des valeurs intermédiaires.}
  \colchunk{Cas d'une fonction strictement monotone.\newline
  $\leftrightarrows$ I: application de l'algorithme de dichotomie à la recherche d'un zéro d'une fonction continue.}
  \colplacechunks
    
  \colchunk{L'image d'un intervalle par une fonction continue est un intervalle.}
  \colchunk{}
  \colplacechunks
\end{parcolumns}

\subsubsubsection{d) Image d'un segment par une fonction continue}
\begin{parcolumns}[rulebetween,distance=\parcoldist]{2}
  \colchunk{Toute fonction continue sur un segment est bornée atteint ses bornes.}
  \colchunk{La démonstration n'est pas exigible.}
  \colplacechunks
    
  \colchunk{L'image d'un segment par une fonction continue est un segment.}
  \colchunk{}
  \colplacechunks
\end{parcolumns}

\subsubsubsection{e) Continuité et injectivité}
\begin{parcolumns}[rulebetween,distance=\parcoldist]{2}
  \colchunk{Toute fonction continue injective sur un intervalle est strictement monotone.}
  \colchunk{La démonstration n'est pas exigible. Présenté en exercice}
  \colplacechunks
    
  \colchunk{La réciproque d'une fonction continue et strictement monotone sur un intervalle est continue.}
  \colchunk{}
  \colplacechunks
\end{parcolumns}


%\subsubsection{C - Dérivabilité}

\subsubsubsection{a) Nombre dérivé, fonction dérivée}
\begin{parcolumns}[rulebetween,distance=\parcoldist]{2}
  \colchunk{Dérivabilité en un point, nombre dérivé.}
  \colchunk{Développement limité à l'orde 1.\newline
  Interprétation géométrique. $\leftrightarrows$ SI: identification d'un modèle de comportement au voisinage d'un point de comportement.\newline
  $\leftrightarrows$ SI: représentation de la fonction sinus cardinal au voisinage de $0$.
  $\leftrightarrows$ I: méthode de Newton.}
  \colplacechunks
  
  \colchunk{La dérivabilité entraîne la continuité.}
  \colchunk{}
  \colplacechunks

  \colchunk{Dérivabilité à gauche, à droite.}
  \colchunk{}
  \colplacechunks
  
  \colchunk{Dérivabilité et dérivée sur un intervalle.}
  \colchunk{}
  \colplacechunks

  \colchunk{Opérations sur les fonctions dérivables et les dérivées: combinaison linéaire, produit, quotient, composition, réciproque.}
  \colchunk{Tangente au graphe d'une réciproque.}
  \colplacechunks
\end{parcolumns}

\subsubsubsection{b) Extremum local et point critique}
\begin{parcolumns}[rulebetween,distance=\parcoldist]{2}
  \colchunk{Extremum local}
  \colchunk{}
  \colplacechunks

  \colchunk{Condition nécessaire en un point intérieur.}
  \colchunk{Un point critique est un zéro de la dérivée.}
  \colplacechunks
\end{parcolumns}

\subsubsubsection{c) Théorème de Rolle et des accroissements finis.}
\begin{parcolumns}[rulebetween,distance=\parcoldist]{2}
  \colchunk{Théorème de Rolle.}
  \colchunk{Utilisation pour établir l'existence de zéros d'une fonction.}
  \colplacechunks

  \colchunk{\'Egalité des accroissements finis.}
  \colchunk{Interprétations géométriques et cinématiques.}
  \colplacechunks

  \colchunk{Inégalité des accroissements finis: si $f$ est dérivable et si $|f'|$ est majorée par $K$, alors $f$ est $K$-lipschitzienne.}
  \colchunk{La notion de fonction lipschitzienne est introduite à cette occasion.\newline
  Application à l'étude des des suites définies par une relation de récurrence $u_{n+1}=f(u_n)$.}
  \colplacechunks

  \colchunk{Caractérisation des fonctions dérivables constantes, monotones, strictement monotones sur un intervalle.}
  \colchunk{}
  \colplacechunks

  \colchunk{Théorème de la limite de la dérivée: si $f$ est continue sur $I$, dérivable sur $I\setminus\{a\}$ et si $\underset{\underset{x\neq a}{x\rightarrow a}}{\lim}f'(x)=l\in\overline{\R}$, alors $\underset{x\rightarrow a}{\lim}\frac{f(x)-f(a)}{x-a}=l$.}
  \colchunk{Interprétation géométrique.\newline
  Si $l\in\R$, alors $f$ est dérivable en $a$ et $f'$ continue en $a$.}
  \colplacechunks

\end{parcolumns}

\subsubsubsection{d) Fonctions de classe $\mathcal{C}^k$}
\begin{parcolumns}[rulebetween,distance=\parcoldist]{2}
  
  \colchunk{Pour $k\in\N \cup \{\infty\}$, fonction de classe $\mathcal{C}^k$.}
  \colchunk{}
  \colplacechunks

  \colchunk{Opérations sur les fonctions de classe $\mathcal{C}^k$: combinaison linéaire, produit (formule de Leibniz), quotient, composition, réciproque.}
  \colchunk{}
  \colplacechunks

  \colchunk{Théorème de classe $\mathcal{C}^k$ par prolongement: si $f$ est de classe $\mathcal{C}^k$ sur $I\setminus\{a\}$ et si $f^{(i)}$ possède une limite finie lorsque $x$ tend vers $a$ pour tout $i\in\{0,\cdots,k\}$, alors $f$ admet un prolongement de classe $\mathcal{C}^k$ sur $I$. }
  \colchunk{}
  \colplacechunks
\end{parcolumns}

\subsubsubsection{e) Fonctions complexes}
\begin{parcolumns}[rulebetween,distance=\parcoldist]{2}
  
  \colchunk{Brève extension des définitions et résultats précédents.}
  \colchunk{Caractérisation de la dérivabilité en termes de parties réelle et imaginaire.}
  \colplacechunks

  \colchunk{Inégalité des accroissements finis pour une fonction de classe $\mathcal{C}^1$.}
  \colchunk{Le résultat, admis à ce stade sera justifé dans le chapitre \og Intégration\fg.}
  \colplacechunks
\end{parcolumns}


\bigskip
\begin{center}
 \textbf{Questions de cours}
\end{center}
Les 9 définitions possibles pour les limites finies ou non.\newline 
\textbf{Démonstrations}\newline
Opérations sur les fonctions admettant une limite finie.\newline
Théorème de la limite monotone. Si $f$ est monotone sur $I$ et si $f$ n'est pas continue en $a$ alors $f(I)$ n'est pas un intervalle.\newline
Théorème des valeurs intermédiaires.\newline
Toute fonction continue sur un segment est bornée et atteint ses bornes.\newline
La réciproque d'une fonction continue et strictement monotone sur un intervalle est continue.

\begin{center}
 \textbf{Prochain programme}
\end{center}

Fonctions dérivables.
\end{document}
