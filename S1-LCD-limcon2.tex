\subsubsection{B - Continuité sur un intervalle}

\subsubsubsection{c) Image d'un intervalle par une fonction continue}
\begin{parcolumns}[rulebetween,distance=\parcoldist]{2}
  \colchunk{Théorème des valeurs intermédiaires.}
  \colchunk{Cas d'une fonction strictement monotone.\newline
  $\leftrightarrows$ I: application de l'algorithme de dichotomie à la recherche d'un zéro d'une fonction continue.}
  \colplacechunks
    
  \colchunk{L'image d'un intervalle par une fonction continue est un intervalle.}
  \colchunk{}
  \colplacechunks
\end{parcolumns}

\subsubsubsection{d) Image d'un segment par une fonction continue}
\begin{parcolumns}[rulebetween,distance=\parcoldist]{2}
  \colchunk{Toute fonction continue sur un segment est bornée atteint ses bornes.}
  \colchunk{La démonstration n'est pas exigible.}
  \colplacechunks
    
  \colchunk{L'image d'un segment par une fonction continue est un segment.}
  \colchunk{}
  \colplacechunks
\end{parcolumns}

\subsubsubsection{e) Continuité et injectivité}
\begin{parcolumns}[rulebetween,distance=\parcoldist]{2}
  \colchunk{Toute fonction continue injective sur un intervalle est strictement monotone.}
  \colchunk{La démonstration n'est pas exigible. Présenté en exercice}
  \colplacechunks
    
  \colchunk{La réciproque d'une fonction continue et strictement monotone sur un intervalle est continue.}
  \colchunk{}
  \colplacechunks
\end{parcolumns}
