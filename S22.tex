%!  pour pdfLatex
\documentclass[a4paper]{article}
\usepackage[hmargin={1.5cm,1.5cm},vmargin={2.4cm,2.4cm},headheight=13.1pt]{geometry}

\usepackage[pdftex]{graphicx,color}
%\usepackage{hyperref}

\usepackage[utf8]{inputenc}
\usepackage[T1]{fontenc}
\usepackage{lmodern}
%\usepackage[frenchb]{babel}
\usepackage[french]{babel}

\usepackage{fancyhdr}
\pagestyle{fancy}

%\usepackage{floatflt}

\usepackage{parcolumns}
\setlength{\parindent}{0pt}
\usepackage{xcolor}

%pr{\'e}sentation des compteurs de section, ...
\makeatletter
%\renewcommand{\labelenumii}{\theenumii.}
\renewcommand{\thepart}{}
\renewcommand{\thesection}{}
\renewcommand{\thesubsection}{}
\renewcommand{\thesubsubsection}{}
\makeatother

\newcommand{\subsubsubsection}[1]{\bigskip \rule[5pt]{\linewidth}{2pt} \textbf{ \color{red}{#1} } \newline \rule{\linewidth}{.1pt}}
\newlength{\parcoldist}
\setlength{\parcoldist}{1cm}

\usepackage{maths}
\newcommand{\dbf}{\leftrightarrows}
% remplace les commandes suivantes 
%\usepackage{amsmath}
%\usepackage{amssymb}
%\usepackage{amsthm}
%\usepackage{stmaryrd}

%\newcommand{\N}{\mathbb{N}}
%\newcommand{\Z}{\mathbb{Z}}
%\newcommand{\C}{\mathbb{C}}
%\newcommand{\R}{\mathbb{R}}
%\newcommand{\K}{\mathbf{K}}
%\newcommand{\Q}{\mathbb{Q}}
%\newcommand{\F}{\mathbf{F}}
%\newcommand{\U}{\mathbb{U}}

%\newcommand{\card}{\mathop{\mathrm{Card}}}
%\newcommand{\Id}{\mathop{\mathrm{Id}}}
%\newcommand{\Ker}{\mathop{\mathrm{Ker}}}
%\newcommand{\Vect}{\mathop{\mathrm{Vect}}}
%\newcommand{\cotg}{\mathop{\mathrm{cotan}}}
%\newcommand{\sh}{\mathop{\mathrm{sh}}}
%\newcommand{\ch}{\mathop{\mathrm{ch}}}
%\newcommand{\argsh}{\mathop{\mathrm{argsh}}}
%\newcommand{\argch}{\mathop{\mathrm{argch}}}
%\newcommand{\tr}{\mathop{\mathrm{tr}}}
%\newcommand{\rg}{\mathop{\mathrm{rg}}}
%\newcommand{\rang}{\mathop{\mathrm{rg}}}
%\newcommand{\Mat}{\mathop{\mathrm{Mat}}}
%\renewcommand{\Re}{\mathop{\mathrm{Re}}}
%\renewcommand{\Im}{\mathop{\mathrm{Im}}}
%\renewcommand{\th}{\mathop{\mathrm{th}}}


%En tete et pied de page
\lhead{Programme colle math}
\chead{Semaine 22 du 23/03/20 au 28/03/20}
\rhead{MPSI B Hoche}

\lfoot{\tiny{Cette création est mise à disposition selon le Contrat\\ Paternité-Partage des Conditions Initiales à l'Identique 2.0 France\\ disponible en ligne http://creativecommons.org/licenses/by-sa/2.0/fr/
} }
\rfoot{\tiny{Rémy Nicolai \jobname}}


\begin{document}

\subsection{Matrices}
\subsubsection{B - Matrices et applications linéaires}
\subsubsubsection{a) Matrice d'une application linéaire dans des bases.}
\begin{parcolumns}[rulebetween,distance=2.5cm]{2}
  \colchunk{Matrice d'une famille de vecteurs dans une base, d'une application linéaire dans un couple de bases.}
  \colchunk{Notation $\Mat _{e,f}(u)$.\\Isomorphisme $u\mapsto \Mat_{e,f}(u)$.}
  \colplacechunks
   \colchunk{Coordonnées de l'image d'un vecteur par une application linéaire.}
  \colchunk{}
  \colplacechunks
   \colchunk{Matrice d'une composée d'applications linéaires. Lien entre matrices inversibles et isomorphismes.}
  \colchunk{Cas particulier des endomorphismes.}
  \colplacechunks
\end{parcolumns}
\subsubsubsection{b) Application linéaire canoniquement associée à une matrice}
\begin{parcolumns}[rulebetween,distance=2.5cm]{2}

   \colchunk{Noyau, image et rang d'une matrice.}
  \colchunk{Les colonnes engendrent l'image, les lignes donnent un système d'équations du noyau. Une matrice carrée est inversible si et seulement si son noyau est réduit au sous-espace nul.}
  \colplacechunks
   \colchunk{Condition d'inversibilité d'une matrice triangulaire. L'inverse d'une matrice triangulaire est une matrice triangulaire.}
  \colchunk{}
  \colplacechunks
\end{parcolumns}

\subsubsubsection{c) Blocs}
\begin{parcolumns}[rulebetween,distance=2.5cm]{2}

   \colchunk{Matrice par blocs.}
  \colchunk{Interprétation géométrique.}
  \colplacechunks
   \colchunk{Théorème du produit par blocs.}
  \colchunk{La démonstration n'est pas exigible.}
  \colplacechunks
\end{parcolumns}

\subsubsection{C - Changements de bases, équivalence et similitude}
\subsubsubsection{a) Changements de bases}
\begin{parcolumns}[rulebetween,distance=2.5cm]{2}
   \colchunk{Matrice de passage d'une base à une autre.}
  \colchunk{La matrice de passage $P_e^{e'}$ de $e$ à $e'$ est la matrice de la famille $e'$ dans la base $e$.\\Inversibilité et inverse de $P_e^{e'}$.}
  \colplacechunks
   \colchunk{Effet d'un changement de base sur les coordonnées d'un vecteur, sur la matrice d'une application linéaire.}
  \colchunk{}
  \colplacechunks
\end{parcolumns}

\subsubsubsection{b) Matrices équivalentes et rang}
\begin{parcolumns}[rulebetween,distance=2.5cm]{2}
   \colchunk{Si $u\in\mathcal L(E,F)$ est de rang $r$, il existe une base $e$ de $E$ et une base $f$ de $F$ telles que $\Mat_{e,f}(u)=J_r$.}
  \colchunk{La matrice $J_r$ a tous ses coefficients nuls à l'exception des $r$ premiers coefficients diagonaux, égaux à $1$.}
  \colplacechunks
   \colchunk{Matrices équivalentes.}
  \colchunk{Interprétation géométrique.}
  \colplacechunks
   \colchunk{Une matrice est de rang $r$ si et seulement si elle est équivalente à $J_r$.}
  \colchunk{Classification des matrices équivalentes par le rang.}
  \colplacechunks
   \colchunk{Invariance du rang par transposition.}
  \colchunk{}
  \colplacechunks
   \colchunk{Rang d'une matrice extraite. Caractérisation du rang par les matrices carrées extraites.}
  \colchunk{}
  \colplacechunks
\end{parcolumns}

\subsubsubsection{c) Matrices semblables et trace}
\begin{parcolumns}[rulebetween,distance=2.5cm]{2}
   \colchunk{Matrices semblables.}
  \colchunk{Interprétation géométrique.}
  \colplacechunks
   \colchunk{Trace d'une matrice carrée.}
  \colchunk{}
  \colplacechunks
   \colchunk{Linéarité de la trace, relation $\tr(AB)=\tr(BA)$, invariance par similitude.}
  \colchunk{Notations $\tr(A)$, $\Tr(A)$}
  \colplacechunks
   \colchunk{Trace d'un endomorphisme d'un espace de dimension finie. Linéarité, relation $\tr(uv)=\tr(vu)$.}
  \colchunk{Notations $\tr(u)$, $\Tr(u)$.\\ Trace d'un projecteur.}
  \colplacechunks
\end{parcolumns}


\bigskip
\begin{center}
 \textbf{Questions de cours}
\end{center}
Matrice d'un vecteur dans une base, d'une application linéaire dans un couple de bases.\newline
Définition des matrices de passage: inversibilité.\newline
Matrices d'une composition d'applications linéaires.\newline
Formule de changement de base sur la matrice d'un vecteur ou d'une application linéaire.\newline
Une matrice est de rang $r$ si et seulement si elle est équivalente à $J_r$.\newline
Invariance du rang par transposition.\newline
Caractérisation du rang par les matrices carrées extraites.

\bigskip
\begin{center}
 \textbf{Prochain programme}
\end{center}
Opérations élémentaires et systèmes linéaires. Sous-espaces affines.
\end{document}
