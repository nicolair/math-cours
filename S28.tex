%!  pour pdfLatex
\documentclass[a4paper]{article}
\usepackage[hmargin={1.5cm,1.5cm},vmargin={2.4cm,2.4cm},headheight=13.1pt]{geometry}

\usepackage[pdftex]{graphicx,color}
%\usepackage{hyperref}

\usepackage[utf8]{inputenc}
\usepackage[T1]{fontenc}
\usepackage{lmodern}
%\usepackage[frenchb]{babel}
\usepackage[french]{babel}

\usepackage{fancyhdr}
\pagestyle{fancy}

%\usepackage{floatflt}

\usepackage{parcolumns}
\setlength{\parindent}{0pt}
\usepackage{xcolor}

%pr{\'e}sentation des compteurs de section, ...
\makeatletter
%\renewcommand{\labelenumii}{\theenumii.}
\renewcommand{\thepart}{}
\renewcommand{\thesection}{}
\renewcommand{\thesubsection}{}
\renewcommand{\thesubsubsection}{}
\makeatother

\newcommand{\subsubsubsection}[1]{\bigskip \rule[5pt]{\linewidth}{2pt} \textbf{ \color{red}{#1} } \newline \rule{\linewidth}{.1pt}}
\newlength{\parcoldist}
\setlength{\parcoldist}{1cm}

\usepackage{maths}
\newcommand{\dbf}{\leftrightarrows}
% remplace les commandes suivantes 
%\usepackage{amsmath}
%\usepackage{amssymb}
%\usepackage{amsthm}
%\usepackage{stmaryrd}

%\newcommand{\N}{\mathbb{N}}
%\newcommand{\Z}{\mathbb{Z}}
%\newcommand{\C}{\mathbb{C}}
%\newcommand{\R}{\mathbb{R}}
%\newcommand{\K}{\mathbf{K}}
%\newcommand{\Q}{\mathbb{Q}}
%\newcommand{\F}{\mathbf{F}}
%\newcommand{\U}{\mathbb{U}}

%\newcommand{\card}{\mathop{\mathrm{Card}}}
%\newcommand{\Id}{\mathop{\mathrm{Id}}}
%\newcommand{\Ker}{\mathop{\mathrm{Ker}}}
%\newcommand{\Vect}{\mathop{\mathrm{Vect}}}
%\newcommand{\cotg}{\mathop{\mathrm{cotan}}}
%\newcommand{\sh}{\mathop{\mathrm{sh}}}
%\newcommand{\ch}{\mathop{\mathrm{ch}}}
%\newcommand{\argsh}{\mathop{\mathrm{argsh}}}
%\newcommand{\argch}{\mathop{\mathrm{argch}}}
%\newcommand{\tr}{\mathop{\mathrm{tr}}}
%\newcommand{\rg}{\mathop{\mathrm{rg}}}
%\newcommand{\rang}{\mathop{\mathrm{rg}}}
%\newcommand{\Mat}{\mathop{\mathrm{Mat}}}
%\renewcommand{\Re}{\mathop{\mathrm{Re}}}
%\renewcommand{\Im}{\mathop{\mathrm{Im}}}
%\renewcommand{\th}{\mathop{\mathrm{th}}}


%En tete et pied de page
\lhead{Programme colle math}
\chead{Semaine 28 du 25/05/20 au 30/05/20}
\rhead{MPSI B Hoche}

\lfoot{\tiny{Cette création est mise à disposition selon le Contrat\\ Paternité-Partage des Conditions Initiales à l'Identique 2.0 France\\ disponible en ligne http://creativecommons.org/licenses/by-sa/2.0/fr/
} }
\rfoot{\tiny{Rémy Nicolai \jobname}}


\begin{document}

\subsection{Espaces préhilbertiens réels}
\begin{itshape}
La notion de produit scalaire a été étudiée d'un point de vue élémentaire
dans l'enseignement secondaire. Les objectifs  de ce chapitre sont les suivants :
\begin{itemize}
\item
généraliser cette notion et  exploiter, principalement à travers l'étude des projections orthogonales, l'intuition acquise dans des situations géométriques en dimension $2$ ou $3$ pour traiter des problèmes posés dans un contexte plus abstrait ;
\item
approfondir l'étude de la géométrie euclidienne du plan, notamment à travers l'étude des isométries vectorielles.
\end{itemize}
Le cours doit être illustré par de nombreuses figures. Dans toute la suite, $E$ est un espace vectoriel réel.
\end{itshape}

\subsubsubsection{a) Produit scalaire}
\begin{parcolumns}[rulebetween,distance=2.5cm]{2}
  \colchunk{Produit scalaire.}
  \colchunk{Notations $\langle x , y \rangle$, $(x|y)$, $x \cdot y$.}
  \colplacechunks

  \colchunk{Produit scalaire canonique sur $\R^n$,\newline
produit scalaire $(f |g)=\int_a^b fg$ sur $\mathcal{C} \big( [ a , b ] , \R \big)$.}
  \colchunk{}
  \colplacechunks
\end{parcolumns}

\subsubsubsection{b) Norme associée à un produit scalaire}
\begin{parcolumns}[rulebetween,distance=2.5cm]{2}
  \colchunk{Norme associée à un produit scalaire, distance.}
  \colchunk{}
  \colplacechunks

  \colchunk{Inégalité de Cauchy-Schwarz, cas d'égalité.}
  \colchunk{Exemples : sommes finies, intégrales.}
  \colplacechunks
  
  \colchunk{Inégalité triangulaire, cas d'égalité.}
  \colchunk{}
  \colplacechunks

  \colchunk{Formule de polarisation :
\begin{displaymath}
 2 \big< x , y \big> = \| x+y \|^2 - \| x \|^2 - \| y \|^2 
\end{displaymath}}
  \colchunk{}
  \colplacechunks
  \end{parcolumns}

\subsubsubsection{c) Orthogonalité}
\begin{parcolumns}[rulebetween,distance=2.5cm]{2}
  \colchunk{Vecteurs orthogonaux, orthogonal d'une partie.}
  \colchunk{Notation $X^\perp$.

  L'orthogonal d'une partie est un sous-espace.}
  \colplacechunks

  \colchunk{Famille orthogonale, orthonormale (ou orthonormée).}
  \colchunk{}
  \colplacechunks
  
  \colchunk{Toute famille orthogonale de vecteurs non nuls est libre.}
  \colchunk{}
  \colplacechunks

  \colchunk{Théorème de Pythagore.}
  \colchunk{}
  \colplacechunks
  
  \colchunk{Algorithme d'orthonormalisation de Schmidt.}
  \colchunk{}
  \colplacechunks
\end{parcolumns}

\subsubsubsection{d) Bases orthonormales}
\begin{parcolumns}[rulebetween,distance=2.5cm]{2}
  \colchunk{Existence de bases orthonormales dans un espace euclidien. Théorème de la base orthonormale incomplète.}
  \colchunk{}
  \colplacechunks

  \colchunk{Coordonnées dans une base orthonormale, expressions du produit scalaire et de la norme.}
  \colchunk{$\dbf$ PC et SI : mécanique et électricité.}
  \colplacechunks

\end{parcolumns}

\subsubsubsection{e) Projection orthogonale sur un sous-espace de dimension finie}
\begin{parcolumns}[rulebetween,distance=2.5cm]{2}
  \colchunk{Supplémentaire orthogonal d'un sous-espace de dimension finie.}
  \colchunk{En dimension finie, dimension de l'orthogonal.}
  \colplacechunks

  \colchunk{Projection orthogonale. Expression du projeté orthogonal dans une base orthonormale.}
  \colchunk{}
  \colplacechunks
  
  \colchunk{Distance d'un vecteur à un sous-espace. Le projeté orthogonal de $x$ sur $V$ est l'unique élément de $V$ qui minimise la distance de $x$ à $V$.}
  \colchunk{Notation $d (x , V)$.}
  \colplacechunks
\end{parcolumns}

\bigskip
\begin{center}
 \textbf{Questions de cours}
\end{center}
Existence d'une base orthogonale dans un espace euclidien. Orthogonalisation de Gram-Schmidt. Dans un espace euclidien, l'orthogonal d'un sous espace $A$ est un supplémentaire de $A$. Expression du projeté orthogonal dans une base orthogonale.
\begin{center}
 \textbf{Prochain programme}
\end{center}
Espace préhilbertien réel (fin) : isométries vectorielles, matrices orthogonales.
\end{document}
