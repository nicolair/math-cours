\subsection{Dénombrement}
\begin{itshape}
Ce chapitre est introduit essentiellement en vue de son utilisation en probabilités ; rattaché aux mathématiques discrètes, le dénombrement interagit également avec l'algèbre et l'informatique. Il permet de modéliser certaines situations combinatoires et offre un nouveau cadre à la représentation de certaines égalités.

Toute formalisation excessive est exclue. En particulier~:
\begin{itemize}
\item  parmi les propriétés du paragraphe a), les plus intuitives sont admises sans démonstration;
\item  l'utilisation systématique de bijections dans les problèmes de dénombrement n'est pas un attendu du programme.
\end{itemize}
\end{itshape}

\subsubsubsection{a) Cardinal d'un ensemble fini}
\begin{parcolumns}[rulebetween,distance=\parcoldist]{2}
  \colchunk{Cardinal d'un ensemble fini.}
  \colchunk{Notations $|A|$, $\card(A)$, $\# A$.}
  \colplacechunks

  \colchunk{Cardinal d'une partie d'un ensemble fini, cas d'égalité.}
  \colchunk{Tout fondement théorique des notions d'entier naturel et de cardinal est hors programme.}
  \colplacechunks

  \colchunk{Une application entre deux ensembles finis de même cardinal est bijective si et seulement si elle est injective, si et seulement si elle est surjective.}
  \colchunk{}
  \colplacechunks

  \colchunk{Cardinal d'un produit fini d'ensembles finis.}
  \colchunk{}
  \colplacechunks


  \colchunk{Cardinal de la réunion de deux ensembles finis.}
  \colchunk{La formule du crible est hors programme.}
  \colplacechunks

  \colchunk{Cardinal de l'ensemble des applications d'un ensemble fini dans un autre.}
  \colchunk{}
  \colplacechunks

  \colchunk{Cardinal de l'ensemble des parties d'un ensemble fini.}
  \colchunk{}
  \colplacechunks
\end{parcolumns}


\subsubsubsection{b) Listes et combinaisons}
\begin{parcolumns}[rulebetween,distance=\parcoldist]{2}
  \colchunk{Nombre de $p$-listes (ou $p$-uplets) d'éléments distincts d'un ensemble de cardinal $n$, nombre d'applications injectives d'un ensemble de cardinal $p$ dans un ensemble de cardinal $n$, nombre de permutations d'un ensemble de cardinal $n$.}
  \colchunk{}
  \colplacechunks

  \colchunk{Nombre de parties à $p$ éléments (ou $p$-combinaisons) d'un ensemble de cardinal $n$.}
  \colchunk{Démonstration combinatoire des formules de Pascal et du binôme.}
  \colplacechunks

\end{parcolumns}
