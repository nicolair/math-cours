%!  pour pdfLatex
\documentclass[a4paper]{article}
\usepackage[hmargin={1.5cm,1.5cm},vmargin={2.4cm,2.4cm},headheight=13.1pt]{geometry}

\usepackage[pdftex]{graphicx,color}
%\usepackage{hyperref}

\usepackage[utf8]{inputenc}
\usepackage[T1]{fontenc}
\usepackage{lmodern}
%\usepackage[frenchb]{babel}
\usepackage[french]{babel}

\usepackage{fancyhdr}
\pagestyle{fancy}

%\usepackage{floatflt}

\usepackage{parcolumns}
\setlength{\parindent}{0pt}
\usepackage{xcolor}

%pr{\'e}sentation des compteurs de section, ...
\makeatletter
%\renewcommand{\labelenumii}{\theenumii.}
\renewcommand{\thepart}{}
\renewcommand{\thesection}{}
\renewcommand{\thesubsection}{}
\renewcommand{\thesubsubsection}{}
\makeatother

\newcommand{\subsubsubsection}[1]{\bigskip \rule[5pt]{\linewidth}{2pt} \textbf{ \color{red}{#1} } \newline \rule{\linewidth}{.1pt}}
\newlength{\parcoldist}
\setlength{\parcoldist}{1cm}

\usepackage{maths}
\newcommand{\dbf}{\leftrightarrows}
% remplace les commandes suivantes 
%\usepackage{amsmath}
%\usepackage{amssymb}
%\usepackage{amsthm}
%\usepackage{stmaryrd}

%\newcommand{\N}{\mathbb{N}}
%\newcommand{\Z}{\mathbb{Z}}
%\newcommand{\C}{\mathbb{C}}
%\newcommand{\R}{\mathbb{R}}
%\newcommand{\K}{\mathbf{K}}
%\newcommand{\Q}{\mathbb{Q}}
%\newcommand{\F}{\mathbf{F}}
%\newcommand{\U}{\mathbb{U}}

%\newcommand{\card}{\mathop{\mathrm{Card}}}
%\newcommand{\Id}{\mathop{\mathrm{Id}}}
%\newcommand{\Ker}{\mathop{\mathrm{Ker}}}
%\newcommand{\Vect}{\mathop{\mathrm{Vect}}}
%\newcommand{\cotg}{\mathop{\mathrm{cotan}}}
%\newcommand{\sh}{\mathop{\mathrm{sh}}}
%\newcommand{\ch}{\mathop{\mathrm{ch}}}
%\newcommand{\argsh}{\mathop{\mathrm{argsh}}}
%\newcommand{\argch}{\mathop{\mathrm{argch}}}
%\newcommand{\tr}{\mathop{\mathrm{tr}}}
%\newcommand{\rg}{\mathop{\mathrm{rg}}}
%\newcommand{\rang}{\mathop{\mathrm{rg}}}
%\newcommand{\Mat}{\mathop{\mathrm{Mat}}}
%\renewcommand{\Re}{\mathop{\mathrm{Re}}}
%\renewcommand{\Im}{\mathop{\mathrm{Im}}}
%\renewcommand{\th}{\mathop{\mathrm{th}}}


%En tete et pied de page
\lhead{Programme colle math}
\chead{Semaine 2 du 23/09/19 au 28/09/19}
\rhead{MPSI B Hoche}

\lfoot{\tiny{Cette création est mise à disposition selon le Contrat\\ Paternité-Partage des Conditions Initiales à l'Identique 2.0 France\\ disponible en ligne http://creativecommons.org/licenses/by-sa/2.0/fr/
} }
\rfoot{\tiny{Rémy Nicolai \jobname}}


\begin{document}
\subsection{Nombres complexes et trigonométrie (fin)}

\subsubsubsection{c) Nombres complexes de module 1 et trigonométrie}
\begin{parcolumns}[rulebetween,distance=2.5cm]{2}
  \colchunk{Cercle trigonométrique. Paramétrisation par les fonctions circulaires.}
  \colchunk{Notation $\U$.\newline
  Les étudiants doivent savoir retrouver les formules du type $\cos(\pi-x)=-\cos x$ et résoudre des équations et inéquations trigonométriques en s'aidant du cercle trigonométrique.}
  \colplacechunks

  \colchunk{Définition de $e^{it}$ pour $t\in\R$. Exponentielle d'une somme. Formules de trigonométrie exigibles: $\cos(a\pm b)$, $\sin(a\pm b)$, $\cos(2a)$, $\sin(2a)$, $\cos a \cos b$, $\sin a \cos b$, $\sin a \sin b$.}
  \colchunk{Les étudiants doivent savoir factoriser des expressions du type $\cos p + \cos q$.}
  \colplacechunks

  \colchunk{Fonction tangente.}
  \colchunk{La fonction tangente n'a pas été introduite au lycée. Notation $\tan$.}
  \colplacechunks

  \colchunk{Formule exigible: $\tan(a\pm b)$.}
  \colchunk{}
  \colplacechunks

  \colchunk{Formules d'Euler.}
  \colchunk{Linéarisation,\newline calcul de $\sum_{k=0}^n\cos(kt)$, de $\sum_{k=0}^n\sin(kt)$.}
  \colplacechunks

  \colchunk{Formule de Moivre.}
  \colchunk{Les étudiants doivent savoir retrouver les expressions de $\cos(nt)$ et de $\sin(nt)$ en fonction de $\cos t$ et $\sin t$.}
  \colplacechunks
  \end{parcolumns}

\subsubsubsection{d) Formes trigonométriques}
\begin{parcolumns}[rulebetween,distance=2.5cm]{2}
  \colchunk{Forme trigonométrique $re^{i\theta}$ avec $r>0$ d'un nombre complexe non nul. Arguments. Arguments d'un produit, d'un quotient.}
  \colchunk{Relation de congruence modulo $2\pi$ sur $\R$.}
  \colplacechunks

  \colchunk{Factorisation de $1\pm e^{it}$.}
  \colchunk{}
  \colplacechunks

  \colchunk{Transformation de $a\cos t + b\sin t$ en $A\cos(t-\varphi)$.}
  \colchunk{$\leftrightarrows$ PC et SI: amplitude et phase.}
  \colplacechunks
\end{parcolumns}

\subsubsubsection{f) Racines $n$-ièmes}
\begin{parcolumns}[rulebetween,distance=2.5cm]{2}
  \colchunk{Description des racines $n$-ièmes de l'unité, d'un nombre complexe non nul donné sous forme trigonométrique.}
  \colchunk{Notation $\U_n$. Représentation géométrique.}
  \colplacechunks

\end{parcolumns}

\subsubsubsection{g) Exponentielle complexe}
\begin{parcolumns}[rulebetween,distance=2.5cm]{2}
  \colchunk{Définition de $e^z$ pour $z$ complexe: $e^z = e^{\Re(z)}e^{i\Im(z)}$.}
  \colchunk{Notation $\exp(z)$, $e^z$.\newline
  $\leftrightarrows$ PC et SI: définition d'une impédance complexe en régime sinusoïdal.}
  \colplacechunks

  \colchunk{Exponentielle d'une somme.}
  \colchunk{}
  \colplacechunks

  \colchunk{Pour tous $z$ et $z'$ dans $\C$, $\exp(z)=\exp(z')$ si et seulement si $z-z'\in 2i\pi \Z$.}
  \colchunk{}
  \colplacechunks

  \colchunk{Résolution de l'équation $\exp(z)=a$.}
  \colchunk{}
  \colplacechunks
\end{parcolumns}

\subsubsubsection{h) Interprétation géométrique des nombres complexes}
\begin{parcolumns}[rulebetween,distance=2.5cm]{2}
  \colchunk{Interprétation géométrique du module et d'un argument de $\frac{c-b}{c-a}$.}
  \colchunk{Traduction de l'alignement, de l'orthogonalité.}
  \colplacechunks

  \colchunk{Interprétation géométrique des applications $z\mapsto az+b$.}
  \colchunk{Similitudes directes. Cas particuliers: translation, homothéties, rotations.}
  \colplacechunks

  \colchunk{Interprétation géométrique de la conjugaison.}
  \colchunk{L'étude générale des similitudes indirectes est hors programme.}
  \colplacechunks
\end{parcolumns}

\subsection{Calculs algébriques (fin)}
Ce chapitre a pour but de présenter quelques notations et techniques fondamentales de calcul algébrique.

\subsubsubsection{b) Coefficients binomiaux et formule du binôme}
\begin{parcolumns}[rulebetween,distance=\parcoldist]{2}
  \colchunk{Factorielle. Coefficients binomiaux.}
  \colchunk{Notation $\binom{n}{p}$.}
  \colplacechunks
  \colchunk{Relation $\binom{n}{p}=\binom{n}{n-p}$}
  \colchunk{}
  \colplacechunks
  \colchunk{Formule et triangle de Pascal.}
  \colchunk{Lien avec la méthode d'obtention des coefficients binomiaux utilisée en Première (dénombrement de chemins).}
  \colplacechunks
  \colchunk{Formule du binôme dans $\C$.}
  \colchunk{}
  \colplacechunks

  \end{parcolumns}

\subsubsubsection{c) Systèmes linéaires}
\begin{parcolumns}[rulebetween,distance=\parcoldist]{2}
  \colchunk{Système linéaire de $n$ équations à $p$ inconnues à coefficients dans $\R$ ou $\C$.}
  \colchunk{$\leftrightarrows$ PC et SI dans le cas $n=p=2$. \newline
  Interprétation géométrique: intersection de droites dans $\R^2$, de plans dans $\R^3$.}
  \colplacechunks
  
  \colchunk{Système homogène associé. Structure de l'ensemble des solutions.}
  \colchunk{}
  \colplacechunks
  
  \colchunk{Opérations élémentaires.}
  \colchunk{Notations $L_i\leftrightarrow L_j$, $L_i\leftarrow \lambda L_i$ ($\lambda \neq 0$), $L_i\leftarrow  L_i +\lambda L_j$.}
  \colplacechunks

  \colchunk{Algorithme du pivot.}
  \colchunk{$\rightleftarrows$ I: pour des systèmes de taille $n>3$ ou $p>3$, on utilise l'outil informatique.}
  \colplacechunks

\end{parcolumns}

\bigskip
Pas de question théorique sur les systèmes linéaires: seulement pratique avec 2 ou 3 équations et inconnues. 
\begin{center}
 \textbf{Questions de cours}
\end{center}
Preuve de l'existence d'un argument d'un nombre complexe non nul à partir du tableau de variation de $\cos$. Preuve de l'énumération des éléments de $\U_n$. Définition récursive des coefficients du binôme par le triangle de Pascal. Preuve de la formule du binôme, de l'expression avec des produits, justification de \og$p$ parmi $n$\fg~ à l'aide de chemins. 

\begin{center}
 \textbf{Prochain programme}
\end{center}
Raisonnements et vocabulaire ensembliste. Inégalités dans $\R$.
\end{document}
