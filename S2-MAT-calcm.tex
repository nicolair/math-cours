\subsubsection{A - Calcul Matriciel}

\subsubsubsection{a) Espaces de matrices}
\begin{parcolumns}[rulebetween,distance=2.5cm]{2}
  \colchunk{Espace vectoriel $\mat np\K$ des matrices à $n$ lignes et $p$ colonnes à coefficients dans $\K$.}
  \colchunk{}
  \colplacechunks
   \colchunk{Base canonique de $\mat np\K$.}
  \colchunk{Dimension de $\mat np\K$.}
  \colplacechunks
\end{parcolumns}

\subsubsubsection{b) Produit matriciel}
\begin{parcolumns}[rulebetween,distance=2.5cm]{2}

   \colchunk{Bilinéarité, associativité.}
  \colchunk{}
  \colplacechunks
   \colchunk{Produit d'une matrice de la base canonique de $\mat np\K$ par une matrice de la base canonique de $\mat pq\K$.}
  \colchunk{}
  \colplacechunks
   \colchunk{Anneau $\matc n\K$.}
  \colchunk{Non commutativité si $n\ge 2$. Exemples de diviseurs de zéro et de matrices nilpotentes.}
  \colplacechunks
   \colchunk{Formule du binôme.}
  \colchunk{Application au calcul de puissances.}
  \colplacechunks
   \colchunk{Matrice inversible, inverse. Groupe linéaire.}
  \colchunk{Notation $\Gl_n(\K)$.}
  \colplacechunks
   \colchunk{Produit de matrices diagonales, de matrices triangulaires supérieures, inférieures.}
  \colchunk{}
  \colplacechunks
\end{parcolumns}

\subsubsubsection{c) Transposition}
\begin{parcolumns}[rulebetween,distance=2.5cm]{2}

   \colchunk{Transposée d'une matrice.}
  \colchunk{Notations $^tA$, $A^T$}
  \colplacechunks
   \colchunk{Opérations sur les transposées~: combinaison linéaire, produit, inverse.}
  \colchunk{}
  \colplacechunks
\end{parcolumns}
