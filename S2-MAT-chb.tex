\subsubsection{C - Changements de bases, équivalence et similitude}
\subsubsubsection{a) Changements de bases}
\begin{parcolumns}[rulebetween,distance=2.5cm]{2}
   \colchunk{Matrice de passage d'une base à une autre.}
  \colchunk{La matrice de passage $P_e^{e'}$ de $e$ à $e'$ est la matrice de la famille $e'$ dans la base $e$.\\Inversibilité et inverse de $P_e^{e'}$.}
  \colplacechunks
   \colchunk{Effet d'un changement de base sur les coordonnées d'un vecteur, sur la matrice d'une application linéaire.}
  \colchunk{}
  \colplacechunks
\end{parcolumns}

\subsubsubsection{b) Matrices équivalentes et rang}
\begin{parcolumns}[rulebetween,distance=2.5cm]{2}
   \colchunk{Si $u\in\mathcal L(E,F)$ est de rang $r$, il existe une base $e$ de $E$ et une base $f$ de $F$ telles que $\Mat_{e,f}(u)=J_r$.}
  \colchunk{La matrice $J_r$ a tous ses coefficients nuls à l'exception des $r$ premiers coefficients diagonaux, égaux à $1$.}
  \colplacechunks
   \colchunk{Matrices équivalentes.}
  \colchunk{Interprétation géométrique.}
  \colplacechunks
   \colchunk{Une matrice est de rang $r$ si et seulement si elle est équivalente à $J_r$.}
  \colchunk{Classification des matrices équivalentes par le rang.}
  \colplacechunks
   \colchunk{Invariance du rang par transposition.}
  \colchunk{}
  \colplacechunks
   \colchunk{Rang d'une matrice extraite. Caractérisation du rang par les matrices carrées extraites.}
  \colchunk{}
  \colplacechunks
\end{parcolumns}

\subsubsubsection{c) Matrices semblables et trace}
\begin{parcolumns}[rulebetween,distance=2.5cm]{2}
   \colchunk{Matrices semblables.}
  \colchunk{Interprétation géométrique.}
  \colplacechunks
   \colchunk{Trace d'une matrice carrée.}
  \colchunk{}
  \colplacechunks
   \colchunk{Linéarité de la trace, relation $\tr(AB)=\tr(BA)$, invariance par similitude.}
  \colchunk{Notations $\tr(A)$, $\Tr(A)$}
  \colplacechunks
   \colchunk{Trace d'un endomorphisme d'un espace de dimension finie. Linéarité, relation $\tr(uv)=\tr(vu)$.}
  \colchunk{Notations $\tr(u)$, $\Tr(u)$.\\ Trace d'un projecteur.}
  \colplacechunks
\end{parcolumns}
