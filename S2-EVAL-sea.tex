\subsubsection{D - Sous-espaces affines d'un espace vectoriel}

\begin{itshape}Le but de cette partie est double~:
\begin{itemize}
\item montrer comment l'algèbre linéaire permet d'étendre les notions de géométrie affine étudiées au collège et au lycée et d'utiliser l'intuition géométrique dans un cadre élargi~;
\item modéliser un problème affine par une équation $u(x)=a$ où $u$ est une application linéaire, et unifier plusieurs situations de ce type déjà rencontrées.
\end{itemize}
Cette partie du cours doit être illustrée de nombreuses figures.\bigskip \end{itshape}

\begin{parcolumns}[rulebetween,distance=\parcoldist]{2}
 \colchunk{Présentation informelle de la structure d'un espace vectoriel~: points et vecteurs.}
  \colchunk{L'écriture $B=A+\vec u$ est équivalente à la relation $\vec{AB}=\vec u$.}
  \colplacechunks
 \colchunk{Translation.}
  \colchunk{}
  \colplacechunks
 \colchunk{Sous-espace affine d'un espace vectoriel, direction. Hyperplan affine.}
  \colchunk{Sous-espace affines de $\R^2$ et $\R^3$.}
  \colplacechunks
 \colchunk{Intersection de sous-espaces affines.}
  \colchunk{}
  \colplacechunks
 \colchunk{Si $u\in\mathcal L(E,F)$, l'ensemble des solutions de l'équation $u(x)=a$ d'inconnue $x$ est soit l'ensemble vide, soit un sous-espace affine dirigé par $\Ker u$.}
  \colchunk{Retour sur les systèmes linéaires, les équations différentielles linéaires d'ordre $1$ et $2$ et la recherche de polynômes interpolateurs.\\La notion d'application affine est hors-programme.}
  \colplacechunks
 \colchunk{Repère affine, coordonnées.}
\end{parcolumns}

