%!  pour pdfLatex
\documentclass[a4paper]{article}
\usepackage[hmargin={1.5cm,1.5cm},vmargin={2.4cm,2.4cm},headheight=13.1pt]{geometry}

\usepackage[pdftex]{graphicx,color}
%\usepackage{hyperref}

\usepackage[utf8]{inputenc}
\usepackage[T1]{fontenc}
\usepackage{lmodern}
%\usepackage[frenchb]{babel}
\usepackage[french]{babel}

\usepackage{fancyhdr}
\pagestyle{fancy}

%\usepackage{floatflt}

\usepackage{parcolumns}
\setlength{\parindent}{0pt}
\usepackage{xcolor}

%pr{\'e}sentation des compteurs de section, ...
\makeatletter
%\renewcommand{\labelenumii}{\theenumii.}
\renewcommand{\thepart}{}
\renewcommand{\thesection}{}
\renewcommand{\thesubsection}{}
\renewcommand{\thesubsubsection}{}
\makeatother

\newcommand{\subsubsubsection}[1]{\bigskip \rule[5pt]{\linewidth}{2pt} \textbf{ \color{red}{#1} } \newline \rule{\linewidth}{.1pt}}
\newlength{\parcoldist}
\setlength{\parcoldist}{1cm}

\usepackage{maths}
\newcommand{\dbf}{\leftrightarrows}
% remplace les commandes suivantes 
%\usepackage{amsmath}
%\usepackage{amssymb}
%\usepackage{amsthm}
%\usepackage{stmaryrd}

%\newcommand{\N}{\mathbb{N}}
%\newcommand{\Z}{\mathbb{Z}}
%\newcommand{\C}{\mathbb{C}}
%\newcommand{\R}{\mathbb{R}}
%\newcommand{\K}{\mathbf{K}}
%\newcommand{\Q}{\mathbb{Q}}
%\newcommand{\F}{\mathbf{F}}
%\newcommand{\U}{\mathbb{U}}

%\newcommand{\card}{\mathop{\mathrm{Card}}}
%\newcommand{\Id}{\mathop{\mathrm{Id}}}
%\newcommand{\Ker}{\mathop{\mathrm{Ker}}}
%\newcommand{\Vect}{\mathop{\mathrm{Vect}}}
%\newcommand{\cotg}{\mathop{\mathrm{cotan}}}
%\newcommand{\sh}{\mathop{\mathrm{sh}}}
%\newcommand{\ch}{\mathop{\mathrm{ch}}}
%\newcommand{\argsh}{\mathop{\mathrm{argsh}}}
%\newcommand{\argch}{\mathop{\mathrm{argch}}}
%\newcommand{\tr}{\mathop{\mathrm{tr}}}
%\newcommand{\rg}{\mathop{\mathrm{rg}}}
%\newcommand{\rang}{\mathop{\mathrm{rg}}}
%\newcommand{\Mat}{\mathop{\mathrm{Mat}}}
%\renewcommand{\Re}{\mathop{\mathrm{Re}}}
%\renewcommand{\Im}{\mathop{\mathrm{Im}}}
%\renewcommand{\th}{\mathop{\mathrm{th}}}


%En tete et pied de page
\lhead{Programme colle math}
\chead{Semaine 16 du 27/01/20 au 01/02/20}
\rhead{MPSI B Hoche}

\lfoot{\tiny{Cette création est mise à disposition selon le Contrat\\ Paternité-Partage des Conditions Initiales à l'Identique 2.0 France\\ disponible en ligne http://creativecommons.org/licenses/by-sa/2.0/fr/
} }
\rfoot{\tiny{Rémy Nicolai \jobname}}


\begin{document}
\subsection{Polynômes et fractions rationnelles (fin)}
\subsubsubsection{h) Fractions rationnelles}
\begin{parcolumns}[rulebetween,distance=\parcoldist]{2}
  \colchunk{Corps $\K (X)$.}
  \colchunk{La construction de $\K (X)$ n'est pas exigible.}
  \colplacechunks

  \colchunk{Forme irréductible d'une fraction rationnelle. Fonction rationnelle.}
  \colchunk{}
  \colplacechunks

  \colchunk{Degré, partie entière, zéros et pôles, multiplicités.}
  \colchunk{}
  \colplacechunks
\end{parcolumns}

\subsubsubsection{i) Décomposition en éléments simples sur $\C$ et sur $\R$}
\begin{parcolumns}[rulebetween,distance=\parcoldist]{2}
  \colchunk{Existence et unicité de la décomposition en éléments simples sur $\C$ et sur $\R$.}
  \colchunk{La démonstration est hors programme.

\noindent On évitera toute technicité excessive.

\noindent La division selon les puissances croissantes est hors programme.}
  \colplacechunks

  \colchunk{Si $\lambda$ est un pôle simple, coefficient de $\dfrac{1}{X - \lambda}$.}
  \colchunk{}
  \colplacechunks

  \colchunk{Décomposition en éléments simples de $\dfrac{P'}{P}$.}
  \colchunk{}
  \colplacechunks
\end{parcolumns}


\subsection{Espaces vectoriels et applications linéaires (1)}

\subsubsection{A - Espaces vectoriels}

\subsubsubsection{Espaces vectoriels}
\begin{parcolumns}[rulebetween,distance=\parcoldist]{2}
  \colchunk{Structure de $\K$ espace vectoriel}
  \colchunk{Espaces $\K^n$, $\K[X]$.}
  \colplacechunks
  
  \colchunk{Produit d'un nombre fini d'espaces vectoriels.}
  \colchunk{}
  \colplacechunks

 \colchunk{Espace vectoriel des  fonctions d'un ensemble dans un espace vectoriel.}
  \colchunk{Espace $\K^\N$ des suites d'éléments de $\K$.}
  \colplacechunks

 \colchunk{Famille presque nulle (ou à support fini) de scalaires, combinaison linéaire  d'une famille de vecteurs.}
  \colchunk{On commence par la notion de combinaison linéaire d'une famille finie de vecteurs.}
  \colplacechunks
  
\end{parcolumns}

\subsubsubsection{Sous-espaces vectoriels}
\begin{parcolumns}[rulebetween,distance=\parcoldist]{2}

 \colchunk{Sous-espace vectoriel~: définition, caractérisation.}
  \colchunk{Sous-espace nul. Droites vectorielles de $\R^2$, droites et plans vectoriels de $\R^3$. Sous-espaces $K_n[X]$ de $\K[X]$.}
  \colplacechunks
 \colchunk{Instersection d'une famille de sous-espaces vectoriels. }
  \colchunk{}
  \colplacechunks
 \colchunk{Sous-espace vectoriel engendré par une partie $X$.}
  \colchunk{Notations $\Vect(X)$, $\Vect{(x_i)}_{i\in I}$.\\Tout sous-espace contenant $X$ contient $\Vect(X)$.}
  \colplacechunks
\end{parcolumns}

\subsubsubsection{Familles de vecteurs}
\begin{parcolumns}[rulebetween,distance=\parcoldist]{2}
 \colchunk{Familles et parties génératrices.}
  \colchunk{}
  \colplacechunks
 \colchunk{Familles et parties libres.}
  \colchunk{}
  \colplacechunks
 \colchunk{Base, coordonnées.}
  \colchunk{Bases canoniques de $K^n$, $K_n[X]$, $\K[X]$.}
  \colplacechunks

\end{parcolumns}
\subsubsubsection{Somme d'un nombre fini de sous-espaces}
\begin{parcolumns}[rulebetween,distance=\parcoldist]{2}

 \colchunk{Somme de deux sous-espaces.}
  \colchunk{}
  \colplacechunks
 \colchunk{Somme directe de deux sous-espaces. Caractérisation par l'intersection.}
  \colchunk{La somme $F+G$ est directe si la décomposition de tout vecteur de $F+G$ comme somme d'un élément de $F$ et d'un élément de $G$ est unique.}

  \colplacechunks
 \colchunk{Sous-espaces supplémentaires.}
  \colchunk{}
  \colplacechunks
 \colchunk{Somme d'un nombre fini de sous-espaces.}
  \colchunk{}
  \colplacechunks
 \colchunk{Somme directe d'un nombre fini de sous-espaces. Caractérisation par l'unicité de la décomposition du vecteur nul.}
  \colchunk{La somme $F_1+\cdots+F_p$ est directe si la décomposition de tout vecteur de $F_1+\cdots+F_p$ sous la forme $x_1+\cdots+x_p$ avec $x_i\in F_i $ est unique.}
  \colplacechunks

\end{parcolumns}


\bigskip
\begin{center}
 \textbf{Questions de cours}
\end{center}

Fractions rationnelles.\newline
Représentants irréductibles d'une fraction.\newline
Définition du degré d'une fraction rationnelle. Degré d'un produit et d'une somme.\newline
Zéro, pôle, multiplicité, valuation en $a\in \C$. Valuation d'un produit, d'une somme.\newline
Exercice traité en classe: dérivation d'une fraction rationnelle, étude du degré de la dérivée.\newline
Avec démonstration:
\begin{itemize}
 \item Forme, existence et unicité de la partie polaire pour $a\in \C$ (démonstration algorithmique).
 \item Existence et unicité de la partie entière.
 \item Une fraction est la somme de sa partie entière et de ses parties polaires.
\end{itemize}

Espaces vectoriels sans dimension.\newline
Définition et caractérisation des couples de sous-espaces supplémentaires.

\begin{center}
 \textbf{Prochain programme}
\end{center}
Espaces vectoriels de dimension finie.
\end{document}
