%!  pour pdfLatex
\documentclass[a4paper]{article}
\usepackage[hmargin={1.5cm,1.5cm},vmargin={2.4cm,2.4cm},headheight=13.1pt]{geometry}

\usepackage[pdftex]{graphicx,color}
%\usepackage{hyperref}

\usepackage[utf8]{inputenc}
\usepackage[T1]{fontenc}
\usepackage{lmodern}
%\usepackage[frenchb]{babel}
\usepackage[french]{babel}

\usepackage{fancyhdr}
\pagestyle{fancy}

%\usepackage{floatflt}

\usepackage{parcolumns}
\setlength{\parindent}{0pt}
\usepackage{xcolor}

%pr{\'e}sentation des compteurs de section, ...
\makeatletter
%\renewcommand{\labelenumii}{\theenumii.}
\renewcommand{\thepart}{}
\renewcommand{\thesection}{}
\renewcommand{\thesubsection}{}
\renewcommand{\thesubsubsection}{}
\makeatother

\newcommand{\subsubsubsection}[1]{\bigskip \rule[5pt]{\linewidth}{2pt} \textbf{ \color{red}{#1} } \newline \rule{\linewidth}{.1pt}}
\newlength{\parcoldist}
\setlength{\parcoldist}{1cm}

\usepackage{maths}
\newcommand{\dbf}{\leftrightarrows}
% remplace les commandes suivantes 
%\usepackage{amsmath}
%\usepackage{amssymb}
%\usepackage{amsthm}
%\usepackage{stmaryrd}

%\newcommand{\N}{\mathbb{N}}
%\newcommand{\Z}{\mathbb{Z}}
%\newcommand{\C}{\mathbb{C}}
%\newcommand{\R}{\mathbb{R}}
%\newcommand{\K}{\mathbf{K}}
%\newcommand{\Q}{\mathbb{Q}}
%\newcommand{\F}{\mathbf{F}}
%\newcommand{\U}{\mathbb{U}}

%\newcommand{\card}{\mathop{\mathrm{Card}}}
%\newcommand{\Id}{\mathop{\mathrm{Id}}}
%\newcommand{\Ker}{\mathop{\mathrm{Ker}}}
%\newcommand{\Vect}{\mathop{\mathrm{Vect}}}
%\newcommand{\cotg}{\mathop{\mathrm{cotan}}}
%\newcommand{\sh}{\mathop{\mathrm{sh}}}
%\newcommand{\ch}{\mathop{\mathrm{ch}}}
%\newcommand{\argsh}{\mathop{\mathrm{argsh}}}
%\newcommand{\argch}{\mathop{\mathrm{argch}}}
%\newcommand{\tr}{\mathop{\mathrm{tr}}}
%\newcommand{\rg}{\mathop{\mathrm{rg}}}
%\newcommand{\rang}{\mathop{\mathrm{rg}}}
%\newcommand{\Mat}{\mathop{\mathrm{Mat}}}
%\renewcommand{\Re}{\mathop{\mathrm{Re}}}
%\renewcommand{\Im}{\mathop{\mathrm{Im}}}
%\renewcommand{\th}{\mathop{\mathrm{th}}}


%En tete et pied de page
\lhead{Programme colle math}
\chead{Semaine 5 du 14/10/19 au 19/10/19}
\rhead{MPSI B Hoche}

\lfoot{\tiny{Cette création est mise à disposition selon le Contrat\\ Paternité-Partage des Conditions Initiales à l'Identique 2.0 France\\ disponible en ligne http://creativecommons.org/licenses/by-sa/2.0/fr/
} }
\rfoot{\tiny{Rémy Nicolai \jobname}}


\begin{document}
\subsection{Techniques fondamentales de calcul en analyse}
\subsubsection{B - Fonction de la variable réelle à valeurs réelles ou complexes (fin)}

\subsubsubsection{e) Dérivation d'une fonction complexe d'une variable réelle.}
\begin{parcolumns}[rulebetween,distance=\parcoldist]{2}
  \colchunk{Dérivée d'une fonction à valeurs complexes.}
  \colchunk{La dérivée est définie par ses parties réelle et imaginaire.}
  \colplacechunks
  
  \colchunk{Dérivée d'une combinaison linéaire, d'un produit, d'un quotient.}
  \colchunk{Brève extension des résultats sur les fonctions à valeurs réelles.}
  \colplacechunks
  
  \colchunk{Dérivée de $\exp(\varphi)$ où $\varphi$ est une fonction dérivable à valeurs complexes.}
  \colchunk{$\leftrightarrows$ PC et SI: électrocinétique.}
  \colplacechunks
  
  \colchunk{Exemple de dérivation d'une fonction à valeurs complexes.}
  \colchunk{Pour $z\in\C\setminus \R$, fonction à valeurs complexes dont la dérivée est $t\mapsto \frac{1}{t+z}$.}
  \colplacechunks
  
\end{parcolumns}


\subsubsection{C - Primitives et équations différentielles linéaires}
\subsubsubsection{a) Calcul de primitives}
\begin{parcolumns}[rulebetween,distance=\parcoldist]{2}
  \colchunk{Primitives d'une fonction définie sur un intervalle à valeurs complexes}
  \colchunk{Description de l'ensemble des primitives d'une fonction sur un intervalle connaissant l'une d'entre elles.\newline
  Les étudiants doivent savoir utiliser les primitives de $x\mapsto e^{\lambda x}$ pour calculer celles de $x\mapsto e^{a x}\cos(bx)$ et de $x\mapsto e^{a x}\sin(bx)$.\newline
  $\leftrightarrows$ PC et SI: cinématique.}
  \colplacechunks

  \colchunk{Primitives des fonctions puissances, trigonométriques et hyperboliques, exponentielle, logarithme,
  \begin{displaymath}
   x\mapsto \frac{1}{1+x^2}, x\mapsto \frac{1}{\sqrt{1-x^2}}
  \end{displaymath}}
  \colchunk{Les étudiants doivent savoir calculer les primitives des fonctions du type
  \begin{displaymath}
   x\mapsto \frac{1}{ax^2+bx+c}
  \end{displaymath}
et reconnaître les dérivées de fonctions composées.}
  \colplacechunks

  \colchunk{Dérivée de $x\mapsto \int_{x_0}^xf(t)dt$ où $f$ est continue.}
  \colchunk{Résultat admis à ce stade.}
  \colplacechunks

  \colchunk{Toute fonction continue admet des primitives.}
  \colchunk{}
  \colplacechunks

  \colchunk{Calcul d'une intégrale au moyen d'une primitive.}
  \colchunk{}
  \colplacechunks

  \colchunk{Intégration par parties pour des fonctions de classe $\mathcal{C}^1$. Changement de variable: si $\varphi$ est de classe $\mathcal{C}^1$ sur $I$ et si $f$ est continue sur $\varphi(I)$, alors, pour tous $a$ et $b$ dans $I$
  \begin{displaymath}
   \int_{\varphi(a)}^{\varphi(b)}f(x)dx = \int_a^bf(\varphi(t))\varphi'(t)dt
  \end{displaymath}  }
  \colchunk{On définit à cette occasion la classe $\mathcal{C}^1$. Application au calcul de primitives.}
  \colplacechunks
  \end{parcolumns}

\subsubsubsection{b) \'Equations différentielles linéaires du premier ordre}
\begin{parcolumns}[rulebetween,distance=\parcoldist]{2}
  \colchunk{Notion d'équation différentielle linéaire du premier ordre:
  \begin{displaymath}
   y'+a(x)y = b(x)
  \end{displaymath}
où $a$ et $b$ sont des fonctions continues définies sur un intervalle $I$ de $\R$ à valeurs réelles ou complexes.}
  \colchunk{\'Equation homogène associée. \newline
  Cas particulier où la fonction $a$ est constante.}
  \colplacechunks

  \colchunk{Résolution d'une équation homogène.}
  \colchunk{}
  \colplacechunks

  \colchunk{Forme des solutions: somme d'une solution particulière et de la solution générale de l'équation homogène.}
  \colchunk{$\leftrightarrows$ PC: régime libre, régime forcé; régime transitoire, régime établi.}
  \colplacechunks

  \colchunk{Principe de superposition.}
  \colchunk{}
  \colplacechunks

  \colchunk{Méthode de la variation de la constante.}
  \colchunk{}
  \colplacechunks

  \colchunk{Existence et unicité de la solution d'un problème de Cauchy.}
  \colchunk{$\leftrightarrows$ PC et SI: modélisation de circuits électriques RC, RL ou de systèmes mécaniques linéaires.}
  \colplacechunks
\end{parcolumns}


\bigskip
On reviendra sur le calcul de primitives au printemps.
\begin{center}
 \textbf{Questions de cours}
\end{center}
Pour $z\in\C\setminus \R$, fonction à valeurs complexes dont la dérivée est $t\mapsto \frac{1}{t+z}$.\newline
Cette année, volontairement, \emph{je n'ai pas formulé de théorème pour le changement de variable}. Mais les étudiants doivent savoir changer de variable pratiquement dans une intégrale.

\textbf{méthodes}\newline
Savoir-faire: on constate qu'ils conduisent au résultat sans justification théorique. 
\begin{itemize}
 \item Primitive de l'inverse d'un trinôme
 \item Primitive d'un polynôme-exponentiel (avec des coefficients indéterminés)
 \item Intégration par parties très simple
 \item Changement de variable très très simple
 \item Equ. diff. lin ordre 1 à coeff. constant avec second membre polynôme-exponentiel
\end{itemize}

\textbf{démonstrations}
\begin{itemize}
 \item Ensemble des solutions d'une eq. dif. lin d'ordre 1 homogène.
 \item Existence d'une sol d'une eq. dif. lin d'ordre 1 (variation constante).
 \item Existence et unicité pour les pbs de Cauchy 
\end{itemize}



\begin{center}
 \textbf{Prochain programme}
\end{center}

\'Equations différentielle ordre 2.\newline
Révision des techniques fondamentales de calcul algébrique, complexe, trigonométrique, en analyse.
\end{document}
