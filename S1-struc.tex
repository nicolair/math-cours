\subsection{Structures algébriques usuelles}
\begin{itshape}Le programme, strictement limité au vocabulaire décrit ci-dessous, a pour objectif de permettre une présentation unifiée
des exemples usuels. En particulier, l'étude de lois artificielles est exclue.

La notion de sous-groupe figure dans ce chapitre par commodité. Le professeur a la liberté de l'introduire plus tard.
\end{itshape}

\subsubsubsection{a) Lois de composition internes}
\begin{parcolumns}[rulebetween,distance=\parcoldist]{2}
  \colchunk{Loi de composition interne. }
  \colchunk{}
  \colplacechunks

  \colchunk{Associativité, commutativité, élément neutre, inversibilité, distributivité.}
  \colchunk{ Inversibilité et inverse du produit de deux éléments inversibles.}
  \colplacechunks

  \colchunk{Partie stable.}
  \colchunk{}
  \colplacechunks

 \end{parcolumns}

\subsubsubsection{b) Structure de groupe}
\begin{parcolumns}[rulebetween,distance=\parcoldist]{2}

  \colchunk{Groupe.}
  \colchunk{Notation $x^n$ dans un groupe multiplicatif, $nx$ dans un groupe additif.

  Exemples usuels : groupes additifs $\Z$, $\Q$, $\R$, $\C$, groupes multiplicatifs $\Q^*$, $\Q_+^*$, $\R^*$, $\R_+^*$, $\C^*$, $\U$, $\U_n$.}
  \colplacechunks

  \colchunk{Groupe des permutations d'un ensemble.}
  \colchunk{Notation $\mathfrak{S}_X$.}
  \colplacechunks

  \colchunk{Sous-groupe : définition, caractérisation.}
  \colchunk{}
  \colplacechunks
 \end{parcolumns}

\subsubsubsection{c) Structures d'anneau et de corps}
\begin{parcolumns}[rulebetween,distance=\parcoldist]{2}
  \colchunk{Anneau, corps. }
  \colchunk{Tout anneau est unitaire, tout corps est commutatif.

  Exemples usuels : $\Z$, $\Q$, $\R$, $\C$.}
  \colplacechunks

  \colchunk{Calcul dans un anneau.}
  \colchunk{Relation $a^n-b^n$ et formule du binôme si $a$ et $b$ commutent.}
  \colplacechunks

  \colchunk{Groupe des inversibles d'un anneau.}
  \colchunk{}
  \colplacechunks
 \end{parcolumns}

