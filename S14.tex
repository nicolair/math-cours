%!  pour pdfLatex
\documentclass[a4paper]{article}
\usepackage[hmargin={1.5cm,1.5cm},vmargin={2.4cm,2.4cm},headheight=13.1pt]{geometry}

\usepackage[pdftex]{graphicx,color}
%\usepackage{hyperref}

\usepackage[utf8]{inputenc}
\usepackage[T1]{fontenc}
\usepackage{lmodern}
%\usepackage[frenchb]{babel}
\usepackage[french]{babel}

\usepackage{fancyhdr}
\pagestyle{fancy}

%\usepackage{floatflt}

\usepackage{parcolumns}
\setlength{\parindent}{0pt}
\usepackage{xcolor}

%pr{\'e}sentation des compteurs de section, ...
\makeatletter
%\renewcommand{\labelenumii}{\theenumii.}
\renewcommand{\thepart}{}
\renewcommand{\thesection}{}
\renewcommand{\thesubsection}{}
\renewcommand{\thesubsubsection}{}
\makeatother

\newcommand{\subsubsubsection}[1]{\bigskip \rule[5pt]{\linewidth}{2pt} \textbf{ \color{red}{#1} } \newline \rule{\linewidth}{.1pt}}
\newlength{\parcoldist}
\setlength{\parcoldist}{1cm}

\usepackage{maths}
\newcommand{\dbf}{\leftrightarrows}
% remplace les commandes suivantes 
%\usepackage{amsmath}
%\usepackage{amssymb}
%\usepackage{amsthm}
%\usepackage{stmaryrd}

%\newcommand{\N}{\mathbb{N}}
%\newcommand{\Z}{\mathbb{Z}}
%\newcommand{\C}{\mathbb{C}}
%\newcommand{\R}{\mathbb{R}}
%\newcommand{\K}{\mathbf{K}}
%\newcommand{\Q}{\mathbb{Q}}
%\newcommand{\F}{\mathbf{F}}
%\newcommand{\U}{\mathbb{U}}

%\newcommand{\card}{\mathop{\mathrm{Card}}}
%\newcommand{\Id}{\mathop{\mathrm{Id}}}
%\newcommand{\Ker}{\mathop{\mathrm{Ker}}}
%\newcommand{\Vect}{\mathop{\mathrm{Vect}}}
%\newcommand{\cotg}{\mathop{\mathrm{cotan}}}
%\newcommand{\sh}{\mathop{\mathrm{sh}}}
%\newcommand{\ch}{\mathop{\mathrm{ch}}}
%\newcommand{\argsh}{\mathop{\mathrm{argsh}}}
%\newcommand{\argch}{\mathop{\mathrm{argch}}}
%\newcommand{\tr}{\mathop{\mathrm{tr}}}
%\newcommand{\rg}{\mathop{\mathrm{rg}}}
%\newcommand{\rang}{\mathop{\mathrm{rg}}}
%\newcommand{\Mat}{\mathop{\mathrm{Mat}}}
%\renewcommand{\Re}{\mathop{\mathrm{Re}}}
%\renewcommand{\Im}{\mathop{\mathrm{Im}}}
%\renewcommand{\th}{\mathop{\mathrm{th}}}


%En tete et pied de page
\lhead{Programme colle math}
\chead{Semaine 14 du 13/01/20 au 18/01/20}
\rhead{MPSI B Hoche}

\lfoot{\tiny{Cette création est mise à disposition selon le Contrat\\ Paternité-Partage des Conditions Initiales à l'Identique 2.0 France\\ disponible en ligne http://creativecommons.org/licenses/by-sa/2.0/fr/
} }
\rfoot{\tiny{Rémy Nicolai \jobname}}


\begin{document}
\subsection{Polynômes et fractions rationnelles (2)}
\subsubsubsection{b) Divisibilité et division euclidienne}
\begin{parcolumns}[rulebetween,distance=\parcoldist]{2}
  \colchunk{Divisibilité dans $\K [X]$, diviseurs, multiples.}
  \colchunk{Caractérisation des couples de polynômes associés.}
  \colplacechunks

  \colchunk{Théorème de la division euclidienne.}
  \colchunk{$\dbf$ I : algorithme de la division euclidienne.}
  \colplacechunks
\end{parcolumns}

\subsubsubsection{c) Fonctions polynomiales et racines}
\begin{parcolumns}[rulebetween,distance=\parcoldist]{2}
  \colchunk{Fonction polynomiale associée à un polynôme.}
  \colchunk{}
  \colplacechunks

  \colchunk{Racine (ou zéro) d'un polynôme, caractérisation en termes de divisibilité.}
  \colchunk{}
  \colplacechunks

  \colchunk{Le nombre de racines d'un polynôme non nul est majoré par son degré.}
  \colchunk{Détermination d'un polynôme par la fonction polynomiale associée.}
  \colplacechunks

  \colchunk{Multiplicité d'une racine.}
  \colchunk{Si $P (\lambda) \ne 0$, $\lambda$ est racine de $P$ de multiplicité $0$.}
  \colplacechunks

  \colchunk{Polynôme scindé. Relations entre coefficients et racines.}
  \colchunk{Aucune connaissance spécifique sur le calcul des fonctions symétriques des racines n'est exigible.}
  \colplacechunks
\end{parcolumns}

\subsubsubsection{d) Dérivation}
\begin{parcolumns}[rulebetween,distance=\parcoldist]{2}

  \colchunk{Dérivée formelle d'un polynôme.}
  \colchunk{Pour $\K = \R$, lien avec la dérivée de la fonction polynomiale associée.}
  \colplacechunks

  \colchunk{Opérations sur les polynômes dérivés : combinaison linéaire, produit. Formule de Leibniz.}
  \colchunk{}
  \colplacechunks

  \colchunk{Formule de Taylor polynomiale.}
  \colchunk{}
  \colplacechunks

  \colchunk{Caractérisation de la multiplicité d'une racine par les polynômes dérivés successifs.}
  \colchunk{}
  \colplacechunks
\end{parcolumns}


\subsection{Arithmétique dans l'ensemble des entiers relatifs}
\begin{itshape}L'objectif de ce chapitre est d'étudier les propriétés de la divisibilité des entiers et des congruences.
\end{itshape}

\subsubsubsection{a) Divisibilité et division euclidienne}
\begin{parcolumns}[rulebetween,distance=\parcoldist]{2}
  \colchunk{Divisibilité dans $\Z$, diviseurs, multiples.}
  \colchunk{Caractérisation des couples d'entiers associés.}
  \colplacechunks

  \colchunk{Théorème de la division euclidienne.}
  \colchunk{}
  \colplacechunks

 \end{parcolumns}

\subsubsubsection{b) PGCD et algorithme d'Euclide}
\begin{parcolumns}[rulebetween,distance=\parcoldist]{2}
  \colchunk{PGCD de deux entiers naturels dont l'un au moins est non nul. }
  \colchunk{Le PGCD de $a$ et $b$ est défini comme étant le plus grand élément (pour l'ordre naturel dans $\N$) de l'ensemble des diviseurs communs à $a$ et $b$.\newline
  Notation $a\wedge b$.}
  \colplacechunks

  \colchunk{Algorithme d'Euclide.}
  \colchunk{L'ensemble des diviseurs communs à $a$ et $b$ est égal à l'ensemble des diviseurs de $a\wedge b$.\newline
$a\wedge b$ est le plus grand élément (au sens de la divisibilité) de l'ensemble des diviseurs communs à $a$ et $b$.}
  \colplacechunks

  \colchunk{Extension au cas de deux entiers relatifs.}
  \colchunk{}
  \colplacechunks
  
  \colchunk{Relation de Bézout.}
  \colchunk{L'algorithme d'Euclide fournit une relation de Bézout.\newline
$\dbf$ I : algorithme d'Euclide étendu.\newline
 L'étude des idéaux de $\Z$ est hors programme.}
  \colplacechunks

  \colchunk{PPCM. }
  \colchunk{Notation $a \vee b$.\newline
Lien avec le PGCD.}
  \colplacechunks

 \end{parcolumns}

\subsubsubsection{c) Entiers premiers entre eux}
\begin{parcolumns}[rulebetween,distance=\parcoldist]{2}
  \colchunk{Couple d'entiers premiers entre eux. }
  \colchunk{}
  \colplacechunks

  \colchunk{Théorème de Bézout. }
  \colchunk{Forme irréductible d'un rationnel.}
  \colplacechunks

  \colchunk{Lemme de Gauss. }
  \colchunk{}
  \colplacechunks

  \colchunk{PGCD d'un nombre fini d'entiers, relation de Bézout. Entiers premiers entre eux dans leur ensemble, premiers entre eux deux à deux. }
  \colchunk{}
  \colplacechunks
 \end{parcolumns}

\subsubsubsection{d) Nombres premiers}
\begin{parcolumns}[rulebetween,distance=\parcoldist]{2}

  \colchunk{Nombre premier. }
  \colchunk{$\dbf$ I : crible d'Eratosthène.}
  \colplacechunks

  \colchunk{L'ensemble des nombres premiers est infini. }
  \colchunk{}
  \colplacechunks

  \colchunk{Existence et unicité de la décomposition d'un entier naturel non nul en produit de nombres premiers. }
  \colchunk{}
  \colplacechunks

  \colchunk{Pour $p$ premier, valuation $p$-adique. }
  \colchunk{Notation $v_p (n)$.

  Caractérisation de la divisibilité en termes de valuations $p$-adiques.

  Expressions du PGCD et du PPCM à l'aide des valuations $p$-adiques.}
  \colplacechunks

 \end{parcolumns}

\subsubsubsection{e) Congruences}
\begin{parcolumns}[rulebetween,distance=\parcoldist]{2}
  \colchunk{Relation de congruence modulo un entier sur $\Z$. }
  \colchunk{Notation $a \equiv b \ [n]$.}
  \colplacechunks

  \colchunk{Opérations sur les congruences : somme, produit.}
  \colchunk{Les anneaux $\Z / n \Z$ sont hors programme.}
  \colplacechunks

  \colchunk{Petit théorème de Fermat.}
  \colchunk{}
  \colplacechunks
 \end{parcolumns}


\bigskip
\begin{center}
 \textbf{Questions de cours}
 \end{center}
 \textbf{Polynômes}:\newline
Pratique de la division euclidienne. Racine et divisibilité. Polynôme scindé: relation entre coefficients et racines. Formule de Taylor.
Caractérisations de la multiplicité.

 \textbf{Arithmétique dans $\Z$}:\newline
Une question de cours orientée informatique: algorithme de calcul du ppcm preuve avec invariant et fonction de terminaison.\newline
Algorithme de calcul du pgcd, application à $\mathcal{D}(a)\cap \mathcal{D}(b) = \mathcal{D}(a\wedge b)$.\newline
Algorithme d'Euclide étendu (avec sa disposition pratique pour calculs à la main) et relation de Bezout.\newline
Théorème de Gauss. \newline
\'Etude de l'équation de Bezout. \newline
Pas de question de cours sur l'existence et l'unicité de la décomposition en facteurs premiers. Les étudiants peuvent utiliser l'expression du pgcd et du ppcm à l'aide des valuations $p$-adiques.\newline
%(Facultatif et difficile) petit théorème de Fermat dans le cas non premier avec l'indicatrice d'Euler.

\begin{center}
 \textbf{Prochain programme}
\end{center}
arithmétique polynomiale, fractions rationnelles, décomposition en éléments simples
\end{document}
