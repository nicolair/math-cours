\input pgmlatex.tex

%En tete et pied de page
\lhead{Programme 2013 maths MPSI}
\rhead{}

\lfoot{\tiny{Cette création est mise à disposition selon le Contrat\\ Paternité-Partage des Conditions Initiales à l'Identique 2.0 France\\ disponible en ligne http://creativecommons.org/licenses/by-sa/2.0/fr/
} }
\rfoot{\tiny{Rémy Nicolai \jobname}}


\begin{document}
\tableofcontents
\section{Premier semestre}
\subsection{Raisonnement et vocabulaire ensembliste}
 Ce chapitre regroupe les différents points de vocabulaire, notations et raisonnement nécessaires aux étudiants pour la conception et la rédaction efficace d’une démonstration mathématique. Ces notions doivent être introduites de manière progressive en vue d’être acquises en fin de premier semestre.
Le programme se limite strictement aux notions de base figurant ci-dessous. Toute étude systématique de la logique ou de la théorie des ensembles est hors programme.

\subsubsubsection{a) Rudiments de logique}
\begin{parcolumns}[rulebetween,distance=\parcoldist]{2}
  \colchunk{Quantificateurs}
  \colchunk{L'emploi de quantificateurs en guise d'abréviation est exclue.}
  \colplacechunks
  
  \colchunk{Implication, contraposition, équivalence}
  \colchunk{Les étudiants doivent savoir formuler la négation d’une proposition.}
  \colplacechunks
  \colchunk{Modes de raisonnement : par récurrence (faible et forte), par contraposition, par l’absurde, par analyse-synthèse.}
  \colchunk{On pourra relier le raisonnement par récurrence au fait que toute partie non vide de $\N$ possède un plus petit élément. Toute construction et toute axiomatique de N sont hors programme.\newline
  Le raisonnement par analyse-synthèse est l’occasion de préciser les notions de condition nécessaire et condition suffisante.}
  \colplacechunks

\end{parcolumns}

\subsubsubsection{b) Ensembles}
\begin{parcolumns}[rulebetween,distance=\parcoldist]{2}
  \colchunk{Ensemble, appartenance, inclusion. Sous-ensemble (ou partie).}
  \colchunk{Ensemble vide.}
  \colplacechunks
  
  \colchunk{Opérations sur les parties d’un ensemble: réunion, intersection, différence, passage au complémentaire. Produit cartésien d’un nombre fini d’ensembles.}
  \colchunk{Notation $A \setminus B$ pour la différence et $E \setminus A$, $\overline{A}$ et  $C^E_A$ pour le complémentaire.}
  \colplacechunks

  \colchunk{Ensemble des parties d’un ensemble.}
  \colchunk{Notation $\mathcal{P}(E)$.}
  \colplacechunks

\end{parcolumns}

\subsubsubsection{c) Applications et relations}
\begin{parcolumns}[rulebetween,distance=\parcoldist]{2}
  \colchunk{Application d’un ensemble dans un ensemble. Graphe d’une application.}
  \colchunk{Le point de vue est intuitif : une application de $E$ dans $F$ associe à tout élément de $E$ un unique élément de $F$. Le programme ne distingue pas les notions de fonction et d’application. Notations $\mathcal{F}(E , F )$ et $\mathcal{F}( E)$.}
  \colplacechunks
  
  \colchunk{Famille d’éléments d’un ensemble.}
  \colchunk{}
  \colplacechunks

  \colchunk{Fonction indicatrice d’une partie d’un ensemble.}
  \colchunk{Notation $1_A$.}
  \colplacechunks

  \colchunk{Restriction et prolongement.}
  \colchunk{Notation $f_{|A}$.}
  \colplacechunks

  \colchunk{Image directe.}
  \colchunk{Notation $f(A)$.}
  \colplacechunks

  \colchunk{Image réciproque.}
  \colchunk{Notation $f^{-1}(B)$. Cette notation pouvant prêter à confusion, on peut provisoirement en utiliser une autre. 
  \newline
  Pour une fonction $f$ de $E$ dans $F$, j'utilise $\Phi(A)$ au lieu de $f(A)$ pour une partie $A$ de $E$ et $\varphi(X)$ au lieu de $f^{-1}(X)$ pour une partie $X$ de $F$.}
  \colplacechunks

  \colchunk{Composition.}
  \colchunk{}
  \colplacechunks

  \colchunk{Injection, surjection. Composée de deux injections, de deux surjections.}
  \colchunk{$f\circ g$ injective implique $g$ injective. $f\circ g$ surjective implique $g$ surjective. }
  \colplacechunks
  
  \colchunk{Bijection, réciproque. Composée de deux bijections, réciproque de la composée.}
  \colchunk{Compatibilité de la notation $f^{-1}$ avec la notation d'une image réciproque.}
  \colplacechunks
  
  \colchunk{Relation binaire sur un ensemble.}
  \colchunk{}
  \colplacechunks
  
  \colchunk{Relation d’équivalence, classes d’équivalence.}
  \colchunk{La notion d’ensemble quotient est hors programme.}
  \colplacechunks
  
  \colchunk{Relations de congruence modulo un réel sur $\R$, modulo un entier sur $\Z$.}
  \colchunk{}
  \colplacechunks
  
  \colchunk{Relation d’ordre. Ordre partiel, total.}
  \colchunk{}
  \colplacechunks
  
\end{parcolumns}

\subsection{Calculs algébriques}
Ce chapitre a pour but de présenter quelques notations et techniques fondamentales de calcul algébrique.

\subsubsubsection{a) Sommes et produits}
\begin{parcolumns}[rulebetween,distance=\parcoldist]{2}
  \colchunk{Somme et produit d'une famille finie de nombres complexes.}
  \colchunk{Notations $\sum_{i\in I}a_i$, $\sum_{i=1}^na_i$, $\prod_{i\in I}a_i$, $\prod_{i=1}^na_i$.\newline
  Sommes et produits télescopiques, exemple de changements d'indice et de regroupements de termes.}  
  \colplacechunks
  
  \colchunk{Expressions simplifiées de $\sum_{k=1}^nk$, $\sum_{k=1}^nk^2$, $\sum_{k=1}^nx^k$.}
  \colchunk{}
  \colplacechunks
  
  \colchunk{Factorisation de $a^n-b^n$ pour $n\in \N^*$.}
  \colchunk{}
  \colplacechunks
  
  \colchunk{Sommes doubles. Produit de deux sommes finies, sommes triangulaires.}
  \colchunk{}
  \colplacechunks

\end{parcolumns}

\subsubsubsection{b) Coefficients binomiaux et formule du binôme}
\begin{parcolumns}[rulebetween,distance=\parcoldist]{2}
  \colchunk{Factorielle. Coefficients binomiaux.}
  \colchunk{Notation $\binom{n}{p}$.}
  \colplacechunks
  \colchunk{Relation $\binom{n}{p}=\binom{n}{n-p}$}
  \colchunk{}
  \colplacechunks
  \colchunk{Formule et triangle de Pascal.}
  \colchunk{Lien avec la méthode d'obtention des coefficients binomiaux utilisée en Première (dénombrement de chemins).}
  \colplacechunks
  \colchunk{Formule du binôme dans $\C$.}
  \colchunk{}
  \colplacechunks

  \end{parcolumns}

\subsubsubsection{c) Systèmes linéaires}
\begin{parcolumns}[rulebetween,distance=\parcoldist]{2}
  \colchunk{Système linéaire de $n$ équations à $p$ inconnues à coefficients dans $\R$ ou $\C$.}
  \colchunk{$\leftrightarrows$ PC et SI dans le cas $n=p=2$. \newline
  Interprétation géométrique: intersection de droites dans $\R^2$, de plans dans $\R^3$.}
  \colplacechunks
  
  \colchunk{Système homogène associé. Structure de l'ensemble des solutions.}
  \colchunk{}
  \colplacechunks
  
  \colchunk{Opérations élémentaires.}
  \colchunk{Notations $L_i\leftrightarrow L_j$, $L_i\leftarrow \lambda L_i$ ($\lambda \neq 0$), $L_i\leftarrow \lambda L_i +\lambda L_j$.}
  \colplacechunks

  \colchunk{Algorithme du pivot.}
  \colchunk{$\rightleftarrows$ I: pour des systèmes de taille $n>3$ ou $p>3$, on utilise l'outil informatique.}
  \colplacechunks

\end{parcolumns}

\subsection{Nombres complexes et trigonométrie}
L’objectif de ce chapitre est de consolider et d’approfondir les notions sur les nombres complexes acquises en classe de 
Terminale. Le programme combine les aspects suivants :
\begin{itemize}
 \item l’étude algébrique du corps $\C$, équations algébriques (équations du second degré, racines $n$-ièmes d'un nombre complexe) ;  
 \item l’interprétation géométrique des nombres complexes et l’utilisation des nombres complexes en géométrie plane;
 \item l’exponentielle complexe et ses applications à la trigonométrie.
\end{itemize}
Il est recommandé d’illustrer le cours par de nombreuses figures.

\subsubsubsection{a) Nombres complexes}
\begin{parcolumns}[rulebetween,distance=\parcoldist]{2}
  \colchunk{Parties réelle et imaginaire.}
  \colchunk{La construction de $\C$ n'est pas exigible.}
  \colplacechunks

  \colchunk{Opérations sur les nombres complexes.}
  \colchunk{}
  \colplacechunks

  \colchunk{Conjugaison, compatibilité avec les opérations.}
  \colchunk{}
  \colplacechunks

  \colchunk{Point du plan associé à un nombre complexe, affixe d'un point, affixe d'un vecteur.}
  \colchunk{On identifie $\C$ au plan usuel muni d'un repère orthonormé direct.}
  \colplacechunks
\end{parcolumns}

\subsubsubsection{b) Module}
\begin{parcolumns}[rulebetween,distance=\parcoldist]{2}
  \colchunk{Module}
  \colchunk{Interprétation géométrique de $|z-z'|$, cercles et disques.}
  \colplacechunks

  \colchunk{Relation $|z|^2=z\bar{z}$, module d'un produit, d'un quotient.}
  \colchunk{}
  \colplacechunks

  \colchunk{Inégalité triangulaire, cas d'égalité.}
  \colchunk{}
  \colplacechunks
\end{parcolumns}

\subsubsubsection{c) Nombres complexes de module 1 et trigonométrie}
\begin{parcolumns}[rulebetween,distance=\parcoldist]{2}
  \colchunk{Cercle trigonométrique. Paramétrisation par les fonctions circulaires.}
  \colchunk{Notation $\U$.\newline
  Les étudiants doivent savoir retrouver les formules du type $\cos(\pi-x)=-\cos x$ et résoudre des équations et inéquations trigonométriques en s'aidant du cercle trigonométrique.}
  \colplacechunks

  \colchunk{Définition de $e^{it}$ pour $t\in\R$. Exponentielle d'une somme. Formules de trigonométrie exigibles: $\cos(a\pm b)$, $\sin(a\pm b)$, $\cos(2a)$, $\sin(2a)$, $\cos a \cos b$, $\sin a \cos b$, $\sin a \sin b$.}
  \colchunk{Les étudiants doivent savoir factoriser des expressions du type $\cos p + \cos q$.}
  \colplacechunks

  \colchunk{Fonction tangente.}
  \colchunk{La fonction tangente n'a pas été introduite au lycée. Notation $\tan$.}
  \colplacechunks

  \colchunk{Formule exigible: $\tan(a\pm b)$.}
  \colchunk{}
  \colplacechunks

  \colchunk{Formules d'Euler.}
  \colchunk{Linéarisation,\newline calcul de $\sum_{k=0}^n\cos(kt)$, de $\sum_{k=0}^n\sin(kt)$.}
  \colplacechunks

  \colchunk{Formule de Moivre.}
  \colchunk{Les étudiants doivent savoir retrouver les expressions de $\cos(nt)$ et de $\sin(nt)$ en fonction de $\cos t$ et $\sin t$.}
  \colplacechunks
  \end{parcolumns}

\subsubsubsection{d) Formes trigonométriques}
\begin{parcolumns}[rulebetween,distance=\parcoldist]{2}
  \colchunk{Forme trigonométrique $re^{i\theta}$ avec $r>0$ d'un nombre complexe non nul. Arguments. Arguments d'un produit, d'un quotient.}
  \colchunk{Relation de congruence modulo $2\pi$ sur $\R$.}
  \colplacechunks

  \colchunk{Factorisation de $1\pm e^{it}$.}
  \colchunk{}
  \colplacechunks

  \colchunk{Transformation de $a\cos t + b\sin t$ en $A\cos(t-\varphi)$.}
  \colchunk{$\leftrightarrows$ PC et SI: amplitude et phase.}
  \colplacechunks
\end{parcolumns}

\subsubsubsection{e) \'Equations du second degré}
\begin{parcolumns}[rulebetween,distance=\parcoldist]{2}
  \colchunk{Résolution des équations du second degré dans $\C$.}
  \colchunk{Calcul des racines carrées d'un nombre donné sous forme algébrique.}
  \colplacechunks

  \colchunk{Somme et produit des racines.}
  \colchunk{}
  \colplacechunks

\end{parcolumns}

\subsubsubsection{f) Racines $n$-ièmes}
\begin{parcolumns}[rulebetween,distance=\parcoldist]{2}
  \colchunk{Description des racines $n$-ièmes de l'unité, d'un nombre complexe non nul donné sous forme trigonométrique.}
  \colchunk{Notation $\U_n$. Représentation géométrique.}
  \colplacechunks

\end{parcolumns}

\subsubsubsection{g) Exponentielle complexe}
\begin{parcolumns}[rulebetween,distance=\parcoldist]{2}
  \colchunk{Définition de $e^z$ pour $z$ complexe: $e^z = e^{\Re(z)}e^{i\Im(z)}$.}
  \colchunk{Notation $\exp(z)$, $e^z$.\newline
  $\leftrightarrows$ PC et SI: définition d'une impédance complexe en régime sinusoïdal.}
  \colplacechunks

  \colchunk{Exponentielle d'une somme.}
  \colchunk{}
  \colplacechunks

  \colchunk{Pour tous $z$ et $z'$ dans $\C$, $\exp(z)=\exp(z')$ si et seulement si $z-z'\in 2i\pi \Z$.}
  \colchunk{}
  \colplacechunks

  \colchunk{Résolution de l'équation $exp(z)=a$.}
  \colchunk{}
  \colplacechunks
\end{parcolumns}

\subsubsubsection{h) Interprétation géométrique des nombres complexes}
\begin{parcolumns}[rulebetween,distance=\parcoldist]{2}
  \colchunk{Interprétation géométrique du module et d'un argument de $\frac{c-b}{c-a}$.}
  \colchunk{Traduction de l'alignement, de l'orthogonalité.}
  \colplacechunks

  \colchunk{Interprétation géométrique desapplications $z\mapsto az+b$.}
  \colchunk{Similitudes directes. Cas particuliers: translation, homothéties, rotations.}
  \colplacechunks

  \colchunk{Interprétation géométrique de la conjugaison.}
  \colchunk{L'étude générale des similitudes indirectes est hors programme.}
  \colplacechunks
\end{parcolumns}


\subsection{Techniques fondamentales de calcul en analyse}
\begin{itshape}Le point de vue adopté dans ce chapitre est principalement pratique : il s’agit, en prenant appui sur les acquis du lycée, de mettre en œuvre des techniques de l’analyse, en particulier celles de majoration. Les définitions précises et les constructions rigoureuses des notions de calcul différentiel ou intégral utilisées sont différées à un chapitre ultérieur. Cette appropriation en deux temps est destinée à faciliter les apprentissages.
Les objectifs de formation sont les suivants :
\begin{itemize}
 \item  une bonne maîtrise des automatismes et du vocabulaire de base relatifs aux inégalités ;
 \item l’introduction de fonctions pour établir des inégalités ;
 \item la manipulation des fonctions classiques dont le corpus est étendu ;
 \item le calcul de dérivées et de primitives ;
 \item la mise en pratique, sur des exemples simples, de l’intégration par parties et du changement de variable ;
 \item l’application des deux points précédents aux équations différentielles.
\end{itemize}
 Les étudiants doivent connaître les principales techniques de calcul et savoir les mettre en pratique sur des cas simples. Le cours sur les équations différentielles est illustré par des exemples issus des autres disciplines scientifiques.
\end{itshape}

\setcounter{subsubsection}{0}
\subsubsection{A - Inégalités dans $\R$}
\begin{parcolumns}[rulebetween,distance=\parcoldist]{2}
  \colchunk{Relation d'ordre sur $\R$. Compatibilité avec les opérations}
  \colchunk{Exemple de majoration et de minoration de sommes, de produit et de quotient.}
  \colplacechunks
  
  \colchunk{Parties positive et négative d'un réel. Valeur absolue. Inégalité triangulaire.}
  \colchunk{Notations $x^+$, $x^-$.}
  \colplacechunks
  
  \colchunk{Intervalles de $\R$.}
  \colchunk{Interprétation sur la droite réelle d'inégalités du type $|x-a|\leq b$.}
  \colplacechunks

  \colchunk{Parties majorées, minorées, bornées.\newline Majorant, minorant, maximum (ou plus grand élément), minimum (ou plus petit élémént).}
  \colchunk{Le \og plus simple des encadrements\fg (terminologie locale) :
\begin{displaymath}
n \min(x_1, \cdots, x_n) \leq \sum_{i=1}^n x_i \leq n \max(x_1, \cdots, x_n)
\end{displaymath}
Les notions de borne supérieure ou inférieure ne seront introduites que lors du cours de présentation axiomatique de $\R$.}

  \colplacechunks
  
  \colchunk{Exemples.}
  \colchunk{Inégalité de Cauchy-Schwarz. Preuve de la divergence de la série harmonique avec des puissances de $2$ et le plus simple des encadrements.}
    
\end{parcolumns}


\subsubsection{B - Fonction de la variable réelle à valeurs réelles ou complexes}
\subsubsubsection{a) Généralités sur les fonctions}
\begin{parcolumns}[rulebetween,distance=\parcoldist]{2}
  \colchunk{Ensemble de définition}
  \colchunk{}
  \colplacechunks
    
  \colchunk{Représentation graphique d'une fonction $f$ à valeurs réelles.}
  \colchunk{Graphes des fonctions $x\mapsto f(x)+a$, $x\mapsto f(x+a)$, $x\mapsto f(a-x)$, $x\mapsto f(ax)$, $x\mapsto af(x)$.\newline
  Résolution graphique d'équations et d'inéquations du type $f(x)=\lambda$ et $f(x)\geq \lambda$.}
  \colplacechunks
    
  \colchunk{Parité, imparité, périodicité.}
  \colchunk{Interprétation géométrique de ces propriétés.}
  \colplacechunks
    
  \colchunk{Somme, produit, composée.}
  \colchunk{}
  \colplacechunks
    
  \colchunk{Monotonie (large et stricte).}
  \colchunk{}
  \colplacechunks
    
  \colchunk{Fonctions majorées, minorées, bornées.}
  \colchunk{Traduction géométrique de ces propriétés. \newline
  Une fonction est bornée si et seulement si $|f|$ est majorée.}
  \colplacechunks    
\end{parcolumns}

\subsubsubsection{b) Dérivation}
\begin{parcolumns}[rulebetween,distance=\parcoldist]{2}
  \colchunk{\'Equation de la tangente en un point.}
  \colchunk{}
  \colplacechunks

  \colchunk{Dérivée d'une combinaison linéaire, d'un produit, d'un quotient, d'une composée.}
  \colchunk{Ces résultats sont admis à ce stade.\newline
  $\rightleftarrows$ SI: étude cinématique.\newline
  $\rightleftarrows$ PC: exemples de calculs de dérivées partielles.\newline
  \`A ce stade, toute théorie sur les fonctions de plusieurs variables est hors programme.}
  \colplacechunks

  \colchunk{Caractérisation des fonctions dérivables constantes, monotones, strictement monotones sur un intervalle.}
  \colchunk{Résultat admis à ce stade. Les étudiants doivent savoir introduire des fonctions pour établir des inégalités.}
  \colplacechunks

  \colchunk{Tableau de variation.}
  \colchunk{}
  \colplacechunks

  \colchunk{Graphe d'une réciproque.}
  \colchunk{}
  \colplacechunks

  \colchunk{Dérivée d'une réciproque.}
  \colchunk{Interprétation géométrique de la dérivabilité et du calcul de la dérivée d'une bijection réciproque.}
  \colplacechunks

  \colchunk{Dérivées d'ordre supérieur.}
  \colchunk{}
  \colplacechunks
  \end{parcolumns}
  
\subsubsubsection{c) \'Etude d'une fonction}
\begin{parcolumns}[rulebetween,distance=\parcoldist]{2}
  \colchunk{Détermination des symétries et des périodicités afin de réduire le domaine d'étude, tableau de variations, asymptotes verticales et horizontales, tracé du graphe.}
  \colchunk{Application à la recherche d'extrémums et à l'obtention d'inégalités.}
  \colplacechunks
  \end{parcolumns}
  
\subsubsubsection{d) Fonctions usuelles}
\begin{parcolumns}[rulebetween,distance=\parcoldist]{2}
  \colchunk{Fonctions exponentielle, logarithme népérien, puissances.}
  \colchunk{Dérivée, variation et graphe.\newline
  Les fonctions puissances sont définies sur $\R_+^*$ et prolongées en $0$ le cas échéant. Seules les fonctions puissances entières sont en outre défines sur $\R_-^*$.\newline
  $\leftrightarrows$ SI: logarithme décimal pour la représentation des diagrammes de Bode.}
  \colplacechunks

  \colchunk{Relations $(xy)^\alpha=x^\alpha y^\alpha$, $x^{\alpha+\beta}=x^\alpha x^\beta$, $(x^\alpha)^\beta=x^{\alpha \beta}$.\newline
  Croissances comparées des fonctions logarithme, puissances et exponentielle.}
  \colchunk{}
  \colplacechunks

  \colchunk{Fonction sinus, cosinus, tangente.}
  \colchunk{$\leftrightarrows$ PC et SI}
  \colplacechunks

  \colchunk{Fonctions circulaires réciproques.}
  \colchunk{Notation $\arcsin$, $\arccos$, $\arctan$.}
  \colplacechunks

  \colchunk{Fonctions hyperboliques.}
  \colchunk{Notations $\sh$, $\ch$, $\th$. \newline
  Seule relation de trigonométrie hyperbolique exigible: $\ch^2x - \sh^2 x =1$.\newline
  Les fonctions hyperboliques réciproques sont hors programmes.}
  \colplacechunks
\end{parcolumns}


\subsubsection{C - Primitives et équations différentielles linéaires}
\subsubsubsection{a) Calcul de primitives}
\begin{parcolumns}[rulebetween,distance=\parcoldist]{2}
  \colchunk{Primitives d'une fonction définie sur un intervalle à valeurs complexes}
  \colchunk{Description de l'ensemble des primitives d'une fonction sur un intervalle connaissant l'une d'entre elles.\newline
  Les étudiants doivent savoir utiliser les primitives de $x\mapsto e^{\lambda x}$ pour calculer celles de $x\mapsto e^{a x}\cos(bx)$ et de $x\mapsto e^{a x}\sin(bx)$.\newline
  $\leftrightarrows$ PC et SI: cinématique.}
  \colplacechunks

  \colchunk{Primitives des fonctions puissances, trigonométriques et hyperboliques, exponentielle, logarithme,
  \begin{displaymath}
   x\mapsto \frac{1}{1+x^2}, x\mapsto \frac{1}{\sqrt{1-x^2}}
  \end{displaymath}}
  \colchunk{Les étudiants doivent savoir calculer les primitives des fonctions du type
  \begin{displaymath}
   x\mapsto \frac{1}{ax^2+bx+c}
  \end{displaymath}
et reconnaître les dérivées de fonctions composées.}
  \colplacechunks

  \colchunk{Dérivée de $x\mapsto \int_{x_0}^xf(t)dt$ où $f$ est continue.}
  \colchunk{Résultat admis à ce stade.}
  \colplacechunks

  \colchunk{Toute fonction continue admet des primitives.}
  \colchunk{}
  \colplacechunks

  \colchunk{Calcul d'une intégrale au moyen d'une primitive.}
  \colchunk{}
  \colplacechunks

  \colchunk{Intégration par parties pour des fonctions de classe $\mathcal{C}^1$. Changement de variable: si $\varphi$ est de classe $\mathcal{C}^1$ sur $I$ et si $f$ est continue sur $\varphi(I)$, alors, pour tous $a$ et $b$ dans $I$
  \begin{displaymath}
   \int_{\varphi(a)}^{\varphi(b)}f(x)dx = \int_a^bf(\varphi(t))\varphi'(t)dt
  \end{displaymath}  }
  \colchunk{On définit à cette occasion la classe $\mathcal{C}^1$. Application au calcul de primitives.}
  \colplacechunks
  \end{parcolumns}

\subsubsubsection{b) \'Equations différentielles linéaires du premier ordre}
\begin{parcolumns}[rulebetween,distance=\parcoldist]{2}
  \colchunk{Notion d'équation différentielle linéaire du premier ordre:
  \begin{displaymath}
   y'+a(x)y = b(x)
  \end{displaymath}
où $a$ et $b$ sont des fonctions continues définies sur un intervalle $I$ de $\R$ à valeurs réelles ou complexes.}
  \colchunk{\'Equation homogène associée. \newline
  Cas particulier où la fonction $a$ est constante.}
  \colplacechunks

  \colchunk{Résolution d'une équation homogène.}
  \colchunk{}
  \colplacechunks

  \colchunk{Forme des solutions: somme d'une solution particulière et de la solution générale de l'équation homogène.}
  \colchunk{$\leftrightarrows$ PC: régime libre, régime forcé; régime transitoire, régime établi.}
  \colplacechunks

  \colchunk{Principe de superposition.}
  \colchunk{}
  \colplacechunks

  \colchunk{Méthode de la variation de la constante.}
  \colchunk{}
  \colplacechunks

  \colchunk{Existence et unicité de la solution d'un problème de Cauchy.}
  \colchunk{$\leftrightarrows$ PC et SI: modélisation de circuits électriques RC, RL ou de systèmes mécaniques linéaires.}
  \colplacechunks
\end{parcolumns}

% fin de TFCA

\input{S1-rsn.tex}

\subsection{Limites, continuité, dérivabilité}
\begin{itshape}
Ce chapitre est divisé en deux parties, consacrées aux limites et à la continuité pour la première, au calcul différentiel
pour la seconde.\newline
Dans de nombreuses questions de nature qualitative, on visualise une fonction par son graphe. Il convient de souligner
cet aspect géométrique en ayant recours à de nombreuses figures.\newline
Les fonctions sont définies sur un intervalle $I$ de $\R$ non vide et non réduit à un point et, sauf dans les paragraphes A-e) et B-f ), sont à valeurs réelles.\newline
Dans un souci d’unification, on dit qu’une propriété portant sur une fonction $f$ définie sur $I$ est vraie au voisinage de $a$ si elle est vraie sur l’intersection de $I$ avec un intervalle ouvert centré sur $a$ si $a$ est réel, avec un intervalle $[A, +\infty[$ si $a=+\infty$, avec un intervalle $]-\infty, A]$ si $a =-\infty$.
\end{itshape}
\input{S1-LCD-limcon}
\subsubsection{C - Dérivabilité}

\subsubsubsection{a) Nombre dérivé, fonction dérivée}
\begin{parcolumns}[rulebetween,distance=\parcoldist]{2}
  \colchunk{Dérivabilité en un point, nombre dérivé.}
  \colchunk{Développement limité à l'orde 1.\newline
  Interprétation géométrique. $\leftrightarrows$ SI: identification d'un modèle de comportement au voisinage d'un point de comportement.\newline
  $\leftrightarrows$ SI: représentation de la fonction sinus cardinal au voisinage de $0$.
  $\leftrightarrows$ I: méthode de Newton.}
  \colplacechunks
  
  \colchunk{La dérivabilité entraîne la continuité.}
  \colchunk{}
  \colplacechunks

  \colchunk{Dérivabilité à gauche, à droite.}
  \colchunk{}
  \colplacechunks
  
  \colchunk{Dérivabilité et dérivée sur un intervalle.}
  \colchunk{}
  \colplacechunks

  \colchunk{Opérations sur les fonctions dérivables et les dérivées: combinaison linéaire, produit, quotient, composition, réciproque.}
  \colchunk{Tangente au graphe d'une réciproque.}
  \colplacechunks
\end{parcolumns}

\subsubsubsection{b) Extremum local et point critique}
\begin{parcolumns}[rulebetween,distance=\parcoldist]{2}
  \colchunk{Extremum local}
  \colchunk{}
  \colplacechunks

  \colchunk{Condition nécessaire en un point intérieur.}
  \colchunk{Un point critique est un zéro de la dérivée.}
  \colplacechunks
\end{parcolumns}

\subsubsubsection{c) Théorème de Rolle et des accroissements finis.}
\begin{parcolumns}[rulebetween,distance=\parcoldist]{2}
  \colchunk{Théorème de Rolle.}
  \colchunk{Utilisation pour établir l'existence de zéros d'une fonction.}
  \colplacechunks

  \colchunk{\'Egalité des accroissements finis.}
  \colchunk{Interprétations géométriques et cinématiques.}
  \colplacechunks

  \colchunk{Inégalité des accroissements finis: si $f$ est dérivable et si $|f'|$ est majorée par $K$, alors $f$ est $K$-lipschitzienne.}
  \colchunk{La notion de fonction lipschitzienne est introduite à cette occasion.\newline
  Application à l'étude des des suites définies par une relation de récurrence $u_{n+1}=f(u_n)$.}
  \colplacechunks

  \colchunk{Caractérisation des fonctions dérivables constantes, monotones, strictement monotones sur un intervalle.}
  \colchunk{}
  \colplacechunks

  \colchunk{Théorème de la limite de la dérivée: si $f$ est continue sur $I$, dérivable sur $I\setminus\{a\}$ et si $\underset{\underset{x\neq a}{x\rightarrow a}}{\lim}f'(x)=l\in\overline{\R}$, alors $\underset{x\rightarrow a}{\lim}\frac{f(x)-f(a)}{x-a}=l$.}
  \colchunk{Interprétation géométrique.\newline
  Si $l\in\R$, alors $f$ est dérivable en $a$ et $f'$ continue en $a$.}
  \colplacechunks

\end{parcolumns}

\subsubsubsection{d) Fonctions de classe $\mathcal{C}^k$}
\begin{parcolumns}[rulebetween,distance=\parcoldist]{2}
  
  \colchunk{Pour $k\in\N \cup \{\infty\}$, fonction de classe $\mathcal{C}^k$.}
  \colchunk{}
  \colplacechunks

  \colchunk{Opérations sur les fonctions de classe $\mathcal{C}^k$: combinaison linéaire, produit (formule de Leibniz), quotient, composition, réciproque.}
  \colchunk{}
  \colplacechunks

  \colchunk{Théorème de classe $\mathcal{C}^k$ par prolongement: si $f$ est de classe $\mathcal{C}^k$ sur $I\setminus\{a\}$ et si $f^{(i)}$ possède une limite finie lorsque $x$ tend vers $a$ pour tout $i\in\{0,\cdots,k\}$, alors $f$ admet un prolongement de classe $\mathcal{C}^k$ sur $I$. }
  \colchunk{}
  \colplacechunks
\end{parcolumns}

\subsubsubsection{e) Fonctions complexes}
\begin{parcolumns}[rulebetween,distance=\parcoldist]{2}
  
  \colchunk{Brève extension des définitions et résultats précédents.}
  \colchunk{Caractérisation de la dérivabilité en termes de parties réelle et imaginaire.}
  \colplacechunks

  \colchunk{Inégalité des accroissements finis pour une fonction de classe $\mathcal{C}^1$.}
  \colchunk{Le résultat, admis à ce stade sera justifé dans le chapitre \og Intégration\fg.}
  \colplacechunks
\end{parcolumns}


\subsection{Analyse asymptotique}
\begin{itshape}
 L’objectif de ce chapitre est de familiariser les étudiants avec les techniques asymptotiques de base, dans les cadres discret
et continu. Les suites et les fonctions y sont à valeurs réelles ou complexes, le cas réel jouant un rôle prépondérant.\newline
On donne la priorité à la pratique d’exercices plutôt qu’à la vérification de propriétés élémentaires relatives aux relations
de comparaison.\newline
Les étudiants doivent connaître les développements limités usuels et savoir rapidement mener à bien des calculs asymptotiques simples. En revanche, les situations dont la gestion manuelle ne relèverait que de la technicité seront traitées à
l’aide d’outils logiciels.
\end{itshape}

\subsubsubsection{a) Relations de comparaison: cas des suites}
\begin{parcolumns}[rulebetween,distance=\parcoldist]{2}
  \colchunk{Relations de domination, de négligeabilité, d'équivalence.}
  \colchunk{Notations $u_n=O(v_n)$, $u_n=o(v_n)$, $u_n\sim v_n$.\newline
  On définit ces relations à partir du quotient $\frac{u_n}{v_n}$ sous l'hypothèse que la suite $(v_n)_{n\in\N}$ ne s'annule pas à partir d'un certain rang.\newline
  Traduction à l'aide du symbole $o$ des croissances comparées des suites de termes généraux $\ln^\beta(n)$, $n^\alpha$, $e^{\gamma n}$.}
  \colplacechunks
  
  \colchunk{Liens entre les relations de comparaison.}
  \colchunk{\'Equivalence des relations $u_n\sim v_n$ et $u_n-v_n = o(v_n)$.}
  \colplacechunks

  \colchunk{Opérations sur les équivalents: produit, quotient, puissances.}
  \colchunk{}
  \colplacechunks

  \colchunk{Propriétés conservées par équivalence: signe, limite.}
  \colchunk{}
  \colplacechunks
\end{parcolumns}


\subsubsubsection{b) Relations de comparaison: cas des fonctions}
\begin{parcolumns}[rulebetween,distance=\parcoldist]{2}
  \colchunk{Adaptation aux fonctions des définitions et résultats précédents.}
  \colchunk{}
  \colplacechunks
\end{parcolumns}


\subsubsubsection{c) Développements limités}
\begin{parcolumns}[rulebetween,distance=\parcoldist]{2}
  \colchunk{Développement limité, unicité des coefficients, troncature.}
  \colchunk{Développement limité en $0$ d'une fonction paire, impaire.}
  \colplacechunks
  
  \colchunk{Forme normalisée d'un développement limité:
  \begin{displaymath}
   f(a+h) \underset{h \rightarrow a}{=} h^p\left(a_0+a_1h+\cdots+a_nh^n+o(h^n) \right)
  \end{displaymath}
  avec $a_0\neq 0$.}
  \colchunk{\'Equivalence $f(a+h)\underset{h\rightarrow 0}{\sim} a_0h^p$; signe de $f$ au voisinage de $a$.}
  \colplacechunks

  \colchunk{Opérations sur les développements limités: combinaison linéaire, produit, quotient.}
  \colchunk{Utilisation de la forme normalisée pour prévoir l'ordre d'un développement.\newline
  Les étudiants doivent savoir déterminer sur des exemples simples le développement limité d'une composée, mais aucun résultat général n'est exigible.\newline
  La division suivant les puissances croissantes est hors programme.}
  \colplacechunks

  \colchunk{Primitivation d'un développement limité.}
  \colchunk{}
  \colplacechunks

  \colchunk{Formule de Taylor-Young: développement limité à l'ordre $n$ en un point d'une fonction de classe $\mathcal{C}^n$.}
  \colchunk{La formule de Taylor-Young peut être admise à ce stade et justifiée dans le chapitre \og Intégration\fg.}
  \colplacechunks

  \colchunk{Développement limité à tout ordre en $0$ de $\exp$, $\sin$, $\cos$, $\sh$, $\ch$, $x\mapsto \ln(1+x)$, $x\mapsto (1+x)^\alpha$, $\arctan$ et de $\tan$ à l'ordre $3$.}
  \colchunk{}
  \colplacechunks

  \colchunk{Utilisation des développements limités pour préciser l'allure d'une courbe au voisinage d'un point.}
  \colchunk{Condition nécessaire, condition suffisante à l'ordre $2$ pour un extrémum local.}
  \colplacechunks
\end{parcolumns}


\subsubsubsection{d) Exemples de développements asymptotiques}
\begin{parcolumns}[rulebetween,distance=\parcoldist]{2}
  \colchunk{}
  \colchunk{La notion de développement asymptotique est présentée sur des exemples simples. \newline
  La notion d'échelle de comparaison est hors programme.}
  \colplacechunks
  
  \colchunk{Formule de Stirling.}
  \colchunk{La démonstration n'est pas exigible.}
  \colplacechunks
\end{parcolumns}

\subsection{Arithmétique dans l'ensemble des entiers relatifs}
\begin{itshape}L'objectif de ce chapitre est d'étudier les propriétés de la divisibilité des entiers et des congruences.
\end{itshape}

\subsubsubsection{a) Divisibilité et division euclidienne}
\begin{parcolumns}[rulebetween,distance=\parcoldist]{2}
  \colchunk{Divisibilité dans $\Z$, diviseurs, multiples.}
  \colchunk{Caractérisation des couples d'entiers associés.}
  \colplacechunks

  \colchunk{Théorème de la division euclidienne.}
  \colchunk{}
  \colplacechunks

 \end{parcolumns}

\subsubsubsection{b) PGCD et algorithme d'Euclide}
\begin{parcolumns}[rulebetween,distance=\parcoldist]{2}
  \colchunk{PGCD de deux entiers naturels dont l'un au moins est non nul. }
  \colchunk{Le PGCD de $a$ et $b$ est défini comme étant le plus grand élément (pour l'ordre naturel dans $\N$) de l'ensemble des diviseurs communs à $a$ et $b$.\newline
  Notation $a\wedge b$.}
  \colplacechunks

  \colchunk{Algorithme d'Euclide.}
  \colchunk{L'ensemble des diviseurs communs à $a$ et $b$ est égal à l'ensemble des diviseurs de $a\wedge b$.\newline
$a\wedge b$ est le plus grand élément (au sens de la divisibilité) de l'ensemble des diviseurs communs à $a$ et $b$.}
  \colplacechunks

  \colchunk{Extension au cas de deux entiers relatifs.}
  \colchunk{}
  \colplacechunks
  
  \colchunk{Relation de Bézout.}
  \colchunk{L'algorithme d'Euclide fournit une relation de Bézout.\newline
$\dbf$ I : algorithme d'Euclide étendu.\newline
 L'étude des idéaux de $\Z$ est hors programme.}
  \colplacechunks

  \colchunk{PPCM. }
  \colchunk{Notation $a \vee b$.\newline
Lien avec le PGCD.}
  \colplacechunks

 \end{parcolumns}

\subsubsubsection{c) Entiers premiers entre eux}
\begin{parcolumns}[rulebetween,distance=\parcoldist]{2}
  \colchunk{Couple d'entiers premiers entre eux. }
  \colchunk{}
  \colplacechunks

  \colchunk{Théorème de Bézout. }
  \colchunk{Forme irréductible d'un rationnel.}
  \colplacechunks

  \colchunk{Lemme de Gauss. }
  \colchunk{}
  \colplacechunks

  \colchunk{PGCD d'un nombre fini d'entiers, relation de Bézout. Entiers premiers entre eux dans leur ensemble, premiers entre eux deux à deux. }
  \colchunk{}
  \colplacechunks
 \end{parcolumns}

\subsubsubsection{d) Nombres premiers}
\begin{parcolumns}[rulebetween,distance=\parcoldist]{2}

  \colchunk{Nombre premier. }
  \colchunk{$\dbf$ I : crible d'Eratosthène.}
  \colplacechunks

  \colchunk{L'ensemble des nombres premiers est infini. }
  \colchunk{}
  \colplacechunks

  \colchunk{Existence et unicité de la décomposition d'un entier naturel non nul en produit de nombres premiers. }
  \colchunk{}
  \colplacechunks

  \colchunk{Pour $p$ premier, valuation $p$-adique. }
  \colchunk{Notation $v_p (n)$.

  Caractérisation de la divisibilité en termes de valuations $p$-adiques.

  Expressions du PGCD et du PPCM à l'aide des valuations $p$-adiques.}
  \colplacechunks

 \end{parcolumns}

\subsubsubsection{e) Congruences}
\begin{parcolumns}[rulebetween,distance=\parcoldist]{2}
  \colchunk{Relation de congruence modulo un entier sur $\Z$. }
  \colchunk{Notation $a \equiv b \ [n]$.}
  \colplacechunks

  \colchunk{Opérations sur les congruences : somme, produit.}
  \colchunk{Les anneaux $\Z / n \Z$ sont hors programme.}
  \colplacechunks

  \colchunk{Petit théorème de Fermat.}
  \colchunk{}
  \colplacechunks
 \end{parcolumns}

\subsection{Structures algébriques usuelles}
\begin{itshape}Le programme, strictement limité au vocabulaire décrit ci-dessous, a pour objectif de permettre une présentation unifiée
des exemples usuels. En particulier, l'étude de lois artificielles est exclue.

La notion de sous-groupe figure dans ce chapitre par commodité. Le professeur a la liberté de l'introduire plus tard.
\end{itshape}

\subsubsubsection{a) Lois de composition internes}
\begin{parcolumns}[rulebetween,distance=\parcoldist]{2}
  \colchunk{Loi de composition interne. }
  \colchunk{}
  \colplacechunks

  \colchunk{Associativité, commutativité, élément neutre, inversibilité, distributivité.}
  \colchunk{ Inversibilité et inverse du produit de deux éléments inversibles.}
  \colplacechunks

  \colchunk{Partie stable.}
  \colchunk{}
  \colplacechunks

 \end{parcolumns}

\subsubsubsection{b) Structure de groupe}
\begin{parcolumns}[rulebetween,distance=\parcoldist]{2}

  \colchunk{Groupe.}
  \colchunk{Notation $x^n$ dans un groupe multiplicatif, $nx$ dans un groupe additif.

  Exemples usuels : groupes additifs $\Z$, $\Q$, $\R$, $\C$, groupes multiplicatifs $\Q^*$, $\Q_+^*$, $\R^*$, $\R_+^*$, $\C^*$, $\U$, $\U_n$.}
  \colplacechunks

  \colchunk{Groupe des permutations d'un ensemble.}
  \colchunk{Notation $\mathfrak{S}_X$.}
  \colplacechunks

  \colchunk{Sous-groupe : définition, caractérisation.}
  \colchunk{}
  \colplacechunks
 \end{parcolumns}

\subsubsubsection{c) Structures d'anneau et de corps}
\begin{parcolumns}[rulebetween,distance=\parcoldist]{2}
  \colchunk{Anneau, corps. }
  \colchunk{Tout anneau est unitaire, tout corps est commutatif.

  Exemples usuels : $\Z$, $\Q$, $\R$, $\C$.}
  \colplacechunks

  \colchunk{Calcul dans un anneau.}
  \colchunk{Relation $a^n-b^n$ et formule du binôme si $a$ et $b$ commutent.}
  \colplacechunks

  \colchunk{Groupe des inversibles d'un anneau.}
  \colchunk{}
  \colplacechunks
 \end{parcolumns}


\subsection{Polynômes et fractions rationnelles}
\begin{itshape}L'objectif de ce chapitre est d'étudier les propriétés de base de ces objets formels et de les exploiter pour la résolution de problèmes portant sur les équations algébriques et les fonctions numériques.

L'arithmétique de $\K[X]$ est développée selon le plan déjà utilisé pour l'arithmétique de
$\Z$, ce qui autorise un exposé allégé. D'autre part, le programme se limite au cas où le corps de base $\K$
est $\R$ ou $\C$.
\end{itshape}

\subsubsubsection{a) Anneau des polynômes à une indéterminée}
\begin{parcolumns}[rulebetween,distance=\parcoldist]{2}
  \colchunk{Anneau $\K [X]$.}
  \colchunk{La contruction de $\K [X]$ n'est pas exigible.

  Notations $\displaystyle \sum_{i=0}^d a_i X^i, \sum_{i=0}^{+ \infty} a_i X^i$.}
  \colplacechunks

  \colchunk{Degré, coefficient dominant, polynôme unitaire.}
  \colchunk{Le degré du polynôme nul est $- \infty$.

  Ensemble $\K_n [X]$ des polynômes de degré au plus $n$.}
  \colplacechunks

  \colchunk{Degré d'une somme, d'un produit.}
  \colchunk{Le produit de deux polynômes non nuls est non nul.}
  \colplacechunks

  \colchunk{Composition.}
  \colchunk{$\dbf$ I : représentation informatique d'un polynôme ; somme, produit.}
  \colplacechunks
\end{parcolumns}

\subsubsubsection{b) Divisibilité et division euclidienne}
\begin{parcolumns}[rulebetween,distance=\parcoldist]{2}
  \colchunk{Divisibilité dans $\K [X]$, diviseurs, multiples.}
  \colchunk{Caractérisation des couples de polynômes associés.}
  \colplacechunks

  \colchunk{Théorème de la division euclidienne.}
  \colchunk{$\dbf$ I : algorithme de la division euclidienne.}
  \colplacechunks
\end{parcolumns}

\subsubsubsection{c) Fonctions polynomiales et racines}
\begin{parcolumns}[rulebetween,distance=\parcoldist]{2}
  \colchunk{Fonction polynomiale associée à un polynôme.}
  \colchunk{}
  \colplacechunks

  \colchunk{Racine (ou zéro) d'un polynôme, caractérisation en termes de divisibilité.}
  \colchunk{}
  \colplacechunks

  \colchunk{Le nombre de racines d'un polynôme non nul est majoré par son degré.}
  \colchunk{Détermination d'un polynôme par la fonction polynomiale associée.}
  \colplacechunks

  \colchunk{Multiplicité d'une racine.}
  \colchunk{Si $P (\lambda) \ne 0$, $\lambda$ est racine de $P$ de multiplicité $0$.}
  \colplacechunks

  \colchunk{Polynôme scindé. Relations entre coefficients et racines.}
  \colchunk{Aucune connaissance spécifique sur le calcul des fonctions symétriques des racines n'est exigible.}
  \colplacechunks
\end{parcolumns}

\subsubsubsection{d) Dérivation}
\begin{parcolumns}[rulebetween,distance=\parcoldist]{2}

  \colchunk{Dérivée formelle d'un polynôme.}
  \colchunk{Pour $\K = \R$, lien avec la dérivée de la fonction polynomiale associée.}
  \colplacechunks

  \colchunk{Opérations sur les polynômes dérivés : combinaison linéaire, produit. Formule de Leibniz.}
  \colchunk{}
  \colplacechunks

  \colchunk{Formule de Taylor polynomiale.}
  \colchunk{}
  \colplacechunks

  \colchunk{Caractérisation de la multiplicité d'une racine par les polynômes dérivés successifs.}
  \colchunk{}
  \colplacechunks
\end{parcolumns}

\subsubsubsection{e) Arithmétique dans $\K[X]$}
\begin{parcolumns}[rulebetween,distance=\parcoldist]{2}

  \colchunk{PGCD de deux polynômes dont l'un au moins est non nul.}
  \colchunk{Tout diviseur commun à $A$ et $B$ de degré maximal est appelé un PGCD de $A$  et $B$.}
  \colplacechunks

  \colchunk{Algorithme d'Euclide.}
  \colchunk{L'ensemble des diviseurs communs à $A$ et $B$ est égal à l'ensemble des diviseurs d'un de leurs PGCD. Tous les PGCD de $A$ et $B$
sont associés ; un seul est unitaire. On le note  $A\wedge B$.}
  \colplacechunks

  \colchunk{Relation de Bézout.}
  \colchunk{L'algorithme d'Euclide fournit une relation de Bézout.

$\dbf$ I : algorithme d'Euclide étendu.

L'étude des idéaux de $\K [X]$ est hors programme.}
  \colplacechunks

  \colchunk{PPCM.}
  \colchunk{Notation $A \vee B$.

  Lien avec le PGCD.}
  \colplacechunks

  \colchunk{Couple de polynômes premiers entre eux. Théorème de Bézout. Lemme de Gauss.}
  \colchunk{}
  \colplacechunks

  \colchunk{PGCD d'un nombre fini de polynômes, relation de Bézout. Polynômes premiers entre eux dans leur ensemble, premiers entre eux deux à deux.}
  \colchunk{}
  \colplacechunks

\end{parcolumns}

\subsubsubsection{f) Polynômes irréductibles de $\C[X]$ et $\R[X]$}
\begin{parcolumns}[rulebetween,distance=\parcoldist]{2}
  \colchunk{Théorème de d'Alembert-Gauss.}
  \colchunk{La démonstration est hors programme.}
  \colplacechunks

  \colchunk{Polynômes irréductibles de $\C [X]$. Théorème de décomposition en facteurs irréductibles dans $\C [X]$.}
  \colchunk{Caractérisation de la divisibilité dans $\C [X]$ à l'aide des racines et des multiplicités.

  Factorisation de $X^n-1$ dans $\C [X]$.}
  \colplacechunks

  \colchunk{Polynômes irréductibles de $\R [X]$. Théorème de décomposition en facteurs irréductibles dans $\R [X]$.}
  \colchunk{}
  \colplacechunks
\end{parcolumns}

\subsubsubsection{g) Formule d'interpolation de Lagrange}
\begin{parcolumns}[rulebetween,distance=\parcoldist]{2}
  \colchunk{Si $x_1, \ldots, x_n$ sont des éléments distincts de $\K$ et $y_1, \ldots, y_n$ des éléments de $\K$, il existe un et un seul $P \in \K_{n-1} [X]$ tel que pour tout $i$ : $\quad P (x_i) = y_i$.}
  \colchunk{Expression de $P$.

Description des polynômes $Q$ tels que pour tout $i$ : $\quad Q (x_i) = y_i$.}
  \colplacechunks
\end{parcolumns}

\subsubsubsection{h) Fractions rationnelles}
\begin{parcolumns}[rulebetween,distance=\parcoldist]{2}
  \colchunk{Corps $\K (X)$.}
  \colchunk{La construction de $\K (X)$ n'est pas exigible.}
  \colplacechunks

  \colchunk{Forme irréductible d'une fraction rationnelle. Fonction rationnelle.}
  \colchunk{}
  \colplacechunks

  \colchunk{Degré, partie entière, zéros et pôles, multiplicités.}
  \colchunk{}
  \colplacechunks
\end{parcolumns}

\subsubsubsection{i) Décomposition en éléments simples sur $\C$ et sur $\R$}
\begin{parcolumns}[rulebetween,distance=\parcoldist]{2}
  \colchunk{Existence et unicité de la décomposition en éléments simples sur $\C$ et sur $\R$.}
  \colchunk{La démonstration est hors programme.

\noindent On évitera toute technicité excessive.

\noindent La division selon les puissances croissantes est hors programme.}
  \colplacechunks

  \colchunk{Si $\lambda$ est un pôle simple, coefficient de $\dfrac{1}{X - \lambda}$.}
  \colchunk{}
  \colplacechunks

  \colchunk{Décomposition en éléments simples de $\dfrac{P'}{P}$.}
  \colchunk{}
  \colplacechunks
\end{parcolumns}


\section{Deuxième semestre}
Le programme du deuxième semestre est organisé autour de trois objectifs :
\begin{itemize}
 \item  introduire les notions fondamentales relatives à l’algèbre linéaire et aux espaces préhilbertiens ;
 \item  prolonger les chapitres d’analyse du premier semestre par l’étude de l’intégration sur un segment et des séries
   numériques, et achever ainsi la justification des résultats admis dans le chapitre \og Techniques fondamentales de
   calcul en analyse \fg ;
 \item consolider les notions relatives aux probabilités sur un univers fini introduites au lycée et enrichir le corpus des
   connaissances sur les variables aléatoires définies sur un tel univers.
\end{itemize}
Le professeur a la liberté d’organiser l’enseignement du semestre de la manière qui lui semble la mieux adaptée.

\subsection{Espaces vectoriels et applications linéaires}
\begin{itshape}Dans tout le cours d'algèbre linéaire, le coprs $\K$ est égal à $\R$ ou $\C$. 
Le programme d'algèbre linéaire est divisé en deux chapitre d'importance comparable, intitulés \og Algèbre linéaire I\fg et \og Algèbre linéaire II\fg. Le premier privilégie les objets géométriques~: espaces, sous-espaces, applications linéaires. Le second est consacré aux matrices. Cette séparation est une commodité de rédaction. Le professeur a la liberté d'organiser l'enseignement de l'algèbre linéaire de la manière qu'il estime la mieux adaptée.

Les objectifs du chapitre \og Espaces vectoriels et applications linéaires\fg sont les suivants~:
\begin{itemize}
\item acquérir les notions de base relatives aux espaces vectoiels et à l'indépendance linéaire~;
\item reconnaître les problèmes linéaires et les modéliser à l'aide des notions d'espace vectoriel et d'application linéaire~;
\item définir la notion de dimension, qui interprête le nombre de degrés de liberté d'un problème linéaire~; il convient d'insister sur les méthodes de calcul de dimension, de faire apparaître que ces méthodes reposent sur deux types de représentations~: paramétrisation linéaire d'un sous-espace, description d'un sous-espace par équations linéaires~;
\item présenter un certain nombre de notions de géométrie affine, de manière à consolider et enrichir les acquis relatifs à la partie affine de la géométrie classique du plan et de l'espace.
\end{itemize}
Il convient de souligner, à l'aide de nombreuses figures, comment l'intuition géométrique permet d'interpréter en petite dimension les notions de l'algèbre linéaire, ce qui facilite leur extension à la dimension quelconque.\end{itshape}

\setcounter{subsubsection}{0}
\subsubsection{A - Espaces vectoriels}

\subsubsubsection{Espaces vectoriels}
\begin{parcolumns}[rulebetween,distance=\parcoldist]{2}
  \colchunk{Structure de $\K$ espace vectoriel}
  \colchunk{Espaces $\K^n$, $\K[X]$.}
  \colplacechunks
  
  \colchunk{Produit d'un nombre fini d'espaces vectoriels.}
  \colchunk{}
  \colplacechunks

 \colchunk{Espace vectoriel des  fonctions d'un ensemble dans un espace vectoriel.}
  \colchunk{Espace $\K^\N$ des suites d'éléments de $\K$.}
  \colplacechunks

 \colchunk{Famille presque nulle (ou à support fini) de scalaires, combinaison linéaire  d'une famille de vecteurs.}
  \colchunk{On commence par la notion de combinaison linéaire d'une famille finie de vecteurs.}
  \colplacechunks
  
\end{parcolumns}

\subsubsubsection{Sous-espaces vectoriels}
\begin{parcolumns}[rulebetween,distance=\parcoldist]{2}

 \colchunk{Sous-espace vectoriel~: définition, caractérisation.}
  \colchunk{Sous-espace nul. Droites vectorielles de $\R^2$, droites et plans vectoriels de $\R^3$. Sous-espaces $K_n[X]$ de $\K[X]$.}
  \colplacechunks
 \colchunk{Instersection d'une famille de sous-espaces vectoriels. }
  \colchunk{}
  \colplacechunks
 \colchunk{Sous-espace vectoriel engendré par une partie $X$.}
  \colchunk{Notations $\Vect(X)$, $\Vect{(x_i)}_{i\in I}$.\\Tout sous-espace contenant $X$ contient $\Vect(X)$.}
  \colplacechunks
\end{parcolumns}

\subsubsubsection{Familles de vecteurs}
\begin{parcolumns}[rulebetween,distance=\parcoldist]{2}
 \colchunk{Familles et parties génératrices.}
  \colchunk{}
  \colplacechunks
 \colchunk{Familles et parties libres.}
  \colchunk{}
  \colplacechunks
 \colchunk{Base, coordonnées.}
  \colchunk{Bases canoniques de $K^n$, $K_n[X]$, $\K[X]$.}
  \colplacechunks

\end{parcolumns}
\subsubsubsection{Somme d'un nombre fini de sous-espaces}
\begin{parcolumns}[rulebetween,distance=\parcoldist]{2}

 \colchunk{Somme de deux sous-espaces.}
  \colchunk{}
  \colplacechunks
 \colchunk{Somme directe de deux sous-espaces. Caractérisation par l'intersection.}
  \colchunk{La somme $F+G$ est directe si la décomposition de tout vecteur de $F+G$ comme somme d'un élément de $F$ et d'un élément de $G$ est unique.}

  \colplacechunks
 \colchunk{Sous-espaces supplémentaires.}
  \colchunk{}
  \colplacechunks
 \colchunk{Somme d'un nombre fini de sous-espaces.}
  \colchunk{}
  \colplacechunks
 \colchunk{Somme directe d'un nombre fini de sous-espaces. Caractérisation par l'unicité de la décomposition du vecteur nul.}
  \colchunk{La somme $F_1+\cdots+F_p$ est directe si la décomposition de tout vecteur de $F_1+\cdots+F_p$ sous la forme $x_1+\cdots+x_p$ avec $x_i\in F_i $ est unique.}
  \colplacechunks

\end{parcolumns}

\subsubsection{B - Espaces de dimension finie}
\subsubsubsection{Existence de bases}
\begin{parcolumns}[rulebetween,distance=\parcoldist]{2}

 \colchunk{Un espace vectoriel est dit de dimension finie s'il possède une famille génératrice finie.}
 \colchunk{}
 \colplacechunks

 \colchunk{Si ${(x_i)}_{1\le i\le n}$ engendre $E$ et si ${(x_i)}_{i\in I}$ est libre pour une certaine partie $I$ de $\{1,\dots,n\}$, alors il existe une partie $J$ de $\{1,\dots,n\}$ contenant $I$ pour laquelle ${(x_j)}_{j\in J}$ est une base de $E$.}
  \colchunk{Existence de bases en dimension finie. \\ Théorème de la base extraite~: de toute famille génératrice on peut extraire une base.\\Théorème de la base incomplète~: toute famille libre peut être complétée en une base.}
  \colplacechunks
\end{parcolumns}

\subsubsubsection{Dimension d'un espace de dimension finie}
\begin{parcolumns}[rulebetween,distance=\parcoldist]{2}

 \colchunk{Dans un espace engendré par $n$ vecteurs, toute famille de $n+1$ vecteurs est liée.}
  \colchunk{}
  \colplacechunks
 \colchunk{Dimension d'un espace de dimension finie. }
  \colchunk{Dimensions de $\K^n$, de $\K_n[X]$, de l'espace des solutions d'une équation différentielle linéaire homogène d'ordre $1$, de l'espace des solutions d'une équation différentielle linéaire homogène d'ordre $2$ à coefficients constants, de l'espace des suites vérifiant une relation de récurrence linéaire homogène d'ordre $2$ à coefficients constants.}
  \colplacechunks
 \colchunk{En dimension $n$, une famille de $n$ vecteurs est une base si et seulement si elle est libre, si et seulement si elle est génératrice.}
  \colchunk{}
  \colplacechunks
 \colchunk{Dimension d'un produit fini d'espaces vectoriels de dimension finie.}
  \colchunk{}
  \colplacechunks
 \colchunk{Rang d'une famille finie de vecteurs.}
  \colchunk{Notation $\rg(x_1,\dots,x_n)$.}
  \colplacechunks

\end{parcolumns}

\subsubsubsection{Sous-espaces et dimension}
\begin{parcolumns}[rulebetween,distance=\parcoldist]{2}


 \colchunk{Dimension d'un sous-espace d'un espace de dimension finie, cas d'égalité.}
  \colchunk{Sous-espace de $\R^2$ et $\R^3$.}
  \colplacechunks
 \colchunk{Tout sous-espace d'un espace de dimension finie possède un supplémentaire.}
  \colchunk{Dimension commune des supplémentaires.}
  \colplacechunks
 \colchunk{Base adaptée à un sous-espace, à une décomposition en somme directe d'un nombre fini de sous-espaces.}
  \colchunk{}
  \colplacechunks
 \colchunk{Dimension d'une somme de deux sous-espaces~; formule de Grassmann. Caractérisation des couples de sous-espaces supplémentaires.}
  \colchunk{}
  \colplacechunks
 \colchunk{Si $F_1,\dots,F_p$ sont des sous-espaces de dimension finie, alors~: $\dim\displaystyle\sum_{i=1}^pF_i\le \displaystyle \sum_{i=1}^p\dim F_i$, avec égalité si et seulement si la somme est directe.}
  \colchunk{}
  \colplacechunks
\end{parcolumns}


\subsubsection{C - Applications linéaires}
\subsubsubsection{Généralités}
\begin{parcolumns}[rulebetween,distance=\parcoldist]{2}
 \colchunk{Application linéaire.}
  \colchunk{}
  \colplacechunks
 \colchunk{Opérations sur les applications linéaires~: combinaison linéaire, composition, réciproque. Isomorphismes.}
  \colchunk{L'ensemble $\mathcal L(E,F)$ est un espace vectoriel.\\ Bilinéarité de la composition.}
  \colplacechunks
 \colchunk{Image et image réciproque d'un sous-espace par une application linéaire. Image d'une application linéaire.}
  \colchunk{}
  \colplacechunks
 \colchunk{Noyau d'un application linéaire. Caractérisation de l'injectivité.}
  \colchunk{}
  \colplacechunks
 \colchunk{Si ${(x_i)}_{i\in I}$ est une famille génératrice de $E$ et si $u\in\mathcal L(E,F)$, alors $\Im g=\Vect(u(x_i),\,i\in I)$.}
  \colchunk{}
  \colplacechunks
 \colchunk{Image d'une base par un endomorphisme.}
  \colchunk{}
  \colplacechunks
 \colchunk{Application linéaire de rang fini, rang. Invariance par composition par un isomorphisme.}
  \colchunk{Notation $\rg(u)$.}
  \colplacechunks
\end{parcolumns}

\subsubsubsection{Endomorphismes}
\begin{parcolumns}[rulebetween,distance=\parcoldist]{2}
 \colchunk{Identité, homothétie.}
  \colchunk{notation $\Id_E$.}
  \colplacechunks
 \colchunk{Anneau $(\mathcal L(E),+,\circ)$.}
  \colchunk{Non commutativité si $\dim E\ge 2$\\ Notation $uv$ pour la composée $u\circ v$.}
  \colplacechunks
 \colchunk{projection ou projecteur, symétrie~: définition géométrique, caractérisation des endomorphismes vérifiant $p^2=p$ et $s^2=\Id$. }
  \colchunk{}
  \colplacechunks
 \colchunk{Automorphismes. Groupe linéaire.}
  \colchunk{Notation $GL(E)$.}
  \colplacechunks
\end{parcolumns}

\subsubsubsection{Détermination d'une application linéaire}
\begin{parcolumns}[rulebetween,distance=\parcoldist]{2}
 \colchunk{Si ${(e_i)}_{i\in I}$ est une base de $E$ et ${(f_i)}_{i\in I}$ une famille de vecteurs de $F$, alors il existe une et une seule application $u\in\mathcal L(E,F)$ telle que pour tout $i\in i:\,u(e_i)=f_i$.}
  \colchunk{En MPSI B, ce résultat est connu sous le nom de \og Théorème de prolongement linéaire\fg. Caractérisation de l'injectivité, de la surjectivité, de la bijectivité de $u$.}
  \colplacechunks
 \colchunk{Classification, à isomorphisme près, des espaces de dimension finie par leur dimension.}
  \colchunk{}
  \colplacechunks
 \colchunk{Une application linéaire entre deux espaces de même dimension finie est bijective si et seulement si elle est injective, si et seulement si elle est surjective.}
  \colchunk{}
  \colplacechunks
 \colchunk{Un endomorphisme d'un espace de dimension finie est inversible à gauche si et seulement si il est inversible à droite.}
  \colchunk{}
  \colplacechunks
 \colchunk{Dimension de $\mathcal L(E,F)$ si $E$ et $F$ sont de dimension finie.}
  \colchunk{}
  \colplacechunks
 \colchunk{Si $E_1,\dots,E_p$ sont des sous-espaces de $E$ tels que $E=\bigoplus_{i=1}^pE_i$ et si $u_i\in \mathcal L(E_i,F)$ pour tout $i$, alors il existe une et une seule application $u\in\mathcal L(E,F)$ telle que $u_{|E_i}=u_i$ pour tout $i$.}
  \colchunk{}
  \colplacechunks
\end{parcolumns}

\subsubsubsection{Théorème du rang}
\begin{parcolumns}[rulebetween,distance=\parcoldist]{2}
 \colchunk{Si $u\in\mathcal L(E,F)$ et si $S$ est un supplémentaire de $\Ker u$ dans $E$, alors $u$ induit un isomorphisme de $S$ sur $\Im u$.}
  \colchunk{}
  \colplacechunks
 \colchunk{Théorème du rang~: $\dim E=\dim \Ker u+\rg(u)$.}
  \colchunk{}
  \colplacechunks
\end{parcolumns}

\subsubsubsection{Formes linéaires et hyperplans.}
\begin{parcolumns}[rulebetween,distance=\parcoldist]{2}
 \colchunk{Forme linéaire.}
  \colchunk{Formes coordonnées relativement à une base.}
  \colplacechunks
 \colchunk{Hyperplan.}
  \colchunk{Un hyperplan est le noyau d'une forme linéaire non nulle. \'Equations d'un hyperplan dans une base en dimension finie.}
  \colplacechunks
 \colchunk{Si $H$ est un hyperplan de $E$, alors pour toute droite $D$ non contenue dans $H$, $E=H\oplus D$. Réciproquement, tout supplémentaire d'une droite est une hyperplan.}
  \colchunk{En dimension $n$, les hyperplans sont exactement les sous-espaces de dimension $n-1$.}
  \colplacechunks
 \colchunk{Comparaison de deux équations d'un même hyperplan.}
  \colchunk{}
  \colplacechunks
 \colchunk{Si $E$ est un espace de dimension finie $n$, l'intersection de $m$ hyperplans est de dimension au moins $n-m$. Réciproquement, tout sous-espace de $E$ de dimension $n-m$ est l'intersection de $m$ hyperplans.}
  \colchunk{droites vectorielles de $\R^2$, droites et plans vectoriels de $\R^3$.\\L'étude de la dualité est hors programme.}
  \colplacechunks
\end{parcolumns}

\subsubsection{D - Sous-espaces affines d'un espace vectoriel}

\begin{itshape}Le but de cette partie est double~:
\begin{itemize}
\item montrer comment l'algèbre linéaire permet d'étendre les notions de géométrie affine étudiées au collège et au lycée et d'utiliser l'intuition géométrique dans un cadre élargi~;
\item modéliser un problème affine par une équation $u(x)=a$ où $u$ est une application linéaire, et unifier plusieurs situations de ce type déjà rencontrées.
\end{itemize}
Cette partie du cours doit être illustrée de nombreuses figures.\bigskip \end{itshape}

\begin{parcolumns}[rulebetween,distance=\parcoldist]{2}
 \colchunk{Présentation informelle de la structure d'un espace vectoriel~: points et vecteurs.}
  \colchunk{L'écriture $B=A+\vec u$ est équivalente à la relation $\vec{AB}=\vec u$.}
  \colplacechunks
 \colchunk{Translation.}
  \colchunk{}
  \colplacechunks
 \colchunk{Sous-espace affine d'un espace vectoriel, direction. Hyperplan affine.}
  \colchunk{Sous-espace affines de $\R^2$ et $\R^3$.}
  \colplacechunks
 \colchunk{Intersection de sous-espaces affines.}
  \colchunk{}
  \colplacechunks
 \colchunk{Si $u\in\mathcal L(E,F)$, l'ensemble des solutions de l'équation $u(x)=a$ d'inconnue $x$ est soit l'ensemble vide, soit un sous-espace affine dirigé par $\Ker u$.}
  \colchunk{Retour sur les systèmes linéaires, les équations différentielles linéaires d'ordre $1$ et $2$ et la recherche de polynômes interpolateurs.\\La notion d'application affine est hors-programme.}
  \colplacechunks
 \colchunk{Repère affine, coordonnées.}
\end{parcolumns}



\subsection{Matrices}
\begin{itshape}
Les objectifs de ce chapitre sont les suivants~:
\begin{itemize}
\item introduire les matrices et le calcul matriciel~;
\item présenter les liens entre applications linéaires et matrices, de manière à exploiter les changements de registres (géométrique, numérique, formel)~;
\item étudier l'effet d'un changement de bases sur la représentation matricielle d'une application linéaire et la relation d'équivalence qui s'en déduit sur $\mat np\K$~;
\item introduire brièvement la relation de similitude sur $\matc n\K$~;
\item étudier les opérations élémentaires et les systèmes linéaires.
\end{itemize}
\end{itshape}


\setcounter{subsubsection}{0}
\subsubsection{A - Calcul Matriciel}

\subsubsubsection{a) Espaces de matrices}
\begin{parcolumns}[rulebetween,distance=2.5cm]{2}
  \colchunk{Espace vectoriel $\mat np\K$ des matrices à $n$ lignes et $p$ colonnes à coefficients dans $\K$.}
  \colchunk{}
  \colplacechunks
   \colchunk{Base canonique de $\mat np\K$.}
  \colchunk{Dimension de $\mat np\K$.}
  \colplacechunks
\end{parcolumns}

\subsubsubsection{b) Produit matriciel}
\begin{parcolumns}[rulebetween,distance=2.5cm]{2}

   \colchunk{Bilinéarité, associativité.}
  \colchunk{}
  \colplacechunks
   \colchunk{Produit d'une matrice de la base canonique de $\mat np\K$ par une matrice de la base canonique de $\mat pq\K$.}
  \colchunk{}
  \colplacechunks
   \colchunk{Anneau $\matc n\K$.}
  \colchunk{Non commutativité si $n\ge 2$. Exemples de diviseurs de zéro et de matrices nilpotentes.}
  \colplacechunks
   \colchunk{Formule du binôme.}
  \colchunk{Application au calcul de puissances.}
  \colplacechunks
   \colchunk{Matrice inversible, inverse. Groupe linéaire.}
  \colchunk{Notation $\Gl_n(\K)$.}
  \colplacechunks
   \colchunk{Produit de matrices diagonales, de matrices triangulaires supérieures, inférieures.}
  \colchunk{}
  \colplacechunks
\end{parcolumns}

\subsubsubsection{c) Transposition}
\begin{parcolumns}[rulebetween,distance=2.5cm]{2}

   \colchunk{Transposée d'une matrice.}
  \colchunk{Notations $^tA$, $A^T$}
  \colplacechunks
   \colchunk{Opérations sur les transposées~: combinaison linéaire, produit, inverse.}
  \colchunk{}
  \colplacechunks
\end{parcolumns}

\subsubsection{B - Matrices et applications linéaires}
\subsubsubsection{a) Matrice d'une application linéaire dans des bases.}
\begin{parcolumns}[rulebetween,distance=2.5cm]{2}
  \colchunk{Matrice d'une famille de vecteurs dans une base, d'une application linéaire dans un couple de bases.}
  \colchunk{Notation $\Mat _{e,f}(u)$.\\Isomorphisme $u\mapsto \Mat_{e,f}(u)$.}
  \colplacechunks
   \colchunk{Coordonnées de l'image d'un vecteur par une application linéaire.}
  \colchunk{}
  \colplacechunks
   \colchunk{Matrice d'une composée d'applications linéaires. Lien entre matrices inversibles et isomorphismes.}
  \colchunk{Cas particulier des endomorphismes.}
  \colplacechunks
\end{parcolumns}
\subsubsubsection{b) Application linéaire canoniquement associée à une matrice}
\begin{parcolumns}[rulebetween,distance=2.5cm]{2}

   \colchunk{Noyau, image et rang d'une matrice.}
  \colchunk{Les colonnes engendrent l'image, les lignes donnent un système d'équations du noyau. Une matrice carrée est inversible si et seulement si son noyau est réduit au sous-espace nul.}
  \colplacechunks
   \colchunk{Condition d'inversibilité d'une matrice triangulaire. L'inverse d'une matrice triangulaire est une matrice triangulaire.}
  \colchunk{}
  \colplacechunks
\end{parcolumns}

\subsubsubsection{c) Blocs}
\begin{parcolumns}[rulebetween,distance=2.5cm]{2}

   \colchunk{Matrice par blocs.}
  \colchunk{Interprétation géométrique.}
  \colplacechunks
   \colchunk{Théorème du produit par blocs.}
  \colchunk{La démonstration n'est pas exigible.}
  \colplacechunks
\end{parcolumns}

\subsubsection{C - Changements de bases, équivalence et similitude}
\subsubsubsection{a) Changements de bases}
\begin{parcolumns}[rulebetween,distance=2.5cm]{2}
   \colchunk{Matrice de passage d'une base à une autre.}
  \colchunk{La matrice de passage $P_e^{e'}$ de $e$ à $e'$ est la matrice de la famille $e'$ dans la base $e$.\\Inversibilité et inverse de $P_e^{e'}$.}
  \colplacechunks
   \colchunk{Effet d'un changement de base sur les coordonnées d'un vecteur, sur la matrice d'une application linéaire.}
  \colchunk{}
  \colplacechunks
\end{parcolumns}

\subsubsubsection{b) Matrices équivalentes et rang}
\begin{parcolumns}[rulebetween,distance=2.5cm]{2}
   \colchunk{Si $u\in\mathcal L(E,F)$ est de rang $r$, il existe une base $e$ de $E$ et une base $f$ de $F$ telles que $\Mat_{e,f}(u)=J_r$.}
  \colchunk{La matrice $J_r$ a tous ses coefficients nuls à l'exception des $r$ premiers coefficients diagonaux, égaux à $1$.}
  \colplacechunks
   \colchunk{Matrices équivalentes.}
  \colchunk{Interprétation géométrique.}
  \colplacechunks
   \colchunk{Une matrice est de rang $r$ si et seulement si elle est équivalente à $J_r$.}
  \colchunk{Classification des matrices équivalentes par le rang.}
  \colplacechunks
   \colchunk{Invariance du rang par transposition.}
  \colchunk{}
  \colplacechunks
   \colchunk{Rang d'une matrice extraite. Caractérisation du rang par les matrices carrées extraites.}
  \colchunk{}
  \colplacechunks
\end{parcolumns}

\subsubsubsection{c) Matrices semblables et trace}
\begin{parcolumns}[rulebetween,distance=2.5cm]{2}
   \colchunk{Matrices semblables.}
  \colchunk{Interprétation géométrique.}
  \colplacechunks
   \colchunk{Trace d'une matrice carrée.}
  \colchunk{}
  \colplacechunks
   \colchunk{Linéarité de la trace, relation $\tr(AB)=\tr(BA)$, invariance par similitude.}
  \colchunk{Notations $\tr(A)$, $\Tr(A)$}
  \colplacechunks
   \colchunk{Trace d'un endomorphisme d'un espace de dimension finie. Linéarité, relation $\tr(uv)=\tr(vu)$.}
  \colchunk{Notations $\tr(u)$, $\Tr(u)$.\\ Trace d'un projecteur.}
  \colplacechunks
\end{parcolumns}

\subsubsection{D - Opérations élémentaires et systèmes linéaires.}
\subsubsubsection{a) Opérations élémentaires}
\begin{parcolumns}[rulebetween,distance=2.5cm]{2}

   \colchunk{Interprétation en termes de produit matriciel. }
  \colchunk{Les opérations élémentaires sont décrites dans le paragraphe \og Systèmes linéaires \fg du chapitre \og Calculs algébriques \fg.}
  \colplacechunks
   \colchunk{Les opérations élémentaires sur les colonnes (resp. lignes) conservent l'image (resp. noyau). Les opérations élémentaires conservent le rang.}
  \colchunk{Application au calcul du rang et à l'inversion de matrices.}
  \colplacechunks
\end{parcolumns}

\subsubsubsection{b) Systèmes linéaires}
\begin{parcolumns}[rulebetween,distance=2.5cm]{2}
   \colchunk{\'Ecriture matricielle d'un système linéaire.}
  \colchunk{Interprétation géométrique~: intersection d'hyperplans affines.}
  \colplacechunks
   \colchunk{Systèmes homogène associé. Rang, dimension de l'espace des solutions.}
  \colchunk{}
  \colplacechunks
   \colchunk{Compatibilité d'un système linéaire. Structure affine de l'espace des solutions.}
  \colchunk{}
  \colplacechunks
   \colchunk{Le système carré $Ax=b$ d'inconnue $x$ possède une et une seule solution si et seulement si $A$ est inversible. Système de Cramer.}
  \colchunk{Le théorème de Rouché-Fontené et les matrices bordantes sont hors programme.}
  \colplacechunks
   \colchunk{\S Algorithme du pivot de Gauss}
  \colchunk{}
 \end{parcolumns}

% fin de MAT

\subsection{Groupe symétrique et déterminants}
\setcounter{subsubsection}{0}
\subsubsection{A - Groupe symétrique}
\begin{itshape}
Le groupe symétrique est introduit exclusivement en vue de l'étude des déterminants.
\end{itshape}

\subsubsubsection{a) Généralités}
\begin{parcolumns}[rulebetween,distance=2.5cm]{2}
  \colchunk{Groupe des permutations de l'ensemble $\big\{ 1 , \ldots , n \big\}$.}
  \colchunk{Notation $S_n$.}
  \colplacechunks

  \colchunk{Cycle, transposition.}
  \colchunk{Notation $(a_1 \ a_2 \ \ldots \ a_p)$.}
  \colplacechunks

  \colchunk{Décomposition d'une permutation en produit de cycles à supports disjoints : existence et unicité.}
  \colchunk{La démonstration n'est pas exigible, mais les étudiants doivent savoir décomposer une permutation.

  Commutativité de la décomposition.}
  \colplacechunks
\end{parcolumns}


\subsubsubsection{b) Signature d'une permutation}
\begin{parcolumns}[rulebetween,distance=2.5cm]{2}
  \colchunk{Tout élément de $S_n$ est un produit de transpositions.}
  \colchunk{}
  \colplacechunks

  \colchunk{Signature : il existe une et une seule application $\varepsilon$ de $S_n$ dans $\{ -1 , 1 \}$ telle que $\varepsilon (\tau) = -1$ pour toute transposition $\tau$ et $\varepsilon (\sigma \sigma') = \varepsilon (\sigma) \varepsilon (\sigma')$ pour toutes permutations $\sigma$ et $\sigma'$.}
  \colchunk{La démonstration n'est pas exigible.}
  \colplacechunks

\end{parcolumns}

\subsubsection{B - Déterminants}
\begin{itshape}
Les objectifs de ce chapitre sont les suivants :
\begin{itemize}
\item  introduire la notion de déterminant d'une famille de vecteurs, en
motivant sa construction par la géométrie;
\item  établir les principales propriétés des déterminants des matrices carrées et des endomorphismes;
\item  indiquer quelques méthodes simples de calcul de déterminants.
\end{itemize}

Dans tout ce chapitre, $E$ désigne un espace vectoriel de dimension finie
$n\geqslant 1$.
\end{itshape}

\subsubsubsection{a) Formes $n$-linéaires alternées}
\begin{parcolumns}[rulebetween,distance=2.5cm]{2}
  \colchunk{Forme $n$-linéaire alternée.}
  \colchunk{La définition est motivée par les notions intuitives d'aire et de volume algébriques, en s'appuyant sur des figures.}
  \colplacechunks

  \colchunk{Antisymétrie, effet d'une permutation.}
  \colchunk{Si $f$ est une forme $n$-linéaire alternée et si $(x_1 , \ldots , x_n)$ est une famille liée, alors $f (x_1 , \ldots , x_n) = 0$.}
  \colplacechunks
\end{parcolumns}


\subsubsubsection{b) Déterminant d'une famille de vecteurs dans une base}
\begin{parcolumns}[rulebetween,distance=2.5cm]{2}
  \colchunk{Si $e$ est une base, il existe une et une seule forme $n$-linéaire alternée $f$ pour laquelle $f (e) = 1$. Toute forme $n$-linéaire alternée est un multiple de $\det_e$.}
  \colchunk{Notation $\det_e$.

  La démonstration de l'existence n'est pas exigible.}
  \colplacechunks

  \colchunk{Expression du déterminant dans une base en fonction des coordonnées.}
  \colchunk{Dans $\R^2$ (resp. $\R^3$), interprétation du déterminant dans la base canonique comme aire orientée (resp. volume orienté) d'un parallélogramme (resp. parallélépipède).}
  \colplacechunks

  \colchunk{Comparaison, si $e$ et $e'$ sont deux bases, de $\det_e$ et $\det_{e'}$.}
  \colchunk{}
  \colplacechunks

  \colchunk{La famille $(x_1 , \ldots , x_n)$ est une base si et seulement si $\det_e (x_1 , \ldots , x_n) \ne 0$.}
  \colchunk{}
  \colplacechunks

  \colchunk{Orientation d'un espace vectoriel réel de dimension finie.}
  \colchunk{$\dbf$ PC : orientation d'un espace de dimension $3$.}
  \colplacechunks
\end{parcolumns}

\subsubsubsection{c) Déterminant d'un endomorphisme}
\begin{parcolumns}[rulebetween,distance=2.5cm]{2}
  \colchunk{Déterminant d'un endomorphisme.}
  \colchunk{}
  \colplacechunks

  \colchunk{Déterminant d'une composée.}
  \colchunk{Caractérisation des automorphismes.}
  \colplacechunks
\end{parcolumns}

\subsubsubsection{d) Déterminant d'une matrice carrée}
\begin{parcolumns}[rulebetween,distance=2.5cm]{2}
  \colchunk{Déterminant d'une matrice carrée.}
  \colchunk{}
  \colplacechunks

  \colchunk{Déterminant d'un produit.}
  \colchunk{Relation $\det (\lambda A) = \lambda^n \det (A)$.

  Caractérisation des matrices inversibles.}
  \colplacechunks

  \colchunk{Déterminant d'une transposée.}
  \colchunk{}
  \colplacechunks
\end{parcolumns}

\subsubsubsection{e) Calcul des déterminants}
\begin{parcolumns}[rulebetween,distance=2.5cm]{2}
  \colchunk{Effet des opérations élémentaires.}
  \colchunk{}
  \colplacechunks

  \colchunk{Cofacteur. Développement par rapport à une ligne ou une colonne.}
  \colchunk{}
  \colplacechunks

  \colchunk{Déterminant d'une matrice triangulaire par blocs, d'une matrice triangulaire.}
  \colchunk{}
  \colplacechunks

  \colchunk{Déterminant de Vandermonde.}
  \colchunk{}
  \colplacechunks
\end{parcolumns}

\subsubsubsection{f) Comatrice}
\begin{parcolumns}[rulebetween,distance=2.5cm]{2}
  \colchunk{Comatrice.}
  \colchunk{Notation $\mbox{Com} (A)$.}
  \colplacechunks

  \colchunk{Relation  $A\,  {}^t \mbox{Com} (A) = {}^t \mbox{Com} (A) A = \det (A) I_n$.}
  \colchunk{Expression de l'inverse d'une matrice inversible.}
  \colplacechunks
\end{parcolumns}

% fin de GSD

\subsection{Espaces préhilbertiens réels}
\begin{itshape}
La notion de produit scalaire a été étudiée d'un point de vue élémentaire
dans l'enseignement secondaire. Les objectifs  de ce chapitre sont les suivants :
\begin{itemize}
\item
généraliser cette notion et  exploiter, principalement à travers l'étude des projections orthogonales, l'intuition acquise dans des situations géométriques en dimension $2$ ou $3$ pour traiter des problèmes posés dans un contexte plus abstrait ;
\item
approfondir l'étude de la géométrie euclidienne du plan, notamment à travers l'étude des isométries vectorielles.
\end{itemize}
Le cours doit être illustré par de nombreuses figures. Dans toute la suite, $E$ est un espace vectoriel réel.
\end{itshape}

\subsubsubsection{a) Produit scalaire}
\begin{parcolumns}[rulebetween,distance=2.5cm]{2}
  \colchunk{Produit scalaire.}
  \colchunk{Notations $\langle x , y \rangle$, $(x|y)$, $x \cdot y$.}
  \colplacechunks

  \colchunk{Produit scalaire canonique sur $\R^n$,\newline
produit scalaire $(f |g)=\int_a^b fg$ sur $\mathcal{C} \big( [ a , b ] , \R \big)$.}
  \colchunk{}
  \colplacechunks
\end{parcolumns}

\subsubsubsection{b) Norme associée à un produit scalaire}
\begin{parcolumns}[rulebetween,distance=2.5cm]{2}
  \colchunk{Norme associée à un produit scalaire, distance.}
  \colchunk{}
  \colplacechunks

  \colchunk{Inégalité de Cauchy-Schwarz, cas d'égalité.}
  \colchunk{Exemples : sommes finies, intégrales.}
  \colplacechunks
  
  \colchunk{Inégalité triangulaire, cas d'égalité.}
  \colchunk{}
  \colplacechunks

  \colchunk{Formule de polarisation :
\begin{displaymath}
 2 \big< x , y \big> = \| x+y \|^2 - \| x \|^2 - \| y \|^2 
\end{displaymath}}
  \colchunk{}
  \colplacechunks
  \end{parcolumns}

\subsubsubsection{c) Orthogonalité}
\begin{parcolumns}[rulebetween,distance=2.5cm]{2}
  \colchunk{Vecteurs orthogonaux, orthogonal d'une partie.}
  \colchunk{Notation $X^\perp$.

  L'orthogonal d'une partie est un sous-espace.}
  \colplacechunks

  \colchunk{Famille orthogonale, orthonormale (ou orthonormée).}
  \colchunk{}
  \colplacechunks
  
  \colchunk{Toute famille orthogonale de vecteurs non nuls est libre.}
  \colchunk{}
  \colplacechunks

  \colchunk{Théorème de Pythagore.}
  \colchunk{}
  \colplacechunks
  
  \colchunk{Algorithme d'orthonormalisation de Schmidt.}
  \colchunk{}
  \colplacechunks
\end{parcolumns}

\subsubsubsection{d) Bases orthonormales}
\begin{parcolumns}[rulebetween,distance=2.5cm]{2}
  \colchunk{Existence de bases orthonormales dans un espace euclidien. Théorème de la base orthonormale incomplète.}
  \colchunk{}
  \colplacechunks

  \colchunk{Coordonnées dans une base orthonormale, expressions du produit scalaire et de la norme.}
  \colchunk{$\dbf$ PC et SI : mécanique et électricité.}
  \colplacechunks
  
  \colchunk{Produit mixte dans un espace euclidien orienté.}
  \colchunk{Notation $\big[ x_1 , \ldots , x_n \big]$.

  Interprétation géométrique en termes de volume orienté, effet d'une application linéaire.}
  \colplacechunks
\end{parcolumns}

\subsubsubsection{e) Projection orthogonale sur un sous-espace de dimension finie}
\begin{parcolumns}[rulebetween,distance=2.5cm]{2}
  \colchunk{Supplémentaire orthogonal d'un sous-espace de dimension finie.}
  \colchunk{En dimension finie, dimension de l'orthogonal.}
  \colplacechunks

  \colchunk{Projection orthogonale. Expression du projeté orthogonal dans une base orthonormale.}
  \colchunk{}
  \colplacechunks
  
  \colchunk{Distance d'un vecteur à un sous-espace. Le projeté orthogonal de $x$ sur $V$ est l'unique élément de $V$ qui minimise la distance de $x$ à $V$.}
  \colchunk{Notation $d (x , V)$.}
  \colplacechunks
\end{parcolumns}

\subsubsubsection{f) Hyperplans affines d'un espace euclidien}
\begin{parcolumns}[rulebetween,distance=2.5cm]{2}
  \colchunk{Vecteur normal à un hyperplan affine d'un espace euclidien. Si l'espace est orienté, orientation d'un hyperplan par un vecteur normal.}
  \colchunk{Lignes de niveau de $M \mapsto \overrightarrow{AM} \cdot \vec{n}$.}
  \colplacechunks

  \colchunk{\'Equations d'un hyperplan affine dans un repère orthonormal.}
  \colchunk{Cas particuliers de $\R^2$ et $\R^3$.}
  \colplacechunks
  
  \colchunk{Distance à un hyperplan affine défini par un point $A$ et un vecteur normal unitaire $\vec{n}$ : $\quad \big| \overrightarrow{AM} \cdot \vec{n} \big|$.}
  \colchunk{Cas particuliers de $\R^2$ et $\R^3$.}
  \colplacechunks
\end{parcolumns}

\subsubsubsection{g) Isométries vectorielles d'un espace euclidien}
\begin{parcolumns}[rulebetween,distance=2.5cm]{2}
  \colchunk{Isométrie vectorielle (ou automorphisme orthogonal) : définition par la linéarité et la conservation des normes, caractérisation par la conservation du produit scalaire, caractérisation par l'image d'une base orthonormale.}
  \colchunk{}
  \colplacechunks
  
  \colchunk{Symétrie orthogonale, réflexion.}
  \colchunk{}
  \colplacechunks


  \colchunk{Groupe orthogonal.}
  \colchunk{Notation $\mbox{O} (E)$.}
  \colplacechunks
\end{parcolumns}

\subsubsubsection{h) Matrices orthogonales}
\begin{parcolumns}[rulebetween,distance=2.5cm]{2}
  \colchunk{Matrice orthogonale : définition ${}^t\!\! A A = I_n$, caractérisation par le caractère orthonormal de la famille des colonnes, des lignes.}
  \colchunk{}
  \colplacechunks

  \colchunk{Groupe orthogonal.}
  \colchunk{Notations $\mbox{O}_n (\R)$, $\mbox{O} (n)$.}
  \colplacechunks
  
  \colchunk{Lien entre les notions de base orthonormale, isométrie et matrice orthogonale.}
  \colchunk{}
  \colplacechunks

  \colchunk{Déterminant d'une matrice orthogonale, d'une isométrie. Matrice orthogonale positive, négative ; isométrie positive, négative.}
  \colchunk{}
  \colplacechunks
  
  \colchunk{Groupe spécial orthogonal.}
  \colchunk{Notations $\mbox{SO} (E)$, $\mbox{SO}_n (\R)$, $\mbox{SO} (n)$.}
  \colplacechunks
\end{parcolumns}

\subsubsubsection{i) Isométries vectorielles en dimension 2}
\begin{parcolumns}[rulebetween,distance=2.5cm]{2}
  \colchunk{Description des matrices orthogonales et orthogonales positives de taille 2.}
  \colchunk{Lien entre les éléments de $\mbox{SO}_2 (\R)$ et les nombres complexes de module 1.}
  \colplacechunks

  \colchunk{Rotation vectorielle d'un plan euclidien orienté.}
  \colchunk{On introduira à cette occasion, sans soulever de difficulté sur la notion d'angle, la notion de mesure d'un angle orienté de vecteurs.

$\dbf$ SI : mécanique.}
  \colplacechunks
  
  \colchunk{Classification des isométries d'un plan euclidien orienté.}
  \colchunk{}
  \colplacechunks
\end{parcolumns}

\subsection{Intégration}
\begin{itshape}
L'objectif majeur de ce chapitre est de définir l'intégrale d'une fonction continue par morceaux sur un segment à valeurs réelles ou complexes et d'en établir les propriétés élémentaires, notamment le lien entre intégration et primitivation. On achève ainsi la justification des propriétés présentées dans le chapitre \og Techniques fondamentales de calcul en analyse\fg.

Ce chapitre permet de consolider la pratique des techniques usuelles de calcul intégral. Il peut également offrir l'occasion de revenir sur l'étude des équations différentielles rencontrées au premier semestre.


La notion de continuité uniforme est introduite uniquement en vue de la construction de l'intégrale. L'étude systématique des fonctions uniformément continues est exclue.

Dans tout le chapitre, \,$\K$ désigne \, $\R$ ou \,$\C$.
\end{itshape}

\subsubsubsection{a) Continuité uniforme}
\begin{parcolumns}[rulebetween,distance=\parcoldist]{2}
  \colchunk{Continuité uniforme.}
  \colchunk{}
  \colplacechunks

  \colchunk{Théorème de Heine.}
  \colchunk{La démonstration n'est pas exigible.}
  \colplacechunks
\end{parcolumns}


\subsubsubsection{b) Fonctions continues par morceaux}
\begin{parcolumns}[rulebetween,distance=\parcoldist]{2}
  \colchunk{Subdivision d'un segment, pas d'une subdivision.}
  \colchunk{}
  \colplacechunks

  \colchunk{Fonction en escalier.}
  \colchunk{}
  \colplacechunks

  \colchunk{Fonction continue par morceaux sur un segment, sur un intervalle.}
  \colchunk{Une fonction est continue par morceaux sur un intervalle $I$ si sa restriction à tout segment inclus dans $I$ est continue par morceaux.}
  \colplacechunks
\end{parcolumns}

\subsubsubsection{c) Intégrale d'une fonction continue par morceaux sur un segment}
\begin{parcolumns}[rulebetween,distance=\parcoldist]{2}
  \colchunk{Intégrale d'une fonction continue par morceaux sur un segment.}
  \colchunk{Le programme n'impose pas de construction particulière.

Interprétation géométrique.

$\dbf$ PC et SI : valeur moyenne.

Aucune difficulté théorique relative à la notion d'aire ne doit être soulevée.

Notations $\displaystyle \int_{[a,b]}f$, $\displaystyle \int_{a}^bf$, $\displaystyle\int_a^bf(t)\,dt$.}
  \colplacechunks

  \colchunk{Linéarité, positivité et croissance de l'intégrale.}
  \colchunk{Les étudiants doivent savoir majorer et minorer des intégrales.}
  \colplacechunks

  \colchunk{Inégalité: $\displaystyle \left\lvert \int_{[a,b]}f \right\rvert \leqslant \int_{[a,b]} \lvert f \rvert$.}
  \colchunk{}
  \colplacechunks

  \colchunk{Relation de Chasles.}
  \colchunk{Extension de la notation $\displaystyle\int_a^bf(t)\,dt$ au cas où $b\leqslant a$. Propriétés correspondantes.}
  \colplacechunks

  \colchunk{L'intégrale sur un segment d'une fonction continue de signe constant est nulle si et seulement si la fonction est nulle.}
  \colchunk{}
  \colplacechunks
\end{parcolumns}

\subsubsubsection{d) Sommes de Riemann}
\begin{parcolumns}[rulebetween,distance=\parcoldist]{2}
  \colchunk{Si $f$ est une fonction continue par morceaux sur le segment $[a,b]$ à valeurs dans $\R$, alors
\[
\frac{b-a}{n}\sum_{k=0}^{n-1}f\left(a+k\frac{b-a}{n}\right) \xrightarrow[n \to +\infty]{} \int_a^bf(t)\,dt.
\]
}
  \colchunk{Interprétation géométrique.

Démonstration dans le cas où $f$ est de classe $\mathcal{C}^1$.

$\dbf$ I : méthodes des rectangles, des trapèzes.}
  \colplacechunks
\end{parcolumns}

\subsubsubsection{e) Intégrale fonction de sa borne supérieure}
\begin{parcolumns}[rulebetween,distance=\parcoldist]{2}
  \colchunk{Dérivation de $\displaystyle x \mapsto \int_a^x f(t)\, dt$ pour $f$ continue. Calcul d'une intégrale au moyen d'une primitive. Toute fonction continue sur un intervalle possède des primitives.}
  \colchunk{Intégration par parties, changement de variable.}
  \colplacechunks
\end{parcolumns}

\subsubsubsection{f) Calcul de primitives}
\begin{parcolumns}[rulebetween,distance=\parcoldist]{2}
  \colchunk{Primitives usuelles.}
  \colchunk{Sont exigibles les seules primitives mentionnées dans le chapitre \og Techniques fondamentales de calcul en analyse\fg.}
  \colplacechunks

  \colchunk{Calcul de primitives par intégration par parties, par changement de variable.}
  \colchunk{}
  \colplacechunks

  \colchunk{Utilisation de la décomposition en éléments simples pour calculer les primitives d'une fraction rationnelle.}
  \colchunk{On évitera tout excès de technicité.}
  \colplacechunks
\end{parcolumns}

\subsubsubsection{g) Formules de Taylor}
\begin{parcolumns}[rulebetween,distance=\parcoldist]{2}
  \colchunk{Pour une fonction $f$ de classe $\mathcal{C}^{n+1}$, formule de Taylor avec reste intégral au point $a$ à l'ordre $n$.}
  \colchunk{}
  \colplacechunks

  \colchunk{Inégalité de Taylor-Lagrange pour une fonction de classe $\mathcal{C}^{n+1}$.}
  \colchunk{L'égalité de Taylor-Lagrange est hors programme.}
  \colplacechunks

  \colchunk{}
  \colchunk{On soulignera la différence de nature entre la formule de Taylor-Young (locale) et les formules de Taylor globales (reste intégral et inégalité de Taylor-Lagrange).}
  \colplacechunks
\end{parcolumns}

\subsection{Séries numériques}
\begin{itshape}
L'étude des séries prolonge celle des suites. Elle permet d'illustrer le chapitre \og Analyse  asymptotique\fg
$\;$ et, à travers la notion de développement décimal  de mieux appréhender les nombres réels.

L'objectif majeur est la maîtrise de la convergence absolue ; tout excès de technicité est exclu.
\end{itshape}

\subsubsubsection{a) Généralités}
\begin{parcolumns}[rulebetween,distance=\parcoldist]{2}
  \colchunk{Sommes partielles. Convergence, divergence. Somme et restes d'une série convergente.}
  \colchunk{La série est notée $\displaystyle{\sum u_n}$. En cas de convergence, sa somme est notée $\displaystyle\sum_{n=0}^{+\infty}u_n$.}
  \colplacechunks

  \colchunk{Linéarité de la somme.}
  \colchunk{}
  \colplacechunks

  \colchunk{Le terme général d'une série convergente tend vers 0.}
  \colchunk{Divergence grossière.}
  \colplacechunks

  \colchunk{Séries géométriques : condition nécessaire et suffisante de convergence, somme.}
  \colchunk{}
  \colplacechunks

  \colchunk{Lien suite-série.}
  \colchunk{La suite $(u_n)$ et la série $\displaystyle \displaystyle\sum_{}^{}(u_{n+1}~-~u_n)$ ont même nature.}
  \colplacechunks
\end{parcolumns}

\subsubsubsection{b) Séries à termes positifs}
\begin{parcolumns}[rulebetween,distance=\parcoldist]{2}
  \colchunk{Une série à termes positifs converge si et seulement si la suite de ses sommes partielles est majorée.}
  \colchunk{}
  \colplacechunks

  \colchunk{Si $0 \leqslant u_n \leqslant v_n$ pour tout $n$, la convergence de $\displaystyle \sum v_n$ implique celle de $\displaystyle \sum u_n$.}
  \colchunk{}
  \colplacechunks

  \colchunk{Si $(u_n)_{n \in \N}$ et $(v_n)_{n \in \N}$ sont positives et si $u_n \sim v_n$, les séries $\displaystyle \sum u_n$ et $\displaystyle \sum v_n$ ont même nature.}
  \colchunk{}
  \colplacechunks
\end{parcolumns}

\subsubsubsection{c) Comparaison série-intégrale dans le cas monotone}
\begin{parcolumns}[rulebetween,distance=\parcoldist]{2}
  \colchunk{Si $f$ est monotone, encadrement des sommes partielles de $\displaystyle \sum f (n)$ à l'aide de la méthode des rectangles.}
  \colchunk{Application à l'étude de sommes partielles et de restes.}
  \colplacechunks

  \colchunk{Séries de Riemann.}
  \colchunk{}
  \colplacechunks
\end{parcolumns}

\subsubsubsection{d) Séries absolument convergentes}
\begin{parcolumns}[rulebetween,distance=\parcoldist]{2}
  \colchunk{Convergence absolue.}
  \colchunk{}
  \colplacechunks

  \colchunk{La convergence absolue implique la convergence.}
  \colchunk{Le critère de Cauchy est hors programme. La convergence de la série absolument convergente $\sum u_n$ est établie à partir de celles de $\sum {u_n}^+$ et $\sum {u_n}^-$.}
  \colplacechunks

  \colchunk{Si $(u_n)$ est une suite complexe, si $(v_n)$ est une suite d'éléments de $\R^+$, si $u_n=O(v_n)$ et si $\displaystyle{\sum v_n}$ converge, alors $\displaystyle{\sum u_n}$ est absolument convergente donc convergente.}
  \colchunk{}
  \colplacechunks
\end{parcolumns}

\subsubsubsection{e) Représentation décimale des réels}
\begin{parcolumns}[rulebetween,distance=\parcoldist]{2}
  \colchunk{Existence et unicité du développement décimal propre d'un réel.}
  \colchunk{La démonstration n'est pas exigible.}
  \colplacechunks

  \colchunk{}
  \colchunk{}
  \colplacechunks

  \colchunk{}
  \colchunk{}
  \colplacechunks

  \colchunk{}
  \colchunk{}
  \colplacechunks
\end{parcolumns}

\subsection{Dénombrement}
\begin{itshape}
Ce chapitre est introduit essentiellement en vue de son utilisation en probabilités ; rattaché aux mathématiques discrètes, le dénombrement interagit également avec l'algèbre et l'informatique. Il permet de modéliser certaines situations combinatoires et offre un nouveau cadre à la représentation de certaines égalités.

Toute formalisation excessive est exclue. En particulier~:
\begin{itemize}
\item  parmi les propriétés du paragraphe a), les plus intuitives sont admises sans démonstration;
\item  l'utilisation systématique de bijections dans les problèmes de dénombrement n'est pas un attendu du programme.
\end{itemize}
\end{itshape}

\subsubsubsection{a) Cardinal d'un ensemble fini}
\begin{parcolumns}[rulebetween,distance=\parcoldist]{2}
  \colchunk{Cardinal d'un ensemble fini.}
  \colchunk{Notations $|A|$, $\card(A)$, $\# A$.}
  \colplacechunks

  \colchunk{Cardinal d'une partie d'un ensemble fini, cas d'égalité.}
  \colchunk{Tout fondement théorique des notions d'entier naturel et de cardinal est hors programme.}
  \colplacechunks

  \colchunk{Une application entre deux ensembles finis de même cardinal est bijective si et seulement si elle est injective, si et seulement si elle est surjective.}
  \colchunk{}
  \colplacechunks

  \colchunk{Cardinal d'un produit fini d'ensembles finis.}
  \colchunk{}
  \colplacechunks


  \colchunk{Cardinal de la réunion de deux ensembles finis.}
  \colchunk{La formule du crible est hors programme.}
  \colplacechunks

  \colchunk{Cardinal de l'ensemble des applications d'un ensemble fini dans un autre.}
  \colchunk{}
  \colplacechunks

  \colchunk{Cardinal de l'ensemble des parties d'un ensemble fini.}
  \colchunk{}
  \colplacechunks
\end{parcolumns}


\subsubsubsection{b) Listes et combinaisons}
\begin{parcolumns}[rulebetween,distance=\parcoldist]{2}
  \colchunk{Nombre de $p$-listes (ou $p$-uplets) d'éléments distincts d'un ensemble de cardinal $n$, nombre d'applications injectives d'un ensemble de cardinal $p$ dans un ensemble de cardinal $n$, nombre de permutations d'un ensemble de cardinal $n$.}
  \colchunk{}
  \colplacechunks

  \colchunk{Nombre de parties à $p$ éléments (ou $p$-combinaisons) d'un ensemble de cardinal $n$.}
  \colchunk{Démonstration combinatoire des formules de Pascal et du binôme.}
  \colplacechunks

\end{parcolumns}


\subsection{Probabilités}
\begin{itshape}
Ce chapitre  a pour objectif de consolider les connaissances  relatives aux probabilités sur un univers fini et aux variables aléatoires définies sur un tel univers présentées dans les classes antérieures. Il s'appuie sur le chapitre consacré au dénombrement.

Ce chapitre  a vocation à interagir avec l'ensemble du programme. Il se prête également à des activités de modélisation de situations issues de la vie courante ou d'autres disciplines.
\end{itshape}
\setcounter{subsubsection}{0}
\subsubsection{A - Probabilités sur un univers fini}
\begin{itshape}
 Les définitions sont motivées par la notion d'expérience aléatoire.
La modélisation de situations aléatoires simples fait partie des capacités attendues des étudiants.
\end{itshape}

\subsubsubsection{a) Expérience aléatoire et univers}
\begin{parcolumns}[rulebetween,distance=2.5cm]{2}
  \colchunk{L'ensemble des issues (ou résultats possibles ou réalisations) d'une expérience aléatoire est
appelé univers.}
  \colchunk{On se limite au cas où cet univers est fini.}
  \colplacechunks

  \colchunk{\'Evénement, événement élémentaire (singleton), événement contraire, événements \og $A$ et $B$ \fg, évènement \og$A$ ou $B$ \fg, événement impossible, événements incompatibles, système complet d'événements.}
  \colchunk{}
  \colplacechunks
\end{parcolumns}

\subsubsubsection{b) Espaces probabilisés finis}
\begin{parcolumns}[rulebetween,distance=2.5cm]{2}

  \colchunk{Une probabilité sur un univers fini $\Omega$ est une application $P$ de $\mathcal{P} (\Omega)$ dans $[ 0 , 1 ]$ telle que $P (\Omega) = 1$ et, pour toutes parties disjointes $A$ et $B$, $P (A \cup B) = P (A) + P (B).$}
  \colchunk{Un espace probabilisé fini est un couple $(\Omega , P)$ où $\Omega$ est un univers fini et $P$ une probabilité sur $\Omega$.}
  \colplacechunks

  \colchunk{Détermination d'une probabilité par les images des singletons.}
  \colchunk{}
  \colplacechunks
  
  \colchunk{Probabilité uniforme.}
  \colchunk{}
  \colplacechunks

  \colchunk{Propriétés : probabilité de la réunion de deux événements, de l'événement contraire, croissance.}
  \colchunk{}
  \colplacechunks
\end{parcolumns}

\subsubsubsection{c) Probabilités conditionnelles}
\begin{parcolumns}[rulebetween,distance=2.5cm]{2}
  \colchunk{Si $P (B) > 0$, la probabilité conditionnelle de $A$ sachant $B$ est définie par : $P (A|B) = P_B (A) = \dfrac{P (A \cap B)}{P (B)}$.}
  \colchunk{On justifiera cette définition par une approche heuristique fréquentiste.

L'application $P_B$ est une probabilité.}
  \colplacechunks

  \colchunk{Formule des probabilités composées.}
  \colchunk{}
  \colplacechunks

  \colchunk{Formule des probabilités totales.}
  \colchunk{}
  \colplacechunks

  \colchunk{Formules de Bayes :
\begin{enumerate}
\item
si $A$ et $B$ sont deux événements tels que $P(A)>0$ et $P(B)>0$, alors
$$P(A\,|\,B)=\frac{P(B\,|\,A)\,P(A)}{P(B)}$$
\item
si $(A_i)_{1\leqslant i\leqslant n}$ est un système complet d'événements de probabilités
non nulles et si $B$ est un événement de probabilité non nulle, alors
$$P(A_j\,|\,B)= \frac{P(B\,|\,A_j)\;P(A_j)}{\displaystyle{\sum_{i=1}^n P(B\,|\,A_i)\;P(A_i)}}$$
\end{enumerate}}
  \colchunk{On donnera plusieurs applications issues de la vie courante.}
  \colplacechunks
\end{parcolumns}

\subsubsubsection{d) \'Evénements indépendants}
\begin{parcolumns}[rulebetween,distance=2.5cm]{2}
  \colchunk{Couple d'événements indépendants.}
  \colchunk{Si $P (B) > 0$, l'indépendance de $A$ et $B$ s'écrit $P (A|B) = P (A)$.}
  \colplacechunks

  \colchunk{Famille finie d'événements mutuellement indépendants.}
  \colchunk{L'indépendance deux à deux des événements d'une famille $(A_1,\dots,A_n)$ n'implique pas l'indépendance mutuelle si $n \geq 3$.}
  \colplacechunks
\end{parcolumns}

\subsubsection{B - Variables aléatoires sur un espace probabilisé fini}
\begin{itshape}
L'utilisation de variables aléatoires pour modéliser des situations aléatoires simples fait partie des capacités attendues des étudiants. La reconnaissance de situations modélisées par les lois usuelles est une capacité attendue des étudiants.
\end{itshape}

\subsubsubsection{a) Variables aléatoires}
\begin{parcolumns}[rulebetween,distance=2.5cm]{2}
  \colchunk{Une variable aléatoire est une application définie sur l'univers $\Omega$ à valeurs dans un ensemble $E$. Lorsque $E\subset\R$, la variable aléatoire est dite réelle.}
  \colchunk{Si $X$ est une variable aléatoire et si $A$
est une partie de $E$, notation $\{X\in A\}$ ou $(X \in A)$ pour l'événement $X^{-1}(A)$.

Notations $P(X\in A)$, $P(X=x)$, $P(X\leqslant x)$.}
  \colplacechunks

  \colchunk{Loi $P_X$ de la variable aléatoire $X$.}
  \colchunk{L'application $P_X$ est définie par la donnée des $P(X=x)$ pour $x$ dans $X(\Omega)$.}
  \colplacechunks

  \colchunk{Image d'une variable aléatoire par une fonction, loi associée.}
  \colchunk{}
  \colplacechunks
\end{parcolumns}

\subsubsubsection{b) Lois usuelles}
\begin{parcolumns}[rulebetween,distance=2.5cm]{2}

  \colchunk{Loi uniforme.}
  \colchunk{}
  \colplacechunks

  \colchunk{Loi de Bernoulli de paramètre $p \in [ 0 , 1 ]$.}
  \colchunk{Notation $\mathcal{B} (p)$.

Interprétation : succès d'une expérience.

Lien entre variable aléatoire de Bernoulli et indicatrice d'un événement.}
  \colplacechunks

  \colchunk{Loi binomiale de paramètres $n \in \N^*$ et $p \in [ 0 , 1]$.}
  \colchunk{Notation $\mathcal{B} (n , p)$.

Interprétation : nombre de succès lors de la répétition de $n$ expériences de Bernoulli indépendantes, ou tirages avec remise dans un modèle d'urnes.}
  \colplacechunks
\end{parcolumns}

\subsubsubsection{c) Couples de variables aléatoires}
\begin{parcolumns}[rulebetween,distance=2.5cm]{2}
  \colchunk{Couple de variables aléatoires.}
  \colchunk{}
  \colplacechunks

  \colchunk{Loi conjointe, lois marginales d'un couple de variables aléatoires.}
  \colchunk{La loi conjointe de $X$ et $Y$ est la loi de $(X , Y)$, les lois marginales de $(X , Y)$ sont les lois de $X$ et de $Y$.

Les lois marginales ne déterminent pas la loi conjointe.}
  \colplacechunks

  \colchunk{Loi conditionnelle de $Y$ sachant $(X=x)$.}
  \colchunk{}
  \colplacechunks

  \colchunk{Extension aux $n$-uplets de variables aléatoires.}
  \colchunk{}
  \colplacechunks  
\end{parcolumns}

\subsubsubsection{d) Variables aléatoires indépendantes}
\begin{parcolumns}[rulebetween,distance=2.5cm]{2}
  \colchunk{Couple de variables aléatoires indépendantes.}
  \colchunk{}
  \colplacechunks

  \colchunk{Si $X$ et $Y$ sont indépendantes : $$P \big( (X , Y) \in A \times B \big) = P (X \in A) \ P (Y \in B).$$}
  \colchunk{}
  \colplacechunks

  \colchunk{Variables aléatoires mutuellement indépendantes.}
  \colchunk{Modélisation de $n$ expériences aléatoires indépendantes par une suite finie $(X_i)_{1\leqslant i\leqslant n}$ de variables aléatoires indépendantes.}
  \colplacechunks

  \colchunk{Si $X_1,\ldots,X_n$ sont des variables aléatoires mutuellement indépendantes, alors quel que soit $\displaystyle (A_1,\ldots,A_n) \in \prod_{i=1}^n \mathcal{P}(X_i(\Omega))$, les événements $(X_i\in A_i)$ sont mutuellement indépendants.}
  \colchunk{}
  \colplacechunks  

  \colchunk{Si $X_1, \ldots, X_n$ sont mutuellement indépendantes de loi $\mathcal{B} (p)$, alors $X_1 + \cdots + X_n$ suit la loi $\mathcal{B} (n , p)$.}
  \colchunk{}
  \colplacechunks  

  \colchunk{Si $X$ et $Y$ sont indépendantes, les variables aléatoires $f (X)$ et $g (Y)$ le sont aussi.}
  \colchunk{}
  \colplacechunks  
\end{parcolumns}

\subsubsubsection{e) Espérance}
\begin{parcolumns}[rulebetween,distance=2.5cm]{2}
  \colchunk{Espérance d'une variable aléatoire réelle.\newline
  Relation : $\displaystyle \quad \textrm{E} (X) = \sum_{\omega \in \Omega} P \big( \{ \omega \} \big) X (\omega)$.}
  \colchunk{Interprétation en terme de moyenne pondérée.
  
  Une variable aléatoire centrée est une variable aléatoire d'espérance nulle.}
  \colplacechunks

  \colchunk{Propriétés de l'espérance : linéarité, positivité, croissance.}
  \colchunk{}
  \colplacechunks

  \colchunk{Espérance d'une variable aléatoire constante, de Bernoulli, binomiale.}
  \colchunk{}
  \colplacechunks  

  \colchunk{Formule de transfert : Si $X$ est une variable aléatoire définie sur $\Omega$ à valeurs dans $E$ et f une fonction définie sur $X(\Omega)$ à valeurs dans $\R$, alors 
  \begin{displaymath}
   \textrm{E} \big( f (X) \big) = \sum_{x \in X (\Omega)} P (X = x) f (x)
  \end{displaymath}
  }
  \colchunk{L'espérance de $f (X)$ est déterminée par la loi de $X$.}
  \colplacechunks  

  \colchunk{Inégalité de Markov.}
  \colchunk{}
  \colplacechunks  

  \colchunk{Si $X$ et $Y$ sont indépendantes : $\quad \textrm{E} (XY) = \textrm{E} (X) \textrm{E} (Y)$.}
  \colchunk{La réciproque est fausse en général.}
  \colplacechunks
\end{parcolumns}

\subsubsubsection{f) Variance, écart type et covariance}
\begin{parcolumns}[rulebetween,distance=2.5cm]{2}
  \colchunk{Moments.}
  \colchunk{Le moment d'ordre $k$ de $X$ est $\textrm{E} (X^k)$.}
  \colplacechunks

  \colchunk{Variance, écart type.}
  \colchunk{La variance et l'écart type sont des indicateurs de dispersion. Une variable aléatoire réduite est une variable aléatoire de variance 1.}
  \colplacechunks

  \colchunk{Relation $\textrm{V}(X)=\textrm{E} (X^2) - \textrm{E} (X)^2$.}
  \colchunk{}
  \colplacechunks

  \colchunk{Relation $\quad \textrm{V} (aX+b) = a^2 \textrm{V} (X)$.}
  \colchunk{Si $\sigma (X) > 0$, la variable aléatoire $\dfrac{X - \textrm{E} (X)}{\sigma (X)}$ est centrée réduite.}
  \colplacechunks  

  \colchunk{Variance d'une variable aléatoire de Bernoulli, d'une variable aléatoire binomiale.}
  \colchunk{}
  \colplacechunks  

  \colchunk{Inégalité de Bienaymé-Tchebychev.}
  \colchunk{}
  \colplacechunks  

  \colchunk{Covariance de deux variables aléatoires.}
  \colchunk{}
  \colplacechunks  

  \colchunk{Relation $\mbox{Cov}(X,Y)= \textrm{E} (XY) - \textrm{E} (X) \textrm{E} (Y)$.  Cas de variables indépendantes.}
  \colchunk{}
  \colplacechunks  

  \colchunk{Variance d'une somme, cas de variables deux à deux indépendantes.}
  \colchunk{Application à la  variance d'une variable aléatoire binomiale.}
  \colplacechunks  

\end{parcolumns}


\end{document}