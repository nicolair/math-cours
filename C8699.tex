\input{courspdf.tex}
\debutcours{Autour du formulaire de trigonométrie}{alpha}

\section{Formulaire}
\index{formulaire de trigonométrie circulaire}\index{formulaire de trigonométrie hyperbolique}
Dans le formulaire présenté sur la page suivante, $a, b, x$ sont réels et $n$ entier relatif.
\twocolumn
\begin{center}
 Trigonométrie circulaire
\end{center}
Définitions
\begin{displaymath}
  \cos a  = \frac{1}{2}(e^{ia}+e^{-ia}) \hspace{0.5cm} \sin a  = \frac{1}{2i}(e^{ia}-e^{-ia})
\end{displaymath}
\begin{displaymath}
  \cos a + i \sin a  = e^{ia} \hspace{0.5cm}\text{(Euler)}
\end{displaymath}
Module
\begin{displaymath}
 (\cos a)^2 + (\sin a)^2 = 1 \hspace{0.5cm} 1+\tan^2a = \frac{1}{\cos^2 a}
\end{displaymath}
Autour de $e^{z+z'}=e^{z}e^{z'}$
\begin{align*}
 (\cos a + i \sin a)^n &= \cos(na) + i \sin(na)\;\text{ (Moivre)}\\
 \cos(a+b) &= \cos a \cos b  - \sin a \sin b\\
 \sin(a+b) &= \sin a \cos b + \cos a \sin b \\
 \tan(a+b) &= \frac{\tan a + \tan b}{1-\tan a \tan b}\\
 \cos(2a) &= \cos^2 a -\sin^2 a \\
          &=2\cos^2 a -1 = 1-2\sin^2a\\
 \sin(2a) &= 2\sin a \cos a \\
 \tan(2a) &= \frac{2\tan a }{1-\tan^2 a}
\end{align*}
Symétries

$\sin$ et $\cos$ sont $2\pi$ périodiques.

$\sin$ impaire, $\cos$ paire.
\begin{align*}
  &\cos(a+\pi) = -\cos a & & \cos (\pi-a) = -\cos a \\
  &\sin(a+\pi) = -\sin a & & \sin (\pi-a) = \sin a \\
  &\tan(a+\pi) = \tan a & & \tan (\pi-a) = -\tan a \\
  &\cos(\frac{\pi}{2}-a) = \sin a & & \cos(\frac{\pi}{2}+a) = -\sin a\\ 
  &\sin(\frac{\pi}{2}-a) = \cos a & & \sin(\frac{\pi}{2}+a) = \cos a\\ 
  &\tan(\frac{\pi}{2}-a) = \frac{1}{\tan a } & & \tan(\frac{\pi}{2}+a) = -\frac{1}{\tan a}\\ 
\end{align*}
Expression avec la tangente de l'arc moitié.\newline
Pour $a\not \equiv \pi \mod (2\pi)$ et $t = \tan \frac{a}{2}$
\begin{align*}
 \cos a  = \frac{1-t^2}{1+t^2} & &
  \sin a = \frac{2t}{1+t^2} & &
  \tan a = \frac{2t}{1-t^2} 
\end{align*}
Produit $\rightarrow$ Somme (linéarisation)
\begin{align*}
 &\cos a \cos b = \frac{1}{2}\left(\cos(a-b) +\cos(a+b)\right)\\
 &\sin a \sin b = \frac{1}{2}\left(\cos(a-b)-\cos(a+b) \right)\\
 &\sin a \cos b = \frac{1}{2}\left(\sin(a-b)+\sin(a+b) \right) \\
 &\cos^2 a = \frac{1}{2}+\frac{1}{2}\cos(2a)\\
 &\sin^2 a = \frac{1}{2}-\frac{1}{2}\cos(2a)
\end{align*}
Somme $\rightarrow$ Produit (factorisation)
\begin{align*}
 &\cos a + \cos b = 2\cos \frac{a+b}{2} \cos \frac{a-b}{2} \\  
 &\cos a - \cos b = -2\sin \frac{a+b}{2} \sin \frac{a-b}{2} \\
 &\sin a + \sin b = 2\sin \frac{a+b}{2} \cos \frac{a-b}{2}
\end{align*}
\'Equations
\begin{align*}
 &\sin x = 0 &\Leftrightarrow x \equiv 0 \mod(\pi) \\
 &\cos x = 0 &\Leftrightarrow x \equiv \frac{\pi}{2} \mod(\pi) \\
 &\sin x = \sin a &\Leftrightarrow
\left\lbrace 
\begin{aligned}
 &x \equiv a \mod(2\pi)\\
&\text{ ou }\\
 &x \equiv \pi - a \mod(2\pi)
\end{aligned}
\right. \\
 &\cos x = \cos a &\Leftrightarrow
\left\lbrace 
\begin{aligned}
 &x \equiv a \mod(2\pi)\\
&\text{ ou }\\
 &x \equiv  - a \mod(2\pi)
\end{aligned}
\right. 
\end{align*}

\begin{center}
\hrulefill \\ 
 Trigonométrie hyperbolique
\end{center}

$\ch$ est la partie \emph{paire} de l'exponentielle réelle.

$\sh$ est la partie \emph{impaire} de l'exponentielle réelle.

\begin{align*}
 &\ch a = \frac{1}{2}\left( e^a + e^{-a}\right)  & & \sh a = \frac{1}{2}\left( e^a - e^{-a}\right) \\
 &e^{a} = \ch a + \sh a &  &e^{-a} = \ch a - \sh a
\end{align*}

\begin{displaymath}
 (\ch a)^2 - (\sh a)^2 = 1
\end{displaymath}

\begin{align*}
 &\ch(a+b) = \ch a \ch b  + \sh a \sh b\\
 &\sh(a+b) = \sh a \ch b + \ch a \sh b \\
 &\th(a+b) = \frac{\th a + \th b}{1+\th a \th b}\\
 &\ch(2a) = (\ch a)^2 + (\sh a)^2 \\
 &\sh(2a) = 2\sh a \ch a \\
 &\th(2a) = \frac{2\th a }{1+\th^2 a}
\end{align*}
\onecolumn
%\clearpage

\section{Commentaires}
\subsection{Trigonométrie circulaire}
\subsubsection{Tangente}
\begin{defi} \index{Fonction tangente: définition}
  La fonction tangente est définie dans $\R \setminus (\frac{\pi}{2} +\pi\Z)$ par :
  \begin{displaymath}
    \forall x \in \R, x \not\equiv \frac{\pi}{2} \mod(\pi): \; \tan x = \frac{\sin x}{\cos x} 
  \end{displaymath}
\end{defi}
\begin{displaymath}
  \cos^2 x + \sin^2 x = 1 \Rightarrow \cos^2 x\left( 1+ \tan^2\right) = 1 
\end{displaymath}

\subsubsection{Formule fondamentale}
Pour des réels $a$ et $b$, on peut exprimer $a\cos t + b\sin t$ comme une fonction trigonométrique avec une amplitude.
\begin{displaymath}
 a\cos t + b\sin t = A \cos(t-\varphi)
 \end{displaymath}
avec $\varphi$ défini modulo $2\pi$ par 
\begin{displaymath}
 \cos \varphi = \frac{a}{\sqrt{a^2 + b^2}}, \hspace{1cm} \sin \varphi = \frac{b}{\sqrt{a^2 + b^2}}
\end{displaymath}


\subsubsection{Valeurs particulières et symétries}
Utiliser la formule fondamentale $e^{z+z'} = e^{z}\, e^{z'}$ et les valeurs particulières
\begin{displaymath}
  e^{2i\pi} = 1, \; e^{i\pi} = -1, \; e^{i\frac{\pi}{2}} = i
\end{displaymath}

\subsubsection{Tangente de l'arc moitié}
\subsubsection{Produit en somme}
\subsubsection{Somme en produit - Linéarisation}
\index{Linéarisation}
\subsubsection{Somme en produit}
\subsubsection{\'Equations}
\subsubsection{Expressions polynomiales}
Expressions polynomiales de $\cos(nt)$ et $\sin(nt)$ en fonction de $\cos(t)$ et $\sin(t)$. Soit directement par la formule du binôme soit par  récurrence avec les formules de transfo de somme en produit.

\subsection{Trigonométrie hyperbolique}
\subsubsection{Définitions}
\subsubsection{Formules}
\end{document}
